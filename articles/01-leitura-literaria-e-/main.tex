\begin{refsection}
    \renewcommand{\thefigure}{\arabic{figure}}
    
    \chapterOneLine
    {Leitura literária e o leitor na era digital}
    \label{chap:leitura}
    
    \articleAuthor
    {Delcimar Francisco de Medeiros}
    {Professor licenciado em Letras (UFRN), especialista em Literatura (UnP), especialista em Língua Portuguesa pelo Instituto de Educação Superior Presidente Kennedy --- IFESP, professor nas redes pública e particular de ensino em Natal-RN. E-mail: dfmedeiros2013@gmail.com. }

    \articleAuthor
    {Arandi Róbson Martins Câmara }
    {Graduado em Letras (UNP). Mestre em Estudos da Linguagem (UFRN). Doutorando em Educação (UFRN). Professor de Língua Portuguesa no IFESP. ID Lattes: 6358.7595.5311.1379. ORCID: 0000-0002-2019-256X. E-mail: arandi@ifesp.edu.br.}
    
    \begin{galoResumo}
        \marginpar{
            \begin{flushleft}
            \tiny \sffamily
            Como referenciar?\\\fullcite{SelfMedeirosAndCamara2021Leitura}\mybibexclude{SelfMedeirosAndCamara2021Leitura}, p. \pageref{chap:leitura}--\pageref{chap:leituraend}, \journalPubDate{}
            \end{flushleft}
        }
        No contexto da cultura digital, esta pesquisa teve como objetivo investigar a leitura literária nos ambientes tecnológicos da informação e as práticas de leitura na formação do leitor na perspectiva do letramento digital e literário. Para tanto, o campo teórico foi o dos estudos da cibercultura aplicados à literatura. Outrossim, a metodologia utilizada foi a bibliográfica, que objetivou conhecer as diferentes contribuições sobre a leitura na era tecnológica. Para isso, na fundamentação teórica, observaram-se as mudanças dos hábitos de leitura dos leitores e os efeitos das tecnologias para a literatura \cite{Ong1996Oralidad, Levy1999Cibercultura, Marcuschi2001Hipertexto}, a literatura na era digital \cite{Xavier2009Era, Almeida2005Letramento}, e as novas práticas de leitura e o perfil do novo leitor literário \cite{Prensky2001Nativos, Santaella2004Navegar}. Foi possível concluir que, no conjunto da observação, o pleno desenvolvimento de um leitor crítico e imersivo, no que tange aos conjuntos do letramento digital e literário, só podem de ser edificadas perante o entrelaçamento por parte dos potenciais de dispositivos móveis e de objetos técnicos mais adequados, como leitores digitais, considerando uma imersão efetiva e consciente na cultura digital e na cibercultura, sem os quais os procedimentos do leitor crítico e do leitor imersivo se tornam precisos. 
    \end{galoResumo}
    
    \galoPalavrasChave{Leitura literária. Cibercultura. Cultura digital. Leitor crítico. Leitor imersivo.}
    
    \begin{otherlanguage}{english}
    
    \fakeChapterOneLine
    {Literary reading and the reader in the Digital Age}
    
    \begin{galoResumo}[Abstract]
        In the context of digital culture, this research investigates literary reading in information technology environments, and reading practices in reader formation from the perspective of digital and traditional literacy. For that, the theoretical field was studies of cyberculture applied to literature. Moreover, the methodology used was the bibliographical one, which aimed to know the different contributions on reading in the Information Era. For this, in the theoretical fundaments, we observed the changes in readers' reading habits and the effects of technologies on literature \cite{Ong1996Oralidad, Levy1999Cibercultura, Marcuschi2001Hipertexto}, literature in the digital age \cite{Xavier2009Era, Almeida2005Letramento}, and the new reading practices and new literary readers profile \cite{Prensky2001Nativos, Santaella2004Navegar}. It was possible to conclude that, in the whole observation, the full development of a critical and immersive reader, as far as the digital and traditional literacy ensembles are concerned, can only be built up in the interweaving by the potential of mobile devices and technical objects as digital readers, considering an effective and conscious immersion in digital culture and cyberculture, without which the procedures of the critical reader and the immersive reader become accurate. 
    \end{galoResumo}
    
    \galoPalavrasChave[Keywords]{Literary reading. Cyberculture. Digital culture. Critical reader. Immersive reader.}
    \end{otherlanguage}
    
    % \flourish
    
    \section{Introdução}

    O nosso trabalho é uma pesquisa bibliográfica que tem como objetivo investigar as possibilidades de associação entre as tecnologias digitais e a literatura e suas práticas de leitura, tema no qual se circunstancia a pesquisa de que tratamos aqui. Desse modo, busca-se entender as novas práticas de leitura e o papel do leitor na era digital. 

    O objeto deste estudo é evidenciar o surgimento de uma nova forma de ler que está ambientada na tecnologia. Diante do que foi lido e pesquisado, percebemos que surgiram vários tipos de ferramentas de leitura, uma vez que o hábito dos novos leitores é bem diferente dos leitores tradicionais. 

    Neste trabalho, ainda podemos observar as transformações que ascenderam com o surgimento da literatura cibernética: a configuração, a forma de escrever do autor e o leitor. Outrossim, a delineação do ciberespaço, uma ambiência virtual em que autor e leitor se misturam. 

    Em cada página que aqui escrevemos, percebemos que existem muitos questionamentos sobre a nova forma de ver o livro na era digital, assim como a maneira de ler em plataformas cibernéticas. Diante disto, as várias análises feitas diante do que foi estudado tenta responder às indagações: como é possível apresentar as transformações dos textos antigos em formatos digitais? Será possível fazer indagações assim que pensamos na virtualização do livro e que o leitor virtual já é uma realidade? 

    Supomos que a realidade literária terá no futuro um desenho bem diferente do contexto atual, ou seja, o livro será eletrônico, consequentemente o leitor se situará em um ambiente de leitura cibernético e virtual. 

    Para abranger melhor o tema em questão, descobrimos caminhos para responder às indagações na pesquisa, apoiamo-nos em alguns teóricos como \textcite{Ong1996Oralidad}, \textcite{Marcuschi2001Hipertexto}, \textcite{Santaella2004Navegar}, \textcite{Gomes2009Nativos}, \textcite{Levy1996Virtual, Levy1999Cibercultura}, \textcite{Ramos2015Fazer}, \textcite{Xavier2009Era}, \textcite{PalfreyAndGasser2011Nascidos}, dentre outros, que por meio de textos e de ponderações, permitiram a investigação, assim como a compreensão do mundo literário nos ambientes virtuais.   

    A fim de nos atentarmos às particularidades da literatura do futuro, procuramos compor este trabalho em quatro partes: a primeira apresenta os principais elementos da literatura na era digital , confirmando de que maneira os textos são reconfigurados pelo emprego do suporte eletrônico; a segunda parte anuncia a cibercultura, a virtualização do ambiente de leitura na contemporaneidade, os cibertextos, focalizando, sobretudo, a adaptação da página impressa para a tela do computador; na terceira, expõem-se exemplos de hipertextos, o texto digital e o contexto, os quais privilegiam a interação e as distintas condições de cooperação do leitor e a quarta analisa o leitor no mundo digital: um novo leitor e finalizando o artigo, o texto traz as considerações finais.

    \section{A literatura na era digital}

    Ano a ano, a literatura digital ganha espaço e se torna inquietação para acadêmicos e professores, ansiosos por apresentar algo novo a seus alunos. Frequentemente, contudo, diante de algo sem parâmetros análogos na história da literatura, a apreciação dá lugar ao encantamento e sobram adjetivos para essa literatura multimídia produzida com o subsídio das novas tecnologias de comunicação, destacando-se o caráter "interativo" desse novo gênero.Alfredo \textcite{Carneiro2011Valor} argumenta que as novas tendências na literatura surgiram a fim de que ela se torne coletiva e afirma ainda que os textos clássicos expostos na internet podem ser lidos, apagados ou até transferidos.

    Acreditamos que o mundo sempre viveu a cultura do livro físico desde que ele surgiu. De tal modo no mundo contemporâneo ainda é comum comprar livros, recebê-los como presentes ou presenteá-los com dedicatória de familiares e de amigos.  Quando lemos a pesquisa de Alfredo Carneiro, observamos que ele assim menciona: 

    \begin{quotation}
        Existem claras vantagens na digitalização. Praticamente uma biblioteca inteira cabe em um pen-drive e novas tecnologias tem surgido tentando dar ao leitor a comodidade do livro, como os \textit{e-readers} e os tablets.  Se podemos baixar de graça o livro O Estrangeiro de Albert Camus e ler em nosso \textit{e-reader}, provavelmente não iremos em uma livraria para comprá-lo \cite[p.~1]{Carneiro2011Valor}.
    \end{quotation}

    Dessa forma, constatamos que a literatura sempre teve uma ligação com recursos tecnológicos. Diante disto, exemplifica-se a poesia oral que nas suas origens necessitava de uma estrutura baseada na repetição, com o objetivo não somente estético, como também nas práticas de memorização que facilitavam a compreensão. 

    Em seu livro \textit{Oralidade e Cultura}, Walter \textcite{Ong1996Oralidad} afirma que a escrita é uma tecnologia disponível para a literatura e sua comunicação. Entendemos que ele considera, assim, que um dos papéis da literacia atual passa por reconstruir a consciência humana primitiva, não letrada. 

    Nesse campo da comunicação literária, \textcite{MartinezArnaldos1990Lenguaje} acrescenta que a tecnologia teve uma acentuada relevância para que os textos literários atingissem um público mais diversificado, não só pela superação dos limites espaço-tempo, bem como pela retirada de obstáculos e de dificuldades na aquisição das obras literárias. Enfatiza, ainda, que muitos livros que já estão esgotados, dificilmente encontrados em livrarias e editoras, só podem ser lidos porque estão em bibliotecas e repertórios virtuais. 

    Além dele, Albaladejo afirma que “o papel das novas tecnologias da ciência digital ou da informática contribuiu decisivamente para a divulgação da literatura, como também contribuiu para a divulgação dos discursos retóricos” \cite[p.~9--18]{Albaladejo2001}.

    Isso tudo nos fez refletir sobre as várias indagações relacionadas à literatura digital, visto que o nosso planeta respira o futuro tecnológico. Diante dessa realidade, os dispositivos de leitura já estão acessíveis a fim de que qualquer pessoa possa fazer downloads de livros e de revistas baixados da internet por meio de aplicativos. Diante disso, pudemos analisar essa situação, porque essa mudança força as editoras a produzirem versões requintadas de \textit{best-sellers}, de manuais, de revistas em quadrinhos e de livros de literatura infanto-juvenil com a utilização de áudios, imagens e interatividade. Isso tem uma finalidade, a de se expandir a competência de absorção e compreensão de uma maneira que não seria admissível apenas com o texto impresso.

    Sabemos que o livro físico\footnote{Um objeto portátil, que é composto por \textit{páginas} encadernadas. Além disso, contém \textit{textos} impressos, \textit{imagens} na formação de uma publicação em unidades (ou que foi concebido como tal), sendo parte principal de um trabalho literário, científico ou outro, formando um volume.} nunca deixará de existir, apenas perderá a primazia, já que o livro digital\footnote{Arquivo que funciona no seu computador, tablet ou celular e que pode ser uma versão eletrônica de um livro que já foi impresso ou um texto original publicado apenas na forma digital. } terá uma maior preferência por parte das gerações vindouras de leitores. Embora isso seja evidente, asseguramos que haverá um público  fiel   ao  livro   impresso.  Essas   pessoas   ainda  gostam   de manusear  as páginas, sentir o cheiro do papel e de sua tinta mais que o uso dos dispositivos eletrônicos. 

    Para entender melhor esse contexto da ambiência virtual, recorremos às explicações do \textcite{Levy1999Cibercultura}. Segundo ele, os recursos tecnológicos poderão ser mais práticos  na criação de textos literários do que qualquer outra convencional. Pudemos observar que, em suas análises, ele afirma ser a plataforma digital uma ferramenta  de uma fecundidade  propriamente cultural, quer dizer, do surgimento de novos gêneros relacionados à interatividade. 

    Para tanto,  foi necessário entender o conceito de literatura digital\footnote{É explorar de forma possível ferramentas que surgiram com o desenvolvimento de tecnologias visuais e sonoras, como o vídeo, o computador e a edição eletrônica de textos. O principal incremento para esse tipo de literatura é a substituição do texto impresso pelo texto em tela, trazendo para a literatura as possibilidades de animação, relacionadas com o cinema e o vídeo. Ocorre, assim, uma integração entre elementos verbais, sonoros e visuais.} (ou eletrônica, conforme o termo é empregado em língua inglesa). Ela nasceu no ambiente digital, como um componente digital de primeira geração criado pelo uso de um computador e (na maioria das vezes) lido em uma tela eletrônica. Na era tecnológica na modernidade, noto que a interrelação das tecnologias computacionais com a literatura avança. Isso adquiriu uma extensão mensurável (em todos os sentidos) muito expressiva, dando possibilidades ao leitor da utilização de novas ferramentas digitais que lhe deram acesso à leitura de jornais, revistas, livros etc. por meios das telas de telefones celulares, \textit{desktops}, \textit{laptops}, \textit{tablets}, \textit{iPad}, \textit{e-readers} e de outras multimídias. Ademais, essa acessibilidade a blogs e a sites de literatura tornou-se produtiva, uma vez que muitos escritores divulgam seus textos nesses ambientes virtuais. Essa situação nos leva a crer que muitos autores utilizam a internet para produzir seus livros. Para isso, criam blogs pessoais onde o leitor interage com a sua obra, podendo mesmo opinar diretamente no processo de criação da história, tornando-se, assim, um coautor\footnote{Tem como relevante sentido da transformação e redimensionamento do espaço da recepção como espaço de interação e transformação, modificando os papéis de emissores e receptores para uma dinâmica relacional, de coautores/criadores.}.   

    \section{O novo ambiente de leitura virtual}

    É possível  entender a era digital na realidade atual a partir da cibercultura.  Ela é a cultura que surgiu, apareceu para o mundo e ainda permanecerá por um futuro longínquo, a partir do uso da internet e de outros recursos cibernéticos através da comunicação virtual, da indústria do entretenimento e do comércio eletrônico, no qual se configura o presente. Isso se confirma a partir do momento que começamos a observar a diversidade de vídeos no Youtube, além da prática de baixar filmes on-line antes de assistir no cinema. Assim, o mundo cibernético avança também no campo da criação literária: a  nova tendência pode ser reconhecida como literatura digital, literatura eletrônica ou cibernética. Além disso, esses diferentes instrumentos designam  a elaboração de obras literárias que são escritas e lidas dentro de um formato digital. 

    Bem claramente, deduzimos o que seja ciberliteratura também designada literatura algorítmica, generativa ou virtual. Denominam-se aqueles textos literários cuja elaboração baseia-se em procedimentos informáticos: combinatórios, multimidiáticos ou interativos. Nela, faz-se o uso das potencialidades do computador como instrumento criativo que propicia o desenvolvimento de estruturas textuais, em estado virtual, atualizando-as até ao infinito.  

    Dessa maneira, procuramos compreender de tal modo que a literatura hoje, diversamente do passado, é muito mais aberta às pessoas visto que ela se encontra acessível no quase infinito espaço cibernético. Na internet, geralmente podemos encontrar milhões de livros que estão disponíveis aos seus leitores virtuais, compreendendo desde os autores desconhecidos aos mais renomados. Ainda admite que os leitores virtuais leiam obras que frequentemente já não se encontram nas prateleiras das livrarias porque estão esgotadas nas editoras.  

    Para tanto, o certo é que consideramos que a literatura tenha adotado uma nova aparência. Diante da disparidade de ferramentas digitais que estão aparecendo em passo acelerado, há um procedimento eficaz de interação e concessão em que se insere a criação literária. 

    Consequentemente, o que entendemos do espaço virtual da ciberliteratura versa em oferecer as possibilidades para manipular a linguagem verbal, admitindo a utilização de recursos de imagens e de sons que nela podem ser inseridos. À vista disso, nesse ambiente cibernético defronto com novas ferramentas para textos, que me proporcionam o surgimento de um novo leitor e de uma nova linguagem.  

    \section{O hipertexto, o texto e as novas abordagens da leitura literária}

    Fizemos uma aprofundada leitura sobre as ponderações de Marcuschi quando em suas palavras se refere ao hipertexto\footnote{Forma de apresentação de informações em uma ferramenta eletrônica de tela, na qual algum elemento (palavra, expressão ou imagem) é destacado e, quando acionado, provoca a exibição de um novo hipertexto com informações relativas ao referido elemento; hipermídia. Ainda, é o termo que remete a um texto ao qual se adicionam outros conjuntos de informação no formato de blocos de textos, palavras, imagens ou sons, cujo acesso se faz por meio de referências exclusivas, no meio digital denominadas \textit{hiperligações}.}. Ele ilustra essa nova estrutura textual como sendo dois caminhos. Isso concede ao leitor a entrada a um número de modo indefinido de outros textos a partir de seleções locais e contínuas no tempo de execução de uma determinada tarefa independente da carga do sistema. Ainda menciona que “permite ao ledor definir interativamente o fluxo de sua leitura a partir de assuntos tratados no texto sem se prender a uma sequência fixa ou a tópicos estabelecidos por um autor” \cite[p.~86]{Marcuschi2001Hipertexto}. Para ele, o hipertexto se apresenta como um procedimento de escrever e ler de forma eletrônica multilinearizado, multisequencial e indeterminado, assim sendo, se realiza em um novo ambiente de escrita.  

    No nosso entender, o hipertexto está associado à pós-modernidade na elaboração de novas narrativas direcionadas ao gosto textual do leitor ou estético do autor. Links e ícones, imagens e efeitos sonoros compõem a interatividade do texto midiático, e o leitor, igualmente, pode escolher o caminho a seguir, dependendo da sua intenção ou alvo, clicando nos hiperlinks que o redirecionarão a alguma página de sua escolha.           

    Esse tipo de leitura se insere também no contexto da nova leitura literária. Mas como diferenciar um hipertexto de um texto impresso tradicional?   Primeiro, a questão não linear que não segue uma regra de começo, meio e   fim, e   pode ser compreendida, lida, feito em qualquer uma de suas estruturas e/ou momentos, ou seja, não há uma resolução ou caminho predefinido a ser adotado. O autor elege o decurso que será seguido, por outro lado, eu, como leitor, navegarei por vários caminhos requeridos no hipertexto sem a ressalva de uma sequência linear.   

    No hipertexto, o arranjo do texto não está amarrado em um eixo centralizado que apoia um conjunto que segue uma hierarquia estabelecida de conhecimentos secundários; assim estabelece que eu possa escolher e solicitar, também, quando decida tanto a ordem de acesso aos díspares segmentos disponibilizados no hipertexto, quanto o eixo coesivo que afere uma definição completa ao texto lido. Segundo, no texto de natureza convencional, há a predominância da linearidade, sequência lógica, por exemplo, que é um predicado das línguas maternas.  

    Há muito o leitor tradicional coexiste com um tipo de livro que amplia   as   ideias em forma linear, em um plano de perfil mental que se associam. Sob outra perspectiva, os leitores pensam conforme construções que ora podem ser lineares, ora não-lineares, porém agem ao mesmo tempo. A escrita não-sequencial do hipertexto admite imaginar um conhecimento que enlaça e profere elementos de natureza diversificada simultaneamente.  

    Outro ponto importante, que pude observar, foi a questão intertextual, visto que o hipertexto é um texto multíplice, que funde e se junta a inúmeros textos. Esses textos que são acessíveis de forma simultânea a um simples toque. Para isso, os links possibilitam uma viagem por múltiplos textos, cujo ajuntamento é determinado pelos programadores por meio de uma palavra. Nisso, o link é uma ponte, um encontro entre produções textuais diferenciadas que propicia o fim das rigorosas fronteiras entre os textos. Assim, os autores dos hipertextos facilitam a leitura/navegação e atraem o leitor para construção ativa do seu próprio caminho. 

    Nesse sentido, o autor argumenta que: 

    \begin{quotation}
        É muito difícil pensar na produção de um texto totalmente inédito, criado a partir do nada. É como se todo texto fosse um hipertexto que possui links explícitos e implícitos com outros. E isso não acontece apenas na modalidade escrita da língua, mas também na oralidade. O fenômeno de intertextualidade pode se dar entre diferentes tipos de textos de uma mesma linguagem (um artigo e uma poesia, por exemplo) e entre textos de diferentes linguagens (um romance e um filme, por exemplo) \cite[p.~11]{Nicola2011Painel}. 
    \end{quotation}

    Desse modo \textcite{Xavier2009Era} também explica:

    \begin{quotation}
        Logo, não seria exagero afirmar que o hipertexto invadiu irreversivelmente a nossa vida. Na Era do Hipertexto, quem resistir a viver sem ele “já era”, ou pelo menos, terá dificuldades de inserção social e profissional. [\dots] Temos que o conhecê-lo cada vez mais para tirar-lhe o máximo do seu potencial comunicativo, socializador, educacional e humano que espera por nossa exploração. É necessário começarmos a dominá-lo, pois o hipertexto é um ponto de partida sem porto de chegada; quanto mais tentamos atravessar suas camadas, mas ele se nos mostra como um novelo infinitamente desdobrável. O desafio está lançado. \cite[p.~17]{Xavier2009Era}.
    \end{quotation}

    Diante do que foi pesquisado sobre a literatura digital e a relação texto e hipertexto, \textcite{Correa2006Literatura} em seu artigo analisa as relações autor/texto/leitor. Ela nos oportuniza entender que essas relações sofreram amplas alterações no decorrer do tempo. Isso teve como consequência as mudanças sociais, históricas e tecnológicas. Acreditamos que, diante do que é apresentado pela autora, muito há de se discutir e descobrir sobre o trinômio autor/texto/leitor na era da informática.  Para ela, os estudos ainda principiam enquanto buscam teorias que embasem, sobretudo, as análises intrínsecas ao hipertexto e em discussões a respeito desses três principais elementos do fazer literário.

    \section{A leitura literária e o novo leitor}

    Nos últimos tempos, contemplamos a disseminação da internet, uma vez que todo tipo de texto escrito passou a ser mais acessível a um maior número de pessoas. Isso permite que pessoas usem seus computadores, tablets ou celulares em conexões à rede mundial e tenham acesso aos mais variáveis gêneros textuais.  

    Em seus textos, \textcite{Ramos2015Fazer} evidencia que a atual Sociedade do Conhecimento é caracterizada por uma mudança constante, fruto de uma revolução tecnológica sem precedentes. Essa situação veio mudar a forma como as pessoas vivem, trabalham, estudam e se divertem. Nisso a promoção da leitura continua a ser vista como um investimento no capital humano.  

    Nesse caso, é importante que se façam leitores capazes de interagirem com a leitura e de terem um posicionamento crítico diante do mundo em que vivem.  Isso sugere, porém, que esse novo leitor esteja consciente de que a promoção da leitura se enquadre num certo contexto temporal e cultural que determina o tipo de leitor que é necessário cativar. \textcite{Ramos2015Fazer} acrescenta que estamos vivendo um novo modelo de cultura, da impressa para a virtual. Nisso, segundo ela, o texto passa a ser observado como uma unidade de comunicação com diferentes formas de expressão, que ainda nos parece de certa forma, insensato limitar o acesso à leitura prazerosa ao suporte impresso ou exclusivamente ao formato literário. Desse modo, não podemos desconhecer o domínio que as tecnologias desempenham sobre o atual leitor, um leitor virtual.  

    Essa nova perspectiva nos leva assim a cultivar os novos gêneros literários que apareceram com a informática, tais quais a poesia eletrônica, narrativas hipertextuais, experiências textuais combinatórias e inúmeros outros. Como exemplo, temos o poema eletrônico que é um procedimento literário atual, que sua produção e ampla divulgação estão sujeitas ao aparato do computador e da internet.           

    Eco destacou   a seguinte afirmação: 

    \begin{quotation}
        É verdade que os objetos literários são imateriais apenas pela metade, pois encarnam-se em veículos que, de hábito, são de papel. Mas houve um tempo em que se incorporavam na voz de quem recordava uma tradição oral ou mesmo em pedra e hoje discutimos sobre o futuro do e-book, que permitiriam ler seja uma coletânea de piadas, seja a Divina Comédia em uma tela de cristal líquido. [\dots] Pertenço, naturalmente, àqueles que, um romance ou um poema, preferem lê-lo em um volume de papel, do qual haverei de recordar até mesmo até mesmo as orelhas e o peso. Dizem, porém, que existe uma geração de hackers que, nunca tendo lido um livro na vida, com o e-book conheceram e provaram agora, pela primeira vez, o Dom Quixote. Quanto proveito para suas mentes e quanta perda para sua vista. Se as gerações futuras chegarem a ter uma boa relação (psicológica ou física) com e-book, o poder de Dom Quixote não mudará. \cite[p.~9--10]{Eco2003Sobre}.
    \end{quotation}


    Isso é uma das vantagens que se vê nos novos dispositivos eletrônicos, tais como smartphones e tablets, que são utilizados também para leitura, justamente a possibilidade de realização de diversas tarefas ao mesmo tempo através de um número incontável de aplicativos. Além do mais, o computador apresenta-se como atrativo em alguns aspectos, especialmente para o jovem leitor de hoje que apresenta um aspecto diferenciado do leitor de antigamente 

    De tal modo, podemos contemplar a tecnologia da computação que veio colaborar para ampliar os estágios de leitura. Até adota um desempenho destacado na formação de sujeitos que têm habilidades com a leitura na tela. Essa sensação do novo os deixa cativados pela nova roupagem literária.  

    A verdade é que mundo está vivendo uma revolução. Esta que pode ser chamada de Revolução Digital, porque abrange níveis tão intensos e vastos da sociedade do nosso planeta que quem sabe nunca se tenha uma compreensão real da grandeza destas mudanças. Tudo isso acontece numa Era em que a humanidade está atingindo um novo plano de evolução cognitiva. E nas crianças e nos jovens digitais isso fica mais evidenciado.  

    \textcite{Prensky2001Nativos} publicou no seu artigo "Nativos Digitais, Imigrantes Digitais", que nos explica esses termos como uma  maneira de entender as profundas diferenças entre os jovens de hoje e muitos de seus antepassados. Escreveu ainda que, embora muitos tenham encontrado os termos úteis, à medida que a humanidade avança no século, quando tudo crescerá na era da tecnologia digital, a distinção entre nativos digitais e os imigrantes digitais tornar-se-á menos relevante. Claramente, enquanto a sociedade trabalha para criar e melhorar o futuro, precisa imaginar um novo conjunto de distinções. Sugeriu, também que, todos pensem em termos de sabedoria digital.   

    Além disto, \textcite{Gomes2009Nativos} nos explica em seu artigo que os nativos digitais são pessoas que nasceram na era cibernética, num mundo globalizado, caracterizados por terem nascido e crescido com as Tecnologias da Informação  e Comunicação. Exemplifica que as pessoas, que nasceram entre 1965 e 1981, pertencem à geração X, considerados imigrantes digitais. Nos ambientes virtuais, se socializam, se anunciam de forma criativa e compartilham pensamentos e novidades. Os autores \textcite[p.~13]{PalfreyAndGasser2011Nascidos}, “caracterizam os colonizadores digitais como pessoas mais velhas, as quais estão desde o início da era digital, mas cresceram em um mundo analógico”. Para eles, essas pessoas colaboram para o desenvolvimento tecnológico e continuam conectadas e sofisticadas no uso das tecnologias, porém baseados nas formas tradicionais e analógicas da interação.  

    Do mesmo modo, surgiu a geração \textit{Millenium}, os nascidos posteriormente a 1980 até a década de 1990. De mais a mais, emerge a geração na segunda metade da década de 1990 até a contemporaneidade. Esses Nativos digitais vivem a explosão das redes sociais e o compartilhamento de ideias, gostos e produtos, manuseando computadores, celulares e notebooks.  

    Portanto, para nós, é nesse panorama que vão desenvolver-se as gerações que viverão até o final do século 21. Eles nasceram numa realidade intercedida pelas tecnologias digitais, que admitem acesso abundoso à informação, comunicação e lazer a qualquer tempo e em qualquer lugar. Não reconhecem outros parâmetros e não compreendem restrições, na escola, em casa ou em qualquer outro lugar, ao manuseio dos smartphones e outras parafernálias digitais que promovem suas vidas.  

    Depois de exaustivas reflexões sobre leitura na era cibernética, chegamos à conclusão que é necessário desfrutar do ato de ler, seja como for.  Para tanto, \textcite{Santaella2004Navegar} apresenta três tipos de leitores: o meditativo, o fragmentado e o imersivo. Diante do assunto discutido até então, escolhi o leitor imersivo que está a todo tempo preparado para coletar e ler novas informações. Ele traça seu próprio caminho em navegações alineares ou multilineares, pois navega por várias dimensões de assuntos por meio dos nós que as une, que pode ter uma leitura que não tem fim. Ainda encruza os dados com outros textos, os compara e gera um terceiro ou um quarto conteúdo.  

    Como também, quando caracteriza o navegador imersivo, em três níveis diferentes de raciocínio: o adutivo, o indutivo e o dedutivo, Santaella explica a aptidão da mente humana que consegue delinear caminhos, bastante percorridos pela arte do imaginário. Ela ilustra que a hipertextualidade conformou um estilo novo de leitura que é de fato uma escritura, pois as consistências vão sendo associadas pelo leitor-produtor.  

    Na realidade atual, todos convivem com a insurreição tecnológica moderna, que transformou o cotidiano das pessoas, e intervém absolutamente na sociedade como um todo. A era digital “transformou os setores da vida individual e da sociedade ao ponto em que ampliou, principalmente, através das redes virtuais o acesso à informação” \cite[p.~26]{Guzzi2010Web}.

    Enfim, mesmo que o tema da literatura digital seja simplesmente novo no nosso meio social, alguns conceitos já se tornam bastante óbvios, como diferenciar o livro eletrônico do livro físico que é digitalizado para as plataformas virtuais. 

    Ademais, o tempo sempre sofrerá mutações, conseguinte a sociedade muda, mesmo assim ainda acreditamos que enquanto existir quem rabisque poemas, utilizando-se de signos e quem os leia, existirá literatura. Não importa de que forma, seja num pedaço de papel como antigamente, grafitado num muro de uma rua, pintado numa parede de uma escola ou nas parafernálias eletrônicas, num ciberespaço. 
    
    \section{Considerações finais}

    Neste trabalho, procuramos averiguar as possibilidades de leitura na era digital. Como também os novos hábitos de leitura a partir de uma literatura ambientada em plataformas cibernéticas. Realizamos, também, de tal modo um breve levantamento sobre a literatura contemporânea nos quais a presença do universo digital apareça, seja como tema, seja como recurso formal.  

    Constatamos ainda que a internet, com seu alto poder de difusão entre as pessoas, faz parte da vida do adolescente, independente   da finalidade do seu uso: chats, correio eletrônico e pesquisas. Como foi pesquisado, percebemos com evidência que os adolescentes a utilizam tanto para o lazer como para o estudo. 

    Diante do que foi pesquisado, notamos ser essencial sobreviver a aspectos deterministas tanto na leitura quanto na teorização sobre obras literárias digitais e outras mostras literárias no ciberespaço. O pesquisador britânico David \textcite{Buckingham2010} chamou atenção para o caso de que, com excessos teóricos que versam sobre letramento digital, atualmente ressaltam a necessidade de aprendizagem de um conjunto ínfimo de competências. Estas que certificam o usuário a operar com eficácia diversos softwares ou a cumprir tarefas básicas de recuperação de informações. Como alternativa a esse tipo de letramento digital funcional, o autor sugere um letramento digital crítico, que deverá auxiliar o leitor a se investigar não apenas pelo manuseio, mas também pelas fontes do conhecimento disponível, pelos interesses de seus elaboradores e   pelos     formatos    como ela    representa o    mundo. Para Buckingham, é importante indagar como esses incrementos tecnológicos estão incluídos a forças sociais, políticas e econômicas mais amplas. 

    Levando-se em consideração esses aspectos, é presumível ponderar que, se possa pensar no aparecimento incipiente de um novo grupo de leitor literário, o ciberleitor: uma analogia ainda em desenvolvimento, contudo que assinala para descrições inexistentes (ou existentes em uma intensidade diferente) na identidade do leitor em base impresso, tais como a interatividade, a sequência não linear, a descentralização, a performatividade, entre outras.   

    Em um nível cultural mais vasto, em contrapartida, quem sabe esteja calhando um enfraquecimento da hierarquia entre a compreensão do autor como possuidor legitimado do conhecimento e do leitor como um aprendiz. Ademais, parece haver ainda a solução ou o destroncamento de identidades intermediárias, como do crítico e do editor, entre outros.  

    De outra forma, mesmo que seja também muito cedo para considerar a abarcamento dessas modificações em um nível amplo, a leitura literária em ambiente virtual pode originar conduções importantes quanto ao modo como se define a identidade do leitor, permitindo a emergência de uma nova identidade, imersiva. “Conectando-se entre nós e nexos, num roteiro multilinear, multisequencial e labiríntico que ele próprio ajudou a construir ao interagir com os nós entre palavras, imagens, documentação, música, vídeo etc.” \cite[p.~33]{Santaella2004Navegar}.

    \printbibliography[heading=subbibliography,notcategory=fullcited]

    \label{chap:leituraend}

\end{refsection}
