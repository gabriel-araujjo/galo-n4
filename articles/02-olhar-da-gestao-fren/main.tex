\begin{refsection}
    \renewcommand{\thefigure}{\arabic{figure}}
    \renewcommand{\thetable}{\arabic{table}}
    \renewcommand{\thequadro}{\arabic{quadro}}
    
    \chapterOneLine
    {O olhar da gestão frente à importância da leitura compreensiva nos anos finais do ensino fundamental}
    \label{chap:olhargestao}
    
    \articleAuthor
    {José de Arimatéia da Paz Albuquerque}
    {Especialista em Gestão de Processos Educacionais do Instituto de Educação Superior Presidente Kennedy --- IFESP. Professor de Língua Portuguesa da rede pública estadual do RN. E-mail: pazcomamor193@gmail.com.}

    \articleAuthor
    {Rozicleide Bezerra de Carvalho}
    {Doutora em Educação (UFRN). Mestra em Ensino de Ciências Naturais e Matemática (UFRN). Licenciada em Ciências Biológicas. Bacharel em Zoologia (UEB). Esp. em Psicopedagogia (UNP). Profª formadora do IFESP. ID Lattes: 1050.4821.8347.3864. ORCID: 0000-0003-0856-3601. E-mail: rozi@ifesp.edu.br.}
    
    \begin{galoResumo}
        \marginpar{
            \begin{flushleft}
            \tiny \sffamily
            Como referenciar?\\\fullcite{SelfAlbuquerqueAndCarvalho2021Olhar}\mybibexclude{SelfAlbuquerqueAndCarvalho2021Olhar}, p. \pageref{chap:olhargestao}--\pageref{chap:olhargestaoend}, \journalPubDate{}
            \end{flushleft}
        }
        O objetivo dessa pesquisa consistiu em investigar como a gestão de uma Escola Pública Estadual do Município de São Gonçalo do Amarante, no Rio Grande do Norte, compreende a importância da leitura compreensiva como responsabilidade das áreas de conhecimento dos anos finais do Ensino Fundamental. Para embasar a pesquisa, foram consideradas as ideias de \textcite{Leffa1996Fatores}; \textcite{Geraldi1991Psicopedagogia}; \textcite{RojasEtAl2009Dificuldades}; \textcite{Lenner2002Ler}; \textcite{DiretrizesCur2013, BaNacCurEF2017}; \textcite{Sole1998Estrategias}; \textcite{BagnoEtAl2002Lingua}, entre outros. Os resultados mostram que os sujeitos da pesquisa compreendem que a leitura compreensiva deve se constituir como compromisso de todas as áreas de conhecimento.
    \end{galoResumo}
    
    \galoPalavrasChave{Leitura Compreensiva. Anos Finais. Ensino Fundamental. Áreas de Conhecimento. Escola Pública Estadual.}
    
    \begin{otherlanguage}{spanish}
    
    \fakeChapterOneLine
    {Opinión de la dirección sobre la importancia de la lectura integral en los últimos años de la educación primaria}
    
    \begin{galoResumo}[Resumen]
        El objetivo de esta investigación fue investigar cómo la gestión de una escuela pública estatal en el municipio de São Gonçalo do Amarante, en Rio Grande do Norte, entiende la importancia de la lectura integral como responsabilidad de las áreas de conocimiento en el años de escuela primaria. Para apoyar la investigación, se consideraron las ideas de \textcite{Leffa1996Fatores}; \textcite{Geraldi1991Psicopedagogia}; \textcite{RojasEtAl2009Dificuldades}; \textcite{Lenner2002Ler}; \textcite{DiretrizesCur2013, BaNacCurEF2017}; \textcite{Sole1998Estrategias}; \textcite{BagnoEtAl2002Lingua}, entre otros. Los resultados muestran que los sujetos de investigación entienden que la lectura integral debe ser un compromiso de todas las áreas del conocimiento.
    \end{galoResumo}
    
    \galoPalavrasChave[Palabras clave]{Lectura comprensiva. Ultimos años. Enseñanza fundamental. Áreas de conocimiento. Escuela pública estatal.}
    \end{otherlanguage}
    
    % \flourish
    
    \section{Introdução}

    Esta seção tem o objetivo de apresentar a concepção de leitura compreensiva assumida pela maioria de autores da linguística e de tecer uma breve discussão acerca da importância da leitura compreensiva como compromisso de todas as áreas de conhecimento e o olhar da gestão frente essa perspectiva. 

    A leitura compreensiva é considerada, pela maioria dos autores que pesquisam esse objeto de estudo, como um processo de significação no qual participam o sujeito e o objeto da leitura \cite{Leffa1996Fatores, VanDijk1997}. Especificamente, pode-se dizer que a leitura é um processo comunicativo que envolve um leitor interpretando as intenções comunicativas de um escritor, registradas em um texto.    

    Na visão de \textcite{JouAndSperb2003Leitura}, na leitura compreensiva devem ser evidenciados dois aspectos importantes na atividade do leitor frente ao texto.  Por um lado, a interação do leitor com o texto, a qual ocorre mediante a construção das macroestruturas e, por outro, a interação do leitor com sua cognição, que ocorre por meio da utilização dos processos metacognitivos (consciência do próprio processo cognoscitivo).  

    Segundo os autores, considera-se que essas duas variáveis estão presentes nos leitores eficientes. Quanto ao primeiro aspecto, a interação do leitor com o texto, torna-se imprescindível analisar, nessa interlocução, a estrutura textual. O texto formal, presente na ação de ler, tem características linguísticas próprias. Tais características constituem as estruturas que determinam a coerência textual, ou seja, a micro, macro e a superestrutura.   

    Diferentes autores pesquisam a influência das estruturas textuais na compreensão da leitura (Alliende \& Condemarín, 1987; Koch \& Travaglia, 1990; van Dijk \& Kintsch, 1983; van Dijk, 1997).  Segundo \textcite{VanDijk1997},  

    \begin{quotation}
        As microestruturas, ou estruturas superficiais, são as proposições individuais e suas relações tornam possível iniciar a compreensão do texto; a macroestrutura é uma representação abstrata da estrutura global de significado de um texto, portanto, de natureza semântica que permite capturar o enredo do mesmo; a superestrutura é também uma estrutura global e possibilita identificar os tipos de textos, como narrativos, argumentativos, etc. \cite[p.~15]{JouAndSperb2003Leitura}.
    \end{quotation}

    O processo de leitura tem sido continuamente fonte de preocupação docente, principalmente, nos anos iniciais do Ensino Fundamental. Nas Diretrizes Curriculares Nacionais Gerais --- DCNG \cite{DiretrizesCur2013}, explicita-se que a leitura deve se constituir como uma das dimensões presentes em todos os componentes curriculares da Educação Básica. Estudos têm mostrado, também tenho observado, na minha experiência como professor (autor desse texto) de Língua Portuguesa da Educação Básica, que estudantes dos anos finais do Ensino Fundamental tem chegado sem ainda saber ler, seja em relação ao conhecimento das letras do alfabeto ou mesmo não ser proficiente em leitura, no que diz respeito às competências comunicativa e discursiva. 

    \textcite{RojasEtAl2009Dificuldades} escrevem que são múltiplas e frequentes as pesquisas que têm como objetivo dar respostas a perguntas corriqueiras que fazemos a nós mesmos em relação à quando iniciar o processo e que metodologias podemos assumir para orientar o ensino-aprendizagem da leitura. São poucos os estudos sobre as dificuldades que encontram os estudantes nesse processo. 

    São diversas as situações-problema que se evidenciam durante o processo de ensino-aprendizagem no espaço escolar em nosso país. Cada vez é mais frequente, no Rio Grande do Norte, em especial, na escola onde leciono, estudantes com dificuldade na leitura e, como consequência, se expressa na escrita. Sendo assim, \textcite{Sanchez2012} enuncia que para abordar o estudo, no que se refere à aprendizagem escolar, faz-se necessário resolver a situação relacionada com sua origem ou natureza.

    A escola que foi selecionada para o desenvolvimento da pesquisa com a equipe gestora, observa-se dificuldades dos estudantes na habilidade de ler de modo compreensivo. 

    Como professor de Língua Portuguesa da referida escola, após alertar a equipe gestora sobre tal situação, ela se expressou acerca das dificuldades dos estudantes, nos anos finais do Ensino Fundamental, em compreender a importância dessa habilidade/atividade em todas as áreas do conhecimento. Em seus discursos, afirmam que se trata de um conteúdo de aprendizagem inerente ao componente curricular Língua Portuguesa, discurso observado nas pesquisas de \textcite{Leffa1996Fatores}; \textcite{Geraldi1991Psicopedagogia}.  

    Diante desse contexto, enuncia-se o problema de pesquisa: Que compreensão tem a equipe gestora da Escola Estadual Padre José Maria Biezinger --- São Gonçalo do Amarante-RN, em relação à importância da leitura compreensiva nas áreas de conhecimento nos anos finais do Ensino Fundamental? 

    O objeto de estudo dessa pesquisa é a importância da leitura compreensiva como prioridade e estratégia didático-pedagógica em todas as áreas de conhecimento dos anos finais do Ensino Fundamental da Escola Estadual Padre José Maria Biezinger --- São Gonçalo do Amarante-RN, na visão dos gestores dessa instituição. 

    O objetivo geral desse trabalho foi o de conhecer como a equipe gestora da Escola Pública Estadual Padre José Maria Biezinger --- São Gonçalo do Amaran\-te-RN --- compreende a importância da leitura compreensiva nas áreas de conhecimento nos anos finais do Ensino Fundamental, de modo a sugerir uma proposta de formação continuada para gestores e professores da referida escola sobre a imprescindibilidade de se trabalhar leitura compreensiva em todas as áreas de conhecimento. 

    Para alcançar esse objetivo, foram estabelecidos os seguintes objetivos específicos:

    \begin{enumerate}
        \item Identificar concepções dos gestores da Escola Estadual Padre José Maria Biezinger sobre a importância da leitura compreensiva como compromisso de todas as áreas de conhecimento e como estratégia didático-pedagógica nos anos finais do Ensino Fundamental;  
        \item Identificar na BNCC \cite{BaNacCurEF2017} a presença da leitura compreensiva como compromisso de todas as áreas de conhecimento nos anos finais do Ensino Fundamental; 
        \item Conhecer a avaliação dos gestores da Escola Estadual Padre José Maria Biezinger --- São Gonçalo do Amarante-RN, acerca da importância do desenvolvimento de um projeto sobre leitura compreensiva como estratégia didático-pedagógica nas áreas de conhecimento dos anos finais do Ensino Fundamental;
        \item Conhecer a opinião dos gestores da Escola Estadual Padre José Maria Biezinger --- São Gonçalo do Amarante-RN, acerca de uma formação continuada sobre leitura compreensiva em todas as áreas de conhecimento.
    \end{enumerate}

    \section{Leitura compreensiva: responsabilidade de todas as áreas de conhecimento}

    Nessa seção, o objetivo é apresentar reflexões sobre a importância da leitura compreensiva como responsabilidade de todas as áreas de conhecimento na Educação Básica, com foco nos anos finais do Ensino Fundamental. 

    Segundo \textcite{Leffa1996Fatores}, a leitura é um processo complexo que envolve fatores diversos. Para ler e compreender um texto, o leitor se utiliza de variados subprocessos, desde o nível inconsciente do processamento gráfico até o nível altamente consciente da atenção, exigida em atividades como a metacompreensão.  O leitor necessita também orquestrar todos esses subprocessos adequadamente, propiciando uma troca constante de informação entre os níveis, a fim de que diferentes subprocessos possam ser incluídos no processo maior da compreensão. 

    A leitura compreensiva é responsabilidade de todas as áreas de conhecimento, portanto, deve se materializar nas salas de aula dos diferentes componentes curriculares da Educação Básica. Em todos os componentes curriculares os estudantes produzem textos, assim, a leitura está presente, seja de maneira implícita ou explícita, na sala de aula. 

    Ler se constitui como habilidade e competência geral na perspectiva da Base Nacional Comum Curricular --- BNCC \cite{BaNacCurEF2017}, que marca uma função essencial --- “talvez a única função --- da escolaridade obrigatória.” \cite[p.~17]{Lenner2002Ler}. Nesse sentido, redefinir o sentido dessa função --- explica, portanto, o significado que se pode atribuir hoje a esses termos tão arraigados a instituição escolar --- é uma tarefa incontestável. \cite[p.~17]{Lenner2002Ler}.

    É relevante se pensar em como a leitura compreensiva pode transformar nossos hábitos institucionais de ensinar a ler de modo compreensivo para aprender a interpretar o mundo dos objetos, bem como saber ler e aprender.

    \textcite{BagnoEtAl2002Lingua} expressam que leitura e escrita (embora, esta última não seja objeto dessa pesquisa, mas essas duas habilidades formam uma unidade) são áreas de especialização de grande abrangência, cada qual com uma ampla literatura específica. Para os autores, há uma ambiguidade básica com o termo ler, bem como defendem que a leitura raramente é ensinada depois dos anos iniciais do Ensino Fundamental; “e se leitura significa uma interpretação-compreensão que vai além do superficial, portanto, ela praticamente, na maioria das vezes, não é ensinada”. \cite[p.~128]{BagnoEtAl2002Lingua} 

    Segundo \textcite[p.~11]{NevesEtAl2003Ler}, ler e escrever é compromisso de todas as áreas de conhecimento, portanto, não se trata apenas de ser conhecimento profissional do professor de Língua Portuguesa. Ler compreensivamente implica saber inferir, interpretar, argumentar, explicar a realidade, entre outras habilidades que devem ser entendidas e que são comuns a todos os componentes curriculares, seja na Educação Básica ou Ensino Superior. Portanto, a leitura compreensiva é condição necessária para escrita proficiente. 

    \textcite{GuedesAndSouza2011NaoApenas} escrevem:

    \begin{quotation}
        A tarefa de ensinar a ler e escrever um texto em história é do professor de história e não do professor de português. A tarefa de ensinar a ler e escrever um texto de ciências é do professor de ciências e não do professor de português. Ler e escrever são tarefas da escola, questões para todas as áreas, uma vez que são habilidades indispensáveis para a formação de um estudante, que é responsabilidade da escola. O processo da leitura envolve diversos aspectos, incluindo não apenas características do texto e do momento histórico em que ele é produzido, mas também características do leitor e do momento histórico em que o texto é lido.  Para ele, o resultado do encontro entre leitor e texto não pode ser descrito, portanto, a partir de um único enfoque. “Uma descrição completa do processo da compreensão deve levar em conta, no mínimo, três aspectos essenciais: o texto, o leitor e as circunstâncias em que se dá o encontro” \cite[p.~1]{GuedesAndSouza2011NaoApenas}.
    \end{quotation}


    Historicamente, no entanto, o estudo da compreensão de leitura tem se caracterizado pela predominância de um ou outro extremo do processo, enfatizando ora o texto ora o leitor, como fator essencial da compreensão. Cada um desses enfoques pressupõe uma explicação diferente para os fatores que intervêm na compreensão. Quando se privilegia o texto, por exemplo, pressupõe-se que a melhoria na compreensão depende de qualidades intrínsecas do texto e que, na medida em que se modificam essas qualidades, está se modificando os níveis de compreensão do leitor. Quando se privilegia o leitor, pressupõe-se que a compreensão do texto aumenta, na medida em que se desenvolvem no leitor as habilidades gerais da leitura \cite[p.~1--2]{Leffa1996Fatores}.

    Compreender a concepção de leitura compreensiva na perspectiva de habilidade é considerar a formação ou desenvolvimento dessa, mediante os passos para saber ler de modo compreensivo, portanto, com domínio e consciência, como defende \textcite{Talizina1988Formacion}.

    Reconhecer a importância da leitura compreensiva em todas as áreas de conhecimento é concebê-la como atividade social e conhecimento profissional inerente a qualquer profissão. Atividade compreendida na perspectiva de interação conforme \textcite{Vygotsky2005Pensamiento} e na perspectiva de \textcite{Leontiev2011Inport} como uma unidade molar, não aditiva, prática, psíquica e valorativa que orienta o sujeito em direção a um ou mais objetivos, portanto, essencial para o desenvolvimento dos sujeitos, por ser essencial para esses estabelecerem relação com os objetos de conhecimento de modo consciente.


    \textcite[p.~20]{Geraldi1991Psicopedagogia} explicita que quanto à linguagem e sua constituição: “Trata-se de pensar a atividade linguística não só a partir das ações que se fazem com a linguagem, mas de pensá-la também a partir das ações que se fazem sobre a linguagem e das ações da linguagem” como por exemplo, a leitura compreensiva.

    Compreende-se nesse trabalho que a leitura compreensiva se constitui como conhecimento e uma das estratégias didático-pedagógicas de suma relevância e compromisso de todas as áreas de conhecimento da Educação Básica, que pode contribuir significativamente para o processo de ensino-aprendizagem dos estudantes nos diferentes componentes curriculares nos anos finais do Ensino Fundamental.

    \section{Gestão democrática: diferentes concepções}

    Esta seção tem o objetivo de apresentar diferentes concepções acerca de gestão democrática. São diferentes autores que discutem sobre esse objeto de conhecimento, os quais, entre eles, podemos citar Mendonça (2000); Drabach e Mousquer (2009); Lück (2007), entre outros. 

    Para Mendonça (2000, Apud DRABACH e MOUSQUER, 2009), a gestão democrática está relacionada a uma determinada abordagem da administração da educação, resultante do enfoque construído nas últimas décadas, em contraponto à ênfase organizacional e tecnicista, bem como ao reducionismo normativista da busca da eficiência pela racionalização de processos. Na visão deste autor, a gestão pode ser entendida no seu sentido amplo como  

    \begin{quotation}
        Um conjunto de procedimentos que inclui todas as fases do processo de administração, desde a concepção de diretrizes de política educacional, passando pelo planejamento e definição de programas, projetos e metas educacionais, até suas perspectivas de implementações e procedimentos avaliativos” (2000, Apud DRABACH e MOUSQUER, p.~69, 2009).
    \end{quotation}

    Neste sentido, pode-se apreender que o autor entende a gestão como uma forma de administração e que, portanto, esta última é mais ampla.  Ainda de acordo com Draback e Mousquer (2009), outros autores, como é o caso de  Lück (2007), embora reconhecendo a prioridade da mudança não só de nomes, mas, preferencialmente, de concepção, preferem demarcar uma distinção entre os termos (administração e gestão) como forma de ressignificar esta prática. A referida autora defende o conceito de gestão escolar como mais apropriado para as demandas do processo educativo atual, por entender que, 

    \begin{quotation}
        A intensa dinâmica da realidade faz com que os fatos e fenômenos mudem de significado ao longo do tempo, de acordo com a evolução das experiências, em vista de que os termos empregados para representá-los, em uma ocasião, deixam de expressar plenamente toda a riqueza dos novos entendimentos e desdobramentos (LÜCK, 2007, p.~47).
    \end{quotation}

    O conceito Gestão Democrática parece apontar para a consolidação da democracia, no Brasil. Surge como uma prática social que vem permeando todas as áreas da sociedade; é como se estivéssemos convocando as pessoas a serem cidadãs, a praticarem a democracia, porque somente com as pessoas participando da vida da “pólis” é possível exercer a cidadania e praticar a política: alternativa à tirania e à barbárie. 

    Cabral Neto (2010), no texto “Mudanças Socioeconômicas e Políticas e suas Repercussões no Campo da Política Educacional” se propõe analisar a relação entre a geopolítica econômica do final do século XX e o seu reflexo na nova configuração da educação no início do século XXI. Nele, é aventado como a economia passou a assumir o controle do estado e a direcionar todas as suas escolhas ao fortalecimento do capital, sobretudo no campo educacional, no qual, além do novo modelo, é implantada também uma nova ideologia. Tais mudanças ocorreram por uma falência do modelo anterior, no qual o estado gerava grandes expectativas para os trabalhadores, além de grande endividamento. Todavia, no intuito de resolver a crise cambial do estado, o capitalismo se apropria da máquina pública, implanta uma nova dinâmica socioeconômica (baseada na meritocracia); de acordo com os interesses do capital.   

    Essa apropriação do estado pelo capital vai se refletir em todos os aspectos da sociedade porque esse “novo governo” é uma espécie de ditadura --- ditadura do capital --- que implanta um novo modelo de gestão, “descentralizado, flexível”, que perpassa do individual ao poder central, tendo como principais repercussões na educação: autonomia na forma de gestão, automatismo do indivíduo (com a supressão das chamadas ciências humanas: filosofia, sociologia, literatura) que agora não precisa mais pensar, porque estará ocupado; terá que aprender a manusear as novas máquinas, visto que a ênfase será na ciência, aliada de primeira hora do capital.  

    A “gestão democrática” que o Brasil tenta consolidar está, todavia, ameaçada por interesses antidemocráticos, capitalistas, que colocam o lucro acima de tudo e de todos. Essa democracia cambiante coloca constantemente em xeque nossa capacidade de caminharmos com nossas próprias pernas; tomarmos nossas próprias decisões. Ao longo da história, a democracia brasileira tem apresentado momentos de estabilidades e rupturas, no entanto, em todos os momentos se faz imperativo a participação cidadã como meio de fortalecer a democracia e isso pode ser traduzido a partir da percepção da Gestão Democrática. 

    Apesar de todos os percalços, a ``Gestão Democrática'' e a “democracia” brasileira parecem se impor; em um país que tem amadurecido, na perspectiva democrática, ao mesmo tempo em que convive com a desigualdade social, a exclusão e a pobreza extrema.  A Gestão Democrática que é praticada da gestão escolar no Brasil prescinde dos princípios democráticos: geralmente, ainda é o gestor quem decide tudo sozinho, sem ouvir, ou comunicar a alguém; não há diálogo ou muito menos transparência.  

    No entanto, a escola deve construir sua Gestão Democrática em conjunto com os atores sociais que dela fazem parte como pais, alunos, direção, professores, funcionários e demais componentes da comunidade. A escolha do diretor escolar pela via da eleição direta e com a participação da comunidade vem se constituindo e ampliando-se como mecanismo de seleção diretamente ligado à democratização da educação e da escola pública, visando assegurar, também, a participação das famílias no processo de gestão da educação de seus filhos (PARENTE; LÜCK, 1999, p. 37).  

    No texto supracitado, é possível perceber, entre outros fatores, que para se construir a autonomia da escola, é necessário criar uma cultura de democratização das relações entre todos os envolvidos no processo educativo. Além disso, é preciso gerir bem os recursos públicos e dele prestar contas. Por fim, é preciso que a escola tenha uma visão, um objetivo, um propósito, um projeto político pedagógico. Ela precisa estar de acordo com as demandas reais da comunidade escolar. É relevante especificar quais os objetivos da escola no curto, médio e longo prazo e qual a educação que se deseja para os educandos. 

    A imprescindibilidade de delegar atividades e trabalhar em equipe parecem ser os pilares de uma gestão aberta e democrática, nos moldes que se prega atualmente. Em significativa parcela de escolas, a centralização de responsabilidades e tarefas, ainda protagonismo das ações, pertence, quase que exclusivamente, à figura do gestor.  

    A importância do trabalho em equipe para o gestor decorre exatamente de ele saber organizar as diferentes funções no espaço que gerencia, sabendo reagrupá-lo na forma de trabalho em equipe. Essa perspectiva, entre outras coisas, permite a todos a coparticipação nas atividades e o sentimento de pertencimento, portanto, permitindo que a equipe trabalhe em direção aos mesmos objetivos, pois há a possibilidade de cada participante ser um defensor em potencial do projeto em pauta. 

    Para que a gestão se constitua democrática e participativa, é necessário que haja a representação de todos os atores que participam direta ou indiretamente da escola, como discentes, docentes, funcionários e comunidade. Ainda na perspectiva de gestão participativa, é importante destacar o papel do “conselho escolar” --- representante dos membros da comunidade escolar. Essa instância pode fazer uma grande diferença para uma gestão ser bem-sucedida porque, de acordo com o senso comum: “várias cabeças pensam melhor que uma”, ou seja, um conselho escolar eficiente pode trazer muitas sugestões para resolver os problemas da escola, podendo optar pela ideia que mais convergir.


    \section{Metodologia da pesquisa}

    Essa seção tem o objetivo de apresentar a metodologia da pesquisa, considerando seus fundamentos metodológicos, o contexto e os participantes, os procedimentos e percurso metodológico, bem como o tratamento dos dados. 

    Quanto à abordagem, a pesquisa é quantitativo/qualitativa; em relação aos objetivos é exploratória, descritiva e analítica. Como referencial metodológico foi utilizado \textcite{LavilleAndDionne1999Construcao}. 

    A pesquisa foi realizada com três gestores (vice-diretor e coordenadoras pedagógicas) de uma Escola Pública do Ensino Fundamental do Município de São Gonçalo do Amarante, no Rio Grande do Norte-RN. 

    O trabalho de pesquisa constitui-se como uma intervenção formativa orientada para uma nova configuração em relação ao olhar da gestão frente à importância da leitura compreensiva em todas as áreas de conhecimento, com um novo enfoque para incorporação dessa habilidade no Projeto Político Pedagógico da escola, de maneira a subsidiar as áreas de conhecimento dos anos finais do Ensino Fundamental.  

    Os instrumentos para coleta de dados foram dois questionários (aplicados no início e final da pesquisa) com perguntas abertas e fechadas e a \textcite{BaNacCurEF2017}. 

    A organização dos dados se deu em quadros e tabelas, o que propiciou a criação de possibilidades e diversos percursos para posterior análise do material coletado. \textcite{LavilleAndDionne1999Construcao} explicitam que esse momento de planejamento (seleção e organização dos dados) da pesquisa comporta três operações principais: codificação, transferência e verificação.  

    Para a análise dos dados coletados na pesquisa, foram realizados procedimentos de análise qualitativa e quantitativa. \textcite{LavilleAndDionne1999Construcao} expressam que a abordagem qualitativa na análise de dados possibilita perceber as inferências, o sentido que está entre as categorias-chave e a relação entre essas categorias.  

    Para a produção do questionário aplicado, inicialmente foi elaborado um plano, conforme Quadro \ref{quad:questionario}, sendo esse instrumento devidamente validado. Para organização e análise dos dados obtidos, realizou-se a codificação, cujo objetivo foi respeitar a identidade dos sujeitos da pesquisa. Como códigos para identificação desses utilizou-se: G1, G2 e G3 = Gestores 1, 2 e 3.

    \begin{longquadro}[t]{ | p{.33\textwidth} p{.60\textwidth} | }
    % \begin{longquadro}{cc}
        \caption{Plano de questionário}
        \label{quad:questionario}\\
       
        \hline
        Objetivos & \hspace{1.75em}Perguntas\\
        \hline
        \endfirsthead

        \multicolumn{2}{c}{\footnotesize \textsf{Quadro \ref{quad:questionario}~--~\textit{continuação da página anterior}}\medskip }\\
        \hline
        Objetivos & \hspace{1.75em}Perguntas\\
        \hline
        \endhead

        \hline
        \multicolumn{2}{r}{\footnotesize \textsf{\textit{continua}}}\\
        \endfoot
       
        \hline
        \caption*{Fonte: elaboração do autor, 2019}
        \endlastfoot
       
        Conhecer o perfil socioprofissional da equipe gestora da Escola Estadual Padre José Maria Biezinger/RN. &
        \begin{enumerate}[series=questionario,nosep,leftmargin=*,after=\vspace{-\baselineskip},before=\vspace{-\baselineskip},label=\textbf{\Roman*.},widest=III]
            \item \textbf{Perfil socioprofissional e o contexto da atividade profissional}
            \begin{enumerate}[series=questoes,nosep,leftmargin=1.25em,label=\arabic*.,widest=10]
                \item Idade:
                \item Sexo:
            \end{enumerate}
            \item \textbf{Experiência na gestão}
            \begin{enumerate}[resume*=questoes]
                \item Tempo que atua na escola atual como gestor:
            \end{enumerate}
        \end{enumerate} \\
        Identificar as concepções da equipe gestora da Escola Estadual Padre José Maria Biezinger/RN sobre a importância da leitura compreensiva nas áreas do conhecimento do Ensino Fundamental.
        & \begin{enumerate}[resume*=questionario]
            \item \textbf{Conhecimento profissional}
            \begin{enumerate}[resume*=questoes]
                \item Qual a sua compreensão sobre a importância da leitura em todas as áreas de conhecimento nos anos finais do Ensino Fundamental? 
                \item Que possibilidades você observa na leitura, de modo a ser contributiva no aprendizado dos estudantes em todas as áreas do conhecimento dos anos finais do Ensino Fundamental?
                \item A gestão desenvolve ou apoia algum projeto de leitura na Escola Estadual Padre José Maria Biezinger? Cite.
                \item Como você compreende a implantação de uma proposta sobre leitura compreensiva como estratégia didático-pedagógica em todas as áreas de conhecimento nos anos finais do Ensino Fundamental na Escola Estadual Padre José Maria Biezinger --- São Gonçalo do Amarante/RN --- à equipe gestora dessa instituição.
            \end{enumerate}
        \end{enumerate} \\
        Conhecer a opinião dos gestores da E. E. Pe. José Maria Biezinger -- São Gonçalo do Amarante/RN -- acerca da importância da leitura compreensiva como estratégia didático-pedagógica nas áreas de conhecimento dos anos finais do Ensino Fundamental. &
        \begin{enumerate}[resume*=questoes,leftmargin=3em,after=\vspace{-\baselineskip},before=\vspace{-\baselineskip},start=8]
            \item Qual o seu olhar enquanto gestor frente à importância da leitura em todas as áreas de conhecimento nos anos finais do ensino fundamental da Escola Estadual Padre José Maria Biezinger?
        \end{enumerate}
        \\
    \end{longquadro}

    A pesquisa foi desenvolvida em três etapas, de acordo com o Quadro \ref{quad:percurso-metodologico}.

    \begin{longquadro}[t]{ | p{.11\textwidth} p{.60\textwidth} p{.22\textwidth} |}
        \caption{Percurso metodológico da pesquisa}
        \label{quad:percurso-metodologico}\\

        \hline
        Etapas & Objetivos específicos & Instrumentos de coleta de dados \\
        \hline
        \endfirsthead

        \hline
        Etapas & Objetivos específicos & Instrumentos de coleta de dados \\
        \hline
        \endhead

        \hline
        \multicolumn{2}{r}{\footnotesize \textsf{\textit{continua}}}\\
        \endfoot
       
        \hline
        \caption*{Fonte: elaboração do autor, 2019}
        \endlastfoot

        1ª etapa &
        Identificar concepções dos gestores da Escola Estadual Padre José Maria Biezinger sobre a importância da leitura compreensiva como estratégia didático-pedagógica nas áreas de conhecimento dos anos finais do Ensino Fundamental.
        & Questionário 1\\
        2ª etapa & 
        Identificar na BNCC \cite{BaNacCurEF2017} a orientação para trabalhar leitura compreensiva nas áreas de conhecimento dos anos finais do Ensino Fundamental.
        & BNCC \cite{BaNacCurEF2017}\\
        3ª etapa & 
        Conhecer a avaliação dos gestores da Escola Estadual Padre José Maria Biezinger/RN acerca da importância do desenvolvimento de um projeto sobre leitura compreensiva como estratégia didático-pedagógica nas áreas de conhecimento do Ensino Fundamental anos finais.

        Conhecer a opinião dos gestores da Escola Estadual Padre José Maria Biezinger/RN acerca de uma formação continuada sobre leitura compreensiva em todas as áreas de conhecimento.  
        & Questionário 2\\


    \end{longquadro}

    A Tabela \ref{tabl:perfil-socioprof} expressa o perfil socioprofissional e o contexto da atividade profissional dos participantes da pesquisa.

    \begin{table}[!ht]
        \centering
        \caption{Perfil socioprofissional e o contexto da atividade profissional}
        \label{tabl:perfil-socioprof}
        
        \begin{tabular}[c]{
            >{\centering\arraybackslash}m{.15\textwidth}
            >{\centering\arraybackslash}m{.10\textwidth} 
            >{\centering\arraybackslash}m{.07\textwidth} 
            >{\centering\arraybackslash}m{.15\textwidth}
            >{\centering\arraybackslash}m{.20\textwidth}
            >{\centering\arraybackslash}m{.23\textwidth}
        }
            \hline
            \multirow{2}{*}{Codificação} & \multicolumn{5}{c}{ Categorias de análise }\\
            \cline{2-6}
            & Idade & Sexo & Experiência na gestão & Tempo atuando na gestão & Pós-graduação \\
            \hline

            G1 & 60 anos & M & 13 anos & 3 anos & Não \\
            G2 & 50 anos & F & 10 anos & 10 anos & Não \\
            G3 & 55 anos & F & 6 anos & 6 anos & Sim, especialista \\
            \hline
            
        \end{tabular}
        \caption*{Fonte: elaboração do autor, 2019}
    \end{table}

    A primeira etapa teve o objetivo de identificar concepções dos gestores da Escola Estadual Padre José Maria Biezinger- São Gonçalo do Amarante/RN - sobre a importância da leitura compreensiva como estratégia didático-pedagógica em todas as áreas de conhecimento. Para esse fim, utilizou-se como instrumento de coleta de dados um questionário com perguntas abertas e fechadas.  

    \textcite{LavilleAndDionne1999Construcao} explicitam que a utilização desse tipo de instrumento constitui-se como uma das formas mais econômicas de coleta de dados quantitativos. São práticos e permitem que os respondentes fiquem à vontade (sem pressões) e no anonimato.  Em relação às desvantagens, pode-se citar, entre outras, o fato de que se as perguntas não estiverem claras, poderá conduzir o respondente a diferentes interpretações, propiciando resultados não confiáveis.

    Para análise da BNCC \cite{BaNacCurEF2017}, que constituiu a segunda etapa, foram estabelecidas duas categorias prévias --- leitura e leitura compreensiva.  

    O objetivo nesta etapa foi identificar se esse documento orientava a utilização da leitura e da leitura compreensiva em todas as áreas de conhecimento, de maneira explícita, tomando como base a análise de conteúdo do tipo fechada, na perspectiva de \textcite{LavilleAndDionne1999Construcao}, conforme Quadro \ref{quad:plano-investigacao}, que expressa o plano de investigação.

    \begin{longquadro}[t]{ | p{.31\textwidth} | p{.31\textwidth} p{.31\textwidth} |}
        \caption{Plano de investigação. Categorias de análise em relação às Áreas de conhecimento}
        \label{quad:plano-investigacao}\\

        \hline
        \multirow{2}{*}{Áreas de conhecimento} & \multicolumn{2}{c|}{Categorias de análise}\\
        \cline{2-3}
        & Leitura & Leitura compreensiva\\
        \hline
        \endfirsthead

        \multicolumn{3}{c}{\footnotesize \textsf{Quadro \ref{quad:plano-investigacao}~--~\textit{continuação da página anterior}}\medskip }\\
        \hline
        \multirow{2}{*}{Áreas de conhecimento} & \multicolumn{2}{c|}{Categorias de análise}\\
        \cline{2-3}
        & Leitura & Leitura compreensiva\\
        \hline
        \endhead

        \hline
        \multicolumn{3}{r}{\footnotesize \textsf{\textit{continua}}}\\
        \endfoot
       
        \hline
        \caption*{Fonte: elaboração do autor, 2019}
        \endlastfoot

        Língua portuguesa & Presença explícita & Presença explícita \\
        Ciências humanas e sociais aplicadas & Presença explícita & Presença explícita \\
        Ciências da natureza & Presença implícita & Presença explícita \\
        Matemática & Presença explícita & Presença explícita \\
        Ensino religioso & Presença implícita & Presença explícita \\

    \end{longquadro}

    Em relação à terceira etapa, dois objetivos foram estabelecidos: a) conhecer a avaliação dos gestores da Escola Estadual Padre José Maria Biezinger --- São Gonçalo do Amarante/RN --- acerca da importância do desenvolvimento de um projeto didático-pedagógico sobre leitura compreensiva como estratégia didático-pedagógica em todas as áreas de conhecimento nos anos finais do Ensino Fundamental; b) conhecer a opinião dos gestores acerca de uma formação continuada sobre leitura compreensiva em todas as áreas de conhecimento. O questionário foi o instrumento para obtenção desses dados.

    Para responder a esses objetivos, foi organizado um plano de investigação, conforme Quadro \ref{quad:pl-investigacao}.

    \begin{longquadro}[t]{ | p{.55\textwidth} | p{.35\textwidth} |}
        \caption{Plano de investigação}
        \label{quad:pl-investigacao}\\

        \hline
        Objetivos & Perguntas\\
        \hline
        \endfirsthead

        \multicolumn{2}{c}{\footnotesize \textsf{Quadro \ref{quad:pl-investigacao}~--~\textit{continuação da página anterior}}\medskip }\\
        \hline
        Objetivos & Perguntas\\
        \hline
        \endhead

        \hline
        \multicolumn{2}{r}{\footnotesize \textsf{\textit{continua}}}\\
        \endfoot
       
        \hline
        \caption*{Fonte: elaboração do autor, 2019}
        \endlastfoot

        Conhecer a avaliação dos gestores da Escola Estadual Padre José Maria Biezinger/RN acerca da importância do desenvolvimento de um projeto sobre leitura compreensiva como estratégia didático-pedagógica nas áreas de conhecimento do Ensino Fundamental anos finais. &
        Qual a sua opinião em relação à importância sobre leitura e leitura compreensiva em todas as áreas do conhecimento do ensino fundamental anos finais? \\[18ex]
        Conhecer a opinião dos gestores da Escola Estadual Padre José Maria Biezinger/RN acerca de uma formação continuada sobre leitura compreensiva em todas as áreas de conhecimento. &
        O que você pensa sobre um projeto de formação continuada (para professores e gestores) sobre leitura e leitura compreensiva?\\
    \end{longquadro}

    \section{Resultados e discussões}

    Nesta seção, apresentaremos os resultados e discussões em relação às informações obtidas nos instrumentos utilizados para obtenção das informações, de acordo com os objetivos específicos, as etapas do percurso metodológico e os instrumentos de coleta de dados utilizados para respondê-los.

    \subsection{Primeira etapa}

    Essa etapa objetiva identificar concepções dos gestores da Escola Estadual Padre José Maria Biezinger sobre a importância da leitura compreensiva como estratégia didático-pedagógica nas áreas de conhecimento dos anos finais do Ensino Fundamental.  

    Os resultados obtidos na pesquisa em relação a identificar concepções dos gestores da Escola Estadual Padre José Maria Biezinger/RN sobre a importância da leitura compreensiva como estratégia didático-pedagógica em todas as áreas de conhecimento estão organizados no Quadro 5.

    \begin{small}
    \begin{longquadro}[t]{ | 
        >{\centering\arraybackslash}p{.1\textwidth} |
        p{.20\textwidth} |
        p{.20\textwidth} |
        p{.20\textwidth} |
        p{.20\textwidth} |
    }
        \caption{Concepções dos gestores em relação à importância da leitura compreensiva em todas as áreas de conhecimento}
        \label{quad:concepcao-gestores}\\

        \hline
        \multirow{2}{*}{Código} & \multicolumn{4}{c|}{ Categorias de análise }\\
            \cline{2-5}
            & Compreensão sobre a importância da leitura 
            & Opinião sobre a importância da leitura em todas as áreas de conhecimento
            & Possibilidades da leitura em todas as áreas de conhecimento 
            & Projetos na escola sobre leitura  \\
        \hline
        \endfirsthead

        \multicolumn{5}{c}{\footnotesize \textsf{Quadro \ref{quad:concepcao-gestores}~--~\textit{continuação da página anterior}}\medskip }\\
        \hline
        \multirow{2}{*}{Código} & \multicolumn{4}{c|}{ Categorias de análise }\\
            \cline{2-5}
            & Compreensão sobre a importância da leitura 
            & Opinião sobre a importância da leitura em todas as áreas de conhecimento
            & Possibilidades da leitura em todas as áreas de conhecimento 
            & Projetos na escola sobre leitura  \\
        \hline
        \endhead

        \hline
        \multicolumn{5}{r}{\footnotesize \textsf{\textit{continua}}}\\
        \endfoot
       
        \hline
        \caption*{Fonte: elaboração do autor, 2019}
        \endlastfoot

        G1
        & Tem importância fundamental para que os alunos se desenvolvam em todas as disciplinas. Traz para o aluno: facilidade na escrita, compreensão de problemas matemáticos, entendimento do mundo, entre outros.
        & É importante para conhecimento prévio dos assuntos estudados, para fixação dos conteúdos e implica numa melhora dos indicadores sociais da comunidade.
        & Traz desenvolvimento e compreensão dos conteúdos de todas as disciplinas, se constituindo assim, primordial para o desenvolvimento dos estudantes.
        & Apoio aos professores que trabalham com leitura. Projetos desenvolvidos pelos professores de Língua Portuguesa de estímulo à leitura. \\

        G2
        & É através da leitura que temos às informações, conhecimento, nos orientamos, nos comunicamos.
        & Nossos alunos ainda têm muitas dificuldades em leitura, portanto, faz-se necessário intensificar o uso da leitura na escola.
        & É essencial para compreensão e aquisição dos conhecimentos das diversas áreas do conhecimento.
        & Apoio aos professores, compra de livros, impressão de textos, incentivo a eventos de produção de textos. \\

        G3
        & A leitura é uma janela para o mundo. É conhecer outros países, outras culturas. Permite adquirir conhecimento durante a vida inteira.
        & Todas as disciplinas são fundamentais, mas o aprendizado da nossa língua nos enriquece, valoriza, identifica culturalmente.
        & A leitura nos possibilita a compreensão em todas as áreas da vida. 
        & Sim. Em parceria com os professores, em harmonia com as demais disciplinas, a escola desenvolve vários projetos de leitura. \\

    \end{longquadro}
    \end{small}

    Embora se observe no discurso dos três gestores, nos resultados coletados no questionário 1, expressos no Quadro \ref{quad:concepcao-gestores}, que a leitura compreensiva é imprescindível em todas as áreas de conhecimento, porém, constata-se que na escola na qual o estudo foi realizado, não há nenhum projeto voltado para leitura compreensiva em todas as áreas de conhecimento, apenas em Língua Portuguesa, trabalho desenvolvido por um único professor desse componente curricular. Segundo \textcite{Leffa1996Fatores}, é comum na prática e no discurso de educadores que a leitura compreensiva é de responsabilidade apenas do professor de Língua Portuguesa.

    \subsection{Segunda etapa}

    Para análise da BNCC \cite{BaNacCurEF2017}, que constitui a segunda etapa, foram estabelecidas duas categorias prévias --- leitura e leitura compreensiva.

    O objetivo nesta etapa foi identificar se esse documento orientava a utilização da leitura e da escrita compreensiva em todas as áreas de conhecimento, de maneira explícita, tomando como base a análise de conteúdo do tipo fechada, na perspectiva de \textcite{LavilleAndDionne1999Construcao}, conforme quadro 6 que expressa o plano de investigação em relação às categorias de análise e as Áreas de Conhecimento presentes na BNCC \cite{BaNacCurEF2017}.

    \subsubsection{Análise da BNCC}

    Embora a BNCC reconheça, em todas as áreas de conhecimento, a importância da leitura, esse termo não é citado de forma explícita quando faz referência aos componentes curriculares Ensino Religioso e Ciências da Natureza.  

    No que se refere à leitura compreensiva, o documento não aborda o termo de maneira explícita, mas, em outras palavras, como compreender, decifrar, depreender, interpretar, analisar, entre outros. Pressupomos que é importante tanto a leitura como a escrita compreensiva no processo de aprendizagem, porque para interpretar, analisar e compreender sentidos e símbolos faz-se necessário a leitura compreensiva, embora possam ser aferidos outros significados a esse termo específico.  

    Na perspectiva de \textcite[p.~26]{Leffa1996Fatores}, “a leitura é um processo complexo que envolve diferentes fatores”. Ainda de acordo com o autor:

    \begin{quotation}
        Para ler e compreender um texto o leitor usa de muitos subprocessos, variando desde o nível inconsciente do processamento gráfico até ao nível altamente consciente da atenção exigida em tarefas como a monitoração da própria compreensão. \cite[p.~26]{Leffa1996Fatores}. 
    \end{quotation}

    \subsection{Terceira etapa}

    Nesta etapa, analisaremos a opinião do G1 sobre um projeto de leitura e leitura compreensiva nas áreas do conhecimento dos anos finais do ensino fundamental. Ainda, a opinião do mesmo sobre uma formação continuada sobre leitura e leitura compreensiva em todas as áreas do conhecimento para gestores e professores da Escola Biezinger. Os demais gestores não puderam responder a esse instrumento, pois se encontravam de licença durante esta etapa da investigação.

    \begin{small}
        \begin{longquadro}[t]{ | 
            >{\centering\arraybackslash}p{.1\textwidth} |
            p{.35\textwidth} |
            p{.35\textwidth} |
        }
            \caption{Categorias de Análise em relação às concepções dos gestores sobre leitura e leitura compreensiva e sobre um projeto de formação continuada para gestores e professores}
            \label{quad:cat-analise-concep}\\
    
            \hline
            \multirow{2}{*}{Código} & \multicolumn{2}{c|}{ Categorias de análise }\\
                \cline{2-3}
                & Opinião dos gestores da escola Biezinger sobre um projeto sobre leitura e leitura compreensiva nas áreas do conhecimento do ensino fundamental anos finais.
                & Opinião dos gestores da escola Biezinger sobre uma formação Continuada sobre leitura e leitura compreensiva em todas as áreas do conhecimento. \\
            \hline
            \endfirsthead
    
            \multicolumn{3}{c}{\footnotesize \textsf{Quadro \ref{quad:cat-analise-concep}~--~\textit{continuação da página anterior}}\medskip }\\
            \hline
            \multirow{2}{*}{Código} & \multicolumn{2}{c|}{ Categorias de análise }\\
                \cline{2-3}
                & Opinião dos gestores da escola Biezinger sobre um projeto sobre leitura e leitura compreensiva nas áreas do conhecimento do ensino fundamental anos finais.
                & Opinião dos gestores da escola Biezinger sobre uma formação Continuada sobre leitura e leitura compreensiva em todas as áreas do conhecimento. \\
            \hline
            \endhead
    
            \hline
            \multicolumn{3}{r}{\footnotesize \textsf{\textit{continua}}}\\
            \endfoot
           
            \hline
            \caption*{Fonte: elaboração do autor, 2019}
            \endlastfoot
    
            G1
            & Considera “um ato de grande importância para o aprendizado do ser humano”. Observa que aprimora o vocabulário e dinamiza o raciocínio e a interpretação. Entende que facilita nossa comunicação com o mundo.
            & Pensa que é uma proposta maravilhosa. Defende que “essas informações seriam bem-vindas ao nosso grupo escolar”. \\
    
            G2 & \hspace*{\fill} - - - \hspace*{\fill} & \hspace*{\fill} - - - \hspace*{\fill} \\
            G3 & \hspace*{\fill} - - - \hspace*{\fill} & \hspace*{\fill} - - - \hspace*{\fill} \\

        \end{longquadro}
    \end{small}

    Acerca da opinião do gestor entrevistado em relação à importância sobre leitura e leitura compreensiva em todas as áreas do conhecimento dos anos finais do ensino fundamental, ele considera “um ato de grande importância para o aprendizado do ser humano”, reconhecendo de maneira enfática o valor da leitura compreensiva. 

    Quanto à ideia de um projeto de formação continuada para gestores e professores da Escola Biezinger, G1 afirma que é “uma proposta maravilhosa”, e que “essas informações seriam bem-vindas ao nosso grupo escolar”.

    \section{Considerações finais}

    O objetivo geral desse trabalho foi o de investigar como a equipe gestora da Escola Pública Estadual Padre José Maria Biezinger --- São Gonçalo do Amarante/RN --- compreende a importância da leitura compreensiva nas áreas de conhecimento nos anos finais do Ensino Fundamental, de modo a sugerir uma proposta de formação continuada para gestores e professores da referida escola sobre a imprescindibilidade de se trabalhar leitura compreensiva em todas as áreas de conhecimento.  

    Embora nos documentos oficiais da educação brasileira se oriente a necessidade e imprescindibilidade de incluir no currículo da escola a leitura em todas as áreas de conhecimento, de modo que o estudante compreenda o que lê, ainda há a concepção, pela maioria dos educadores que trabalham em sala de aula que esta habilidade ou competência é de responsabilidade do professor de Língua Portuguesa. 

    Nos dados obtidos nesta pesquisa, observa-se que embora os gestores tenham expressado que concordam que a leitura compreensiva deva fazer parte da prática docente em todas as áreas, constata-se que não há nenhum projeto que envolva essa habilidade nos diferentes componentes curriculares ou área de conhecimento. Apenas o professor de Língua Portuguesa desenvolve uma proposta nessa perspectiva. Diante dos resultados sobre aprendizagem em leitura no Rio Grande do Norte, que apontam para ineficiência de estudantes nessa habilidade, não há dúvida sobre a necessidade de sua inclusão no currículo escolar. 

    Os resultados obtidos mostram que os gestores participantes da pesquisa entendem que a leitura compreensiva deve perpassar todas as áreas de conhecimento no currículo da Educação Básica porque tem se apresentado nas salas de aula significativa quantidade de estudantes com dificuldade em compreender o que leem. Espera-se com essa investigação que a comunidade escolar perceba a importância da leitura compreensiva como prática social, portanto, imprescindível no currículo da Educação Básica. \textcite{Leffa1996Fatores} destaca que em termos de intervenção pedagógica há atualmente uma preocupação maior com o processo do que com o produto da leitura.  

    É importante considerar durante as aulas que fazer perguntas e levar o próprio estudante a questionar sua leitura possibilita o desenvolvimento do pensamento metacognitivo, além de o próprio gestor questionar suas ações antes, durante e depois da sua realização. Esses saberes devem se constituir como conhecimento profissional inerente não apenas do professor, mas também do gestor, visto que ele está à frente no direcionamento da escola. 

    É mais do que urgente se pensar em uma política de incentivo mais arrojada para a formação de gestores e de professores, tendo como referência suas necessidades formativas. Mas não apenas centrando-se na formação continuada em serviço, mas também investimento na infraestrutura dos espaços escolares como, por exemplo, acervo da biblioteca atualizado e profissionais com competência para ocupar este espaço.

    \printbibliography[heading=subbibliography,notcategory=fullcited]

    \mybibexclude{Sanchez2012}
    \nocite{Flores2004Ensino}
    \nocite{RamalhoEtAl2003Ler}
    \mybibexclude{VanDijk1997}

    \label{chap:olhargestaoend}

\end{refsection}
