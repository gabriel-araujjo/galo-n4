\begin{refsection}
\renewcommand{\thefigure}{\arabic{figure}}

\chapterTwoLines
{De como as letras formam um cidadão}
{os ritos e símbolos da Primeira República na cidade de Parelhas-RN (1928--1930)}
\label{chap:decomoasletras}

\articleAuthor
{Laísa Fernanda Santos de Farias}
{Mestranda pelo Programa de Pós-Graduação em História UFRN-CERES, professora da rede privada. ID Lattes: 4075.8724.6125.7574. ORCID: 0000-0002-2025-1259. E-mail: nandafarias07@gmail.com.}

\articleAuthor
{Sebastião Genicarlos}
{Mestre em antropologia, UFRN, professor da rede pública. ID Lattes: 7476.4847.6789.2937. ORCID: 0000-0003-3529-2851. E-mail: sebastiaosantos710@gmail.com.}

\begin{galoResumo}
    \marginpar{
        \begin{flushleft}
        \tiny \sffamily
        Como referenciar?\\\fullcite{SelfFariasAndGenicarlos2021}\mybibexclude{SelfFariasAndGenicarlos2021}, p. \pageref{chap:decomoasletras}--\pageref{chap:decomoasletrasend}, \journalPubDate{}
        \end{flushleft}
    }
    O trabalho que se apresenta, elenca enquanto temática principal a análise de como o Plano de Propaganda Contra o Analfabetismo, criado no primeiro mandato do prefeito Florêncio Luciano na cidade de Parelhas, entre os anos de 1928 a 1930, trouxe para esse município algumas noções de progresso e civilidade impostos pela Primeira República. Assim, objetivamos com este texto, explorar os desejos de desenvolvimento para a cidade pensados por Florêncio Luciano, compreender em que medida este projeto educativo promoveu novas formas de sociabilidades, e apontar como o discurso desse prefeito estava ligado a uma rede de contatos e influências que pensavam a educação também a nível estadual e nacional.  Logo, essa investigação foi possibilitada pela exploração do discurso presente no relatório de mandato, apresentado em 1930 pelo prefeito supracitado, onde foram apontadas algumas intenções e alcances trazidos pelo seu projeto educativo para os alunos parelhenses. Desta feita, com os resultados do aprofundamento da fonte aqui explorada, é perceptível elencar que a partir da leitura e reflexão dessa parte da documentação do plano, aliada à bibliografia consultada, constatamos a composição do espaço educacional parelhense alinhada à ideia de modernização e inserção de novas sociabilidades. 
\end{galoResumo}

\galoPalavrasChave{Alfabetização. Cartografia educacional. Modernização. Sertão.}

\begin{otherlanguage}{english}

\fakeChapterTwoLines
{How the letters form a citizen}
{The rites and symbols of the First Republic in the city of Parelhas-RN (1928--1930)}

\begin{galoResumo}[Abstract]
    The work that is giving to know, presents as its main theme an analysis of how the Propaganda Against Illiteracy Plan, created during the first term of mayor Florêncio Luciano in the town of Parelhas, between the years 1928 and 1930, brought to that municipality some notions of progress and civility imposed by the First Republic. Thus, with this text, we aim to explore the development wishes Florêncio Luciano thought for the town; to understand to what extent that educational project promoted new forms of sociability; and to point out how that mayor's speech was linked to a network of contacts and influences that also thought education at the level of the state and of the nation. Therefore, this investigation was made possible by the exploration of the discourse present in the mandate report, presented in 1930 by the aforementioned mayor, which pointed some intentions and the scope his educational project brought for the students from Parelhas. This time, with the results of the deepening of the source explored here, it is noticeable that from the reading and reflection of these sources combined with the consulted bibliography, we can see the constitution of the educational space of the town of Parelhas aligned with the idea of modernization and the insertion of new sociability.
\end{galoResumo}

\galoPalavrasChave[Keywords]{Literacy. Educational cartography. Modernization. Sertão.}
\end{otherlanguage}

\section{Os prelúdios do plano}

\label{chap:decomoasletrasend}

\end{refsection}
