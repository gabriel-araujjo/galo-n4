\begin{refsection}
    \renewcommand{\thefigure}{\arabic{figure}}
    
    \chapterOneLine{A institucionalização da Matriz de Santa Luzia na cidade de Mosso\-ró-RN}
    \label{chap:institucionalizacao}
    
    \articleAuthor
    {Arthur Ebert Dantas dos Santos}
    {Discente do curso de Licenciatura em História da UERN. ID Lattes: 1007.7175.0723.9191. ORCID: 0000-0002-8150-0706. E-mail: arthur.ebert40@gmail.com.}

    \articleAuthor
    {Jackson Luiz Fernandes Adelino}
    {Discente do curso de Licenciatura em História da UERN. ID Lattes: 9611.7379.1926.1408. E-mail: jacksonadelino@alu.uern.br.}

    \articleAuthor
    {Lara Raquel de Souza e Maia}
    {Discente do curso de Licenciatura em História da UERN. ID Lattes: 3966.1343.1569.5403. E-mail: laramaia@alu.uern.com.}

    \articleAuthor
    {Valdeci dos Santos Júnior}
    {Professor Adjunto IV da UERN. ID Lattes: 5748.3825.9902.4802. ORCID: 0000-0002-5314-4943. E-mail: valdecisantosjr@hotmail.com.}
    
    \begin{galoResumo}
        \marginpar{
            \begin{flushleft}
            \tiny \sffamily
            Como referenciar?\\\fullcite{SelfSantosAtAl2021}\mybibexclude{SelfSantosAtAl2021}, p. \pageref{chap:institucionalizacao}--\pageref{chap:institucionalizacaoend}, \journalPubDate{}
            \end{flushleft}
        }
        Partindo de motivos culturais e religiosos, a maioria dos municípios brasileiros escolhe um santo católico para apadrinhar aquele local, assim, abençoando-os com sua determinada ``graça''. A cidade de Mossoró, do estado do Rio Grande do Norte tem como padroeira Santa Luzia. O presente artigo pretende realizar a discussão acerca da institucionalização da matriz de Santa Luzia, no município de Mossoró-RN, e de que forma esse processo esteve relacionado à elevação do povoado à condição de freguesia. A metodologia utilizada consiste na análise de discursos historiográficos produzidos acerca do tema, bem como a utilização de fontes eclesiásticas, como documentos de transações financeiras e construções de capelas, a fim de compreender quais os aspectos socioculturais que contribuíram à consolidação da referida santa como padroeira de Mossoró.
    \end{galoResumo}
    
    \galoPalavrasChave{Mossoró. História Eclesiástica. Rio Grande do Norte. Século XIX. Santa Luzia.}
    
    \begin{otherlanguage}{english}
    
    \fakeChapterOneLine
    {The institutionalization of Matriz de Santa Luzia in the city of Mossoró-RN}

    \begin{galoResumo}[Abstract]
        Thanks to cultural and religious reasons, most Brazilian municipalities choose a Patron Saint for blessing the place with their ``grace''. The city of Mossoró, in the state of Rio Grande do Norte, has Saint Lucy as its patron. This article aims to discuss the institutionalization of Matriz de Santa Luzia (St.~Lucy's Mother Church), in the municipality of Mossoró-RN, and how that process related to the elevation of the village to the status of \textit{freguesia}. The used methodology consists on analysis of historiographic speeches related to the topic, as well as the use of ecclesiastical sources in order to understand which socio-cultural aspects contributed to the consolidation of the aforementioned saint as patron of Mossoró.
    \end{galoResumo}
    
    \galoPalavrasChave[Keywords]{Mossoró. Ecclesiastical History. Rio Grande do Norte. 19th Century. Santa Luzia.}
    \end{otherlanguage}

    \printbibliography[heading=subbibliography,notcategory=fullcited]

    \hfill Recebido em 13 jan. 2021.

    \hfill Aprovado em 16 abr. 2021.

    \label{chap:institucionalizacaoend}

\end{refsection}
