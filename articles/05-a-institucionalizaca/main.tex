\begin{refsection}
    \renewcommand{\thefigure}{\arabic{figure}}
    
    \chapterOneLine{A institucionalização da Matriz de Santa Luzia na cidade de Mosso\-ró-RN}
    \label{chap:institucionalizacao}
    
    \articleAuthor
    {Arthur Ebert Dantas dos Santos}
    {Discente do curso de Licenciatura em História da UERN. ID Lattes: 1007.7175.0723.9191. ORCID: 0000-0002-8150-0706. E-mail: arthur.ebert40@gmail.com.}

    \articleAuthor
    {Jackson Luiz Fernandes Adelino}
    {Discente do curso de Licenciatura em História da UERN. ID Lattes: 9611.7379.1926.1408. E-mail: jacksonadelino@alu.uern.br.}

    \articleAuthor
    {Lara Raquel de Souza e Maia}
    {Discente do curso de Licenciatura em História da UERN. ID Lattes: 3966.1343.1569.5403. E-mail: laramaia@alu.uern.com.}

    \articleAuthor
    {Valdeci dos Santos Júnior}
    {Professor Adjunto IV da UERN. ID Lattes: 5748.3825.9902.4802. ORCID: 0000-0002-5314-4943. E-mail: valdecisantosjr@hotmail.com.}
    
    \begin{galoResumo}
        \marginpar{
            \begin{flushleft}
            \tiny \sffamily
            Como referenciar?\\\fullcite{SelfSantosAtAl2021}\mybibexclude{SelfSantosAtAl2021}, p. \pageref{chap:institucionalizacao}--\pageref{chap:institucionalizacaoend}, \journalPubDate{}
            \end{flushleft}
        }
        Partindo de motivos culturais e religiosos, a maioria dos municípios brasileiros escolhe um santo católico para apadrinhar aquele local, assim, abençoando-os com sua determinada ``graça''. A cidade de Mossoró, do estado do Rio Grande do Norte tem como padroeira Santa Luzia. O presente artigo pretende realizar a discussão acerca da institucionalização da matriz de Santa Luzia, no município de Mossoró-RN, e de que forma esse processo esteve relacionado à elevação do povoado à condição de freguesia. A metodologia utilizada consiste na análise de discursos historiográficos produzidos acerca do tema, bem como a utilização de fontes eclesiásticas, como documentos de transações financeiras e construções de capelas, a fim de compreender quais os aspectos socioculturais que contribuíram à consolidação da referida santa como padroeira de Mossoró.
    \end{galoResumo}
    
    \galoPalavrasChave{Mossoró. História Eclesiástica. Rio Grande do Norte. Século XIX. Santa Luzia.}
    
    \begin{otherlanguage}{english}
    
    \fakeChapterOneLine
    {The institutionalization of Matriz de Santa Luzia in the city of Mossoró-RN}

    \begin{galoResumo}[Abstract]
        Thanks to cultural and religious reasons, most Brazilian municipalities choose a Patron Saint for blessing the place with their ``grace''. The city of Mossoró, in the state of Rio Grande do Norte, has Saint Lucy as its patron. This article aims to discuss the institutionalization of Matriz de Santa Luzia (St.~Lucy's Mother Church), in the municipality of Mossoró-RN, and how that process related to the elevation of the village to the status of \textit{freguesia}. The used methodology consists on analysis of historiographic speeches related to the topic, as well as the use of ecclesiastical sources in order to understand which socio-cultural aspects contributed to the consolidation of the aforementioned saint as patron of Mossoró.
    \end{galoResumo}
    
    \galoPalavrasChave[Keywords]{Mossoró. Ecclesiastical History. Rio Grande do Norte. 19th Century. Santa Luzia.}
    \end{otherlanguage}

    \section{Introdução}

    O impacto da Igreja Católica ao longo da história da humanidade é enorme. Desde a Idade Medieval ela se estabeleceu como uma instituição de forte poder e influência onde estivesse. Mesmo durante a sua crise com as Reformas Protestantes, ela se manteve firme em seu papel de domínio, principalmente nas colônias do período. A América, por exemplo, foi um local que ela conseguiu se firmar bem (tanto nas colônias espanholas quanto nas portuguesas), já que as Metrópoles, no início da colonização, faziam questão de utilizar a religião como forma de domínio sobre os nativos. 

    Por ser um órgão tão presente no cotidiano colonial, acabou influenciando na forma como a dinâmica social se estruturou, incluindo o ``desenho'' das cidades nas colônias. Isso aconteceu por, muitas vezes, a Coroa, em especial a portuguesa, não se preocupou tanto em elaborar um planejamento urbano para aquela região, e a Igreja Católica como já possuía suas diretrizes, e atrelada às suas missões de evangelização, acabavam, intencionalmente ou não, formando povoados que posteriormente se transformaram em cidades. 

    Porém, apesar de serem da mesma instituição, a atuação da Igreja Católica na formação do espaço colonial variava de acordo com a Metrópole (Espanha ou Portugal). Nas colônias espanholas, por exemplo, a Coroa e a Igreja Católica conjuntamente se preocupavam com a organização da cidade, e seguiam o plano xadrez determinado pela Lei das Índias\footnote{Considerada a primeira legislação urbanística da Idade Moderna, a lei instituída por Filipe II, no ano de 1573, fez uma ``associação entre os princípios das ideais renascentistas, as influências do Tratado de Vitrúvio e as realizações concretizadas na América''. Ver mais em: \fullcite{Dantas2004Cidades}.}. Segundo \textcite{Medeiros2010Igreja}:

    \begin{quotation}
        Nas cidades hispano-americanas, as igrejas e os prédios oficiais são grandes, muito maiores e mais elaborados do que do lado brasileiro, superdimensionados para o tamanho e a população das cidades. O resultado é uma cidade densa, compacta, pequena, e com pouca vegetação. \cite[p.~59]{Medeiros2010Igreja}.
    \end{quotation}

    Já no caso das colônias portuguesas (incluindo o Brasil), a Igreja influenciava muito na urbanização, mas não da forma como acontecia na América Espanhola, e sim de uma forma desordenada. Como a Coroa Portuguesa não possuía a preocupação de fazer um planejamento urbano, e também por contar com dinheiro insuficiente para investimentos assim, esse papel ficou com a Igreja Católica, que possuía seus regimentos para a construção de templos religiosos, e com os beneficiários responsáveis pelas suas Sesmarias, que deveriam desenvolver a região. A Igreja possuía já em 1707 (quando foi redigido, mas publicado apenas em 1719), um planejamento para quando fosse necessário construir paróquias novas:

    \begin{quotation}
        Conforme direito Canônico, as Igrejas se devem fundar, e edificar em lugares decentes, e acommodados, pelo que mandamos, que havendo-se de edificar de novo alguma Igreja parochial em nosso Arcebispado, se edifique em sítio alto, e lugar decente, livre de humidade, e desviado, quando possível de lugares imundos e sórdidos\dots\footnote{Documento ``Constituiçoens primeyras do Arcebispado da Bahia feytas\dots'', publicado em 1719. \apud[p.~22]{Marx1991Cidade}[p.~195--196]{Parente1998Papel}.}
    \end{quotation}

    Ou seja, dessa forma, como as igrejas eram construídas nos lugares com um padrão habitacional mais bem definido, e com a Coroa não se preocupando com isso, ``era a partir da igreja que surgia às ruas, e não o contrário'' \cite[p.~195]{Parente1998Papel}.

    \textcite{Medeiros2010Igreja}, baseando-se em Bittar, Mendes e Veríssimo (2007) divide as ocupações religiosas no Brasil em três fases: a primeira (século XVI até o XVII) foi predominada pela presença dos Colégios de Jesuítas na formação de cidades; a segunda (início do século XVII até o século XVIII) teve como característica os grandes conventos das ordens religiosas; a última (todo o século XVIII) se caracterizou pela construção de igrejas e capelas no interior como forma de abrir os cultos de irmandades e confrarias, consolidando assim a dominação do interior do território brasileiro.

    No caso da Capitania do Rio Grande (atualmente, Rio Grande do Norte) essa prática de construção de capelas se tornou muito comum no século XVIII. O sertão da Capitania começou a ser mais povoado nesse período, pois estavam acontecendo muitos conflitos por sesmeiros de outras regiões (como a Bahia, por exemplo) estavam querendo terras na região e isso estava desagradando os fazendeiros. Então, intensificou-se a cobrança para que os sesmeiros beneficiados com a terra ocupassem-a e produzissem nela. Assim, com a fomentação no comércio da região foi-se criando várias rotas e pontos comerciais que posteriormente tornaram-se vilas \cite{Monteiro2000Caminhos}.

    Uma cidade que provavelmente passou por esses dois processos (o religioso e o comercial) foi Mossoró, como destaca \textcite[p.~83]{Monteiro2000Caminhos}. Tendo em vista que a fundação de Mossoró data de 1852, um ponto muito importante de se destacar é o papel que a capela de Santa Luzia teve dentro desse processo, tendo em vista que ela data de 1772. A formação da própria capela também sugere a que a teoria apresentada por Monteiro é plausível, tendo em vista que ela surgiu no sítio que levava o nome da Santa:

    \begin{quotation}
        Na ribeira de Mossoró em 1739, já era conhecido pelo nome de Santa Luzia o sítio onde se acha edificado a cidade de Mossoró, provando-se isto por uma carta de data e Sesmaria concedida ao Capitão João do Vale Bezerra, de uma das terras em um córrego grande que deságua no rio Mossoró chamado Saco Grande (hoje Açude do Saco) junto de Santa Luzia, em abril daquele ano. \cite[p.~52]{Souza2010Historia}.
    \end{quotation}

    De acordo com \textcite{Souza2010Historia}, a capela de Santa Luzia foi elevada à nomeação de ``freguesia independente'' em 1842:

    \begin{quotation}
        Em 1842, em virtude da Lei Provincial Nº 87 de 27 de outubro, a capela de Santa Luzia de Mossoró, filial da do Apodi, foi declarada Freguesia independente; e  posta em concurso, o Padre Antônio Joaquim, que ainda era diácono, submetendo-se a dito concurso, foi aprovado e promovido Pároco colado da nova Freguesia da qual só tomou posse em 1844, assistida pelos seus irmãos de hábito os padres Francisco Longino Guilherme de Melo, Leonardo de Freitas Costa, José Antônio Lopes da Silveira e Florêncio Gomes de Oliveira."\:\cite[p.~131]{Souza2010Historia}.
    \end{quotation}

    É interessante notarmos que o processo de formação da localidade de Mossoró sempre esteve envolto a questões religiosas: "Em 1852, pela lei Provincial nº 246 de 15 de março, foi a povoação de Santa Luzia elevada à categoria de Vila e Criado município, para cujo ato muito influiu o Padre Rodrigues, que já era conhecido na Província, como político de prestígio na localidade de sua residência."\:\cite[p.~135]{Souza2010Historia}. O autor ainda versa sobre como era a situação da referida freguesia, bem como as dinâmicas sociais e econômicas aqui presentes:

    \begin{quotation}
        A povoação de Santa Luzia consistia em um pequeno quadro de casas de construção péssima e sem arquitetura, a maior parte casas de taipa, em frente da pequena Capela, um pouco deteriorada, com o teto quase todo abaixo e qual havia sido construída em 1772 pelo Sargento-mór Antônio de Souza Machado. A nova Freguesia era pobre; o comércio quase nulo; os poucos negociantes que haviam traziam do Aracati as mercadorias em costas de animais; agricultura pouca, consistindo a sua maior riqueza na indústria pastoril, cujo principais fazendeiros eram os membros das famílias denominadas --- `Camboa', `Guilherme' e `Ausentes' --- as mais numerosas do lugar segundo a tradição."\:\cite[p.~132]{Souza2010Historia}.
    \end{quotation}

    Como podemos perceber na citação acima, a povoação de Santa Luzia era pequena, pouco desenvolvida, com poucas habitações e, consequentemente, poucos habitantes. Em relação a sua economia, sua arrecadação era insuficiente fazendo com que o crescimento da vila ficasse comprometido. Como já citado, sua formação está ligada a questões religiosas, visto que assim como aconteceu em cidades como São Roque, cidade do Estado de São Paulo, e Caruaru, cidade do Estado da Paraíba, a cidade de Mossoró, até então vila, também surgiu nos arredores de uma capela.

    Sobre a influência exercida pela religião em grupos menos abastados, \textcite[p.~9]{Faco1972Cangaceiros} dirá que: ``A única forma de consciência do mundo, da natureza, da sociedade, da vida, que possuíam as populações interioranas, era dada pela religião ou por seitas nascidas nas próprias comunidades rurais, variantes do catolicismo.''. Assim, percebe-se que a estruturalização das moradias em torno da Capela de Santa Luzia, foi um ato baseado, e guiado, pela fé de seus moradores, sendo este uma ação bastante comum no interior do Nordeste, assim como o ``apadrinhamento'' da localidade por um santo, no caso de Mossoró, a padroeira tornou-se, no século XX, Santa Luzia, a virgem de Siracusa.

    \section{Desenvolvimento}

    A localidade do sítio de Santa Luzia teve dois principais donos, o capitão Teodorico da Rocha, anterior a 1739, e em 1754 até 1770, o sargento-mór da Ribeira, José de Oliveira Leite \cite[p.~53]{Souza2010Historia}. Em 1770 quem assumiu o sítio de Santa Luzia foi o Sargento-mór Antônio de Souza Machado; no mesmo ano, ele e sua esposa, D. Rosa Fernandes, começaram a buscar a construção da capela, que levaria o mesmo nome do sítio. De acordo com Fausto de Souza, a construção da capela data do ano de 1772:

    \begin{quotation}
        Em 5 de agosto de 1772 por Provisão assinada pelo reverendo Padre Inácio de Araújo Gondim, Vigário Colado da Freguesia de Santo Amaro de Jaboatão, de Pernambuco, então visitador dos sertões do Norte, foi concedida licença ao mesmo Sargento mór António de Sousa Machado e sua mulher, por assim haverem requerido, para erigirem uma capela tendo como invocação Santa Luzia, na ribeira de Mossoró da mencionada freguesia, autorizando o referido visitador, na aludida Provisão [\dots]. Essa capela foi construída, de pedra e cal, no mesmo ano de 1772, e no mesmo lugar aonde se acha hoje edificada a matriz de Mossoró pelo referido Sargento-mór que com ela despendeu a quantia de 590\$770\,{}rs. \cite[p.~54]{Souza2010Historia}.
    \end{quotation}

    No momento de sua fundação, a igreja era pertencente à Freguesia de Nossa Senhora da Conceição e de São João Batista das Várzeas do Apodi. 

    No ano de 1842, 120 anos após a capela ser erguida, a mesma é elevada ao status de Matriz \cite[p.~57]{Souza2010Historia}, todavia, apesar do sítio e a capela levarem o nome de Santa Luzia, a mesma não era a padroeira ou figura religiosa da localidade ainda, visto que a escolha de um padroeiro dependia da sensibilidade dos moradores aliados ao simbolismo que o santo representava para aquela localidade. A seguir podemos ler uma citação da obra de Fausto de Souza, onde ele nos expõe a transcrição de um documento da Igreja de Mossoró:

    \begin{quotation}
        Até 1854 só existia em Mossoró uma irmandade religiosa --- a de Nossa Senhora do Rosário --- dos homens pretos, criada segundo presumimos em 1786. Em 1855, porém, o Padre Antonio Joaquim criou a irmandade da Senhora Santa Luzia, Padroeira da Freguesia, conforme se vê no documento que se segue, cópia de um livro da igreja de Mossoró: ``Ata da primeira reunião e institucionalização da irmandade da Senhora Santa Luzia, Orago desta Freguesia da Vila de Mossoró, com a baixo se declara: --- Aos dois dias do mês de Fevereiro do ano de mil oitocentos e cinquenta e cinco, no corpo da Igreja Matriz desta Freguesia de Mossoró, pelas nove horas e meia da manhã do sobredito dia, antes da missa Conventual, se acha reunido grande concurso de povo, conforme o convite feito pelo Reverendo Vigário da Freguesia Antonio Joaquim Rodrigues á estação de varias missas conventuais a fim de organizar-se uma irmandande da Senhora Santa Luzia, Orago desta Freguesia de Mossoró, e achando o mesmo vigário boa vontade em seus fregueses para o fim convocado, mandou colocar no corpo da Igreja Matriz, mesa e assentos e tomou assento com grande parte de seus fregueses, e em seguida se procedeu ao alisamento em caderno, o qual deve ser  transferido para um livro, logo que haja e foram escritos  ou alistados no caderno mais de duzentos nomes de indivíduos de um e outro sexo; concluindo o alistamento tratou-se do encargo de organizar os artigos de compromisso, e foram todos concordes que, querendo o respectivo Vigário encarregar-se desse trabalho, ficarão satisfeitos pelo que foi o trabalho aceito. [\dots]''\:\cite[p.~147--148]{Souza2010Historia}.
    \end{quotation}

    Analisando o documento em questão, Fausto de Souza afirma que existia em Mossoró a irmandade religiosa de Nossa Senhora do Rosário dos homens pretos, criada, de acordo com o autor, aproximadamente no ano de 1786 \cite[p.~147]{Souza2010Historia}. Como podemos interpretar, a irmandade existia antes da fundação da Irmandade da Senhora Santa Luzia, o que muito provavelmente deveria representar a antiga padroeira da Freguesia, antes da escolha de Santa Luzia.  

    O grande ponto é: por que a capela havia deixado de ser uma freguesia de Apodi e se tornado uma matriz? Como foi esse processo e quem esteve por trás? O estudioso Luís Câmara Cascudo defende a tese de que o processo de elevação da capela para Matriz foi um ponto importantíssimo para que posteriormente Mossoró se tornasse um município de fato \cite[p.~54]{Cascudo2010Notas}. Isso se sustenta tendo em vista que elevar-se ao posto de Matriz depende de um comércio pelo menos estável para sustentar o local. Então, mais que um marco religioso, tornar-se Matriz era também um marco econômico, visto que as Matrizes, geralmente, eram construídas em cidades que possuíam certa influência comercial para sua região.

    No Acervo Oswaldo Lamartine\footnote{Disponível em \url{https://colecaomossoroense.org.br/site/acervo-oswaldo-lamartine/}.}, presente na Coleção Mossoroense, é possível achar o livro História de Mossoró, escrito por Francisco Fausto de Souza. Esse livro traz alguns documentos muito importantes que podem auxiliar na pesquisa, para que seja possível entender como era essa relação de poder que o sucesso da igreja traria ao vilarejo e vice-versa. 

    Inicialmente, para a construção da capela gastou-se um grande valor \cite[exatos 590\$770 rs, como consta no livro de ][p.~54]{Souza2010Historia}. Com o falecimento do Sargento-mór Antonio de Souza Machado no fim do século, foi feita uma visita e nela analisada as contas da capela, onde foi constatado que o fundador estava devendo 24\$400 réis à capela (ou seja, também à Matriz em Apodi), enquanto a capela estava devendo ao seu administrador 41\$290 réis. Basicamente, ao fim do século XVIII a Capela de Santa Luzia estava ``em crise'', com um saldo negativo. A partir desta informação podemos identificar o quanto se apostava na construção e fundação de uma capela que representasse a ``diocese local'', além de que mesmo com a edificação da estrutura, a capela estava passando por uma crise, talvez representando a falta de identificação dos fiéis da freguesia para com a Santa.

    Porém, de acordo com os dados que \textcite[p.~58]{Souza2010Historia} traz, a capela teve pelo menos outros três administradores antes de ser elevada ao patamar de Matriz. Com o passar dos anos, e analisando as contas, foi possível ver que houve uma melhora financeira, passando a receber muitas doações, e até mesmo pessoas ``devendo'' à paróquia. E essa melhoria não se dava somente em forma de dinheiro, mas também em terras e animais para a pecuária, como é possível ver no trecho da reportagem de Francisco Fausto para ``O Nordeste'', publicada em 15 de maio de 1929:

    \begin{quotation}
        O patrimônio da Capela de Santa Luzia, em 1842 quando foi elevada a categoria de Matriz, era o seguinte: uma légua de terra em quadro no sítio ``Canto do Junco'', doada por Domingos Fernandes e sua mulher; um pedaço de terra no sítio ``Santa Luzia'', a começar do Córrego da Calheira (que hoje chamam da Rua dos Cavalcanti) até Macacos, arrendando as terras do defunto José da Costa de Oliveira Barca, doada em 1801 por D. Rosa Fernandes, viúva do Sargento-mór Souza Machado; uma porção de terra no lugar Macacos, deixadas em testamento à mesma santa, por Manoel da Costa de Oliveira Barca, vulgo Manoel Ferreira, falecido em Recife; uma sorte de terras no Riacho Grande do Juazeiro que deu a pagamento à santa, Francisco da Costa Correia; uma casa na povoação, junto da capela doada por Manoel Ferreira, a qual em 1820, segundo uma declaração do Procurador João Joaquim de Melo servia de moradia do capelão. Também tinha a santa, gados situados em Santa Luzia e na barra de Mossoró, etc.
    \end{quotation}

    Outro ponto importante para que fosse possível a elevação para Matriz eram os materiais que estavam dispostos na capela para a realização de eventos religiosos no local. A partir dos relatórios de controle da Matriz, relacionados ao tempo em que esta era Capela, relata-se que desde a criação da capela até o ano de 1816 aconteceram sete visitas de clérigos para constatar a situação da capela, como estava as alfaias e ornamentos. Na primeira visita registrada, em 1775, a relação feita pelo Visitador Alexandre Bernardino dos Reis constava uma imagem da milagrosa Senhora Santa Luzia, um crucifixo para o altar, um permanente xamalete branco, uma alva de pano de linho, dois corporais e um sanguinho, uma toalha, uma cálice de prata, um missal novo com mola, uma imagem do Senhor, uma imagem de São Gonçalo, uma toalha para o altar e uma pedra d'ara \cite[p.~66]{Souza2010Historia}.

    Já na segunda visita, em 1779, realizada por Joaquim Monteiro da Rocha, mostrou um inventário com bem mais recursos: uma igreja feita de pedra e cal, uma imagem no altar-mór de Santa Luzia, uma do Senhor Crucificado, uma imagem de Nossa Senhora do Rosário em seu altar, uma pedra d'ara, uma frontal do chamaleto do altar-mór, duas toalhas do dito altar, de bertanha; um ornamento do xamaleto, com alva, um m. cordão, um frontal do xamalete e toalha de bertanha no altar o Rosário, um cálice prata, dois copos e dois saguinhos, um par de galhetas de estanho e um vaso de comunhão, três vaqueras de ferro; uma imagem de Santa Luzia.

    Quando foram apresentadas as petições para que Santa Luzia de Mossoró fosse uma Freguesia independente, o Bispo Dom João da Purificação Marques Perdigão respondeu que seria, sim, possível à criação da Freguesia, desde que ``esta igreja estiver preparada para ser Matriz, possuindo ao mesmo tempo as utensilias, alfaias e paramentos necessários para a administração dos Sacramentos'' \cite[p.~62]{Cascudo2010Notas}. Apesar de não haver as documentações que foram apresentadas por Antônio Francisco Fraga Júnior, é sabido que foram suficientes para a criação da Freguesia de Santa Luzia, no ano de 1842.

    \section{Elevação à freguesia}

    \textcite{Cascudo2010Notas} aponta que um dos principais nomes para a concretização da capela de Santa Luzia em Matriz foi Antônio Francisco Fraga Júnior, conhecido como Fraguinha, personalidade de confiança dos moradores. Fraga recorre ao Bispo Diocesano, dom João da Purificação Marques Perdigão. Todavia, este último não podia criar Freguesia, apenas poderia aprovar criações, visto que, como aponta Fausto de Souza, a criação de Freguesias era um assunto legislativo, e por consequência, o processo foi direcionado à Comissão de Negócios Eclesiásticos e mais partes, que pertencia à Assembléia \cite[p.~50]{Cascudo2010Notas}. O processo tramitava entre a Assembléia Legislativa e a aprovação ou desaprovação do Bispo Diocesano. Então Fraga, como versa Câmara Cascudo, ``na agonia do sonho'' \cite[p.~51]{Cascudo2010Notas} apresenta uma petição à Assembleia no ano de 1839, a qual se encontra transcrita na obra de Cascudo:

    \begin{quotation}
        Os habitantes da Povoação de S. Luzia do Mossoró desta Província, representados na presente Petição por Antonio Francisco Fraga Junior chegam a esta Assembleia reclamando um benefício que a vista das circunstâncias parece merecer a justiça de seus legisladores, o qual passa a expor: Distando quinze léguas da Matriz da Vila do Apodi, cuja Freguesia pertence aquela Povoação, lhe fica por esta distância e mau caminho, máxime pelo inverno assaz penoso os recursos espirituais, tendo sucedido já pela demora deles se finarem pessoas sem receberem esses alimentos que caracterizam o Cristo que tem abraçado a Doutrina Católica, Apostólica, Romana, que felizmente professa-se no Império do Brasil. [\dots] Para a boa tranquilidade das consciências dos habitantes da mesma povoação, é de suma necessidade que a Capela ali existente seja elevada a Matriz com a nominação de Freguesia de Santa Luzia do Mossoró- pela razão de que sendo Matriz há de ter Pároco, e tendo Pároco, os recursos são prontos, por não embaraçar a isto os motivos que vêem de ponderar. A Capela por seu asseio, decência, apartamentos e mais necessários é digna de ser elevada a categoria de Matriz, e conquanto a Câmara respectiva afirme ocupar só setecentas almas de Comunhão, os habitantes da Povoação sem temer de errarem, dizem que fixada a divisa da Freguesia pelos pontos que passam a indigitar, pode conter pouco mais ou menos quatro mil almas. [\dots] Enfim os habitantes de Mossoró contam com o apoio da assembleia e lisojeam-se com a expedição do pedido. \cite[p.~51--52]{Cascudo2010Notas}.
    \end{quotation}

    A petição se estende por várias linhas, mas os pontos que desejamos ressaltar foram expostos na citação acima. Por se tratar de um assunto religioso, Câmara Cascudo aponta que as Câmaras Municipais interessadas nesse processo não podiam se opor, mas podiam dar suas opiniões em relação aos ``inconvenientes da causa, falando sobre limites propostos, número de almas de comunhão que lhe ficam pertencendo'' \cite[p.~50]{Cascudo2010Notas}, tal qual podemos conferir no discurso de defesa que Fraga coloca, onde ele faz referência a questão das almas de comunhão. Bem como podemos conferir a questão da religiosidade e da economia, quando o mesmo cita que se elevada à categoria de Matriz, a capela contará com a presença de um Pároco, e por consequência, a chegada de recursos advindos do poder eclesiástico se torna mais fácil, tendo em vista que até aquele momento esses recursos iam para a Matriz em Apodi, e de lá eram encaminhados para a capela. 

    Na primeira petição, o parecer foi negativo, dado cinco dias após o pedido, tendo o argumento de que isso envolveria a mudança de limites das Freguesias de Apodi, Campo Grande e Princesa \cite[p.~53]{Cascudo2010Notas}. Porém, Fraga Júnior apresentou, no ano seguinte, outra petição com a base argumentativa praticamente igual à do ano anterior. Mas, paralelamente, outros moradores do povoado em comunidades próximas também fizeram uma petição contra argumentando que não era necessária essa nova Freguesia, tendo em vista que, para eles, as funções exercidas pela Matriz de Apodi estavam sendo cumpridas devidamente. Ambas as petições foram apuradas, mas o parecer para o pedido de Fraga Júnior foi negado novamente, mas dessa vez por não haver a permissão do Bispo. Assim como foi ressaltado, muito provavelmente, as razões que levaram os moradores de povoados vizinhos a construir uma nova petição, agora desaprovando a elevação da Capela, seria uma possível separação de territórios, o que poderia influir no contingente que a Matriz de Apodi conseguia reunir, visto que sues fieis estavam satisfeitos com a atuação da Matriz.

    Seguiram-se as discussões na Assembleia. Um parecer positivo surge em nome de Dom João da Purificação Marques Perdigão, em 1841:

    \begin{quotation}
        Ilmº, Snrº. Examinando o requerimento o documento que V.S. menciona no Ofício que Nos dirigis em data de 17 d´outubro de 1840. que ora Nos foi apresentado, à cerca da criação de uma nova Freguesia de Santa Luzia de Mossoró, parece-nos que pode ser criada esta Freguesia, depois que esta igreja estiver preparada para ser Matriz, possuindo ao mesmo tempo as utensilias alfaias, e paramentos necessárias para a administração dos Sacramentos.

        Queira V. S. levar ao conhecimento d´Assembleia Legislativa dessa Província este Nosso parecer para sua inteligência.   

        Deus Guarde a V.S. muitos anos. Seminário Episcopal d´Olinda 23 de Novembro de 1841. \cite[p.~61--62]{Cascudo2010Notas}
    \end{quotation}

    Importante ressaltar que, em um primeiro momento, foram negadas as petições que tinham como objetivo elevar a Vila a Freguesia, pois aparentemente uma Freguesia precisa possuir certo aparato e dispor de uma estrutura que suporte as atividades desempenhadas na Matriz, como a celebração rotineira de missas e o acolhimento de um maior número de fieis. Entretanto, como podemos ver na citação de Câmara Cascudo, a Assembleia deu parecer positivo ao constatar que a localidade em questão tinha chances de alcançar os patamares aceitáveis para tal criação da nova Freguesia, logo, elevando a capela à Matriz.

    Podemos constatar a partir da citação de que transformar a igreja em uma Matriz era essencial à elevação do povoado em Freguesia, e que também era de suma importância a futura Matriz possuir aporte material para a administração das atividades sacramentais. Fraga Júnior reuniu os documentos necessários e solicitou a aprovação que foi conseguida, como podemos ver no trecho a seguir: 

    \begin{quotation}
        O Alferes Alexandre de Souza Rocha, Administrador da Igreja Capela de S. Luzia de Mossoró em virtude da Lei; Atesto que a Capela de S. Luzia desta povoação tem todos os paramentos e alfaias que necessite uma Matriz e não precisa de reparos nenhum, já se acha pronta, e decente, é quanto sei, e posso atestar em fé de verdade. Mossoró 8 de setembro de 1842.

        \vspace{12pt}

        \noindent{}ALEXANDRE DE SOUZA ROCHA 

        \vspace{12pt}

        \noindent{}N. 187. \\
        Pg Cento e vinte reis de Selo. \\
        Natal 19 de setembro de 1842. \\
        ALCOSTA. SIABRA DE MELLO. \cite[p.~65]{Cascudo2010Notas}.
    \end{quotation}

    Após esse parecer positivo, Fraga Júnior em 1842, e no dia 24 de outubro é elevada à categoria de Matriz a antiga Capela de Santa Luzia, como podemos ver na citação a seguir:

    \begin{quotation}
        D. Manoel d'Assis Mascarenhas, Presidente da Província do Rio Grande do Norte. Faço saber a todos os seus Habitantes que a Assembleia Legislativa Provincial Decretou, e eu Sancionei a Resolução seguinte: Art. 1º - Fica desmembrada da Freguesia do Apodi, e elevada à Categoria de Matriz a Filial Capela de S. Luzia de Mossoró, conservando a mesma Fábrica, e Guizamento, que a Matriz de que é desmembrada. \cite[p.~66]{Cascudo2010Notas}.
    \end{quotation}

    Vale citar também outros pontos importantes presentes nesta resolução, como os limites do povoado de Mossoró e sua incorporação ao Município da Vila da Princesa, sendo, atualmente, o município de Assú. Em 15 de março de 1852, através da Lei nº 246, o até então povoado de Mossoró adquire sua autonomia e se torna município, assim, se desincorporando da Comarca de Assú. Passados quase dois séculos, a influência religiosa, e econômica, exercida pela Matriz, hoje Catedral, de Santa Luzia, continuam sendo poderosos fatores dentro da cultura da cidade. Onde, a festa de celebração à Padroeira é um dos maiores eventos religiosos do Estado do Rio Grande do Norte, movimentando milhares de fiéis de todo o Brasil e aquecendo a economia mossoroense.

    \section{Conclusão}

    Através da análise da documentação e dos discursos relacionados à transformação do povoado em Freguesia independente e a Capela de Santa Luzia em Matriz, podemos afirmar que ambos os processos estiveram extremamente associados, pois a elevação da capela à condição de matriz foi um fator que implicou diretamente para o sítio de Santa Luzia recebesse o título de Freguesia independente em 1842. Vale ressaltar que além de auxiliar na condição de Freguesia independente, o status de Matriz conferiu ao então povoado de Mossoró uma maior atenção econômica, e independência em vários âmbitos antes atrelados à Freguesia do Apodi.

    A conquista da nomeação de Freguesia independente se deu através da reivindicação por parte dos moradores do povoado na Assembleia Legislativa, tendo esta que passar pela aprovação da Assembleia, do Bispo e das outras Freguesias e povoados.  Sendo assim, nota-se que o caso de Mossoró segue a questão afirmada por Gomes Parente, de que era a partir da igreja que surgia a cidade \cite[p.~195]{Parente1998Papel}.

    No quesito metodológico, é de suma importância ressaltar as contribuições das produções de Câmara Cascudo e Francisco Fausto de Souza, ambas do ano de 2010, para a escrita deste trabalho, visto que os documentos apresentados no artigo são citações de transcrições presentes nas obras de ambos.

    \printbibliography[heading=subbibliography,notcategory=fullcited]

    \hfill Recebido em 13 jan. 2021.

    \hfill Aprovado em 16 abr. 2021.

    \label{chap:institucionalizacaoend}

\end{refsection}
