\begin{refsection}
    \renewcommand{\thefigure}{\arabic{figure}}
    
    \chapter[``Vencido o {\itshape New Look}'': {\itshape resistências femininas a Christian Dior e as suas modas (Natal/RN, 1948--1953)}]{``VENCIDO O {\itshape NEW LOOK}\footnote{Em tradução literal, \textit{new look} significa ``novo olhar'', tal termo foi designado por Carmel Snow, repórter norte-americana da revista Vogue, quando foi cobrir a coleção de lançamento da marca Dior. Termo que será abordado de forma mais profunda no decorrer do trabalho.}''\\Resistências femininas a Christian Dior e as suas modas (Natal/RN, 1948--1953)}

    \label{chap:vencidonewlook}
    
    \articleAuthor
    {João Vieira Neto}
    {Discente do curso de Licenciatura em História. Bolsista IC/Propesp/UFRN. ID Lattes: 9127.3680.8474.7839. ORCID: 0000-0001-7244-4573. E-mail: jvieiran00@gmail.com.}

    \articleAuthor
    {Joel Carlos de Souza Andrade}
    {Professor do Departamento de História/CERES/UFRN e do Mestrado em História dos Sertões/UFRN. ID Lattes: 6752.7281.1456.8336. ORCID: 0000-0003-2141-0212. E-mail: jocadesoan@yahoo.com.br.}
    
    \begin{galoResumo}
        \marginpar{
            \begin{flushleft}
            \tiny \sffamily
            Como referenciar?\\\fullcite{SelfVieiraNetoAndAndrade2021}\mybibexclude{SelfVieiraNetoAndAndrade2021}, p. \pageref{chap:vencidonewlook}--\pageref{chap:vencidonewlookend}, \journalPubDate{}
            \end{flushleft}
        }
        Discute o estilo \textit{``new look''} desenhado pelo estilista francês Christian Dior em 1947, e como este foi apropriado e criticado pelas moças em Natal durante os anos de 1948--1953, a partir do periódico potiguar \textit{Diário de Natal}. No caso do francês, suas reportagens destacaram a sua ousadia em relação à criação visto que a situação mundial estava desfavorável à moda em virtude da economia fragilizada, causada pela Segunda Guerra Mundial. Entretanto, as intempéries econômicas não foram suficientes para que Dior suspendesse suas criações, e apresentasse uma nova silhueta, a qual acentuava a feminilidade. Entre as notícias, é possível perceber resistências através do discurso jornalístico, devido os problemas econômicos. Metodologicamente, o estudo constitui-se a partir de consultas ao \textit{Diário de Natal}, e análise de autores como Gilles Lipovetsky (1989), Medeiros Filho (2014) e Luca (2009), essenciais para a tessitura do estudo em questão. As informações sobre a moda feminina eram refletidas entre 1948--1950 no periódico na seção ``Notícias da Moda'' e expunha notícias acerca do modismo vigente, marcado pela criação de Christian Dior. A despeito do descontamento das natalenses, no contexto pós Segunda Guerra Mundial, as criações do estilista não foram interrompidas e bem como suas apropriações por tais moças resistentes.
    \end{galoResumo}
    
    \galoPalavrasChave{Imprensa. História da Moda. Cidade do Natal.}

    \filbreak

    \begin{otherlanguage}{english}

    \fakeChapterTwoLines
    {The \textit{New Look} vanquished}
    {women's resistance to Christian Dior and his fashions (Natal/RN, 1948--1953)}

    \begin{galoResumo}[Abstract]
        It discusses the ``new look'' style designed by the French designer Christian Dior in 1947, and how it was appropriated and criticized by the girls in Natal during the years 1948--1953, from the Rio Grande do Norte newspaper \textit{Diário de Natal}. In the case of the French, his reports highlighted his boldness in relation to creation, since the world situation was unfavorable to fashion due to the fragile economy, caused by the Second World War. However, the economic maelstrom was not enough for Dior to suspend his creations, and to present a new silhouette, which accentuated femininity. Among the news, it is possible to perceive resistance through the journalistic discourse, due to economic problems. Methodologically, the study is constituted by consultations with the Diário de Natal, and analysis of authors such as Gilles Lipovetsky (1989), Medeiros Filho (2014) and Luca (2009), who are essential for the design of the study in question. Information about women's fashion was reflected between 1948--1950 in the newspaper, at the section ``Notícias da Moda'' (News of Fashion), which exposed news about the fad of the time, marked by the creation of Christian Dior. Despite the Natalense discontent, in the post-World War II context, the stylist's creations were not interrupted and neither were his appropriations by such resistant girls. 
    \end{galoResumo}
    
    \galoPalavrasChave[Keywords]{Press. History of Fashion. City of Natal.}
    \end{otherlanguage}

    

    \printbibliography[heading=subbibliography,notcategory=fullcited]

    \hfill Recebido em 15 abr. 2021.

    \hfill Aprovado em 19 abr. 2021.

    \label{chap:vencidonewlookend}

\end{refsection}
