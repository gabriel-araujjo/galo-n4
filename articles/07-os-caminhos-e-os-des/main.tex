\begin{refsection}
    \renewcommand{\thefigure}{\arabic{figure}}
    
    \chapterOneLine{Os caminhos e os desdobramentos da vida, trajetória política e dos discursos e pronunciamentos de Dinarte Mariz}
    \label{chap:caminhosdesdo}
    
    \articleAuthor
    {Larisse Santos Bernardo}
    {Mestranda em História dos Sertões (UFRN-CERES, Caicó). ID Lattes: 8111.2333.0951.3952. ORCID: 0000-0001-8427-7303. E-mail: larissesantosbernardo@yahoo.com.br.}

    \articleAuthor
    {Jailma Maria de Lima}
    {Professora de História do Departamento de História na UFRN-CERES, Caicó. ID Lattes: 7070.0102.8841.6835. ORCID: 0000-0001-8689-1753. Orientadora da referida pesquisa.}

    \begin{galoResumo}
        \marginpar{
            \begin{flushleft}
            \tiny \sffamily
            Como referenciar?\\\fullcite{SelfBernardoAndLima2021}\mybibexclude{SelfBernardoAndLima2021}, p. \pageref{chap:caminhosdesdo}--\pageref{chap:caminhosdesdoend}, \journalPubDate{}
            \end{flushleft}
        }
        O presente trabalho tem por desígnio traçar os percursos analíticos da trajetória de vida e política, como também das reverberações acerca dos discursos e pronunciamentos a partir da figura eminente de Dinarte de Medeiros Mariz. Político este que se firmou no cenário da política nacional e, que por sua vez, deixou marcas registradas na política potiguar a partir de suas raízes no Seridó norte-rio-grandense. Nessa perspectiva o desenvolvimento da narrativa se dará através de um viés que traçará sua vida política mostrando as alianças criadas, os degraus percorridos por Dinarte, levando em consideração que antes de se tornar figura política, ele foi comerciante e agropecuarista, experiências que corroboraram para o seu fortalecimento na vida política. Assim, a partir desses encaminhamentos busca-se concernir o porquê da manutenção no imaginário político e social em torno da figura pública, uma vez que o mesmo conquistou impulsividade pública ocupando vários cargos, entre os quais, prefeito da cidade de Caicó, governador do Estado do Rio Grande do Norte e Senador da República.
    \end{galoResumo}
    
    \galoPalavrasChave{Dinarte Mariz. Vida. Trajetória Política. Discursos.}
    
    \begin{otherlanguage}{english}
    
    \fakeChapterOneLine
    {Ways and unfoldings of Dinarte Mariz's life, political trajectory, speeches, and pronouncements.}

    \begin{galoResumo}[Abstract]
        This work aims to trace the analytical paths of life and politics, as well as the reverberations about the speeches and pronouncements by the eminent figure of Dinarte de Medeiros Mariz. Politician who was established in the national politics scene and, in turn, left his mark etched in the potiguar politics starting from its roots in Seridó of Rio Grande do Norte. In this perspective, the narrative development will take place through a bias that will trace his political life showing created alliances, the steps taken by Dinarte, taking into account that, before becoming a political figure, he was a trader and farmer, experiences that corroborated for strengthening of his political life. Thus, based on these referrals, we seek to concern the reason why the political and social imaginary maintains around that public figure, since he gained public impulsiveness by occupying various positions, among which, mayor of Caicó, governor of the State of Rio Grande do Norte and Senator of the Republic.
    \end{galoResumo}
    
    \galoPalavrasChave[Keywords]{Dinarte Mariz. Life. Political trajectory. Speeches.}
    \end{otherlanguage}

    \section{Introdução}

    O presente trabalho tem por desígnio traçar os percursos analíticos da trajetória de vida e política, como também das reverberações acerca dos discursos e pronunciamentos a partir da figura eminente de Dinarte de Medeiros Mariz. Político este que se firmou no cenário da política nacional e, que por sua vez, deixou marcas registradas na política potiguar a partir de suas raízes no Seridó norte-rio-grandense.  

    Para isso, é necessário primordialmente apresentar a pessoa de Dinarte de Medeiro Mariz\footnote{Ver em: \textcite[p.~220]{Maia2005Dinarte}}, assim registrado, mais conhecido por seu segundo sobrenome Mariz. Observando a vida e sua trajetória política, o artigo intitulado \textit{Período Republicano} da fundação José Augusto, o mesmo descreve que \textit{Dinarte de Medeiros Mariz}, nasceu na Fazenda Solidão em Serra Negra-RN\footnote{[\dots] O professor Vergniaud Lamartine Monteiro explica o nome da região no semi-árido nordestino: ``Os primeiros batedores da região, localizados na fralda sudeste da serra, verificaram serem as suas encostas acentuadas noruegas, as quais davam à serra aquele lugar, ao tempo imerso em vegetação sombria e matarias virgens, um aspecto negro''. Na região, além da vegetação de menor porte, convivem o juazeiro, a oiticica, a jurema, o angico e o pau-d'arco. É o coração do Seridó. \cite[p.~39--40]{Lima2003Solidao}.} no dia 23 de agosto de 1903, filho de Manuel Mariz Filho e de Maria Cândida de Medeiros Mariz o quinto entre quatorze filhos do casal. Seu avô, José Bernardo de Medeiros, foi constituinte em 1891 e ocupou uma cadeira no Senado Federal de 1890 a 1907. Com vinte e um anos de idade, Dinarte Mariz contrai matrimônio com Diva Wanderley, filha de Virgolino Pereira Monteiro, comerciante no setor pecuário e político de Campina Grande-PB.

    Ainda em se tratando sobre a trajetória de vida de Dinarte Mariz, é necessário discorrer brevemente sobre sua escolaridade, uma vez que, o mesmo não chegou a cursar o ensino superior, o que levou a afirmar várias vezes que ele era formado na escola da vida. Assim, Mariz teve:

    \begin{quotation}
        O seu primeiro professor foi Arthéfio Bezerra, no Grupo Escolar Coronel Mariz, em Serra Negra. Estudou aritmética, leitura e análise sintática, que na época se dizia análise lógica. Foi para Caicó e, no Grupo Escolar Senador Guerra, concluiu o primário com o professor Pedro Gurgel de Amaral. Foi sempre o primeiro aluno da classe. Lá, aprendeu cantar o hino de Sant'Ana, a história da cidade, viu a beleza plástica no desfile da irmandade do Rosário, viveu o encanto místico da região. \cite[p.~41]{Lima2013Sertao}.
    \end{quotation}

    Ao concebermos sua descendência política, compreendemos que a mesma integra parte de uma parentela que conseguiu se manter no poder. Segundo Linda Lewin:

    \begin{quotation}
        A parentela está associada a uma organização social e estava subjacente à base da rede de parentes e amigos de um político. O núcleo dos seguidores políticos que a ele se vinculam de maneira personalística, constituindo os membros de sua parentela. Os membros deste grupo de base familiar organizavam localmente o eleitorado para fornecer-lhe os votos, defendiam seus interesses partidários em seu município natal e os serviam lealmente em que ingressavam por nomeação. \cite[p.~113]{Lewin1993Politica}.
    \end{quotation}

    Antes de adentrarmos sobre a vida política de Mariz, é de bom grado explanar, por sua vez, a ocupação do mesmo antes de enveredar na carreira política. Então, Dinarte de Medeiros Mariz foi um remanescente da cultura algodoeira e da pecuária, ou seja, um comerciante que por sua vez comandava política e economicamente a região do Seridó. Mediante esse contexto, ficou evidente que sucedeu partir da região habitada pelo referido acima, precisamente, da cidade de Caicó, que fica localizada no Estado do Rio Grande do Norte (RN), na região Seridó\footnote{O Seridó é uma civilização solidária. Desde que consideremos civilização num conceito menos amplo que os que se aplicam à nação. Região desfavorecida pelo clima, nuvens e chão, é beneficiada pelo homem, sua vontade, sua decisão. E pelas bênçãos de Deus. \cite[p.~57]{Lima2003Solidao}.

    A civilização do Seridó é uma herança cultural que se baseia em vontade coletiva, impossível de ser medida, incomensurável. Tem por base suas propriedades rurais que são historicamente unidades autônomas, sustentadas pela produção de gado, pela produção agrícola e pelos peixes que povoam as centenas de pequenos açudes cavados pela mão do homem. \cite[p.~57--58]{Lima2003Solidao}. }. Assim, foi através desse lugar que propiciou a criação de um grupo oligárquico\footnote{A oligarquia se compõe necessariamente daquele grupo minoritário que, por meio da divisão organizacional do poder, logra ocupar posições institucionais que lhe permitem tomar decisões que afetam os interesses coletivos de forma infensa a controle. \cite[p.~48]{Couto2012Oligarquias}.

    Essa concepção da ``classe política'' é importante na construção de um conceito descritivo de oligarquia porque é ela que permite pensar nos ``oligarcas'' como um grupo de poder específico e na ``oligarquia'' como a forma de predomínio desse grupo, que se distingue dos demais não por sua origem de classe, mas pelo papel organizacional específico que desempenha. \cite[p.~48]{Couto2012Oligarquias}.
    
    Aqueles que se profissionalizam como dirigentes partidários, retirando dessa condição seus ganhos e seu status, mas também desfrutando de condições diferenciadas de poder organizacional, rapidamente adquirem as condições para se formarem uma oligarquia. O que permite a sua transformação em oligarcas não é apenas a sua conversão em profissionais da política (embora esta seja uma condição necessária), mas a detenção de um poder na organização não desfrutado pelos demais. Noutros termos, a organização é capturada pelos dirigentes, e isto é o que lhes converte em oligarcas. \cite[p.~48]{Couto2012Oligarquias}.} --- familiar, que apareceu com o desenvolvimento da cotonicultura, representado pelo seu líder maior, o coronel, José Bernardo de Medeiros.\footnote{Ver em: \textcite[p.~187]{Lamartine2003Personagens}}

    Seu primeiro ato político se deu no ano de 1927, com apenas 24 anos de idade, quando solicitou a intendência, ou seja, a prefeitura da cidade de Serra Negra do Norte. Esta reivindicação, por sua vez, não obteve resultados, uma vez que, a família na época pensou que não era a vez dele, fato que o deixou bastante angustiado, caso este que foi confirmado por Olavo de Medeiros Filho:

    \begin{quotation}
        Conversando certa vez no alpendre da fazenda Solidão, perguntei a Dinarte Mariz quais os motivos que o teriam levado a participar da Revolução de 1930, sendo ele parente e conterrâneo do Governador Juvenal Lamartine de Faria, deposto pela referida Revolução, dando uma risada, afirmou-me Dinarte Mariz que tudo teria origem em um pedido que ele fizera a Juvenal Lamartine propondo-se a ser prefeito de sua querida cidade Serra Negra do Norte. O pedido provocou gargalhadas em Juvenal, que descartou a pretensão do parente, pessoa que, segundo ele, não preenchia as condições exigidas para ocupar a chefia da edilidade. \cite[p.~166]{Lima2003Solidao}.
    \end{quotation}

    Então em 1929, durante o governo de Washington Luís (1926--1930), era comerciante de algodão em Caicó (RN), e ingressou na Aliança Liberal\footnote{A Aliança Liberal foi formada em 1929 por setores dissidentes da oligarquia paulista e mineira insatisfeita com o sistema excludente. (SPINELLI, sd., p. 15). } --- agrupamento político oposicionista formado basicamente pelos partidos republicanos mineiro e gaúcho, pelo Partido Democrático (PD) paulista e pelo situacionismo paraibano apoiando a candidatura de Getúlio Vargas e João Pessoa à presidência e vice-presidência da República nas eleições de março de 1930. Contudo, o candidato eleito foi Júlio Prestes, apoiado pelo presidente Washington Luís. A derrota de Vargas, aliada ao assassinato de João Pessoa no mês de julho em Recife, provocou a eclosão do movimento revolucionário de outubro de 1930, ao qual o então Dinarte Mariz sob o comando do capitão do exército Abelardo Torres da Silva Castro participou da revolução no Rio Grande do Norte. 

    Adentrando nos caminhos políticos já introduzidos anteriormente acima, Dinarte de Medeiros Mariz deu continuidade aos seus engajamentos na política, uma vez que, passou a se posicionar favoravelmente a Revolução de 1930, participou ativamente de todos os movimentos armados a posteriori como a Revolução Constitucionalista de 1932 em São Paulo, a Intentona Comunista de 1935\footnote{Em março de 1935 foi criada no Brasil a Aliança Nacional Libertadora (ANL), organização política cujo presidente de honra era o líder comunista Luís Carlos Prestes. Inspirada no modelo das frentes populares que surgiram na Europa para impedir o avanço do nazi-fascismo, a ANL defendia propostas nacionalistas e tinha como uma de suas bandeiras a luta pela reforma agrária. Em agosto, a organização intensificou os preparativos para um movimento armado com o objetivo de derrubar Vargas do poder e instalar um governo popular chefiado por Luís Carlos Prestes. Iniciado com levantes militares em várias regiões, o movimento deveria contar com o apoio do operariado, que desencadearia greves em todo o território nacional. O primeiro levante militar foi deflagrado no dia 23 de novembro de 1935, na cidade de Natal. No dia seguinte, outra sublevação militar ocorreu em Recife. No dia 27, a revolta eclodiu no Rio de Janeiro, então Distrito Federal.}, combatendo os comunistas no Rio Grande do Norte, esteve presente nas conspirações contra a ditadura varguistas e outro momento importante foi a participação da fase preparatória da Revolução de 1964. Dessa forma, segundo Jailma Maria de Lima ``[\dots] A ênfase dada a sua própria trajetória é a de um revolucionário, que articula nos bastidores, mas também que está na linha de frente de alguns episódios [\dots]''. \cite[p.~11]{Lima2013Memoria}.

    \section{Dinarte Mariz: vida, político e seus discursos}

    Dinarte Mariz assim como ficou conhecido, ingressou na vida pública com seu primeiro cargo político como prefeito da cidade de Caicó, aos 27 anos de idade, cargo do qual se afastou após dois anos em face de seu apoio ao Movimento Constitucionalista de 1932, o que lhe valeu três prisões no Rio de Janeiro. Como homem bem articulado e inquieto que era, de volta ao seu estado natal fundou o jornal ``A Razão'' e foi um dos fundadores do Partido Popular ao tempo em que prosperavam seus negócios com o algodão. Mediante ao desenvolvimento do mesmo nos atos políticos, e através de seu forte desempenho como figura política exerceu o mandato de senador por quatro vezes, tendo sido a última por escolha indireta do presidente da República. Nessa linha de influência por mais de uma vez, foi 1º secretário do Senado, um dos cargos mais importantes daquela Casa legislativa. 

    Nessa trajetória e dentro desse contexto, Dinarte Mariz não parou e seguiu em frente com suas articulações e artimanhas para conseguir seus objetivos. Com alianças familiares e proximidades políticas, principalmente, com o seu primo José Augusto Bezerra de Medeiros\footnote{Este por sua vez, ex-governador do Estado do Rio Grande do Norte e com forças políticas na região do Seridó.}, fundou o partido da União Democrática Nacional (UDN), este instituído oficialmente em nível nacional em 7 de abril de 1945, que por sua vez, congregava forças diversas e até antagônicas, em uma extensa frente de oposição ao governo Vargas. Sobre a composição e as alianças do partido:

    \begin{quotation}
        Presidente: José Augusto; vice-presidente: Dinarte Mariz; secretários: Luiz Antonio dos Santos Lima e Djalma Marinho; tesoureiro: Severino Alves Bila. O diretório possuía ainda uma comissão de articulação com o interior e uma comissão de imprensa. Os Diários Associados estavam representados por Edilson Varela e Américo de Oliveira Costa. (O Diário, Natal, 6 de jul. 1945, p. 1).
    \end{quotation}

    Então, nesse período para além da criação e fundação do partido da UDN, Dinarte Mariz lançou-se a candidato ao Senado pela UDN, este por sua vez, não alcançou êxito no pleito eleitoral de 1945. Portanto, Dinarte Mariz foi um homem destemido, com personalidade forte e que não media esforços para alcançar seus objetivos. No ano de 1950 lança sua candidatura ao governo do Estado do Rio Grande do Norte, mas, em acordo com José Augusto Varela não concorreu as eleições, e sim, voltando a concorrer ao Senado Federal na legenda da UDN, que por sua vez, saiu novamente derrotado.  

    No pleito de 1954 Dinarte Mariz, favorecido por um acordo firmado com seu adversário, o pessedista Georgino Avelino, elegeu-se senador pelo Rio Grande do Norte como candidato da coligação UDN-PSP-PSD. Pouco tempo depois de assumir a cadeira como senador, Mariz em fevereiro de 1955 lançou sua candidatura ao governo do Estado do RN com o apoio do Presidente da República João Café Filho, obtendo resultados positivos. Portanto, no pleito de outubro de 1955 foi eleito governador do Rio Grande do Norte.\footnote{Disponível em \url{http://www.fgv.br/cpdoc/acervo/dicionarios/verbete-biografico/dinarte-de-medeiros-mariz}. Acesso em 03 abr. 2021.}

    A partir desse contexto, outro fator importante ao se debruçar e dialogar é acerca de outros rumos que desembocaram na trajetória política de Mariz. É nessa perspectiva que cabe aqui abordar o que existiu de desavenças e disputas no Rio Grande do norte. Nesse contexto, Dinarte Mariz foi o precursor de uma disputa ferrenha que existe no Rio grande do Norte, rivalidade essa conhecida através dos partidos, Vermelho x Verde. A origem desta rixa de cores de partido veio na década de 60, quando o Governo do Estado foi disputado por Aluízio Alves e Dinarte Mariz, período que surgiu a Ditadura Militar e, consequentemente, os partidos MDB e Arena. Portanto, partindo dessa premissa o referido político se constituiu de forma precisa com objetivos formados que permitiu alcançar visibilidade não somente na política como também construindo uma representação de homem do povo. Com base em Chartier:

    \begin{quotation}
        As percepções do social não são de forma alguma discursos neutros: produzem estratégias e práticas (sociais, escolares, políticas) que tendem a impor uma autoridade à custa de outros, por elas menosprezados, a legitimar um projecto reformador ou a justificar para os próprios indivíduos, as suas escolhas e condutas. [\dots] As lutas de representações rem tanta importância como as lutas econômicas para compreender os mecanismos pelos quais um grupo impõe, ou tenta impor, a sua concepção do mundo social, os valores que são os seus, e o seu domínio. Ocupar-se dos conflitos de classificações ou de delimitações não e, portanto, afastar-se do social --- como julgou durante muito tempo uma história de vistas demasiado curtas ---, muito pelo contrário, consiste em localizar os pontos de afrontamento tanto mais decisivos quanto menos imediatamente materiais. \cite[p.~17]{Chartier1990Historia}.
    \end{quotation}

    Ao discorrer e se propor analisar a construção da representação da figura pública do ex-governador do Rio Grande do Norte e do ex-senador da República, que por sua vez ficou viva nos seridoenses e potiguares, faz-se necessário discorrer a respeito do conceito de representação desenvolvido por Roger Chartier. Para isso, Chartier aborda o conceito de representação coletiva, ele destaca que ambos se associam a três aspecto com o mundo social:

    \begin{quotation}
        Primeiro, o trabalho de classificação e de recorte que produz as configurações intelectuais múltiplas pelas quais a realidade é contraditoriamente construída pelos diferentes grupos que compõem uma sociedade; em seguida, as práticas que visam a fazer reconhecer uma identidade  social, a exibir uma maneira própria de estar no mundo, a significar simbolicamente um estatuto e uma posição; enfim, as formas institucionalizadas e objetivadas graças às quais ``representantes'' (instâncias coletivas ou indivíduos singulares) marcam de modo visível e perpetuado a existência do grupo, da comunidade ou da classe. \cite[p.~73]{Chartier1990Historia}.
    \end{quotation}

    Na busca do entendimento das representações, a história cultural afasta-se de uma história social baseada nas lutas econômicas, e busca estudar a sociedade das estratégias baseadas simbólicas manifestadas pelas classes, grupos e meios sociais. Sobre o conceito de representação, vemos que a princípio demostrava o que estava ausente, posteriormente vemos a associação dessa ao que está presente. Sobre a função da representação, o autor aborda que:

    \begin{quotation}
        Todas visam, com efeito, a fazer com que a coisa não tenha existência senão na imagem que a exibe, com que a representação mascare ao invés de designar adequadamente o que é seu referente. A relação de representação é assim turvada pela fragilidade da imaginação, que faz com que se tome o engodo pela verdade, que considera os sinais visíveis como indícios seguros de uma realidade que não existe. Assim desviada, a representação transforma- se em máquina de fabricar respeito e submissão, em um instrumento que produz uma imposição interiorizada, necessária lá onde falta o possível recurso à força bruta. \cite[p.~75]{Chartier1990Historia}.
    \end{quotation}

    Para além da construção que se edificou de Dinarte Mariz e a priori discutida, um outro fator importante a respeito da conjuntura política nos pleitos eleitorais são os discursos e pronunciamentos feitos pelo então estadista Mariz. São diversos momentos que observamos o posicionamento político do qual Dinarte possuiu, uma vez que, o mesmo em uma de suas várias entrevistas que deu, diz a seguinte frase: ``O mundo é grande de se ver e eu já vi de tudo. Mas eu vejo tudo a partir de Caicó''\footnote{Vem em: \textcite[p.~70]{Lima2003Solidao}}. Com essa frase ele quis dizer que mesmo sendo um homem viajado, conhecedor do mundo não esqueceu o lugar que lhe ensinou a voar por outros ares, assim não desmemoriando suas raízes. 

    Para compreendermos o tema acerca da construção e a trajetória político e social de Dinarte Mariz, é por sua vez necessário versar acerca dos seus pronunciamentos e posicionamentos políticos para com o povo potiguar. Para tanto, é imprescindível que se tenha um olhar voltado para os discursos, uma vez que, os mesmos perpassam pela sociedade e são entendidos e assimilados de formas diferentes, de maneira que se tem a necessidade de serem abordados e explorados para que possam, por sua vez, servirem de elementos textuais e embasamento para o desenvolvimento na composição dos caminhos percorridos e alcançados pelo referido político Dinarte Mariz. 

    Nesse contexto, pensar as sociedades e seus diferentes discursos produzidos é segundo Foucault ``[\dots] ao mesmo tempo controlada, selecionada, organizada e redistribuída por certo número de procedimentos que têm por função conjurar seus poderes e perigos, dominar seu acontecimento aleatório, esquivar sua pesada e temível materialidade''. \cite[p.~8--9]{Foucault1996Ordem}. Isso parte da premissa da qual temos conhecimento da nossa sociedade e de que há posicionamentos de exclusão e de impedimentos quanto aos posicionamentos de um determinado discurso, visto que nem todos tem o direito de dizer tudo, como também falar de tudo em qualquer âmbito.  

    Então diante dessa conjuntura, a sociedade se priva de pôr em prática certos tipos de discursos uma vez que elas se comportam de maneira em que estão pautados os discursos que são favoráveis a determinado tipo de assunto. Então, segue assim um ritual de direito privilegiado ou de exclusão quanto ao sujeito que quer falar, e esta exclusão da qual se encontra dentro dos discursos é compreendida por Foucault como um tipo de interdição. Dentro desse caminho de afastamento e o seu ligamento com o impedimento, é possível observar que tais discursos estão embasados e voltados para a sexualidade, como também sobre política. Segundo Foucault:

    \begin{quotation}
        [\dots] como se o discurso, longe de ser elemento transparente ou neutro no qual a sexualidade se desarma e a política se pacifica, fosse um dos lugares onde elas exercem, de modo privilegiado, alguns de seus mais temíveis poderes. Por mais que o discurso seja aparentemente bem pouca coisa, as interdições que o atingem revelam logo rapidamente, sua ligação com o desejo e com o poder. Nisto não há nada de espantoso, visto que o discurso --- como a psicanálise nos mostrou --- não é simplesmente aquilo que manifesta (ou oculta) o desejo; é, também, aquilo que é o objeto do desejo; e visto que --- isto a história não cessa de nos ensinar --- o discurso não é simplesmente aquilo que traduz as lutas ou os sistemas de dominação, mas aquilo porque, pelo que se luta, o poder do qual nos queremos apoderar. \cite[p.~9--10]{Foucault1996Ordem}.
    \end{quotation}

    Ainda em se tratando da análise do discurso, outro ponto discorrido por Foucault associado ao discurso é por sua vez, o ato de exclusão dos discursos, uma vez que, este está na oposição entre o verdadeiro e o falso. Essa problematização aparece a partir do momento em que surgem questionamentos acerca do qual foi, qual é através dos levantamentos sobre os discursos, e estes no que lhes diz respeito estiveram presentes por vários séculos da nossa história. Então, segundo Foucault ``[\dots] é um tipo de separação que rege nossa vontade de saber, então é talvez algo como um sistema de exclusão (sistema histórico, institucionalmente constrangedor) que vemos desenhar-se''. \cite[p.~14]{Foucault1996Ordem}. Dessa forma, a exclusão e a separação se constituíram com total certeza, isso com base nos discursos dos poetas gregos do século VI que buscavam a veracidade dos discursos.

    \begin{quotation}
        [\dots] o discurso verdadeiro no sentido forte e valorizado do termo, o discurso verdadeiro pelo qual se tinha respeito e terror, aquele ao qual era preciso submeter-se porque ele reinava, era o discurso pronunciado por quem de direito e conforme o ritual requerido; era o discurso que pronunciava a justiça e a atribuía a cada qual sua parte; era o discurso que, profetizando o futuro, não somente anunciava o que ia se passar, mas contribuía para a sua realização, suscitava a adesão dos homens e se tramava assim com destino. [\dots] Hesíodo e Platão uma certa divisão e estabeleceu, separando o discurso verdadeiro e discurso falso; separando nova visto que, doravante, o discurso verdadeiro não é mais o discurso precioso e desejável, visto que não é mais o discurso ligado ao exercício do poder. \cite[p.~14--15]{Foucault1996Ordem}.
    \end{quotation}

    Nessa perspectiva, para além das análises acerca dos discursos com base na exclusão, interdição e separação a partir das problematizações dos mesmos através dos discursos do exterior e do interior, Foucault aponta que tem outra existência de um grupo do qual está presente nos seus procedimentos, e este por sua vez, proporciona o controle dos discursos, não no ponto de controlar o seu poder e nem de conjurar suas aparições, mas ``[\dots] trata-se de determinar as condições de seu funcionamento, de impor aos indivíduos que os pronunciam certo número de regras e assim de não permitir que todo mundo tenha acesso a eles. [\dots] ninguém entrará na ordem do discurso se não satisfazer a certas exigências ou se não for, de início, qualificado para fazê-lo''. \cite[p.~37]{Foucault1996Ordem}.

    Dessa forma, as análises e abordagens pautadas por Michel Foucault discorre sobre os discursos presentes nas diferentes sociedades a partir de suas várias vertentes como a exclusão, a interdição, a separação e os seus procedimentos de como eles devem ser vistos e analisados. Diante disso, Foucault mostra que:

    \begin{quotation}
        O discurso nada mais é do que a reverberação de uma verdade nascendo diante de seus próprios olhos; e, quando tudo pode, enfim, tomar a forma do discurso, quando tudo pode ser dito e o discurso pode ser dito a propósito de tudo, isso se dá porque todas as coisas, tendo manifestado e intercambiado seu sentido, podem voltar à interioridade silenciosa da consciência de si. \cite[p.~49]{Foucault1996Ordem}.
    \end{quotation}

    Portanto, os discursos seguem o caminho da verdade a qual se encontra dentro de qualquer manifestação das sociedades que os utilizam para demonstrar seus posicionamentos e ensejos de uma determinada situação. Assim, pode-se dizer que todo discurso tem seu próprio significado. 

    Atrelado aos discursos e pronunciamentos de Dinarte Mariz, no livro intitulado ``Dinarte Mariz vida e luta de um potiguar'' de Agaciel da Silva Maia, traz em seu contexto os memoráveis discursos feitos no Senado Federal, dos quais muito bem elogiado pelo referido autor, e que tem como conteúdo os mais diversos que são entre eles: reverenciando a memórias de ex-companheiros políticos, questões voltadas para a seca no Nordeste, bem como a problematização da economia nordestina e por fim comemorações aos 80 anos do mesmo e da criação da Universidade Federal do Rio grande do Norte.\footnote{``[\dots] ano de 1958 tornar-se-ia um marco referencial na cultura e na educação norte-rio-grandense, pois foi no dia 5 de junho desse ano que Dinarte Mariz criou a Universidade Do Rio Grande do Norte, depois federalizada. E não hesitou em construir modernos centros educacionais em Mossoró e em Caicó, dotando-os dos mais avançados recursos pedagógicos da época.'' \cite[p.~38]{Maia2005Dinarte}.} Assim, diz Dinarte:

    \begin{quotation}
        Devo dizer a todos que esta casa foi para mim mais do que uma universidade, porque talvez se tivesse passado por uma universidade, não teria conseguido aprender tanto, receber tantos ensinamentos capazes de me tornar um servidor, um cativo da coisa pública, em defesa do povo brasileiro e, sobretudo, da democracia, sempre cambaleante, que nos oferece momentos, às vezes, de euforia, mas que foge quando pensamos em construir um patrimônio para as gerações que vêm. \cite[p.~173--174]{Maia2005Dinarte}.
    \end{quotation}

    Dinarte Mariz, em seus pronunciamentos trazia também um discurso que pairava críticas e questionamentos para os seus colegas políticos, dos quais apresentava diretamente ao Senado Federal, quando assim apontava que todos aqueles que faziam parte daquela referida casa deveriam lutar e buscar o melhor para o povo e para o Brasil, e não serem apenas aqueles políticos individualistas. Assim em um de seus discursos no Senado Federal diz:

    \begin{quotation}
        Congresso amortecido não é congresso; Congresso que briga por coisas pequeninas, sem pensar no futuro do país, não é congresso; Congresso só se afirma quando defende idéias, princípios e as grandes causas quando a nação está em risco. Este é o Congresso que eu gostaria de ver. Este é o Congresso que nós precisamos, nesta hora, convocar. Os partidos políticos estão aí, as brigas são internas, mas há uma coisa maior do que as brigas dentro dos partidos: é o interesse maior, é o interesse da Nação. Porque se não nos capacitarmos disso, pior do que tem acontecido acontecerá.  E então nós cairemos diante do povo, sem poder dar uma explicação e muito menos encontrar caminhos para que, amanhã o povo possa crer e voltar as vistas para nos apoiar, prejudicando as gerações que hão de chegar para a grande caminhada do futuro. \cite[p.~185]{Maia2005Dinarte}.
    \end{quotation}

    Diante disso, dialogando com Foucault a partir do conceito de discurso por ele desenvolvido, ao se deparar com a apropriação do discurso, vemos que Foucault cita a educação como sendo um dos veículos dessa manifestação, tendo vista que é um modo ``democrático'' de ventilar os pensamentos e embates políticos de determinadas épocas e espaços. Assim, o autor completa o raciocínio com um pensamento que ``todo o sistema de educação é uma maneira política de manter ou de modificar a apropriação dos discursos, com os saberes e os poderes que eles trazem consigo'' \cite[p.~44]{Foucault1996Ordem}.

    Portanto, nesse mesmo contexto, Foucault analisa o ritual de ventilação das palavras, que por sua vez, o autor aponta que as sociedades do discurso e os grupos doutrinários andam imbricados, servindo assim de apoio um para o outro, para poderem alcançar o objetivo maior que é as apropriações sociais desses pensamentos considerados verdadeiros. Buscando a definição do discurso, o autor mostra que esta parte da leitura do mundo, ou seja, é mesmo a releitura dos pensamentos que circulam sobre a sociedade. Desse modo, vemos que:

    \begin{quotation}
        O discurso nada mais é do que a reverberação de uma verdade nascendo diante de seis próprios olhos; e, quando tudo pode, enfim, tomar a forma de discurso, quando tudo pode ser dito e o discurso pode ser dito a propósito de tudo, isso se dá porque todas as coisas, tendo manifestado e intercambiado seu sentido, podem voltar à interioridade silenciosa da consciência de si. \cite[p.~49]{Foucault1996Ordem}.
    \end{quotation}

    Como anteriormente discutido, Foucault aborda a educação como sendo o meio mais sensato para se pôr em prática os discursos na sociedade. Sabemos que se tem uma extensa significação da palavra educação, uma vez que, a mesma traz um significado amplo o que aqui permite explanar o posicionamento acerca de educação a partir dos discursos de Dinarte Mariz, tendo em vista que, o mesmo era um admirador ferrenho da educação. No entanto, mesmo não tendo concluído o ensino superior, concluiu penas o curso primário, ele tinha convicção de que sem a educação não há povo desenvolvido. Segundo Maia, ``[...] Reconhecia na educação a chave para o progresso social. São suas estas palavras: ``Só na educação uma nação encontrará caminhos para a solução dos seus problemas e felicidade de seu povo.'' \cite[p.~39]{Maia2005Dinarte}.

    Dando continuidade aos discursos a respeito da educação, um outro discurso feito por Dinarte Mariz no Senado Federal em 18 de outubro de 1983, sobre os 25 anos da Universidade Federal do Rio Grande do Norte. Para o mesmo, a criação da Universidade foi de grande relevância para o Estado do Rio Grande do Norte, visto que teria formação superior e uma notável instituição pública, porquanto, ``Não sei se devemos considerar mais importante o projeto ou o processo, se a Universidade é, primacialmente, uma fonte produtora de conhecimento de uma força geradora de inquietação intelectual, condição primeira para a renovação do saber e a tecnologia.'' \cite[p.~200]{Maia2005Dinarte}. Assim, fica perceptível a relevância da mesma e Dinarte Mariz se posicionou: ``Importa acrescentar que a Universidade não apenas forma pesquisadores, mas apresenta resultados materiais e insofismáveis do esforço de pesquisa atualmente desenvolvido.'' \cite[p.~200]{Maia2005Dinarte}.

    Concomitantemente, Dinarte Mariz discursava a partir das observações e dos anseios que para ele estava inserido dentro das necessidades que o povo carecia, pois via nesse trajeto de comunicação uma ponte positiva para seus objetivos. Assim, através do embasamento na educação, envereda a construção da sua imagem, uma vez que, ao pensar na edificação de uma sociedade diz que ``Entregue-se às universidades regionais a tarefa de pesquisar as nossas riquezas e identificar a vocação do nosso povo, construtor de uma civilização tropical.'' \cite[p.~67]{Lima2003Solidao}. Desse modo, é compreensível que no decurso dos seus posicionamentos a educação é a chave principal para cuidar da saúde, cultura e economia, pois ``Só na educação uma nação encontrará caminhos para a solução dos seus problemas e felicidade do seu povo.'' \cite[p.~66]{Lima2003Solidao}.

    \section{Considerações finais}

    Corroborando com os trabalhos já desenvolvidos sobre ``Dinarte Mariz'', o referido escrito aqui desenvolvido, busca levar o leitor a um momento de reflexão sobre um homem que tem suas origens na Fazenda Solidão município de Serra Negra do Norte, localizada no interior do Seridó e que por determinação, convicção de seus objetivos e muito esforço, se tornou um protagonista da história política no período das oligarquias, comprovando assim a relevância no cenário potiguar e nacional.  No decorrer da construção dessa narrativa, temos a certeza que muito mais está para ser descrito e pesquisado sobre a figura eminente de Dinarte Mariz, visto que, essa obra é uma escrita inicial que mostra a proposta de como será importante descrever a história desse líder político, e, é sabido relatar que por se tratar de um personagem conhecido no meio político, esse trabalho irá apenas ser mais um contribuinte na formação de ideias que ressaltam a importância que esse homem público possuiu para o Rio Grande do Norte.

    Por fim, notadamente percebe-se que Dinarte Mariz até os tempos atuais é um personagem político que deixou marcas importantes na administração pública, visto que além de ser bem articulado politicamente, foi exemplo de respeito, solidariedade. Dessa forma, fica claro através dos seus registros apresentados em alguns de seus discursos e de seus escritos. Portanto, cabe salientar que é uma figura tão marcante que em cada cidade do interior iremos encontrar uma rua com seu nome, ou uma estátua, ou um busto em praça pública, a história de Dinarte é cultura, memória, identidade.

    \nocite{Azevedo2004Figuras}
    \nocite{Ferreira2011Historia}
    \nocite{Machado2000Perfil}
    \nocite{Mariz1980Vida}
    \nocite{Medeiros1998Dinarte}
    \nocite{Moraes2003Sertao}
    \nocite{Nunes2003Sertao}
    \nocite{Thompson2009Historia}
    \nocite{MarizCPDOC}

    \printbibliography[heading=subbibliography,notcategory=fullcited]

    \hfill Recebido em 30 mar. 2021.

    \hfill Aprovado em 24 abr. 2021.

    \label{chap:caminhosdesdoend}

\end{refsection}
