\begin{refsection}
    \renewcommand{\thefigure}{\arabic{figure}}
    
    \chapterOneLine
    {Uma reflexão diante do pedagogo empresarial como gestor da ouvidoria }
    \label{chap:reflexao-pedagogo}

    \articleAuthor
    {Tâmara Maria Soares de Medeiros de Cavalcanti}
    {Graduada em Pedagogia pelo Instituto de Educação Superior Presidente Kennedy (IFESP). Especialista em Gestão em Processos Educacionais. E-mail: tmsoares@globo.com.}
    
    \articleAuthor
    {Ilsa Fernandes de Queiroz}
    {Graduada em Ciências Sociais (UFRN). Mestra em Ciências Sociais (UFRN). Professora Formadora do IFESP. ID Lattes: 6966.1039.5536.5909. E-mail: ilsafe13@yahoo.com.br.}
    
    \begin{galoResumo}
        \marginpar{
            \begin{flushleft}
            \tiny \sffamily
            Como referenciar?\\\fullcite{SelfCavalcanteAndQueirozAndesde2021Uma}\mybibexclude{SelfCavalcanteAndQueirozAndesde2021Uma}, p. \pageref{chap:reflexao-pedagogo}--\pageref{chap:reflexao-pedagogoend}, \journalPubDate{}
            \end{flushleft}
        }
        O Presente estudo pretende evidenciar a atuação do Pedagogo em um ambiente empresarial na gestão da Ouvidoria de uma Instituição Financeira Pública. Apresenta uma pesquisa bibliográfica além de uma pesquisa de campo por meio de uma rigorosa observação nos canais de comunicação da ouvidoria dentre eles: o Fale Conosco, o telefone 0800 e o e-mail. Nesse sentido, a pesquisa aponta a importância da presença do pedagogo na ouvidoria, contribuindo para um trabalho mais humanizado, com possibilidades de análises mais seguras, buscando colocar à disposição do cidadão um serviço de qualidade. No primeiro item, apresenta-se o que é a empresa ouvidoria, de que forma se apresenta do ponto de vista econômico, tendo o Governo do RN como sócio majoritário. No segundo, detém-se no foco do trabalho do Pedagogo como especialista em aprendizagem e Educação. No terceiro, o pedagogo como líder que recebe, analisa e oferece respostas conclusivas, após eventual encaminhamento para instrução junto aos setores responsáveis. No quarto, compreende-se a presença do pedagogo de grande relevância para fortalecer práticas de acolhimento dos cidadãos fomentando convivências geradoras de exercícios mais democráticos. 
    \end{galoResumo}
    
    \galoPalavrasChave{Pedagogia. Ouvidor. Gestão.}
    
    \begin{otherlanguage}{english}

    \fakeChapterOneLine
    {A reflection before business educators as a manager of the ombudsman}

    \begin{galoResumo}[Abstract]
        This study intends to highlight the role of the Pedagogue in a business environment in the management of the Ombudsman of a Public Financial Institution. It presents bibliographical research in addition to a field research through a rigorous observation of the ombudsman's communication channels, among them: Fale Conosco, the 0800 telephone and the e-mail. In this sense, the research points out the importance of the presence of the pedagogue in the ombudsman, contributing to a more humanized work, with safer analysis possibilities, seeking to make a quality service available to the citizen. The first item presents what the ombudsman company is, how it presents itself from the economic point of view, with the Government of RN as the majority partner. In the second, it focuses on the work of the Pedagogue as a specialist in learning and education. In the third, the pedagogue as a leader who receives, analyzes and offers conclusive answers, after an eventual referral for instruction with the responsible sectors. In the fourth, it is understood the presence of a highly relevant pedagogue to strengthen citizen welcoming practices, fostering interactions that generate more democratic exercises. 
    \end{galoResumo}
    
    \galoPalavrasChave[Keywords]{Pedagogy. Ombudsman. Management.}
    \end{otherlanguage}


    \section{Introdução}

    O estudo descreve sobre a pesquisa do Curso de Especialização em Gestão de Processos Educacionais, do Instituto de Educação Superior Presidente Kennedy. Busca relatar a atuação do Pedagogo em ambiente empresarial na gestão da Ouvidoria de uma Instituição Financeira Pública.  

    A Empresa é a associação de pessoas, explorando uma atividade com objetivo definido, liderada pelo empresário, pessoa empreendedora, que dirige e lidera a atividade com o fim de atingir ideais e objetivos também definidos. Enquanto a Pedagogia é a ciência que estuda e aplica doutrinas e princípios visando a um programa de ação em relação à formação, aperfeiçoamento e estímulo de todas as faculdades da personalidade das pessoas, de acordo com ideais e objetivos definidos.  

    Nesse sentido, a Empresa e a Pedagogia agem em direção à realização de ideais e objetivos definidos no trabalho de provocar mudanças no comportamento das pessoas. A gestão das ouvidorias suscita debates que interessam não apenas aos profissionais que atuam nessa função, mas também aos gestores de empresas e instituições públicas que anseiam pelo aumento da eficiência. Este estudo descreve a atuação do pedagogo como gestor da Ouvidoria de uma Instituição Financeira Pública.  

    Os saberes do Pedagogo e a efetividade profissional vão ao encontro dos objetivos da Empresa, que busca a excelência da qualidade no atendimento ao público. O pedagogo também pode contribuir para harmonizar, humanizar e permitir o espaço empresarial mais motivador, prazeroso e acolhedor. Após a conclusão do trabalho, serão identificados os pontos críticos existentes e serão apresentadas alternativas para solução dos problemas vivenciados, buscando excelência na qualidade dos serviços públicos.  

    A atuação do Pedagogo na Ouvidoria da empresa é comprovada em relatórios semestrais publicados no site da empresa e em outros instrumentos de avaliação realizados pela internet, com o intuito de demonstrar o nível de satisfação do cidadão. A partir das informações trazidas por todos os cidadãos, a ouvidoria identifica os pontos críticos existentes, apresenta alternativas para a solução dos problemas vivenciados, possibilitando a visão holística das demandas apresentadas de forma a obter subsídios e informações importantes para o aprimoramento dos serviços prestados pela empresa.  

    Os fatos apresentados nesta pesquisa apontam o Pedagogo como líder gestor na Instituição. Ele parte da condição de professor, trazendo uma série de capacidades, de virtudes e facilidades para promover o relacionamento interpessoal. Portanto, desempenha com eficiência e eficácia as ações apresentadas na Ouvidoria. A presença do pedagogo tende a fortalecer práticas de acolhimento dos cidadãos fomentando convivências geradoras de exercícios possivelmente mais democráticos. Essa pesquisa permite compreender o papel do pedagogo de forma mais precisa, clara, objetiva e principalmente, com bases teóricas de comprometimento científico. Percebemo-nos outro ouvidor, mais preparado para os desafios postos a nossa frente, com uma perspectiva mais acadêmica e com mais possibilidades de análises diante dos desafios postos. 

    A empresa pesquisada apresenta-se sob a forma de economia mista de capital fechado com participação acionária majoritária do Governo Estadual do Rio Grande do Norte e de sócios minoritários privados, com destaque para as Federações da Indústria, do Comércio e da Agricultura. A Empresa tem como missão fomentar as atividades econômicas localizadas no estado através de programas de financiamento e investimentos, além da gestão de fundos e da prestação de serviços financeiros com esses instrumentos. Busca promover o desenvolvimento e apoiar a geração de emprego e renda no Estado do Rio Grande do Norte.  

    Este estudo apresenta quatro itens em que no primeiro apresenta-se sucintamente a discussão sobre o que é a empresa ouvidoria, de que forma se apresenta do ponto de vista econômico, tendo o Governo do Rio Grande do Norte como sócio majoritário. Destacam-se a missão da ouvidoria, seus instrumentos e seu papel diante do governo do RN no sentido de apoiar o emprego e a renda.     

    Historiamos brevemente sobre o surgimento da Ouvidoria e tratamos do papel do ouvidor, no século XIX na Suécia. No Brasil, no período colonial aplicando a Lei da metrópole, se reportava a colônia. Era uma representação do cidadão. Só na década de 1980, muito recentemente, após a constituição de 1988, surge na ouvidoria instâncias e mecanismos que permitem aos cidadãos participarem direta em diversas etapas das políticas públicas.       

    Sobre o papel que o pedagogo exerce no atendimento ao público, recebendo, registrando, instruindo, analisando e dando tratamento formal e adequado às reclamações dos clientes e usuários de produtos e serviços que não foram solucionados pelo atendimento habitual realizado por suas agências e quaisquer outros canais de atendimento. Portanto, o pedagogo tem uma função preponderante no exercício cotidiano da ouvidoria. No sentido não só de ouvir, mas ouvir e buscar alternativas de atendimento aos cidadãos.  

    No segundo item fazemos uma discussão diante da proximidade entre a Pedagogia e a Empresa, elas possuem o mesmo objetivo em relação às pessoas atualmente. A Empresa é a associação de pessoas, explorando uma atividade com objetivo definido, liderada pelo empresário, pessoa empreendedora, que dirige e lidera a atividade com o fim de atingir ideais e objetivos também definidos. 

    Enquanto a Pedagogia é a ciência que estuda e aplica doutrinas e princípios visando um programa de ação em relação à formação, aperfeiçoamento e estímulo de todas as faculdades da personalidade das pessoas, de acordo com ideais e objetivos definidos.  

    Nesse sentido, a Empresa e a Pedagogia agem em direção a realização de ideais e objetivos definidos no trabalho de provocar mudanças no comportamento das pessoas. Contribuindo para o exercício de um cidadão mais pleno na sua relação com a empresa.   

    No terceiro item fazemos uma discussão envolvendo a liderança do Pedagogo em espaços diversos, nos escolares, nos hospitais. Nesse sentido, o pedagogo na ouvidoria consegue conquistar novos saberes. Através de treinamentos passaram a ocupar diversas áreas, compromissados com a missão da Instituição e os valores éticos e morais.    

    No quarto item apresentam-se as considerações nas quais constata-se a importância da presença do pedagogo na ouvidoria, contribuindo para um trabalho mais humanizado, com possibilidades de análises mais seguras, buscando colocar à disposição do cidadão um serviço mais qualificado e preciso.  


    \section{Breve história da ouvidoria}

    A origem do Ouvidor (Ombudsman) se deu no início do século XIX na Suécia numa demonstração clara de fortalecimento dos direitos do cidadão diante do poder do Estado. \textcite{FERREIRA2004Novo} relata que a palavra deriva do sueco e do inglês, sendo composta por: \textit{ombud} (sueco): representante ou deputado; \textit{man} (inglês): homem. Contudo, a ideia de ombudsman é muito anterior a essa data.

    No Brasil, o cargo de ouvidor surgiu no período colonial e conforme Vismona \cite[2001, p.~11 apud][p.~11]{PEREIRA2013Ouvidoria} tinha a função de “aplicar a lei da metrópole, ou seja, exercia não uma representação do cidadão diante do Órgão público, mas o inverso atendia ao titular do poder, reportando o que ocorria na colônia.”   

    Após a redemocratização no Brasil na década de 1980, notadamente após a promulgação da constituição de 1988, surgiram muitas instâncias e mecanismos que possibilitaram a participação direta do cidadão nas diversas etapas das políticas públicas, desde sua formulação, passando pela implementação, até o seu monitoramento e a avaliação.  

    Em 1986, foi criada a primeira Ouvidoria Pública Brasileira na Prefeitura de Curitiba, no estado do Paraná, sendo crescente o surgimento de novas Ouvidorias públicas a cada ano. Em 1991, o Estado do Paraná criou o primeiro ouvidor-geral estadual \cite{PINTO1998Ombudsman, Ouvidoria2003}. Em 1992, criou-se a primeira ouvidoria pública federal, a Ouvidoria-Geral da República, vinculada ao Ministério da Justiça \cite{Ouvidoria2003}. Conforme decreto n. 8.243/14, Art. 2º, Inciso V:

    \begin{quotation}
        As Ouvidorias públicas são instâncias de controle e participação social, responsáveis pelo tratamento das reclamações, solicitações, denúncias, sugestões e elogios relativos às políticas e aos serviços públicos, prestados sob qualquer forma ou regime, com vistas ao aprimoramento da gestão pública \cite[p.~1]{Decreto8243-2014}.  
    \end{quotation}

    A Ouvidoria da empresa estudada estabelece um canal de relacionamento com os usuários dos serviços oferecidos pela mesma, a fim de atender e superar as suas expectativas e necessidades, buscando a excelência da qualidade nos serviços públicos. Acreditamos que, ao ampliar a oportunidade de participação da sociedade, as informações decorrentes propiciam a melhoria dos serviços prestados. Assim sendo, a Ouvidoria busca exercer suas funções com agilidade a fim de responder às manifestações e, através de princípios legais e éticos, fortalecendo a cidadania. 

    Das características inerentes à pedagogia, a condução do comportamento das pessoas em direção a um objetivo determinado está diretamente vinculada ao perfil para o exercício da Ouvidoria. Considerando que o pedagogo atua no atendimento ao público, recebendo, registrando, instruindo, analisando e dando tratamento formal e adequado às reclamações dos clientes e usuários de produtos e serviços, que não foram solucionados pelo atendimento habitual realizado por suas agências e quaisquer outros canais de atendimento. Percebem-se as interações do trabalho do ouvidor com a atuação de um educador. Segundo Carlos \textcite[p.~1]{BRANDÃO1981que}:

    \begin{quotation}
        Ninguém escapa da educação. Em casa, na rua, na igreja, ou na escola, de um modo ou de muitos, todos nós envolvemos pedaços da vida com ela: para aprender, para ensinar, para aprender-e-ensinar. Para saber, para fazer, para ser ou para conviver, todos os dias misturamos a vida com a educação. Com uma ou com várias: educação! Educações. [\dots] Não há uma forma única nem um único lugar em que ela acontece e talvez nem seja o melhor; o ensino escolar não é a unia prática, e o professor profissional não é seu único praticante.
    \end{quotation}

 

    A prática pedagógica acontece a partir do momento em que o ouvidor mantém contato com o cliente. De acordo com \textcite[p.~27]{LIBÂNEO2005campo}. 

    \begin{quotation}
        De fato, vem se acentuando o poder pedagógico de vários agentes educativos formais e não-formais. Ocorrem ações pedagógicas não apenas na família, na escola, mas também nos meios de comunicação, nos movimentos sociais e outros grupos humanos organizados, em instituições não escolares\dots{}~nas empresas, há atividades de supervisão do trabalho, orientação de estagiários, formação profissional em serviço. As empresas reconhecem a necessidade de formação geral como requisito para enfrentamento da intelectualização do processo produtivo. 
    \end{quotation}


 

    Verifica-se, pois, uma ação pedagógica múltipla na sociedade, extrapolando o âmbito escolar formal, abrangendo esferas mais amplas da educação informal e não-formal. \cite{LIBÂNEO2005campo}. 

    Dessa forma, os princípios que orientam a ação do Pedagogo precisam traduzir uma política de ação dessa área que permeia todos os atos e momentos da práxis educativa. Só através destes olhares, é que percebemos que há um imenso mundo para além ou aquém do mundo que espreitamos fora de nós.  

    \textcite[p.~124]{FRANCO2008Pedagogia}, ao refletir que  

    \begin{quotation}
        [\dots] no processo de formação do pedagogo deve construir profunda intimidade com as questões da docência, do ensino, mas será inconcebível subsumir a formação de pedagogos, ao exclusivo exercício docente”, pois o contexto atual exige uma formação ampla, além do espaço escolar.   
    \end{quotation}

    Assim, os saberes do pedagogo se expandem a todo e qualquer contexto que envolva trocas de experiências e produção de saber. Pois, a pedagogia sempre estará relacionada à condução das práticas humanas, promovendo as condições necessárias para o desenvolvimento integral das pessoas em um ambiente salutar e favorável ao crescimento. 

    \section{Pedagogia empresarial}

    Pedagogia e Empresa possuem o mesmo objetivo em relação às pessoas atualmente. A Empresa é a associação de pessoas, explorando uma atividade com objetivo definido, liderada pelo empresário, pessoa empreendedora, que dirige e lidera a atividade com o fim de atingir ideais e objetivos também definidos. 

    Enquanto, a Pedagogia é a ciência que estuda e aplica doutrinas e princípios visando um programa de ação em relação à formação, aperfeiçoamento e estímulo de todas as faculdades da personalidade das pessoas, de acordo com ideais e objetivos definidos. Portanto a Empresa e a Pedagogia agem em direção a realização de ideais e objetivos definidos no trabalho de provocar mudanças no comportamento das pessoas. Esse processo de mudança provocada no comportamento das pessoas em direção a um objetivo chama-se aprendizagem. E aprendizagem é a especialidade da Pedagogia e do Pedagogo. 

    Para Holtz \cite[1999 apud][p.~7]{OLIVEIRA2012Pedagogia} diz que, a “Pedagogia empresarial designa as atividades de estímulo ao desenvolvimento profissional e pessoal realizadas dentro das empresas”. A nossa sociedade está inserida em meio aos acontecimentos pedagógicos, ocorrendo de forma sistematizada ou natural, além dos espaços escolares.

    O Foco do trabalho do Pedagogo como especialista em Educação é atuar na Empresa através de ações educativas que garantam a manutenção do ambiente positivo e agradável, estimulador de produtividade. Suas aspirações e objetivos devem corresponder a uma questão social e ética.  

    Para \textcite[p.~20]{CADINHA2007Conceituando}, “o Pedagogo é um estudioso das ações educativas que ocorrem em todas as vidas sociais, culturais e intelectuais do sujeito inserido em uma sociedade na qual contribui para o seu desenvolvimento”. Dessa maneira, a Pedagogia Empresarial se configura como mais uma possibilidade de atuação do pedagogo como profissional responsável em sistematizar a Pedagogia dentro da empresa. 

    A década de 1980 constitui um período de mudança nos conceitos de liderança.  

    \begin{quotation}
        O líder deixa de ser encarado como aquele que conduz, de forma mecânica, hierárquica e prescritiva, o processo de influenciar os outros a atingir objetivos pré-definidos, para começar a ser visto como um gestor de sentido, ou seja, alguém que define a realidade organizacional através da articulação entre uma visão (que é reflexo da maneira como ele define a missão da organização) e os valores que lhe servem de suporte \cite[BRYMAN, 1996, p.~280 apud][p.~21]{COSTAAndCASTANHEIRA2015liderança}.  
    \end{quotation}

    O Pedagogo produz uma forma de ação no meio de ações ainda não definidas, e a Instituição necessita deste aporte pedagógico que se constrói a partir de um olhar diferencial em relação a outros.  


    \section{O pedagogo como líder gestor na ouvidoria}

    As funções dos pedagogos anteriormente se concentravam em espaços escolares e hospitalares. Com a reestruturação produtiva no espaço empresarial percebeu-se a necessidade de promover uma relação mais humanizada entre empresa e empregados. Para isso, o pedagogo como profissional da educação correspondia a esse perfil o que possibilitou a ampliação a ampliação de suas ações no campo das ciências humanas. Assim o pedagogo pode atuar nas Empresas, públicas ou privadas, na Gestão de Pessoas, Gestão Pública, Educação Social, Ouvidoria entre outras áreas afins.  

    Com a participação dos Pedagogos nas empresas, novos saberes foram conquistados. Através de treinamentos, passaram a ocupar diversas áreas, compromissados com a missão da Instituição e os valores éticos e morais. Sobre isso, \textcite[p.~18]{BASTOS2008política} diz que: 

    \begin{quotation}
        Mudar a cultura organizacional constitui um processo complexo que os dirigentes subestimam. Quanto mais consistente for a cultura, mais difícil será a sua mudança em relação oposta a seus valores, uma vez que ela funciona como anteparo que afasta a organização de tais inovações. 
    \end{quotation}

    No contexto de busca do aprimoramento de procedimentos e rotinas da Instituição, a Ouvidoria surge como ferramenta estratégica para a gestão organizacional de excelência, uma vez que estabelece canal permanente de comunicação entre a empresa e os cidadãos, possibilitando avaliações gerenciais e ajustes, a partir da demanda da sociedade e da satisfação do usuário. 

    Ao ampliar a oportunidade de participação da sociedade, as informações propiciam a melhoria dos serviços prestados. A Ouvidoria pesquisada, gerenciada por pedagogo busca exercer suas funções com agilidade a fim de responder as manifestações, através de princípios legais e éticos fortalecendo a cidadania. 

    A Ouvidoria é o canal disponível para solicitar informações, dar sugestões, registrar reclamações ou elogios sobre os serviços prestados pela Instituição, ou ainda encaminhar denúncias. 

    A referida Ouvidoria foi instituída por determinação do Conselho Monetário Nacional (CMN) e do Banco Central do Brasil (BCB) em cumprimento ao Art. 2º da Resolução nº 4.433 de 23 de julho de 2015. Devendo receber e tratar as reclamações dos respectivos clientes e usuários que não forem solucionados pelo atendimento habitual, realizado pela Instituição ou por quaisquer outros pontos ou canais de atendimento, entre outras atribuições.  

    São atribuições da Ouvidoria, segundo o art. 6º da Resolução Conselho Monetário Nacional 4.433/2015 do Banco Central do \textcite[p.~3]{Resolucao4433-2015}, 

    \begin{quotation}
        \noindent\begin{enumerate}[series=lei,label=\Roman*~---]%
            \item atender, registrar, instruir, analisar e dar tratamento formal e adequado às demandas dos clientes e usuários de produtos e serviços;%
            \item prestar esclarecimentos aos demandantes acerca do andamento das demandas, informando o prazo previsto para resposta;%
            \item encaminhar resposta conclusiva para a demanda no prazo previsto;%
        \end{enumerate}
    \end{quotation}

    Dentre os princípios e regras de comportamento da gestão pública, a Ouvidoria observada é pautada pela legalidade, moralidade, transparência, legitimidade, independência, imparcialidade, probidade, isenção e publicidade. 

    O prazo de respostas para as demandas recebidas não pode ultrapassar dez dias úteis, podendo ser prorrogado, excepcionalmente e de forma justificada, uma única vez, por igual período, limitado o número de prorrogações a 10% (dez por cento) do total de demandas no mês, devendo o demandante ser informado sobre os motivos da prorrogação.  

    A referida Instituição no seu Art. 8º Resolução 4.433/2015 dispõe sobre a constituição e o funcionamento da Ouvidoria, que garante o acesso gratuito dos clientes e dos usuários, por meio de canais ágeis e eficazes como o telefone 0800, amplamente divulgado nas dependências da empresa e sítios eletrônicos na Internet acessível pela sua página inicial. Por meio do formulário on-line, disponível 24 horas; e-mail da ouvidoria; carta protocolada, direcionada a Ouvidoria; e o atendimento presencial realizado na empresa durante o expediente. 

    O Pedagogo gestor participa de exame de certificação organizado por instituição de reconhecida capacidade técnica abrangendo temas relacionados à ética, aos direitos e defesa do consumidor e à mediação de conflitos. Dentre as atividades desenvolvidas pelo Pedagogo gestor na referida Ouvidoria podem-se destacar: 

    \begin{itemize}
        \item Receber e analisar manifestações dos usuários relacionadas aos serviços prestados pela empresa, oferecendo respostas conclusivas, após eventual encaminhamento para instrução junto aos setores responsáveis; 

        \item Promover, quando possível, a mediação e a conciliação, ou outras ações para a solução pacífica de conflitos entre o usuário e a empresa; 
    
        \item Subsidiar a avaliação das políticas e dos serviços públicos a partir do processamento das informações obtidas com a análise das manifestações recebidas pelos diversos canais da Ouvidoria; 
    
        \item Formular e manter atualizada a Carta de Serviços ao Usuário da empresa, atendendo a Lei nº 13.460/2017; 
    
        \item Produzir, disponibilizar, analisar dados e informações através de relatórios visando avaliar a eficiência, a eficácia e a efetividade da atuação da empresa por semestre;  
    
        \item Participar de cursos para atualização periódica dos conhecimentos da Ouvidoria e temas relacionados à transparência publica;  
    
        \item Disseminar na empresa os conhecimentos adquiridos realizando treinamentos.  
    \end{itemize}

    A Ouvidoria da empresa observada, está interligada a Rede E-SIC RN Sistema de Informação ao Cidadão, instituída pela Lei Federal nº 12.527, de 18 de novembro de 2011 (Lei de Acesso à informação), Lei Estadual nº 9.963, de 27 de julho de 2015 e Decreto Estadual nº 25.399, de 31 de julho de 2015.

    \nocite{Lei12527-2011}

    A Lei nº 13.460, de 26 de junho de 2017 dispõe sobre participação, proteção e defesa dos direitos do usuário dos serviços públicos da administração pública. No artigo 13 inclui como atribuição da ouvidoria, a promoção da participação do usuário na administração pública, em cooperação com outras entidades de defesa. Assim, compreende-se a ouvidoria não apenas como instância de participação social, mas promotora desse direito por meio da realização de ações pedagógicas, campanhas e eventos de ouvidoria ativa. Portanto, a ouvidoria pública através das ações pedagógicas vai ao encontro do usuário.  

    A atuação do pedagogo na ouvidoria apresenta caráter pedagógico, propositivo e resolutivo: É pedagógico porque desempenha importante processo educativo ao esclarecer aos usuários sobre seus direitos e responsabilidades. Expressar desejos e necessidades, expor conflitos, construir argumentos, formular propostas, ouvir outros pontos de vista, reagir, debater e chegar ao consenso são atitudes que transformam aqueles que integram os processos participativos. 

    É propositivo porque as ouvidorias identificam todas as manifestações que recebem como matéria-prima para a elaboração de informações, que são direcionadas às instâncias de gestão dentro das organizações e para os demais órgãos de controle. É resolutivo porque busca a solução de problemas trazidos pelos usuários e identifica falhas que possibilitem ajustar e melhorar o oferecimento de serviços públicos à sociedade. 

    A linguagem usada pelo pedagogo é personalizada dirigindo-se ao usuário pelo nome. Devendo ser cidadã e inclusiva. A linguagem cidadã é clara, acessível, de fácil compreensão, evitando siglas e termos técnicos. A linguagem inclusiva não usa expressões preconceituosas ou ofensivas a indivíduos ou grupos.       

    A Lei nº 13.460/2017 determina em seu art. 23, que os órgãos e entidades públicas deverão avaliar os serviços prestados sob os seguintes aspectos: satisfação do usuário com o serviço prestado; qualidade do atendimento prestado ao usuário; cumprimento dos compromissos e prazos definidos para a prestação dos serviços; quantidade de manifestação de usuários; e medidas adotadas pela administração pública para melhoria e aperfeiçoamento da prestação do serviço. 

    A ouvidoria em foco atende os requisitos da lei 13.460/2017 e Resolução 4.433/2015 CMN, apresentando ao órgão regulador Banco Central do Brasil relatórios semestrais, abordando os aspectos qualitativos e quantitativos acerca das atividades desenvolvidas pela ouvidoria no cumprimento de suas atribuições. \cite{Lei13460-2017}.  

    Os Relatórios são apresentados ao Conselho de Administração, à diretoria executiva, e às auditorias internas e externas. Sendo eles, aplicados no processo de gestão da Instituição, servindo para facilitar o planejamento; medir os resultados da atuação da organização; embasar os processos de tomada de decisão; contribuir para melhoria contínua dos processos dentro da organização e fazer análise comparativa do desempenho da organização em períodos diferentes. 

    A interação entre a Ouvidoria e os demais setores da empresa acontece de maneira sólida e estreita, resultando eficácia pela porcentagem das manifestações recebidas dentro do prazo e com alto nível de resolutividade.  

    O nível de satisfação dos usuários demonstrada pela pesquisa do site mede a efetividade da Ouvidoria, e a redução do número de dias para responder ao usuário confirma eficiência do processo. 

    A Ouvidoria em destaque visa ser referência como canal efetivo de participação do cidadão, propiciando o aperfeiçoamento contínuo dos serviços prestados pela Instituição sendo agente da participação do cidadão no aprimoramento dos serviços prestados, considerando Ética, Respeito ao Cidadão, Transparência, Imparcialidade e Foco em Resultados. 

    A atuação da Ouvidoria destaca-se como pilar no projeto de Governança Corporativa em curso e Política de Transparência da Empresa. Nesse sentido, as demandas registradas pelos cidadãos contribuem para que as áreas técnicas identifiquem oportunidades de aprimoramento dos processos e consequentemente dos serviços e produtos oferecidos à sociedade. A avaliação registrada no site da empresa pelo canal Fale Conosco é consequência de mudança na filosofia de gestão da administração pública para o atendimento das necessidades do usuário-cidadão. 

    O pedagogo interliga a unidade organizacional Ouvidoria às outras unidades da Instituição. Podendo identificar possibilidade de melhoria, propor mudanças dentre as quais a promoção da agilidade no atendimento, contribuindo para o desempenho da Empresa.  

    \textcite[p.~44]{BOLONGNA2011desenvolvimento} diz que “a possibilidade de utilização do tempo, como um fator essencial de planejamento é uma característica do líder contemporâneo. Lidera melhor no mundo contemporâneo quem se relaciona melhor com o tempo”. Portanto, o pedagogo exerce a liderança na Ouvidoria, cumprindo todas as atribuições determinadas pela legislação vigente. 

    \section{Considerações}

    Os saberes do Pedagogo, a efetividade profissional vai ao encontro dos objetivos da Empresa que busca a excelência da qualidade no atendimento ao público. Portanto, o pedagogo contribui para harmonizar, humanizar e permitir o espaço empresarial mais motivador, prazeroso e acolhedor.  

    O canal Fale Conosco é um mecanismo de enorme contribuição no processo de construção da qualidade desse atendimento mais eficaz ao público, dialogando e fortalecendo a excelência e a qualidade desse atendimento.   

    Um aspecto de grande importância na ouvidoria é a linguagem usada pelo pedagogo, que é personalizada dirigindo-se ao usuário pelo nome. Na nossa compreensão, esse tratamento contribui mais ainda para a identidade como proposta cidadã e inclusiva. Novamente reafirmamos que linguagem cidadã é clara, acessível, de fácil compreensão, evitando siglas e termos técnicos. A linguagem inclusiva não usa expressões preconceituosas ou ofensivas a indivíduos ou grupos. Essa comunicação aproxima o cidadão da ouvidoria.  

    A atuação do Pedagogo na Ouvidoria da referida empresa, é comprovada nos relatórios semestrais publicadas no site da empresa, e em outros instrumentos de avaliação realizados pela internet, demonstrando o nível de satisfação do cidadão. 

    A partir das informações trazidas por todos os cidadãos, a ouvidoria identifica os pontos críticos existentes, apresenta alternativas de solução dos problemas vivenciados, possibilitando a visão holística das demandas apresentadas de forma a obter subsídios e informações importantes para o aprimoramento dos serviços prestados pela empresa. O pedagogo é um líder contemporâneo, compreendendo o tempo como fator preponderante para o exercício do seu planejamento.  

    Os fatos apresentados nesta pesquisa apontam o Pedagogo como líder gestor na Instituição. Ele parte da condição de professor, trazendo uma série de capacidades, de virtudes e facilidade de relacionamento interpessoal. Portanto, desempenha com eficiência e eficácia as ações apresentadas na Ouvidoria. A presença do pedagogo tende a fortalecer práticas de acolhimento dos cidadãos fomentando convivências geradoras de exercícios possivelmente mais democráticos. 

    Essa pesquisa permitiu compreender o papel do pedagogo de forma mais precisa, clara, objetiva e principalmente, com bases teóricas de comprometimento científico. Percebe-se outro ouvidor, mais preparado para os desafios postos à frente, com uma perspectiva mais acadêmica e com possibilidades de análises mais seguras. 

    \printbibliography[heading=subbibliography,notcategory=fullcited]

    \label{chap:reflexao-pedagogoend}

\end{refsection}
