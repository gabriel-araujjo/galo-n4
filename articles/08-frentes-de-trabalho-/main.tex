\begin{refsection}
    \renewcommand{\thefigure}{\arabic{figure}}
    
    \chapterTwoLines
    {Frentes de trabalho e ligas camponesas}
    {movimentos populares, conflitos e sobrevivência (1960--1976)}
    \label{chap:frentestrab}
    
    \articleAuthor
    {João Paulo de Lima Silva}
    {Graduado em História (UFRN-CERES, Caicó), Especialista em História dos Sertões (UFRN-CERES, Caicó), mestrando no Programa de Pós-Graduação em História dos Sertões do (UFRN-CERES, Caicó). ID Lattes: 8111.2333.0951.3952. ORCID: 0000-0002-4254-8571. E-mail: joaopaulojp31@hotmail.com. Sob orientação da Prof.ª Drª. Jailma Maria de Lima}

    \begin{galoResumo}
        \marginpar{
            \begin{flushleft}
            \tiny \sffamily
            Como referenciar?\\\fullcite{SelfSilva2021}\mybibexclude{SelfSilva2021}, p. \pageref{chap:frentestrab}--\pageref{chap:frentestrabend}, \journalPubDate{}
            \end{flushleft}
        }
        Este artigo procura abordar as ações estabelecidas nas Frentes de Trabalho no Rio Grande do Norte e as Ligas Camponesas em Pernambuco. Espaços de resistência e conflitos, onde os retirantes da seca a partir do programa norte americano Aliança para o Progresso se estabelecem como agentes modernizadores através de suas funções voltadas à construção de obras emergenciais. A partir da obra de Henrique Alonso, ``Criar Ilhas de Sanidade: os Estados Unidos e a Aliança para o Progresso no Brasil (1961--1966)'', o documentário norte americano ``The Troubled Land'' de 1961 que aborda o tema das Ligas Camponesas no interior pernambucano, e matérias publicadas no Diário de Natal entre 1960 e 1970, foi possibilitada a ampliação do repertório investigativo sobre tal temática. Os resultados mostram que as muitas intervenções realizadas entre as décadas de 1960 e meados de 1970 surgiram como paliativas diante das intempéries climáticas nordestinas, além do reflexo da proposta modernizadora estabelecida pelos Estados Unidos da América aos países que aderissem ao programa. Diante dos fatos, a população necessitada, que buscava aflita por um meio de sobrevivência, encontrou nas ações da elite oligárquica, ainda que não em definitivo, uma solução.
    \end{galoResumo}
    
    \galoPalavrasChave{Aliança para o Progresso. Modernização. Nordeste.}
    
    \begin{otherlanguage}{english}
    
    \fakeChapterTwoLines
    {Work fronts and peasant leagues}
    {popular movements, conflicts and survival (1960--1976)}

    \begin{galoResumo}[Abstract]
        This article seeks to address the actions established in the Work Fronts in Rio Grande do Norte and the Peasant Leagues in Pernambuco. Resistance and conflicts spaces where migrants fleeing the drought, due to the U.S. program Aliança para o Progresso, establish themselves as modernizing agents through their functions aimed at the construction of emergency works. Based on the Henrique Alonso's work ``Criar Ilhas de Sanidade: os Estados Unidos e a Aliança para o Progresso no Brasil (1961--1966)'', on the 1961 U.S. documentary film ``The Troubled Land'', that addresses the Peasant Leagues in the interior of Pernambuco, and on articles published in the Diário de Natal between 1960 and 1970, it was possible to expand the investigative repertoire on the topic. The results show the many interventions carried out between the 1960s and the mid-1970s emerged as palliatives in the face of northeastern climatic hazards, in addition to the modernizing proposal established by the United States of America for countries that joined the program. In view of the facts, the needy population, who in distress were looking for means of surviving, found in oligarchic elite actions, even if not definitively, a solution.
    \end{galoResumo}
    
    \galoPalavrasChave[Keywords]{Alliance for Progress. Modernization. North East.}
    \end{otherlanguage}

    \section{Introdução}

    No início da década de 1960, quando a América Latina tornou-se a primeira prioridade da agenda externa dos Estados Unidos haja vista tenha sido considerada como a ``região mais perigosa do mundo'', a administração do então presidente John Fitzgerald Kennedy utilizou-se fartamente daquela construção discursiva para criar a Aliança para o Progresso \cite[p.~25--26]{Pereira2005Criar}.

    A Aliança para o Progresso surgiu propagandeada como um programa de ajuda humanitária, onde regiões mais empobrecidas receberam ajuda alimentícia e financeira. Esses países assumiram como compromisso quitar parte dos empréstimos realizados a médio ou longo prazo, como também, cumprir metas nas áreas da educação e construção. Ampliar o número de salas de aulas e construir açudes e estradas apareceu no âmago dessas negociações. 

    Se a América Latina era vista como a região mais perigosa do mundo, devido sua importância geopolítica, o Nordeste brasileiro ficou conhecido como uma região explosiva, não apenas por ser a região mais empobrecida do país, como também o lugar onde a ameaça comunista era mais fortemente estabelecida. Perigoessemais evidente em Pernambuco, onde as Ligas Camponesas e o governador Miguel Arraes não escondiam uma forte postura antiamericana. 

    A partir desse enlace, objetivamos uma abordagem que envolva e torne compreensíveis as ações estabelecidas nas Frentes de Trabalho no Rio Grande do Norte e as Ligas Camponesas em Pernambuco. Uma vez que esses espaços de resistência e conflitos, ao tornarem os retirantes da seca os principais agentes mobilizadores dos conflitos, também, a partir do programa norte americano Aliança para o Progresso e as ações ditas modernizadoras, os tornam vítimas e, ao mesmo tempo um entrave para a sociedade. Muitas lacunas precisam ser preenchidas. 

    Eric Hobsbawm descreve o sentimento antiamericano como uma forma de preocupação da identidade nacional. A chegada de muitos imigrantes aos Estados Unidos na metade do século XIX estimulou a criação de uma imagem do cidadão norte-americano. O ``bom americano'' deveria demonstrar seu patriotismo através de rituais formais e informais afirmando todo tipo de ideal convencional e institucional estabelecido como característica que reafirmasse sua condição \cite[p.~288]{Hobsbawm1984Producao}.

    Ao observarmos a fala do autor, percebemos o surgimento desse sentimento antiamericano tendo início no interior do próprio país norte americano, onde para ser visto como tal bastava que houvesse um pensamento de oposição a qualquer que fosse o termo da política estadunidense. Atitude essa que atravessou suas fronteiras e foi demonstrada também de forma muito hostil por vários países por onde passaram suas comitivas. 

    A produção historiográfica que informa sobre o tema\footnote{Sobre os impactos da Aliança para o Progresso no Nordeste do Brasil, ver informações extraídas de \fullcite{Pereira2005Criar}. Neste trabalho, o autor examina a política externa norte-americana para a América Latina em geral, em particular para o Brasil durante a década de 1960. O foco do trabalho foi a Aliança para o Progresso no Brasil, com destaque para a atuação do programa na região Nordeste e no Rio Grande do Norte. O capítulo cinco realiza o estudo dos conflitos e aproximações entre a Aliança para o Progresso e a Política e o governo Aluísio Alves.}, ao ser confrontada com matérias do jornal Diário de Natal das décadas entre 1960 e 1970 se tornou insuficiente, uma vez que ambos os recursos apontam informações distintas, de certo modo tal fato se torna relevante, uma vez que nos é possibilitada uma ampliação de fatos e contestações sobre o tema discorrido. 

    A partir da pesquisa que fizemos podemos verificar que, as muitas intervenções realizadas entre as décadas de 1960 e meados de 1970 surgem como um paliativo diante das intempéries climáticas nordestinas, além do reflexo da proposta de modernização estabelecida pelos Estados Unidos da América aos países que aderissem ao programa. 

    O Brasil foi o país latino-americano que mais recebeu investimentos do então novo programa de política externa e o Nordeste foi o alvo principal da Aliança no Brasil \cite[p.~6]{Pereira2005Criar}.

    Certamente os Estados Unidos observavam a situação de miséria formulada pela seca um campo fértil para a proliferação de seus ideais. À medida que o convênio beneficiava a população necessitada e, de certo modo a classe política, uma ampla ocupação norte americana ia acontecendo gradualmente nas áreas atendidas por programas.


    \section{O antiamericanismo nas Ligas Camponesas}

    Recife, a capital de Pernambuco está localizada na Zona Metropolitana e no início dos anos de 1960 a cidade foi considerada como ``o centro dos grandes problemas relacionados à pobreza encontrados no Nordeste'' \cite[p.~30]{Page1972Revolucao}. 

    Entre as várias dificuldades que assolavam Recife, talvez o mais grave fosse o habitacional. Boa parte da população mais carente era composta de retirantes provenientes da Zona da Mata e Sertão que perambulavam na esperança de melhores condições de vida.  

    Ao chegarem à cidade essas pessoas eram apresentadas a um cenário sem grandes expectativas. Fatores como a desigualdade social, e as condições precárias de trabalho acentuavam ainda mais a péssima condição dos menos favorecidos, que viam como último refúgio buscar emprego na região do açúcar, localizada na Zona da Mata, e onde se encontrava grande parte do latifúndio responsável pela exploração do trabalhador rural. 

    O problema foi agravado com a modernização do campo ou quando os poderes dos senhores de engenho começaram a ser dividido com os usineiros. A instalação das indústrias de açúcar na região transformou os engenhos em fornecedores de matéria-prima para as usinas. O refino industrializado provocou a venda de muitos engenhos. Desta forma, os industriais do açúcar passaram a acumular poderes econômicos e políticos em Pernambuco \cite[p.~37]{Page1972Revolucao}. 

    A estrutura fundiária provocou diretamente o problema da fome. Josué de Castro descreve que isto foi resultado da organização socioeconômica instalada não só em Pernambuco como em todo Nordeste: 

    \begin{quotation}
        O que se verifica no Nordeste açucareiro é que a fome de que sofrem suas populações é produto exclusivo do seu tipo de organização econômica, da exploração econômica de tipo colonial [\dots] em torno da monocultura do açúcar. A fome aparecendo como uma espécie de subproduto da economia da cana e os famintos como uma forma de bagaço de sua estrutura social: o bagaço humano do latifúndio açucareiro \cite[p.~73]{Castro1975Sete}.
    \end{quotation}

    O autor aborda o problema da fome como um resultado da monocultura do açúcar, pois a região oferecia condições climáticas e de solo propícias para o cultivo de gêneros destinados à alimentação da população. Aliado a isso, essa monocultura renunciou a produção de outros alimentos agravando a situação local.  

    A insatisfação por parte dos trabalhadores serviu de estopim para o início dos movimentos de revolta encabeçados por líderes que emergiam contra os grandes latifundiários e as ações do governo americano por observarem nisso um conjunto de ideias que cada vez mais aprisionava e empobrecia os trabalhadores. 

    Nesse contexto destacaram-se os trabalhadores rurais, que liderados por Francisco Julião, um atuante advogado, político, filho de pessoas influentes na agricultura e inspirados por Fidel Castro e Mao-Tsé-Tung, começou a chamar a atenção de governos estrangeiros, e principalmente grandes latifundiários locais que viram aos poucos os seus antigos regimes de poder sendo menosprezados. 

    A repercussão foi tanta que em setembro de 1960 o jornalista do \textit{The New York Times}, Tad Szulc, desembarcou no Recife para coletar informações sobre os desdobramentos das Ligas Camponesas em Pernambuco. Procurou conhecer a atmosfera local através de visitas in loco e coletando dados durante uma semana. Ao retornar aos Estados Unidos publicou as informações coletadas no jornal entre os meses de outubro e novembro sendo a primeira de muitas reportagens reproduzida em capa. O tom sensacionalista reproduzido pelo autor apontou para uma situação caótica em pleno desenvolvimento assinalando para uma possível situação revolucionária cada vez mais latente em toda vastidão pobre do Nordeste \cite[p.~62]{Barros2017Pobreza}.

    O documentário ``The Troubled Land'' retratou os trabalhadores que compunham as Ligas Camponesas como homens ignorantes por natureza, pertencentes a uma estatística onde trabalhar para sobreviver era o único direito que possuíam. Em alguns momentos a fala dos personagens traduz a educação como algo muito distante, uma realidade totalmente estagnada, onde o discurso do senhor do latifúndio era a única lei.

    Michel Foucault aponta a questão educacional como um dos meios pelo qual chegamos à apropriação social dos discursos. Entender a educação como o instrumento articulador para que todo indivíduo, em uma sociedade como a nossa, possa ter acesso a qualquer tipo de discurso, torna a sua utilização indispensável nas mais diversas áreas, e, também, no âmbito das lutas sociais. Nesse sentido, a educação seria uma maneira política de modificar a realidade dos trabalhadores, porém o medo era uma constante entre eles \cite[p.~43--44]{Foucault2012Ordem}.

    Durante determinada cena, os tiros de Constâncio Maranhão diante da câmera mostraram bem o poder secular do latifúndio. É contra isso que Francisco Julião vai se posicionar negativamente, e, buscar a partir de um discurso humanitário e persuasivo, realizar um chamamento revolucionário. É mais um político que, assim como os estadunidenses, descobriu a importância das Ligas para as contendas políticas.  

    As ações do governo norte americano voltadas para impedir qualquer que fosse a iniciativa de cunho comunista, sempre estiveram voltadas mais diretamente para Recife. Como já citado, o estado de Pernambuco possuía certa postura antiamericana, fosse através do comportamento das ligas camponesas e seus líderes ou do próprio governador. Fato esse que pedia uma maior vigilância levando, por exemplo, à instalação do mais importante escritório da Aliança para o Progresso estar situado em Recife, o que culminava com a frequente visita de comitivas norte-americanas. Tal fato não ocorreu no Rio Grande do Norte, uma vez que, o governo potiguar nutria de certa proximidade ideológica com os Estados Unidos da América.

    \section{A seca pede, a força do latifúndio ordena}

    Era contra toda essa situação que Francisco Julião se posicionava, alertando os trabalhadores e despertando a ira dos grandes latifundiários.  

    A historiadora norte-americana Jan Knippers Black afirma que a ajuda norte-americana ao Nordeste do Brasil não foi motivada pela pobreza existente na região, mas, resultado das mobilizações sociais das Ligas Camponesas e do seu líder Francisco Julião \cite[p.~161]{Black2009Penetracao}.

    Muito convincente essa afirmação se realmente analisarmos o fato de que, qualquer projeto estabelecido no Nordeste seria estreitamente relacionado à seca. Uma vez que geralmente eram apresentados como solução para os problemas advindos da estiagem. Existia sempre uma apropriação pelas classes proprietárias de modo a buscar, através de seus discursos ocultos, manter seus privilégios locais e assegurar espaços ameaçados, tendo em vista a ascensão de outros grupos. 

    As ações da Aliança voltadas à agricultura brasileira vivenciaram ao longo da década de 1960 uma série de processos, que vão desde a ascensão dos movimentos sociais rurais cada vez mais organizados, passando pelas ferrenhas discussões em torno da reforma agrária e da modernização, desembocando na formação do Complexo Agroindustrial Brasileiro (CAI). O programa estadunidense lançado por Kennedy integrou de forma mais direta ou indireta, todos esses processos, deixando a marca de sua ingerência na agricultura brasileira \cite[p.~21]{Natividade2018Alianca}.

    A temática envolvendo a reforma agrária e consequentemente seu desenvolvimento, foi algo que gerou muitos conflitos entre os líderes dos movimentos sociais, seus adeptos e os grandes latifundiáros no Nordeste. Tais conflitos ocorreram com tanta frequencia e brutalidade que despertaram a atenção da crítica norteamenricana. Para melhor argumentar tais fatos, temos como fonte audiovisual o já citado documentário ``The Troubled Land'', traduzido para o português como ``A terra problemática'', o documentário produzido no Brasil a mando do governo estadunidense em 1961, mostrou de forma concreta como era o tratamento entre os chamados coronéis da terra e seus empregados.

    Filmado em Pernambuco, o filme mostrou como era a vida dos cortadores de cana-de-açúcar na fazenda do latifundiário Constâncio Maranhão, um homem com arma em punho, que se dizia simples, e ao mesmo tempo dava tiros para o ar enquanto falava que, ``aquilo'', era o que teriam aqueles trabalhadores que o desobedecessem.  

    Em cenas posteriores, Francisco Julião, grande nome das ligas camponesas, sempre é visto discursando em locais de grande impacto popular: feiras livres e canaviais. Sempre utilizando uma fala em tom encorajador, ele se dirige aà classe trabalhadora para que esses se libertassem dos abusos de seus senhores. 

    O documentário foi produzido para a rede de televisão estadunidense ABC, que escolheram o Nordeste como cenário perfeito para documentar o suposto surgimento de uma ``Nova Cuba'', os perigos da atuação de Francisco Julião sobre a massa camponesa. Uma atitude premeditada, uma vez que, posteriormente tais fatos registrados justificarião perante a opinião pública, o apoio dado ao golpe militar no Brasil, fato ocorrido no mesmo ano de exibição do filme. 

    \section{Frentes de Trabalho resistência e sobrevivência}

    O Nordeste desde a implantação da Aliança para o Progresso no Brasil ocupou espaço privilegiado nas agendas dos governos brasileiro e norte-americano. Entretanto, muitos anos antes do início daquele programa de política externa, a região já havia recebido atenção prioritária, tanto nos Estados Unidos como no Brasil \cite[p.~288]{Pereira2005Criar}. 

    Desde o final do século XIX, o Nordeste tornou-se um problema de repercussão nacional, os efeitos da seca eram o que caracterizavam essa região. Visto como um campo fértil para a realização de futuros enlaces políticos, diversos investimentos foram realizados e muitas instituições criadas com o intuito de gerir toda a situação.  

    Entretanto, como afirma Celso Furtado em uma de suas obras sobre o tema, essa ação do Estado não resultou em melhorias para a população que era vitimada pelas secas. Nesse sentido, como observa o autor, os investimentos federais no Nordeste desde a década de 1950 para combater o problema da seca ``foi desviado de seu autêntico objetivo social para transformarem-se em instrumento de consolidação dos latifúndios de pecuária, ameaçados em suas próprias bases pelas grandes calamidades sociais em que se haviam transformado as secas'' \cite[p.~22]{Furtado1985Fantasia}. 

    No Nordeste, o Rio Grande do Norte foi o ponto preferencial de atuação da Aliança para o Progresso. Visto como a principal ``Ilha de sanidade'', expressão criada pelo embaixador norte-americano Lincoln Gordon, para nomear os benefícios que os Estados Unidos através do Programa Aliança para o Progresso poderiam ofertar para o Nordeste, Brasil e América Latina \cite[p.~27]{Pereira2005Criar}.

    O estado possuía uma localização geográfica vista como estratégica no que diz respeito à prevenção e possíveis investiduras comunistas. Esse fato foi decisivo para que a comitiva responsável pela articulação do programa norte-americano visualizasse a referida região como a melhor porta de entrada e possível campo de permanência das tropas e ideais anticomunistas, oportunizando com isso, um elo junto à administração Aluízio Alves (1961--1966) que colheu significativos frutos políticos. 

    No Rio Grande do Norte as manifestações populares, compostas também por trabalhadores insatisfeitos se mostraram mais tímidas, tendo em vista que essas funcionavam como uma forma de ocupação para esses flagelados que viam nas Frentes de Trabalho, talvez de forma ingênua, uma área em crescimento, além de seu único refúgio de sobrevivência, de início tudo funcionou de forma organizada e pacífica. Sobre a função dessas organizações trabalhista e consequentemente emergenciais, Duarte nos explica:

    \begin{quotation}
        As medidas de enfrentamento dos efeitos da escassez de recursos hídricos seguiam três frentes: a intensificação na construção de açudes e outras obras complementares, o aumento da construção de estradas de rodagem e de ferro e o incentivo à emigração para outros estados, principalmente nas áreas onde o desemprego assumiu grandes proporções, garantindo a ocupação e os meios de subsistência da população \cite[p.~33]{Duarte2002Seca}.
    \end{quotation}

    Constam nos acervos da Paróquia de Santana em Caicó, no Rio Grande do Norte, documentos referentes à administração das frentes de trabalho por parte do Departamento Nacional de Obras Contra as Secas (DNOCS) e pelo Departamento de Estradas de Rodagens (DER). Neles, encontramos a composição das Frentes. Isso partindo do montante de operários, as condições de trabalho, desde as atividades desenvolvidas, até como estes eram conceituados a partir do trabalho. 

    Em um relatório produzido no ano de 1976, encontramos dado como, jornada de trabalho de 8 horas diárias registra com assinatura de ponto, salário líquido da época referente a 502 cruzeiros, sendo estes já descontados impostos referentes ao Instituto Nacional de Previdência Social (INPS), pago quinzenalmente em moeda, na própria Frente de Trabalho e não na cidade onde estivesse localizada. Isso certamente se deu pela dificuldade de locomoção desses trabalhadores até a zona urbana.\footnote{O documento manuscrito (relatório) encontra-se no Acervo de documentos da Paróquia da Diocese de Caicó. No primeiro andar do Centro Pastoral Dom Wagner, depositados em pastas plásticas e armários de ferro. Os documentos do acervo não se encontram enumerados por se encontrarem em processo de catalogação. Acesso em 02 maio 2018.}

    Havia uma lista de itens a ser evitado, e entre eles destacava-se a presença de menores/escolaridade, certamente alegando que os menores de acordo com suas idades deveriam estar frequentando as séries escolares e não atuando nas Frentes.

    Podemos observar no documento as prioridades que deveriam ser respeitadas ao se compor os grupos de trabalhadores. Entre elas estão, que os homens deveriam ser casados, solteiros arrimo\footnote{Diz-se da pessoa que dá proteção, auxílio ou amparo. No caso do documento, quando o homem é solteiro e responsável pelo sustento de uma família e, quando se é menor, mas está protegido por seus familiares. Ver: \fullcite{Bueno2016Minidicionario}\mybibexclude{Bueno2016Minidicionario}.} de família, e menores com mais de 14 anos/arrimo.

    Nesses relatórios o homem era tratado como um sujeito que possui valores reconhecidos pelos chefes, sendo estes; respeitador, dóceis, pacíficos, trabalhadores, inteligentes e que, se sentem honrados pelo trabalho, não podendo por isso ficar parados. Essas eram as impressões que deveriam ser enviadas aos escritórios que geriam o programa, quando na verdade a situação nas Frentes de Trabalho começava a se agravar.

    Criadas a partir do Programa Alimentos para a Paz para buscar de soluções para os problemas da pobreza, no entanto, com o passar dos anos e, diante da seca cada vez mais frequente, estas passaram a ser não só o único refúgio dos mais necessitados, bem como um grande comércio para os mais abastados. Assim sendo, qualquer que fosse a ameaça de fim destas, a população de flagelados ficava temerosa e angustiada por notícias, a demora por declarações gerou o clima de tensão que se alastrou pelo Estado.  

    Boatos de que o trabalho estava sendo improdutivo e que suas atividades iriam parar fez com que tivesse início uma onda de invasões e saques por várias cidades do Nordeste. A situação se agravou, pois o salário se tornou insuficiente para o sustento, e até mesmo o alimento recebido como parte do pagamento chegou a ser vendido e os filhos desses trabalhadores passaram a virar pedintes como forma de complementar uma renda para tantas necessidades. 

    A situação se tornou cada vez mais inflamável. A seca deixou de ser um problema natural e começou a ser tratado como um problema econômico. No meio de tudo isso, uma massa desesperada que não media esforços para obter uma solução a seu favor. As manchetes dos jornais eram claras e cada vez mais traziam fatos que representavam o desespero dos flagelados e a falta de interesse dos políticos em resolver a situação. 

    Diante da situação, Edward Thompson colabora com suas ideias ao nos dizer que:

    \begin{quotation}
        Em todas as sociedades, naturalmente, há um duplo componente essencial: o controle político e o protesto, ou mesmo a rebelião. Os donos do poder representam seu teatro de majestade, superstição, poder, riqueza e justiça sublime. Os pobres encenam seu contrateatro, ocupando o cenário das ruas dos mercados e empregando o simbolismo do protesto e do ridículo \cite[p.~239--234]{Thompson2001Historia}.
    \end{quotation}

    Um dos muitos fatos que narram essa história e compuseram as páginas dos jornais locais, foi quando no ano de 1967, não suportando mais a situação de miséria e descaso, 700 homens que há dias aguardavam em Santa Cruz no interior do Rio Grande do Norte ser alistados pelo escritório da SUDENE, não obtendo resposta que deveria vir da cidade de Natal, como informou o prefeito Clodoaldo Medeiros, por volta das 15 horas iniciou os saques à cidade. Um depósito no centro foi o primeiro saqueado, suas portas foram arrombadas, pessoas foram pisoteadas e durante 15 minutos as pessoas subtraíram todos os gêneros alimentícios do local. Dalí a população faminta buscou novos lugares para repetir a cena, e só encerrariam o episódio após intervenção da polícia que por ordem do delegado controlou os saques.\footnote{\fullcite{OperacaoSaque1967}. Acesso em 17 ago. 2018.}

    O cenário composto para o futuro imediato do Nordeste era apocalíptico. As pressões sociais tornaram-se explosivas e evidenciaram ser um risco para a segurança interna do Brasil. Para o governo norte-americano, a conjuntura atual favorecia a possibilidade de uma segunda Revolução Cubana, desta vez em solo brasileiro. 

    As Frentes de Trabalho continuaram sendo mantidas com recursos vindos da Aliança destinados aos flagelados, no entanto, agora as Frentes conviviam, não só com a fome, a má administração e a exploração, também aumentaram os problemas de saúde resultado das péssimas condições às quais estavam expostos os trabalhadores nesses locais. 

    Mesmo diante de tantas irregularidades e fatalidades, o homem sertanejo teve de ser apresentado a aquele que seria o seu maior medo, o fim das Frentes de Trabalho. Em novembro de 1970, foi noticiado que a SUDENE iniciaria as dispensas. O Ministro do Interior José Costa Cavalcante declarou que, agora os flagelados teriam mais um dia de folga remunerado, para que pudessem preparar as suas terras para o inverno que segundo eles se aproximava. Ele afirmou ao Diário de Natal que: 

    \begin{quotation}
        ``a desmobilização das frentes de trabalho em todo o Nordeste será em ordem progressiva, dependendo apenas, das chuvas que forem caindo em toda a região. Inicialmente, vamos dar mais um dia de folga ao flagelado para ele ter melhores condições de preparar sua terra e aguardar o inverno de 1971.''\footnote{\fullcite{SudenoComecou1970}. Acesso em 17 ago. 2018.}
    \end{quotation}

    Não demorou muito, o Diário de Natal noticiou a desarticulação de todas as Frentes de Trabalho existentes no Rio Grande do Norte desde o dia anterior. As notícias foram que tudo aconteceu normalmente, que as últimas 13 frentes existentes no Estado, englobando um total de 10 mil homens foram desmobilizadas na mais completa ordem, e que agora, o Estado seria beneficiado com bens, como por exemplo, viaturas que antes serviam à SUDENE como forma de apoio. Os discursos oficiais surgidos acerca dos flagelados encerraram como que quisessem calar uma longa jornada de fome, dor e morte. As Frentes de Trabalho não deixaram de existir em definitivo, pois com o agravamento de outras secas, essas tornaram a surgir em menor número e proporções, porém ainda abastecidas pelo Programa Alimentos Para a Paz, que mesmo com o período de desarticulação das Frentes, ainda enviavam os mantimentos para que estes fossem utilizados junto às comunidades carentes\footnote{\fullcite{SemAnormalidades1971}. Acesso em 17 ago. 2018.}.

    O governo agora estava sob a tutela de Cortez Pereira. Candidato também eleito com a promessa de modernizar o estado e promover apoio aos movimentos populares nacionalistas. Os mesmos que apoiaram o governo Aluízio Alvese tempos depois foram apontados como subversivos. Ao mesmo tempo em que Aluízio rompeu com esses movimentos populares que o apoiaram em 1960, foi retomando as velhas práticas conservadoras e oligárquicas que tanto tinha condenado durante a campanha eleitoral. O fato é que essa prática ainda perdurou o suficiente, mostrando que no momento da desarticulação das Frentes de Trabalho, nem uma relação em prol dessa massa de desempregados foi tomada.

    \section{Considerações Finais}

    Estudar lugares de resistência e conflitos populares entre as décadas de 1960 e 1970 nos remete à observação de como tantos arranjos e desarranjos que compuseram essas ações, em grande parte do percurso, se atrelaram às imagens de poder e autoridade por agentes que se sobrepuseram aos marginalizados. 

    A literatura memorial e acadêmica encontrada sobre esse período nos mostra personagens oligárquicos e populistas, como é o caso de Aluízio Alves e Constâncio Maranhão, considerados ``tradicionais'' cada um no seu respectivo meio. Além disso, trata da modernização elaborada a partir dos investimentos por parte da Aliança para o Progresso, mas também, nos mostra uma modernização meramente emergencial e paliativa.

    Ainda dispomos de poucas fontes acessíveis onde possamos nos aprofundar nos pormenores desse período. Porém, o que possuímos se faz suficiente para, de início, ampliarmos a pesquisa de novas contribuições para essa literatura, sempre problematizando dúvidas e aprimorando conceitos.

    \nocite{TroubledLand1961}
    \nocite{RelatorioDNOC1976}
    \nocite{Pereira2007Estados}

    \printbibliography[heading=subbibliography,notcategory=fullcited]

    \hfill Recebido em 18 mar. 2021.

    \hfill Aprovado em 13 abr. 2021.

    \label{chap:frentestrabend}

\end{refsection}
