\begin{refsection}
    \renewcommand{\thefigure}{\arabic{figure}}
    
    \chapterOneLine
    {Reflexões sobre o fazer pedagógico dos professores do Centro Estadual de Educação Profissional Doutor Ruy Pereira dos Santos}
    \label{chap:reflexao-fazer-pedagogico}

    \begin{otherlanguage}{spanish}

        \fakeChapterOneLine
        {Reflexiones sobre el quehacer pedagógico de los docentes del Centro Estadual de Educação Profissional Doutor Ruy Pereira dos Santos}
    
    \end{otherlanguage}

    \articleAuthor
    {Roseane Idalino da Silva}
    {Mestra em Educação (IFRN), Especialista em Gestão de Processos Educacionais (IFESP) e em Metodologia do Ensino Religioso (FDHS), Pedagoga (UFRN) e licenciada em Ciências da Religião (UERN). Coordenadora Pedagógica do Centro de Educação Profissional Dr. Ruy Pereira dos Santos (SEEC/RN). ID Lattes: 9983.1767.6259.6239. ORCID: 0000-0002-5353-7309. E-mail: roseaneidalino@gmail.com.}
    
    \articleAuthor
    {Mariza Silva de Araújo}
    {Licenciada e Bacharela em História (UFRN). Especialista em História da Cultura (UFRN). Mestra em Ciências Sociais (UFRN). Dra. em Educação (UFPB). Professora Formadora do Instituto de Educação Superior Presidente Kennedy (IFESP). ID Lattes: 8524.1038.0661.1287. ORCID: 0000-0002-5322-306X. E-mail: mariza@ifesp.edu.br.}
    
    \begin{galoResumo}
        \marginpar{
            \begin{flushleft}
            \tiny \sffamily
            Como referenciar?\\\fullcite{SelfSilvaAndAraújo2021Reflexões}\mybibexclude{SelfSilvaAndAraújo2021Reflexões}, p. \pageref{chap:reflexao-fazer-pedagogico}--\pageref{chap:reflexao-fazer-pedagogicoend}, \journalPubDate{}
            \end{flushleft}
        }
        O presente artigo pondera sobre o fazer pedagógico dos professores do Centro de Educação Profissional Doutor Ruy Pereira dos Santos, localizado em São Gonçalo do Amarante/RN, com vistas a refletir sobre as práticas pedagógicas desenvolvidas nessa instituição nos dois primeiros anos de seu funcionamento. Utilizamos como referenciais teóricos: \textcite{CIAVATTA2005formação}, \textcite{RAMOS2005Possibilidades,RAMOS2008Concepção}, \textcite{DANTE2010AlgumasPossibilidades} e \textcite{FREIRE1996Pedagogia}. A metodologia consistiu na observação participante com base em \textcite{MINAYO2007desafio}, para compreender como se deu o processo inicial de funcionamento dos Centros de Educação Profissional no Rio Grande do Norte e também para observar o trabalho pedagógico dessa instituição de ensino; além disso, recorremos à pesquisa bibliográfica. A pesquisa tem uma abordagem qualitativa com base nas observações feitas in locus. Essa pesquisa é relevante, pois apresenta um panorama do funcionamento e das práticas pedagógicas presentes no Centro bem como uma reflexão sobre o Ensino Médio Integrado à Educação Profissional.  Os resultados apontam que o fazer pedagógico do CEEP Dr. Ruy Pereira dos Santos busca atender a uma dimensão do trabalho, da ciência, da cultura e da tecnologia, visando desenvolver um trabalho na perspectiva da formação humana integral. 
    \end{galoResumo}
    
    \galoPalavrasChave{Ensino Médio Integrado, Educação Profissional, Práticas Pedagógicas.}
    
    \begin{otherlanguage}{spanish}

    \begin{galoResumo}[Resumen]
        El presente artículo pondera sobre el quehacer pedagógico de los docentes del Centro Estadual de Educação Profissional Doutor Ruy Pereira dos Santos, situado en la ciudad de São Gonçalo do Amarante, en el Estado del Rio Grande do Norte, para reflexionar sobre las prácticas pedagógicas desarrollados en esa institución en los primeros años de su funcionamiento. Utilizamos como referenciales teóricos: \textcite{CIAVATTA2005formação}, \textcite{RAMOS2005Possibilidades,RAMOS2008Concepção}, \textcite{DANTE2010AlgumasPossibilidades} y \textcite{FREIRE1996Pedagogia}. La metodología consistió en la observación participante con base en \textcite{MINAYO2007desafio}, para comprender como se dio el proceso inicial de funcionamiento de los Centros de Educación Profesional en el Estado del Rio Grande do Norte; y también para observar el trabajo pedagógico de esa institución de enseñanza; además de esto, recurrimos a la búsqueda bibliográfica. La búsqueda bibliográfica tiene un enfoque cualitativo con base en las observaciones realizadas in locus. Esa búsqueda es relevante, pues presenta un panorama del funcionamiento y de las prácticas pedagógicas presentes en el Centro, así como es una reflexión sobre la Enseñanza Media Integrada a la Educación Profesional. Los resultados apuntan que el quehacer pedagógico CEEP Dr. Ruy Pereira dos Santos busca atender a una dimensión del trabajo, de la ciencia, de la cultura y de la tecnología, visando desarrollar un trabajo con la perspectiva de la formación humana integral.  
    \end{galoResumo}
    
    \galoPalavrasChave[Palabras-clave]{Enseñanza Media Integrada, Educación Profesional, Prácticas Pedagógicas.}
    \end{otherlanguage}

    \section{Introdução}

    O Ensino Médio Integrado à Educação Profissional se tornou um dos maiores desafios na educação atual e se constitui em uma das propostas que visam romper com a dualidade histórica que acompanha essa modalidade de ensino no Brasil. Assim, refletir sobre o fazer pedagógico de instituições que ofertam o Ensino Médio Integrado à Educação Profissional leva-nos a compreender como o processo de ensino-aprendizagem se dá nessa modalidade de ensino. O presente artigo apresenta uma reflexão sobre o fazer pedagógico dos professores do Centro de Educação Profissional Doutor Ruy Pereira dos Santos, localizado em São Gonçalo do Amarante/RN, com o objetivo de refletir sobre as práticas pedagógicas desenvolvidas nessa instituição nos dois primeiros anos de funcionamento. 

    Integrante da equipe pedagógica do Centro de Educação Profissional Doutor Ruy Pereira dos Santos, localizado em São Gonçalo do Amarante/RN, acompanhamos esse processo e percebemos que este não pode ser vivido sem ter um olhar de pesquisa e de intervenção, sabendo que o registro sistemático e o retorno a essa escrita é uma experiência de formação continuada e que deve compor o fazer pedagógico dos professores. 

    O ingresso na equipe pedagógica do referido Centro se deu por meio de processo seletivo interno, em 2017, realizado pela Secretaria de Estado da Educação e da Cultura (SEEC)\footnote{Atualmente denomina-se Secretaria de Estado, da Educação, do Esporte e do Lazer (SEEL).}, para preenchimento das vagas abertas para a composição de sua equipe gestora. Esta, atualmente, tem a seguinte composição: Diretor, Vice-Diretor, Coordenador Pedagógico e Coordenador Financeiro.  

    Desde então, os questionamentos com relação à implementação e ao desenvolvimento dos trabalhos pedagógicos nos Centros têm sido as nossas inquietações, em especial do CEEP Dr. Ruy Pereira dos Santos. A primeira intenção de estudo seria ter contato com a realidade de implementação e do fazer pedagógico de todos eles, ou, ao menos, daqueles situados na região metropolitana de Natal, que hoje dispõe de cinco Centros em funcionamento --- CEEP Prof. João Faustino Ferreira (Natal); CEEP Dr. Ruy Pereira dos Santos (São Gonçalo do Amarante); CEEP Professora Lourdinha Guerra (Parnamirim); CEEP Ruy Antunes Pereira (Ceará Mirim); CEEP Professor Hélio Xavier de Vasconcelos (Extremoz). 

    Para uma melhor compreensão da temática abordada, faz-se necessário, no entanto, delimitar o \textit{locus} da pesquisa e, para tal, definimos que iremos dialogar apenas com o CEEP Dr. Ruy Pereira dos Santos, compreendendo que a sua realidade pode ser um reflexo dos demais Centros Estaduais, em especial daqueles que atenderam até o fim de 2017 com o tempo semi-integral. Trata-se, dessa maneira, de um estudo de caso.  

    Para este artigo, nos deteremos às práticas pedagógicas que compõem o fazer pedagógico dos professores no Centro Estadual de Educação Profissional Doutor Ruy Pereira dos Santos, no decorrer dos seus dois primeiros anos de funcionamento, a partir da observação participante. No âmbito das práticas, optou-se por analisar o currículo integrado. 

    Temos como objetivo principal refletir sobre o fazer pedagógico dos professores no CEEP Dr. Ruy Pereira dos Santos --- São Gonçalo do Amarante, RN, considerando a efetivação do currículo integrado. Nessa perspectiva, alguns questionamentos nortearam a nossa pesquisa, levando em consideração, inicialmente, as práticas pedagógicas de um Centro que se propõem a ofertar o Ensino Médio Integrado à Educação Profissional, tais como: \textit{Quais dificuldades foram enfrentadas pela equipe do Centro Estadual de Educação Profissional (CEEP) Dr. Ruy Pereira dos Santos para sua implementação? Quais foram as maiores dificuldades pedagógicas enfrentadas pela equipe, para garantir um Ensino Médio Integrado à Educação Profissional?}

    A metodologia adotada se constituiu de pesquisas bibliográficas sobre o tema. Trata-se de uma pesquisa qualitativa, com estudo de caso, baseada nas orientações de \textcite{MINAYO2007desafio}; para nossa investigação, usamos, como instrumento, a observação participante, a fim de compreender como se deu o processo inicial de funcionamento do CEEP Dr. Ruy Pereira dos Santos, visando dialogar com o trabalho pedagógico dessa instituição de ensino. As atividades realizadas diariamente como coordenadora pedagógica da instituição contribuíram significativamente para este trabalho, pois possibilitou o pensar sobre a ação, tendo em vista aperfeiçoar o trabalho realizado nas escolas e Centros que ofertam o Ensino Profissional, de forma a contribuir com a produção de conhecimento na área de Educação Profissional.  

    Essa pesquisa é relevante, pois apresenta um panorama do funcionamento e das práticas pedagógicas presentes no CEEP Dr. Ruy Pereira dos Santos, pois compreender como tem sido esse processo contribuirá para elaboração de estratégias de formação continuada da equipe pedagógica e administrativa bem como traçar estratégias de investimento financeiro que visem à melhoria do trabalho pedagógico na rede estadual de ensino.  

    Acreditamos também que essa pesquisa nos dará uma visão geral da realidade dos demais Centros Profissionais da rede estadual de ensino, embora cada um tenha as suas especificidades, enquanto fazer pedagógico. 

    \section{A educação profissional no Brasil e a realidade de ensino médio integrado à educação profissional no Rio Grande do Norte}

    Na atualidade, vivemos os desafios relacionados aos avanços tecnológicos, às dinâmicas do mundo do trabalho e às preocupações com a formação de um ser humano integral. As mudanças que têm ocorrido na sociedade moderna são significativas e, automaticamente, exigem de nós uma profissionalização de qualidade, que vise contribuir com o desenvolvimento das múltiplas habilidades presentes no ser humano. 

    As profissões se modificam para atender a uma demanda do mercado; o tempo, o trabalho, a produção e, em especial, a formação ganham perfis e significados diferentes gerando exigências na sociedade que, para atender a essa demanda, até então não se via.  

    O Brasil, ao longo de sua história, tem passado por mudanças que interferem na educação escolar. Nos primeiros anos de organização sistematizada, tivemos influência dos padres Jesuítas que já desenvolviam as formações específicas para os conhecimentos propedêuticos e para os conhecimentos manuais. Esse tipo de educação reforça que o ensino profissional no Brasil, desde sua colonização, se consolida para atender à demanda de formação específica para uma determinada parcela da população, com perfil determinado para tal. De acordo com \textcite[p.~32]{FRIGOTTOAndCIAVATTAAndRAMOS2005Ensino}, “No Brasil, o dualismo se enraíza em toda a sociedade através de séculos de escravismo e discriminação do trabalho manual.”, reforçando que o ensino de atividades manuais seria destinado para pessoas pobres.  

    A divisão entre trabalho manual e trabalho intelectual é confirmada nos modelos de ensino que se configurou no Brasil, ficando estabelecido que os conhecimentos propedêuticos deveriam ser voltados para os filhos da classe mais abastada da sociedade enquanto o ensino da técnica seria destinado aos filhos da classe trabalhadora, ou seja, o do fazer, ficando essa parcela social isenta da necessidade de uma formação que lhes preparasse para além do fazer manual. Sobre isso, \textcite[p.~4]{CIAVATTA2005formação} diz: 

    \begin{quotation}
        Esse dualismo toma um caráter estrutural especialmente a partir da década de 1940, quando a educação nacional foi organizada por leis orgânicas, segmentando a educação de acordo com os setores produtivos e as profissões, e separando os que deveriam ter o ensino secundário e a formação propedêutica para a universidade e os que deveriam ter formação profissional para a produção. 
    \end{quotation}

    Recentemente, algumas mudanças ocorreram e foram significativas para que o Ensino Profissional no Brasil tivesse um maior alcance para a população de modo geral. Destaca-se a Reforma da Educação Profissional ocorrida no Governo Luiz Inácio Lula da Silva, que desencadeou várias ações, entre elas o Programa Brasil Profissionalizado (2007) cuja adesão, por parte dos Estados, possibilitou a ampliação do ensino profissional nas redes estaduais de ensino.  

    O Ensino Médio Profissional, dentro dessa conjuntura, retoma a proposta de Ensino Integrado, apesar de o ensino concomitante e o subsequente ainda serem garantidos pelo Decreto nº 5.154 de 2004, que vai reger as modificações no Ensino Profissional, proposta que leva em consideração a formação do ser humano em uma perspectiva integral, com uma ampla formação no que se refere ao mundo do trabalho, garantindo conhecimento para o trabalho laboral e, além disso, para a vida.  

    Compreendendo que os Estados passam a ampliar as suas propostas dentro do contexto do Decreto nº 5.154/04, o Ensino Profissional ganha amplitude no cenário do Estado do Rio Grande do Norte, portanto, para atender a essas novas demandas --- os investimentos financeiros são redimensionados para a reestruturação das escolas que já ofertam ou ofertarão o Ensino Médio Integrado à Educação Profissional, modificando as dinâmicas de trabalho e o fazer pedagógico.  

    Tais mudanças já estão em andamento, embora esses órgãos, em sua composição, ainda apresentem dificuldades para garantir um Ensino Médio Profissional de qualidade, processo que vai desde a organização das estruturas das escolas que o irão ofertar --- ou que já o estão ofertando ---, no que se refere a maquinários e equipamentos específicos para a formação dos alunos, até a formação inicial e continuada dos profissionais envolvidos no processo de ensino-aprendizagem.  

    Nessa dimensão, compreendemos que o fazer pedagógico se caracteriza por todas as práticas de ensino, planejadas ou não, vivenciadas dentro do espaço escolar. Essas práticas, pensadas e repensadas em seu fazer, compõem o processo formativo do professor e da equipe pedagógica. 

    O fazer pedagógico dos professores aponta para as mudanças tecnológicas que modificaram as formas de produção da sociedade para a sua sobrevivência. As revoluções que ocorreram no mundo, ao longo desses últimos séculos, levaram, sem dúvida, a uma transformação do homem que atua no mundo do trabalho. Quando nos referimos ao mundo do trabalho, não nos limitamos ao laboral, mas fazemos referência também a sua ação como ser pertencente ao mundo, que produz vida, cultura, ciência \cite{CIAVATTA2005formação}.  

    Essas transformações trazem, à Educação, novas demandas, de forma rápida e sem muito diálogo. A escola, então, passa a ser vista como o espaço responsável pela produção de um novo homem/profissional, o qual deve atuar em um contexto diverso, já que lhe são cobradas múltiplas competências. Nessa realidade, emergem políticas públicas que irão conduzir o trabalho pedagógico das escolas, com um olhar para esse mundo do trabalho, impactando os direitos e deveres dos agentes desse processo para uma formação que visa ao preparo do estudante para essa perspectiva. 

    De acordo com \textcite[p.~2]{NASCIMENTOAndSILVA2017Políticas}, essas políticas seguem os interesses e preferências que provavelmente serão desenvolvidas nos governos, ou seja, temos políticas de governo e não política de Estado, as quais estão a serviço de quem está no poder. Nesse contexto, fragilizam-se as perspectivas de políticas públicas, pois, sem continuidade, é impossível garantir avanços na Educação Profissional que necessita de ações efetivas no que diz respeito à formação continuada de professores licenciados e não licenciados bem como investimentos em pesquisas em nível de Ensino Médio.  

    O Ensino Profissional no Brasil está mergulhado nesta conjuntura de forma tão profunda que, em muitos casos, uma política impede o desenvolvimento de outra. Os projetos e programas apresentados pelos governos, em sua grande maioria, são para atender as suas necessidades e não a da população de fato. De acordo com \textcite[p.~3]{NASCIMENTOAndSILVA2017Políticas},  

    \begin{quotation}
        [\dots] a educação profissional no país está passível de descontinuidade, já que projetos e programas são reinventados com base em uma releitura das ações governamentais anteriormente promovidas, inclusive, apresentados com outros nomes, adjetivos e no aparato legal. Desse modo, constatamos que as (re)apresentações de projetos e programas se dão com foco nas disputas de poder e não na promoção de melhorias para educação.
    \end{quotation}


    O Decreto nº 5.154/2004 institui uma nova organização ao Ensino Profissional, permitindo diferentes intervenções por meio da rede pública de ensino e pela rede privada. Salientamos que essa disponibilidade abre precedentes para várias situações no ensino profissional e que cumprir o que está no decreto se torna algo desafiador em nossas realidades de ensino.  

    Diante das modificações ocorridas depois do Decreto nº 5.154/2004, podemos destacar como ponto positivo e desafiador a expansão da Educação Profissional nas redes estaduais, seguindo uma proposta de ensino na forma integrada. Vale salientar que a Educação Profissional pode ser ofertada na forma articulada, concomitante ou subsequente. Na forma articulada, o processo de ensino-aprendizagem se dá com a integração entre o ensino das disciplinas propedêuticas\footnote{Disciplinas preparatórias para a continuidade dos estudos em nível universitário que se configuram nas disciplinas gerais de formação como Matemática, Geografia, História etc.} e as disciplinas técnicas na mesma instituição de ensino, possibilitando uma formação integral dos sujeitos. Na forma concomitante, a oferta das disciplinas técnicas acontece separada das propedêuticas, em instituições de ensino diferentes; e, por fim, a subsequente se dá após a conclusão do ensino médio.  

    O Rio Grande do Norte, em 2018, apresenta um quadro significativo de escolas que oferecem a Educação Profissional em sua rede de ensino, com 54 escolas conveniadas para oferta de Ensino Médio Integrado à Educação Profissional e a construção de 10 Centros Estaduais de Educação Profissional. Entre estes, destacamos os sete Centros Estaduais de Educação Profissional, que tiveram suas atividades iniciadas em 2017, além do CENEP, já em funcionamento desde 2008, distribuídos em várias cidades do RN, ampliando o quadro das diretorias regionais de ensino (DIRECs), que tem escolas e centros profissionais atendendo com o ensino de tempo integral ou semi-integral\footnote{Dados coletados do site da Secretaria da Educação e da Cultura: \url{http://www.educacao.rn.gov.br/}.}.

    Alguns Centros iniciaram suas atividades sem uma prévia preparação das equipes que iriam atuar na rede --- quadro incompleto de funcionários e professores e sem a devida estrutura física. Vale salientar que esse fato ocorreu com três Centros que não iriam atuar em tempo integral. 

    O Centro Estadual de Educação Profissional Dr. Ruy Pereira dos Santos, em 2017, atendia a 160 estudantes; no ano de 2018, contava com um total de 340 estudantes distribuídos em oito turmas, com dois cursos técnicos vigentes --- Técnico em Segurança do Trabalho e Técnico em Edificações --- e passou de um total de nove para 18 professores, todos responsáveis por várias disciplinas. 

    A estrutura curricular aprovada pela Secretaria de Educação do Estado, em dezembro de 2018, sistematiza o ensino em disciplinas e o estabelece da seguinte maneira: Base Nacional Curricular e parte diversificada --- Linguagens, Matemática, Ciências da Natureza e Ciências Humanas; e a Formação Técnica e Profissional --- Núcleo Articulador e Núcleo Tecnológico. Para o ano letivo de 2018, a estrutura curricular já foi alterada para atender à proposta do Ensino Médio Integrado à Educação Profissional, atuando em tempo integral.  

    Podemos salientar, ainda, que, ao fim de 2017, os três Centros que não atendiam em tempo integral, passaram, em 2018, a funcionar nessa conjuntura, sendo reorganizado todo o fazer pedagógico para atender a uma nova proposta de ensino orientada pela metodologia de êxito intitulada \textit{Escola da Escolha}, o que trouxe uma dinâmica diferente no trabalho pedagógico que já havia sido sistematizado em 2017. 

    Por estarmos acompanhando todo o processo de organização do trabalho pedagógico nessa instituição, podemos traçar um panorama do fazer pedagógico realizado pela equipe de docentes para sistematizar as mudanças ocorridas nesse período.  

    \section{Reflexões sobre o ensino médio integrado e as práticas pedagógicas no CEEP Dr. Ruy Pereira dos Santos}

    A expansão da Educação Profissional ocorrida nas últimas décadas possibilita a construção dos Centros Estaduais de Ensino Profissional, no Rio Grande do Norte, com recursos federais que visam atender à população, garantindo o Ensino Médio integrado à Educação Profissional, com uma dinâmica de atuação que valoriza a formação integral do aluno e a integração curricular, ressaltando avanços para desconstruir uma proposta de dualidade histórica no Ensino Médio com a Educação Profissional. 

    Nesse contexto, é preciso compreender melhor o que viria a ser ensino integrado. De acordo com \textcite[p.~2]{CIAVATTA2005formação}: 

    \begin{quotation}
        É tornar íntegro, tornar inteiro, o que [sic]? No caso da formação integrada ou do ensino médio integrado ao ensino técnico, queremos que a educação geral se torne parte inseparável da educação profissional em todos os campos onde se dá a preparação para o trabalho: seja nos processos produtivos, seja nos processos educativos como a formação inicial, como o ensino técnico, tecnológico ou superior. 
    \end{quotation}


    Compreender o ensino integrado dentro da realidade de escola que temos hoje é desafiador, pois os processos de ensino-aprendizagem estão arraigados de experiências de ensino que não levavam em consideração a dimensão integral no processo formativo do estudante. Os nossos profissionais nem sempre conseguem ter uma visão do estudante como ser integral, que deve ser formado em sua totalidade, pois a formação dos professores, por muito tempo, não levou em conta tal dimensão. Assim, o ensino da disciplina era o que se deveria ter em vista, e não a dimensão formativa desse processo.  

    É preciso, portanto, compreender que a escola é lugar de integrar. Sobre isso, \textcite[p.~43]{MACHADO2013avaliação} afirma que é nesse espaço que o indivíduo realiza o encontro com o outro e com o conhecimento, reforçando a compreensão de que o conhecimento não é dissociado, ou hierarquizado, mas que vive uma dialética para possibilitar uma formação mais completa.  

    Na conjuntura da organização dos Centros Estaduais da rede estadual de ensino do Rio Grande do Norte, essa compreensão é algo que preocupa as equipes diretivas em esfera superior. Observamos que, quando iniciamos um trabalho de formação que tem como foco o Ensino Médio Integrado à Educação Profissional, se faz necessário um processo de entendimento, de recepção dessa proposta para que ela venha efetivamente integrar --- se não houver logo de início esse processo de formação inicial, os ajustes deverão acontecer, inevitavelmente, dentro do processo, o que pode dificultar os diálogos e, consequentemente, a integração curricular.  

    Também destacamos que o Ensino Médio Integrado tem como base para o currículo integrado: o trabalho, a ciência, a cultura e a tecnologia – compreendendo que esses princípios são indissociáveis \cite{RAMOS2008Concepção} e que devem compor todo o fazer pedagógico.  

    O fazer pedagógico do professor deve ser respaldado em ações de reflexão sobre a prática, levando-se em consideração que esse fazer exige planejamento, pesquisa, valorização dos conhecimentos dos educandos e da comunidade local \cite{FREIRE1996Pedagogia}, garantindo, assim, um efetivo significado nas práticas de ensino realizadas no espaço escolar.  

    Dentro dessa perspectiva, para que o currículo integrado seja uma realidade, o \textit{trabalho} é compreendido como princípio educativo na educação básica; compreende a produção científica e tecnológica, ao longo dos anos, como patrimônio da humanidade, como algo que é adquirido no convívio social, visando à melhoria nas condições de trabalho, reforçando a ideia de que somos seres de trabalho, conhecimento e cultura. \textcite{RAMOS2008Concepção} ainda reforça essa noção e afirma que se trata de se ver: 

    \begin{quotation}
        O trabalho compreendido como realização humana inerente ao ser (sentido ontológico) e como prática econômica (sentido histórico associado ao respectivo modo de produção); a ciência compreendida como os conhecimentos produzidos pela humanidade que possibilita o contraditório avanço produtivo; e a cultura, que corresponde aos valores éticos e estéticos que orientam as normas de conduta de uma sociedade \cite[p.~3]{RAMOS2008Concepção}.
    \end{quotation}

    Também somos produtores de ciência, sistematizando os conhecimentos formulados em pesquisas e no cotidiano, visando compreender os fenômenos naturais e suas variações. Dentro dessa dinâmica, passamos a criar cultura, em seu sentido mais amplo, entendendo-a como “[\dots] tanto a produção ética quanto estética de uma sociedade” \cite[p.~7]{RAMOS2008Concepção}.

    Tendo esses princípios educativos como base, o desafio de atuar em uma escola de ensino médio profissional é, sem dúvida, um processo de compreensão de que ensinar não é transmitir conhecimento, mas sim permitir que o aluno se faça crítico, sistematizar o ensino por meio da pesquisa, valorizando os conhecimentos prévios dos alunos, do respeito ao próximo, da vivência, da cultura \cite{FREIRE1996Pedagogia}, enfim, todo um processo conjuntivo que leva a uma formação integral. 

    Tendo essa dimensão como base, podemos compreender que o trabalho desenvolvido no CEEP Dr. Ruy Pereira dos Santos, no decorrer desses dois anos, buscou garantir que o conhecimento fosse compreendido em sua totalidade, buscando realizar a interdisciplinaridade nas ações cotidianas, com base nas observações realizadas ao longo da pesquisa. 

    No ano de 2017, o CEEP atendia à comunidade ofertando o Ensino Médio Integrado à Educação Profissional em tempo semi-integral. O perfil dos alunos nesse período era de jovens que buscavam um curso técnico para adentrar no mercado de trabalho, viam no ensino médio integrado a possibilidade de obter essa formação.

    O trabalho pedagógico foi vivenciado, inicialmente, com a ausência de professores da área técnica. Essa realidade pendurou até o fim do terceiro bimestre. Com isso, foi possível perceber uma pequena evasão, tendo em vista que o objetivo de muitos era a formação técnica, e a ausência desses profissionais gerava sentimento de incompletude.  

    Os professores das disciplinas propedêuticas realizaram, entretanto, um trabalho sempre voltado para a formação do estudante, compreendendo a dimensão da formação integral. Nesse ano, foram realizadas atividades interdisciplinares --- podemos destacar o 1º Simpósio do CEEP Dr. Ruy Pereira dos Santos, atividade que teve o intuito de aproximar o estudante a vivências acadêmicas proporcionadas pela realização de palestras, mesas redondas e produção de artigos para serem apresentados em Grupos de Trabalhos. Com base em nossas observações, concluímos que a atividade teve caráter científico e o objetivo de contribuir com o processo formativo do aluno, que deve ter acesso às estratégias para o desenvolvimento de pesquisas voltadas para as demandas da comunidade local.  

    O simpósio 2017 se organizou com o tema \textit{Meio Ambiente e Sustentabilidade} --- um convite ao cuidado com a casa comum. Na organização geral, as pesquisas foram divididas em quatro Grupos de Trabalho: 1)~Meio Ambiente, Literatura e Artes; 2)~Meio Ambiente, Matemática e Formação Tecnológica; 3)~Meio Ambiente e Interdisciplinaridade: desafios e perspectivas no cotidiano escolar; 4)~Meio Ambiente e Ciências Humanas. 

    Podemos perceber que os movimentos de interdisciplinaridade foram vivenciados nesse processo, possibilitando uma formação crítica e reflexiva na tentativa de solucionar problemas vivenciados na sociedade ou de refletir sobre eles, e, em especial, sobre a comunidade de que a escola faz parte. Nessa dimensão, de acordo com \textcite{YARED2008Interdisciplinaridade}:  

    \begin{quotation}
        A palavra interdisciplinaridade evoca a "disciplina" como um sistema constituído ou por constituir, e a interdisciplinaridade sugere um conjunto de relações entre disciplinas abertas sempre a novas relações que se vai descobrindo. Interdisciplinar é toda interação existente entre duas ou mais disciplinas no âmbito do conhecimento, dos métodos e da aprendizagem das mesmas [sic]. Interdisciplinaridade é o conjunto das interações existentes e possíveis entre as disciplinas nos âmbitos indicados. \cite[SUERO, 1986, p.~18--19 apud][p.~161]{YARED2008Interdisciplinaridade}
    \end{quotation}

    Entendemos que essa concepção de Yared é pertinente e tem em vista possibilitar uma formação mais completa para o estudante, com uma perspectiva de formação integral. Dessa maneira, o conhecimento que, historicamente, foi sistematizado e organizado de forma disciplinar, nesse novo contexto de formação, necessita ser reestruturado e reorganizado em uma perspectiva integrada, que percebe a necessidade de compreender o processo de ensino-aprendizagem na dimensão da tecnologia, da ciência, da cultura e do trabalho como eixos que garantam essa integração.  

    No ano seguinte, em 2018, a oferta do Ensino Médio Integrado à Educação Profissional passou a ser em tempo integral, modificando totalmente o trabalho que vinha sendo realizado na instituição. A primeira mudança foi em relação à quase totalidade dos professores que ali atuavam. Por motivos de organização administrativa e em virtude da proposta do tempo integral, a equipe que fosse atuar no CEEP também precisava trabalhar em tempo integral, ou seja, com um regime de 40 horas de trabalho, todas realizadas na escola.  

    Junto com essa mudança também aconteceram modificações na estrutura curricular dos cursos ofertados --- Técnico em Segurança do Trabalho e Técnico em Edificações. Essas mudanças alteram as dinâmicas de trabalho cotidiano. A instituição passa a ofertar disciplinas como: Projeto de Vida, Componentes eletivos, Estudo Orientado, Preparação Pós-Médio, Avaliação Semanal, Atividades pré-experimentais e experimentais, Informática Básica, além da disciplina de Empreendedorismo. 

    As orientações dadas pelo Instituto de Corresponsabilidade pela Educação (ICE), com base na proposta pedagógica Escola da Escolha, passam a ser a base do trabalho pedagógico realizado nas instituições de ensino de tempo integral, orientando e sistematizando seu trabalho pedagógico.  

    Nessa nova dinâmica, o trabalho interdisciplinar tomou proporções bem maiores, passando a acontecer nas atividades do cotidiano, situação que, até então, só acontecia em eventos, salvo algumas atividades específicas.  

    Com base em nossos registros, observamos que os componentes eletivos e as avaliações semanais, duas atividades que acontecem semanalmente, ganham uma dimensão interdisciplinar, desde a sua organização até a sua execução. Além dessas atividades, podemos destacar que os professores das áreas de conhecimento dialogam constantemente para alinhar as suas atividades. Os momentos de diálogo acontecem no dia de planejamento coletivo, coordenado por um professor que tenha o perfil para exercer a função.  

    Verificamos que, de modo geral, o fazer pedagógico do Centro Estadual de Educação Profissional Dr. Ruy Pereira dos Santos busca atender a uma dimensão do trabalho, da ciência, da cultura e da tecnologia, visando desenvolver um trabalho que pense na perspectiva da formação humana integral, contribuindo significativamente nesse processo com ações de reflexão e vivências diferenciadas no espaço escolar, para a oferta do Ensino Médio Integrado à Educação Profissional. 

    Vale salientar que a oferta de Educação Profissional ainda se configura em um caminho de ajustes, no que concerne à formação continuada dos professores, ao ajuste das estruturas curriculares e à compreensão do Currículo Integrado.

    \section{Considerações finais}

    Finalizamos este trabalho, que teve como propósito refletir sobre o Ensino Médio Integrado à Educação Profissional, considerando como referência o trabalho pedagógico realizado no Centro Estadual de Educação Profissional Dr. Ruy Pereira dos Santos, no Rio Grande do Norte.  

    As intervenções que hoje são aliadas a esse contexto, como a ação do Instituto de Corresponsabilidade pela Educação, são estratégias que, em sua conjuntura, tentam fazer com que o Ensino Médio Integrado em tempo integral seja uma realidade consolidada na rede, embora ainda seja insuficiente no que diz respeito a estratégias específicas para a Educação Profissional. 

    Inferimos que o Ensino Médio Integrado à Educação Profissional, no Rio Grande do Norte, ainda provocará grandes reflexões sobre a prática pedagógica. Já antevemos que, a partir dessas reflexões, adaptações precisarão ser feitas para garantir um bom resultado acadêmico e uma formação que tenha o Currículo Integrado como caminho para o trabalho pedagógico realizado no âmbito escolar.  

    \nocite{Decreto5154-2004}
    \printbibliography[heading=subbibliography,notcategory=fullcited]

    \label{chap:reflexao-fazer-pedagogicoend}

\end{refsection}
