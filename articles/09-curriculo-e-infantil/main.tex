\begin{refsection}
    \renewcommand{\thefigure}{\arabic{figure}}
    
    \chapterTwoLines
    {Currículo na Educação Infantil}
    {desafios encontrados pelos professores}
    \label{chap:curriculo-ed-infant}

    \articleAuthor
    {Josefa Maria de Medeiros }
    {Graduada em Pedagogia pelo Instituto de Educação Superior Presidente Kennedy. Especialista em Gestão de Processos Educacionais. E-mail: zefa\textunderscore{}galega@hotmail.com.}
    
    \articleAuthor
    {Antônia Zélia de Assis Dantas}
    {Graduada em Pedagogia (UFRN). Mestra em Gestão Educacional pela Universidade Internacional de Lisboa. Especialista em Educação (UFPB), Especialista em Comunicação Educacional pela Universidade Regional do Nordeste, em Campina Grande, PB. Professora Formadora do Instituto de Educação Superior Presidente Kennedy --- IFESP. ID Lattes: 3949.5792.0898.6115. ORCID: E-mail: antoniazdantas@gmail.com.}
    
    \begin{galoResumo}
        \marginpar{
            \begin{flushleft}
            \tiny \sffamily
            Como referenciar?\\\fullcite{SelfMedeirosAndDantas2021Currículo}\mybibexclude{SelfMedeirosAndDantas2021Currículo}, p. \pageref{chap:curriculo-ed-infant}--\pageref{chap:curriculo-ed-infantend}, \journalPubDate{}
            \end{flushleft}
        }
        O presente artigo faz uma análise acerca dos desafios encontrada pelos professores na elaboração do currículo na Educação infantil. Adota como metodologia as pesquisas bibliográficas e de campo, que procuram conhecer os principais desafios encontrados quanto à elaboração de currículos no seu nível de concretização da sala de aula. Os estudos realizados apontam a necessidade de dar voz às crianças nessa construção e a necessidade de repensar as posturas dos professores.
    \end{galoResumo}
    
    \galoPalavrasChave{Educação. Currículo. Educação Infantil.}
    
    \begin{otherlanguage}{english}

    \fakeChapterTwoLines
    {Curriculum in early childhood education}
    {challenges teachers encounter}

    \begin{galoResumo}[Abstract]
        This article analyzes the challenges teachers face in the curriculum development in early childhood education. It adopts bibliographic and field research as a methodology, which seeks to understand the main challenges encountered in the development of curricula at their level of implementation in the classroom. The studies carried out point to the need to give children a voice in this construction and the need to rethink the attitudes of teachers. 
    \end{galoResumo}
    
    \galoPalavrasChave[Keywords]{Education. Resume. Child education.}
    \end{otherlanguage}

    \section{Introdução}

    O presente trabalho tem por tema “Currículo na Educação Infantil: desafios encontrados pelos professores”, e partiu do seguinte problema de pesquisa: quais são os principais desafios encontrados pelo professor quanto à organização do currículo na Educação Infantil?  

    Admitindo a riqueza do contexto de lutas e determinações e discussões sobre as quais a Educação Infantil vem sendo construída, foi que pensamos definir o objeto de estudo dessa pesquisa, organização do currículo na Educação Infantil, temática que despertou curiosidade enquanto educadora, que observa na prática algumas dificuldades dos professores quanto à elaboração da proposta curricular, mesmo com as formações que atualmente são disponibilizadas. Sem generalizar a questão, pois sei que este problema pode não se estender a todos os professores, no entanto, devido à complexidade do currículo escolar, acredita-se que muito tem para ser esclarecido ou compreendido. 

    Os objetivos do artigo pautaram-se em explorar alguns conceitos de currículo à luz dos teóricos que discutem o currículo, identificar a concepção do currículo na Educação Infantil e na legislação educacional e analisar os desafios encontrados pelos professores que atuam nessa etapa do ensino quanto a elaboração do currículo. 

    Como procedimentos metodológicos adotamos a Pesquisa Bibliográfica, onde foram consultadas referências que tratem do tema em questão, para ter uma visão geral do assunto; e a Pesquisa de Campo, com observação e aplicação de questionário aos professores, objetivando compreender e explicar o problema pesquisado. 

    Acerca da organização do artigo, além da Introdução que abarca o tema, a justificativa, os objetivos e a metodologia, o trabalho está dividido em seções: “Educação Infantil: o que diz a legislação” (ECA, \citefield{ECA1990}{year}; LDB, \citefield{LDB1996}{year}; RCNEI, \citefield{RCNEI1998}{year}); “Dimensões conceituais do currículo” \cite{MoreiraAndSilva2005Currículo,CAVALCANTI2011Currículo,SACRISTÁN1999Poderes};“Concepções de currículo na Educação Infantil” \cite{OLIVEIRA2013Novas,BARBOSA2009Práticas,KRAMER2003Formação} e “Os desafios dos professores na elaboração do currículo”, análise dos dados da pesquisa. 

 

    \section{Educação Infantil: o que diz a legislação}

    O contexto histórico da Educação Infantil, durante séculos, foi marcado pela responsabilidade da família e principalmente da mãe que era a cuidadora do lar e que tinha como função procriar mais filhos e zelar pela educação deles, uma vez que os maridos trabalhavam nas lavouras e engenhos para o sustento da casa. \cite{OLIVEIRA2002Educação}. 

    Era também por meio da participação nas tradições e no convívio com os adultos que as crianças aprendiam as normas e regras de sua cultura, e a infância durava até os sete anos de idade e a partir daí a criança era vista como um adulto em miniatura e exercia os mesmos trabalhos que os adultos. 

    Analisando a História Constitucional do Brasil, iniciando pela “Carta Imperial”, de 1824, até os dias atuais sob a égide da “Carta Cidadã”, de 1988, vê-se como a educação é tratada a nível constitucional, ou seja, se a política educacional brasileira atende aos verdadeiros anseios do cidadão. 

    Atualmente a Educação Infantil vem se caracterizando em cenário de maior destaque e também de algumas mudanças, principalmente quando se refere ao atendimento às crianças desde a função assistencialista até à função educacional, aliando o educar e o cuidar. 

    Mesmo com os avanços obtidos a partir da Constituição de 1988, considerando os direitos sociais das crianças atendidas em creches e pré-escolas e as promulgações de novas leis, assim como a Lei de Diretrizes e Bases da Educação \citeyear{LDB1996}, muitos são, ainda, os desafios enfrentados pela Educação Infantil, apesar do seu reconhecimento como dever do Estado e da família. 

    Houve um avanço significativo da legislação quanto está reconheceu a criança como cidadã, com direito à educação de qualidade desde o seu nascimento. 

    Outro avanço que reforça os direitos já conquistados e avança no sentido da proteção integral e dos direitos sociais das crianças e adolescentes, é o Estatuto da Criança e do Adolescente (ECA), promulgado em 1990. Nele, a concepção de infância se intensificou ainda mais, dando possibilidade de transformar um novo pensamento acerca do que venha a ser criança passando a ser compreendida como uma pessoa que também possui direitos, e liberdade plena de expressar o seu pensamento como cidadã. 

    Em 1996 entra em cena a nova Lei de Diretrizes e Bases da Educação Nacional (LDBEN 9.394/96) e, com ela, ocorrem significativas mudanças acerca do atendimento institucional às crianças pequenas, dentre as quais o processo de inserção das creches no sistema de ensino. 

    A nova LDBEN se caracterizou como um marco histórico importantíssimo para a Educação Infantil, pois responsabilizou os municípios no atendimento das crianças de 0 a 6 anos e estabeleceu um curto espaço de tempo para que eles se organizassem e assumissem a Educação Infantil em seus respectivos sistemas de ensino. Para dar a tônica da qualidade do atendimento a partir das orientações do Ministério da Educação e não mais da Assistência Social como secularmente era submetida, foram lançados os Referencias Curriculares Nacionais para a Educação Infantil (RCNEI) que trouxe uma definição específica sobre o conceito de criança: “Sujeito social e histórico e faz parte de uma organização familiar que está inserida em uma sociedade, com uma determinada cultura, em um determinado momento histórico. É profundamente marcada pelo meio social em que se desenvolve” \cite[p.~20]{RCNEI1998}.  

    Nesse meio tempo, as Diretrizes Curriculares Nacionais de Educação Infantil (DCNEI) assumiram a missão de substituir os RCNEI que por mais de uma década subsidiaram as práticas pedagógicas da Educação Infantil.  

    Para isso, traz uma nova definição de currículo e de proposta pedagógica que considera esses novos processos educativos que devem estar comprometidos com a democracia e a cidadania, com a dignidade da pessoa humana, com o reconhecimento da necessidade de defesa do meio ambiente e com o rompimento de relações de dominação etária, socioeconômica, étnico-racial, de gênero, regional, linguística e religiosa. 

    Em 2014 o Plano Nacional de Educação (PNE) foi promulgado e a universalização da pré-escola e ampliação da creche está explícita na meta 1 com prazos definidos. O PNE estabelece o limite de até um ano para que os municípios adaptem a partir dele, seus planos municipais de educação e, executem-nos conforme exposto nos planos. A meta 1 traz desdobramentos definidos como estratégias que facilitarão os caminhos para que a meta seja alcançada. 

    Nesse sentido, consoante aos marcos legais citados anteriormente, o PNE afirma a importância de uma base nacional comum curricular para o Brasil, com o foco na aprendizagem como estratégia para fomentar a qualidade da Educação Básica em todas as etapas e modalidades, referindo-se a direitos e objetivos de aprendizagem e desenvolvimento. 

    Com o Parecer CNE/CP nº 15/2017, aprovado em 15 de dezembro de 2017, a Educação Infantil foi inserida na Base Nacional Comum Curricular, debatida num processo amplo de discussão e democratização realizado em todo país por diferentes representações sociais, de classe, entidades, instituições, entre outras representações. Com a inclusão da Educação Infantil na BNCC, mais um importante passo é dado nesse processo histórico de sua integração ao conjunto da Educação Básica. 

    A concepção de Educação Infantil que veicula nesse documento alia o educar ao cuidar, entendendo o cuidado como algo indissociável do processo educativo. Nesse contexto, as creches e pré-escolas, ao acolher as vivências e os conhecimentos construídos pelas crianças no ambiente da família e no contexto social, e articulá-los em suas propostas pedagógicas, têm o objetivo de ampliar o universo de experiências, conhecimentos e habilidades dessas crianças, diversificando e consolidando novas aprendizagens.  

    Por fim, conclui-se que muitos são os avanços obtidos na Educação Infantil para crianças de 0 a 6 anos por intermédio da Constituição Federal (1988), Estatuto da Criança e do Adolescente (1990), Lei de Diretrizes e Bases (1996), entre outros marcos legais que reconheceram a criança como um sujeito de direitos, reforçaram e ampliaram a perspectiva da Educação Infantil como primeira etapa da Educação Básica. 

    
    \section{Dimensões conceituais do currículo}

    O currículo é um tema muito importante, sendo fundamental no debate atual sobre educação, pois deixou de ser apenas um conjunto de disciplinas que constituem um curso de qualquer nível, para denominar a inteira participação da escola no processo da experiência professor-aluno.  

    Até a década de 1960 currículo significava grade de disciplinas. A partir de então, houve a necessidade de se estudar esse tema, a fim de procurar uma ressignificação conceitual. Na década de 1970 as bases dos paradigmas críticos começam a ser estabelecidas, e o currículo deixou de ser grade e assumiu novos significados. 

    Durante os anos de 1980, o currículo passa a ser reconhecido como campo de lutas e contradições, passando a absorver o que acontece no mundo. Na década de 1990, com o apogeu da Lei de Diretrizes e Bases (1996), as discussões conceituais se expandem significativamente e ganham corpo com a aprovação das Diretrizes Curriculares Nacionais para a Educação Infantil (2009), dentre outros documentos que foram sendo elaborados. 

    Sob esse entendimento observamos que o campo das discussões curriculares é um campo aberto. Como afirma \textcite{PACHECO2007Currículo} existem diversas concepções de currículo e não existe um consenso acerca de sua definição, como podemos notar nas definições de \textcite{MoreiraAndSilva2005Currículo}, \textcite{CAVALCANTI2011Currículo}, \textcite{SACRISTÁN1999Poderes}, \textcite{MASETTO2003Competência}, \textcite{OLIVEIRA2013Novas}, entre outros. 

    Para \textcite[p.~8]{MoreiraAndSilva2005Currículo}, 

    \begin{quotation}
        [\dots] o currículo não é um elemento inocente e neutro de transmissão desinteressada do conhecimento social. O currículo está implicado em relações de poder, o currículo transmite visões sociais particulares e interessadas, o currículo produz identidades individuais e sociais particulares. O currículo não é um elemento transcendente e atemporal --- ele tem uma história, vinculada as formas específicas e contingentes de organização da sociedade e da educação. 
    \end{quotation}

    \textcite{CAVALCANTI2011Currículo} sugere o currículo constituído em um campo complexo onde os limites que o conceituam são amplos. Essa conotação pode ser vista na definição complexa trazida por Silva, em que o currículo extrapola a ideia de um corpo de disciplina. Porém, Cavalcante se distancia de Silva quando faz menção a existência de duas concepções de currículo. Uma que ele define como comum, onde o currículo compreende um “elenco e sequência de matérias ou disciplinas para todo o sistema escolar” (p.~173); e outra que a denomina de típica, vista como um “conjunto de experiências educativas vividas pelos alunos, sob a tutela da escola” \cite[p.~173]{CAVALCANTI2011Currículo}. 

    Contribuindo com esta análise, \textcite[p.~61]{SACRISTÁN1999Poderes} faz abordagem do currículo numa perspectiva cultural e o reconhece como: 

    \begin{quotation}
        [\dots] a ligação entre a cultura e a sociedade exterior à escola e à educação; entre o conhecimento e cultura herdados e a aprendizagem dos alunos; entre a teoria e a prática possível, dadas determinadas condições. 
    \end{quotation}

    Contudo, essas discussões não se permitem seguir apenas por esse caminho conceitual. \textcite{MASETTO2003Competência} ao definir currículo traz o foco na aprendizagem, mantendo a ideia de que ela seja adquirida mediante atividades e práticas planejadas.  

    \textcite{SOBRAL2007Propostas} produz a ideia de currículo numa perspectiva dialógica e defende que a construção de um currículo deve se constituir no diálogo entre o documento que orienta a prática e a prática que mobiliza a reelaboração do documento num processo contínuo de reflexão e ação com a participação ativa e democrática de toda a comunidade escolar.  

    \textcite{OLIVEIRA2013Novas} aproximadas mesmas ideias, ou seja, da dinâmica do trabalho curricular nas escolas e acha pertinente uma nova compreensão de currículo, que não o considere como um produto pré‐estabelecido, mas um processo por meio do qual os praticantes do currículo resinifiquem suas experiências a partir das redes de saberes e fazeres das quais participam.  

    Assim, considerando, o professor e o aluno passam a ser de suma importância no processo de implementação do currículo. Os professores precisam sistematicamente ser na concepção de \textcite[p.~12]{SCOCUGLIA2014Paulo}, “reeducados na própria prática reflexiva de construir/reconstruir currículo, junto com os demais sujeitos do processo educativo”, ou seja, constantemente é preciso buscar meios e caminhos para que a aprendizagem dos alunos seja facilitada, não devendo, entretanto, ser negado o esforço da busca, o que faz com que realmente a aprendizagem ocorra. 

    \textcite[p.~10]{MORGADO2005Currículo}, complementa que os professores “constituem a principal força propulsora da mudança educativa e do aperfeiçoamento da escola”, uma vez que deles depende, em grande parte, as formas como se idealizam e concretizam os processos educativos. 

    De acordo com o que foi visto podemos entender que apesar de tantas concepções formuladas o currículo termina centralizando o seu foco no percurso que leva a aprendizagem. Logo é importante que o ambiente escolar e os professores formulem de acordo com as necessidades e as limitações dos estudantes, pois, o mesmo é o principal sujeito do ambiente educacional.


    \section{Concepções de currículo na Educação Infantil}

    Pesquisas demonstram que ao final da década de 1970 e início de 1980, debates mais amplos acerca da Educação Infantil começam a aparecer, uma vez que a área conquista alguns espaços importantes, em especial no período pré-constituíste, favorecendo o delineamento de um novo projeto pedagógico para esse campo educativo. \cite{OLIVEIRA2002Educação}. 

    A Constituição de 1988 prevê a adoção do currículo escolar por todas as instituições de ensino do país, o que visa garantir que todos os estudantes do Brasil tenham acesso a uma série de conteúdos fixos, que são considerados mínimos para a Formação Básica. 

    O Art. 210 da CF/88 determina como dever de o Estado para com a educação fixar “conteúdos mínimos para o Ensino Fundamental, de maneira a assegurar a formação básica comum e respeito aos valores culturais e artísticos, nacionais e regionais”. Dessa forma, foram elaborados e distribuídos pelo MEC, a partir de 1995, os Referenciais Curriculares Nacionais para a Educação Infantil – RCNEI (1998), os Parâmetros Curriculares Nacionais - PCN’s para o Ensino Fundamental (1997) e os Referenciais Curriculares para o Ensino Médio (2006). 

    A trajetória da construção de uma proposta curricular para a Educação Infantil também emergiu com a nova Lei de Diretrizes e Bases da Educação (LDB) 9394/96, que compreende a Educação Infantil em seu artigo 29, “[\dots] como a primeira etapa da Educação Básica e tem como finalidade o desenvolvimento integral da criança considerando o aspecto psicológico, intelectual e social”; e, com as novas Diretrizes Curriculares Nacionais da Educação Infantil (DCNEIs) aprovadas pelo Conselho Nacional de Educação em 2009, que representam uma valiosa oportunidade para se pensar como e em que direção atuar junto às crianças a partir de determinados parâmetros e como articular o processo de ensino-aprendizagem na Escola Básica. 

    No contexto das DCNEIs, o currículo é visto como:  

    \begin{quotation}
        Um conjunto de práticas que buscam articular as experiências e os saberes das crianças com os conhecimentos que fazem parte do patrimônio cultural, artístico, ambiental, científico e tecnológico, de modo a promover o desenvolvimento integral de crianças de 0 a 5 anos de idade \cite[p.~12]{Resolução5-2009}. 
    \end{quotation}


    Nessa perspectiva, compreendendo as instituições de Educação Infantil como ambientes não só do cuidar, mas também do educar, o currículo deve ser pensado no desenvolvimento da criança nos aspectos físico, moral e intelectual. Desse modo, o trabalho pedagógico organizado em creche ou pré-escola, deve integrar o cuidar e o educar, criando um ambiente em que a criança se sinta segura, satisfeita em suas necessidades, acolhida em sua maneira de ser, onde possa construir hipóteses sobre o mundo e elaborar sua identidade. 

    A esse respeito, \textcite[p.~76]{KRAMER2003Formação} enfatiza que:  

    \begin{quotation}
        [\dots] Não é possível educar sem cuidar [\dots] há atividades que uma criança pequena não faz sozinha [\dots] há atividades de cuidado que são específicas da educação infantil, contudo, no processo de educação, em qualquer nível de ensino, cuidamos sempre do outro. Ou deveríamos cuidar! [\dots] já não será hora de assumir o educar, entendendo que abrange as duas dimensões. 
    \end{quotation}

    Considerando as características da faixa etária compreendida em zero e cinco anos e suas formas específicas de aprender criou-se categorias curriculares para organizar os conteúdos a serem trabalhados nas instituições de Educação Infantil.

    \begin{quotation}
        Esta organização visa abranger diversos e múltiplos espaços de elaboração de conhecimentos e de diferentes linguagens, a construção da identidade, os processos de socialização e o desenvolvimento da autonomia das crianças que propiciam, por sua vez, as aprendizagens consideradas essenciais \cite[p.~45]{RCNEI1998}.
    \end{quotation}

    As particularidades de cada proposta curricular devem estar vinculadas às características socioculturais da comunidade na qual a instituição de Educação Infantil está inserida e às necessidades e expectativas da população atendida, pois, quando se constrói o currículo para a Educação Infantil é preciso pensar a criança como um sujeito social e histórico que se desenvolve através da interação com o outro, “pois é a criança a origem e o centro de toda atividade escolar” \cite[p.~53]{MoreiraAndSilva2005Currículo}.  

    Segundo \textcite{BARBOSA2009Práticas}, o currículo, nessa perspectiva, precisa estar articulado às práticas culturais de determinado grupo social. Devem estar presentes, também, os contextos sociais e familiares em que se inserem essas crianças, sendo que essa articulação se faz necessária para que haja uma compatibilidade de valores e uma compreensão em torno das transformações dos valores. 

    \textcite{KISHIMOTO1994Currículo} pondera que os alunos possuem individualidades distintas, interpretam e vivenciam as situações de modo variado. Para isso requer reconhecimento que “[\dots]os hábitos, costumes e valores presentes na sua família e na localidade mais próxima interferem na sua percepção do mundo e na sua inserção” \cite[p.~22]{KRAMER2003Formação}. 

    Assim, entende-se que é preciso considerar os conhecimentos que a criança já possui e suas várias experiências culturais, para, portanto, pensar uma ação pedagógica que proponha para as crianças um mundo de interação, o que contribuirá para o desenvolvimento emocional e social, fundamentando-as nas formações, e na realidade de cada um. \textcite{OLIVEIRA2013Novas} colabora e diz que a preocupação deve estar em oferecer à criança “o direito de ampliar sua perspectiva de mundo”. 

    Nesse sentido, para ouvir as vozes infantis, faz‐se necessário, como salienta \textcite[p.~6]{OLIVEIRA2013Novas}, superar alguns desafios para a elaboração curricular, bem como sua efetivação no cotidiano escolar, assim, a proposta pedagógica, “[\dots] deve “[\dots] transcender a prática pedagógica centrada no professor”. 

    Enfim, criar e dar vida ao currículo na Educação Infantil requer uma mudança de paradigmas nas relações do professor, melhorando suas ações pedagógicas e criando um ambiente cooperativo, como as autoras \textcite[p.~51]{DEVRIESAndZAN2004currículo} afirmam: “[\dots] Criar uma atmosfera sócio moral cooperativa, consultando as crianças e dando a elas uma significativa quantidade de poder para determinar o que ocorre em sala de aula”.  


    \section{Os desafios dos professores na elaboração do currículo}

    Nessa seção da pesquisa analisamos os dados coletados à luz do referencial construído e a partir deste fazemos uma reflexão acerca dos resultados obtidos, considerando a natureza do seu objeto de estudo. Para isso optamos por realizar uma pesquisa com seis (6) professoras de um Centro Municipal de Educação Infantil, situado no bairro das Quintas, em Natal/RN. Nele funcionam os turnos matutino (níveis I, II e III) e vespertino (níveis II, III e IV), com 138 alunos no total. 

    No que tange a técnica de coleta de dados escolhemos como instrumento a aplicação de um questionário com ênfase em questões abertas, totalizando onze (11) perguntas, as quais foram analisadas através da abordagem qualitativa com tratamento fundados nos procedimentos da análise de conteúdo, na perspectiva de \textcite{BARDIN2011Análise} que a define como: 

    \begin{quotation}
        Um conjunto de técnicas de análise das comunicações visando a obter, por procedimentos sistemáticos e objetivos de descrição do conteúdo das mensagens, indicadores (quantitativos ou não) que permitam a inferência de conhecimentos relativos às condições de produção/recepção (variáveis inferidas) destas mensagens \cite[p.~47]{BARDIN2011Análise}.  
    \end{quotation}

    Sendo assim, trataremos da análise dos resultados obtidos perante as informações das professoras pesquisadas, mediante as seguintes questões:

    \begin{itemize}
        \item Caracterização, formação acadêmica e atuação profissional;  
        \item Atuação na Educação Infantil;  
        \item A oferta de Formação Continuada;  
        \item A elaboração do projeto de Formação Continuada na escola;  
        \item Expectativas da Formação Continuada;  
        \item Contribuições das formações para elaboração do currículo;  
        \item Concepção de currículo;  
        \item A importância de organizar o currículo na Educação Infantil; 
        \item Os desafios apontados para executar o currículo.  
    \end{itemize}

    Essas questões são colocadas a seguir considerando como foco de análise os desafios enfrentados pelos dos professores diante a elaboração do currículo. 

    \paragraph{Caracterização, formação acadêmica e atuação profissional} Sobre a formação dos professores participantes da pesquisa a instituição abriga: um professor Especialização em Educação Infantil, outro Especialização em Linguística, um com o curso do Magistério, dois com Magistério e Pedagogia e um com Serviço Social e Pedagogia. 

    No tocante à qualificação das professoras notamos, apesar de formação no campo da docência, elas não apresentam formação específica na área da Educação Infantil, exceto um professor que possui especialização. 

    \paragraph{Atuação na Educação Infantil} Analisando as falas das professoras pesquisadas, identificadas nos relatos pelas letras de Aa F, observamos os sentimentos de pertencimento delas com o trabalho que desenvolvem, apesar de revelar alguns desafios presentes no cotidiano dos espaços da Educação Infantil, manifestam:

    \begin{quotation}
        \noindent\negpar[-1.5em]{}Professora A\quad{}\\Sinto-me realizada em sala de aula, adoro lidar com as crianças. 
        \medskip
    
        \negpar[-1.5em]{}Professora B\quad{}\\Feliz, apesar das dificuldades estou em sala de aula com 22 crianças, sem auxiliar, é muito difícil, mesmo assim amo ser educadora. 
        \medskip

        \negpar[-1.5em]{}Professora C\quad{}\\Sinto-me uma privilegiada, pois estou sempre aprendendo com as crianças. 
        \medskip

        \negpar[-1.5em]{}Professora D\quad{}\\Bem, gosto muito do que faço porém os desafios são inúmeros. 
        \medskip

        \negpar[-1.5em]{}Professora E\quad{}\\Realizada, gosto do meu trabalho [\dots] sou consciente que meu aprendizado é contínuo e satisfatório. 
        \medskip

        \negpar[-1.5em]{}Professora F\quad{}\\Realizo-me, todos os dias vê que meu aprendizado é constante. Procuro melhorar minha “educação de qualidade”. 
    \end{quotation}

    Fica, então, estabelecido o anúncio feito logo acima, as professoras demostram felicidade e dizem gostar do que fazem e ainda evidenciam ser conscientes do seu papel diante do binômio ensinar e aprender. 

    Esta é a posição que a maioria assume posturas significativas perante a elaboração do currículo. Entretanto, quando falam dos desafios encontrados para desempenhar o seu trabalho, implicitamente revelam fragilidades perante a prática curricular. 

    \paragraph{A oferta de Formação Continuada} Quanto à existência da oferta de formação para os professores, as participantes da pesquisa disseram participar sempre que essas são proporcionadas. Porém, a Professora D comentou que ainda são poucos os encontros e enaltece a sua importância.
    
    \begin{quotation}
        \noindent\negpar[-1.5em]{}Professora A\quad{}\\Participo dos encontros coletivos na escola. 
        \medskip
    
        \negpar[-1.5em]{}Professora B\quad{}\\Sim, quando tem. 
        \medskip

        \negpar[-1.5em]{}Professora C\quad{}\\Participo, dando o melhor e procurando aprender mais para ensinar com qualidade. 
        \medskip

        \negpar[-1.5em]{}Professora D\quad{}\\Quando é proporcionado o momento, sim. 
        \medskip

        \negpar[-1.5em]{}Professora E\quad{}\\Sim, nos momentos de discussões das propostas pedagógicas de quando surgem questionamentos novos, visando melhor aprimoramento na área. 
        \medskip

        \negpar[-1.5em]{}Professora F\quad{}\\Sim, estamos sempre nos reunindo para discutir e pôr em prática novos temas e propostas que surgem dentro da educação infantil [\dots] no momento estamos estudando a BNCC. 
    \end{quotation}
    

    Percebemos nas falas que as professoras valorizam a Formação Continuada e se sentem motivas em participar desses momentos formativos tendo em vista que proporcionam o seu aperfeiçoamento profissional e contribuem com o processo de conscientização da importância de sua prática docente. 

    \textcite{FREIRE1996Pedagogia} colabora com o assunto quando afirma que somos eternos aprendizes e que devemos estar em formação permanente. A perspectiva de ser eterno aprendiz também nos remete à consciência do inacabamento quando Freire afirma que, “na verdade, o inacabamento do ser ou sua inconclusão é próprio da experiência vital. Onde há vida, há inacabamento” \cite[p.~50]{FREIRE1996Pedagogia}, sendo esta característica, própria do ser humano, a constância da aprendizagem. 

    \paragraph{A elaboração do projeto de Formação Continuada da escola} No que se refere à questão quatro (4) as professoras afirmaram desconhecer o Projeto e apenas duas (2) disseram ter participado da sua elaboração, e alegaram que a escola tem gestão democrática.  

    \begin{quotation}
        \noindent\negpar[-1.5em]{}Professora F\quad{}\\Todos foram convidados a colaborar da construção e da realização de uma educação de qualidade para as crianças.  
        \medskip
    
        \negpar[-1.5em]{}Professora E\quad{}\\É importante todos participarem dessa construção apontando os pontos positivos e os eventuais negativos, objetivando a melhoria e o crescimento da escola. 
    \end{quotation}

    Das narrativas das professoras fica a dúvida, houve uma boa divulgação da atividade? A gestão preocupou-se em sensibilizar a sua equipe para elaboração do projeto? As professoras não partícipes ignoram a atividade? Essas questões ficam nebulosas, muito embora, sabemos da importância da coletividade diante as decisões tomadas na instituição de ensino. Como diz \textcite{MORGADO2005Currículo}, os professores são a força propulsora de mudança na escola. A ausência de sua participação poderá comprometera qualidade dos trabalhos realizados. 

    \paragraph{Expectativas da Formação Continuada} Quanto a essa questão três (3) professoras disseram não e três (3) disseram sim. Para não, as professoras justificaram: “A secretaria oferece poucos cursos e em horários não compatíveis com a nossa realidade”. Entretanto, não especificam que realidade é essa. Também: “Às vezes a temática não condiz com a realidade dos CMEIs [\dots] falam de inclusão, mas a clientela não é atendida, por exemplo,”.  

    Sabemos que as discussões que norteiam a formação dos professores, as que acontecem no interior da escola ou fora dela, têm uma importância muito grande quando se trata da elaboração do currículo formal. Observando essas conotações \textcite{OLIVEIRA2013Novas} se aproxima dessas ideias quando fala da dinâmica do trabalho curricular, ou seja, para o autor, não considere o currículo como um produto pré‐estabelecido, mas um processo em que todos da escola participar, pois só assim, o real terá significado, já que as experiências dos professores serão escutadas e consideradas. 

    \paragraph{Contribuições das formações para a elaboração do currículo} Quanto ao questionamento sugerido as Professoras A, D, E e F responderam que sim, conforme os relatos: 

    \begin{quotation}
        \noindent\negpar[-1.5em]{}Professora A\quad{}\\Norteia as ações definindo o que desejamos fazer, como desejamos fazer e com qual finalidade desejamos fazer. 
        \medskip
    
        \negpar[-1.5em]{}Professora D\quad{}\\Ajuda no desenvolvimento do meu trabalho. 
        \medskip

        \negpar[-1.5em]{}Professora E\quad{}\\Os temas abordados são pertinentes, sendo voltada para a realidade existente. 
        \medskip

        \negpar[-1.5em]{}Professora F\quad{}\\Tem temas atuais que embasam a nossa prática durante todo o ano letivo. 
    \end{quotation}

    Conforme a Professora E, o currículo escolar deve estar voltado para a realidade dos alunos e articulado às suas práticas culturais, pois, os contextos social e familiar em que se inserem essas crianças são ricos de saberes e valores. Em \textcite{BARBOSA2009Práticas}, podemos encontrar respaldo para a posição dessa professora, como também para o que coloca a Professora F, quando ele compreende o currículo como práticas culturais que se articula a diversos contextos. 

    \paragraph{Concepção de currículo} As concepções de currículo das professoras se diferem. Concebido como um sistema complexo e aberto que articulam em uma dinâmica interativa diversos fatores o currículo é possui posicionamento político, intencionalidades, contextos, valores, redes de conhecimentos e saberes e aprendizagens \cite{CAVALCANTI2011Currículo}.  

    \begin{quotation}
        \noindent\negpar[-1.5em]{}Professora A\quad{}\\Currículo é o modo de organizar as práticas educativas [\dots] refere-se aos espaços, a rotina, as experiências com as linguagens verbais e não verbais. 
        \medskip
    
        \negpar[-1.5em]{}Professora B\quad{}\\Dados relativos à vida da criança. 
        \medskip

        \negpar[-1.5em]{}Professora C\quad{}\\É muito importante pois é através dele que nos dá condições de oferecermos os conteúdos necessários. 
        \medskip

        \negpar[-1.5em]{}Professora D\quad{}\\Conhecimentos culturais, artísticos, sociais e emocionais, que proporcionamos aos nossos alunos. 
        \medskip

        \negpar[-1.5em]{}Professora E\quad{}\\São as experiências de aprendizagens implementadas pelas escolas de forma que deve contribuir para a formação do aluno de modo crítico. 
        \medskip

        \negpar[-1.5em]{}Professora F\quad{}\\São aspectos ligados a seleção dos conteúdos, mas também referentes aos métodos, procedimentos, técnicas, recursos empregados na educação escolar. 
    \end{quotation}

    Nas respostas das professoras fica clara a compreensão que têm sobre a concepção do currículo escolar. Em essência, o currículo deve contribuir para a total e plena construção da identidade dos alunos e, além disso, deve também estimular as capacidades, as competências, o discernimento e a análise crítica dos alunos, essa fala podemos encontrar dentre os relatos das professoras, a exemplo a Professora E. 

    \textcite{SILVA2010Documentos} ampara essa descrição quando diz que ao se assumir uma postura crítica do currículo amplia-se e modifica-se a visão e a concepção de currículo. Dessa maneira, um currículo com uma visão crítica contribui para a construção das identidades, para o desenvolvimento das potencialidades, da criatividade e da subjetividade dos alunos. 

    \paragraph{A importância de organizar o currículo na Educação Infantil} Quando indagadas sobre essa questão as professoras responderam que: 

    \begin{quotation}
        \noindent\negpar[-1.5em]{}Professora A\quad{}\\É preciso construir uma proposta pedagógica pensando no desenvolvimento integral da criança e observar os aspectos do cuidar, educar e brincar. 
        \medskip
    
        \negpar[-1.5em]{}Professora C\quad{}\\Respeitar a faixa etária do aluno, para que atenda às necessidades básicas da aprendizagem. 
        \medskip

        \negpar[-1.5em]{}Professora D\quad{}\\Proporcionar o contato dos alunos com as diversas linguagens, inserindo-os em um contexto e patrimônio cultural. 
        \medskip

        \negpar[-1.5em]{}Professora E\quad{}\\Utilizar temas atuais que estejam presentes na realidade do aluno com as participações de grupos dos interessados na comunidade.  
        \medskip

        \negpar[-1.5em]{}Professora F\quad{}\\Utilizar temas diversos e a participação democrática de todos os envolvidos. 
    \end{quotation}

    Construir uma proposta pedagógica e curricular para a Educação Infantil exige o entendimento sobre o desenvolvimento integral da criança e os aspectos do cuidar e do educar, além do meio social que a criança está inserida, sua prática social, a família e a estrutura da instituição de ensino, como foi citado na fala da Professora A.  

    \paragraph{Os desafios apontados pelas professoras para executar o currículo} Nesse questionamento as professoras apontaram os principais desafios encontradas ao executar o currículo em sala de aula. Nota-se que são muitas e que por vezes tais desafios atrapalham o processo educativo.   

    \begin{quotation}
        \noindent\negpar[-1.5em]{}Professora A\quad{}\\Falta de interesse dos pais na educação escolar de seus filhos, que atribuem a escola um papel que não é dela. Existe também outros fatores que interferem para gerar dificuldades, como o descaso do poder público, a falta de materiais didático-pedagógicos. 
        \medskip
    
        \negpar[-1.5em]{}Professora C\quad{}\\Falta de material adequado para atender as dificuldades tanto do professor como da criança.  
        \medskip

        \negpar[-1.5em]{}Professora D\quad{}\\Estrutura precária e pouco investimento, ausência de auxiliar de sala. 
        \medskip

        \negpar[-1.5em]{}Professora E\quad{}\\Interesse dos pais, que alegam falta de tempo. Outro fator importante é a parte estrutural do prédio, faltando acessibilidade para alunos cadeirantes [...] profissionais que tenham habilidades nas diversas deficiências (surdez e visão). 
        \medskip

        \negpar[-1.5em]{}Professora F\quad{}\\Estrutura física. 
    \end{quotation}

    Percebemos durante a aplicação do questionário a angústia de muitas professoras, pois os desafios diários da profissão são muitos, e o ato de educar é muito grandioso, pois vai além de passar os conhecimentos e alcançar metas. É preciso que os educadores tenham em mente qual é o seu papel e o que esperam de seus alunos e se realmente querem formar cidadãos.  

    As professoras pesquisadas têm a consciência da importância no desenvolvimento e na educação dessas crianças e que todos se empenham e dão o melhor de si para formar cidadãos críticos/reflexivos empenhados na construção de um mundo mais humano e melhor para todos. Contudo, se isentam da sua participação diante desses desafios, pois em nenhum momento elas tratam das questões pedagógicas.  

    Assim, a elaboração de uma proposta curricular exige compromisso, estudo e planejamento da instituição de ensino, incluindo todos os segmentos que a compreende. Não se esquecendo da família do aluno.


    \section{Considerações finais}

    O estudo em questão teve como objetivo buscar respostas referentes aos principais desafios encontrados pelo professor quanto à organização do currículo na Educação Infantil. 

    Para esse entendimento ficou esclarecido que o currículo é um tema importante e fundamental no debate atual sobre educação, pois deixou de ser apenas um conjunto de disciplinas que constituem um curso de qualquer nível, para denominar a inteira participação da escola no processo da experiência professor-aluno.  

    Como resultado da pesquisa deste trabalho entendemos que o currículo é amplo e diversificado, não pode ser visto como um documento pronto e, está sempre aberto a novas reformulações, já que a cada dia surgem novos assuntos para serem discutidos. Pelo fato de estar se adequando a realidade local, social e econômica na vivência dos alunos, de forma que o currículo tem que estar presente no dia a dia da criança e, da instituição escolar, demonstra que a aprendizagem envolve uma sequência didática que começa desde a Educação Infantil e vai até a formação do educando que envolve o passado, presente e, reflete no futuro do mesmo.  

    Por fim, os relatos apreendidos na pesquisa de campo enriqueceram o trabalho, ficando claro que não adianta só oferecer Formação Continuada para os professores de Educação Infantil se, não são ofertados os meios para que o currículo se concretize em sua totalidade, levando os profissionais de educação ao improviso para alcançar os objetivos do cuidar e do educar com êxito nos resultados esperados pela instituição de ensino, família e sociedade em geral. 

    \nocite{ConstituiçãoBrasil1988}
    \nocite{Parecer15-2017}
    \nocite{PNE2014}
    \nocite{PASCHOALAndMACHADO2009história}

    \printbibliography[heading=subbibliography,notcategory=fullcited]

    \label{chap:curriculo-ed-infantend}

\end{refsection}
