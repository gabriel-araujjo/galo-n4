\begin{refsection}
    \renewcommand{\thefigure}{\arabic{figure}}
    
    \chapterTwoLines
    {O Diário de Natal}
    {o papel da imprensa potiguar na circulação das notícias do Projeto Baixo-Açu (1975--1979)}
    \label{chap:diarionatal}
    
    \articleAuthor
    {Maiara Brenda Rodrigues de Brito}
    {Mestranda em História dos Sertões (UFRN-CERES, Caicó). ID Lattes: 7133.7861.6580.5891. ORCID: 0000-0003-3708-6533. E-mail: maiara.brendaaa@hotmail.com.}

    \begin{galoResumo}
        \marginpar{
            \begin{flushleft}
            \tiny \sffamily
            Como referenciar?\\\fullcite{SelfBritoMBR2021}\mybibexclude{SelfBritoMBR2021}, p. \pageref{chap:diarionatal}--\pageref{chap:diarionatalend}, \journalPubDate{}
            \end{flushleft}
        }
        O presente texto fala do uso da mídia local para a divulgação do Projeto Baixo Açu, no Estado do Rio Grande do Norte. Considerando as possibilidades de análises históricas permitidas pela fonte jornalística, o presente trabalho faz uso do periódico \textit{O Diário de Natal} (1975--1979) para entender algumas informações e discursos que circularam sobre o projeto modernizador, que objetivava combater às secas dos sertões do Estado potiguar. Metodologicamente, o trabalho fundamenta-se na revisão da literatura, de cunho acadêmico e regional sobre o Projeto Baixo Açu, incluindo a seleção de textos sobre modernidade e sertões. Houve a consulta online do periódico \textit{O Diário de Natal} e tudo será analisado via Análise do Discurso.
    \end{galoResumo}
    
    \galoPalavrasChave{Periódicos. Discurso. Modernidade.}
    
    \begin{otherlanguage}{english}
    
    \fakeChapterTwoLines
    {O Diário de Natal}
    {the role of Rio Grande do Norte press in the circulation of news of Baixo-Açu project (1975--1979)}

    \begin{galoResumo}[Abstract]
        This text talks about the use of local media for the dissemination of Baixo-Açu project in the state of Rio Grande do Norte. Considering the possibilities of historical analysis allowed by the journalistic source, the present work makes use of the newspaper \textit{O Diário de Natal} (1975--1979) to understand some information and speeches that circulated about the modernizing project that aimed to combat drought in the backlands of the state of Rio Grande do Norte. Methodologically, the work is based on a literature review of academic and regional nature about the Baixo-Açu project, including the selection of texts about modernity and backcountry. There was an online consultation of the journal \textit{O Diário de Natal} and everything will be analyzed via Discourse Analysis.
    \end{galoResumo}
    
    \galoPalavrasChave[Keywords]{Periodical. Speech. Modernity.}
    \end{otherlanguage}

    \section{Introdução}

    O jornal é um importante meio de comunicação social, pois apresenta diferentes informações ao leitor como: economia, política, educação, anúncios, assuntos populares e policias. Os periódicos foram reconhecidos como fonte histórica a partir de um movimento crítico e intelectual promovido pela Escola do Annales\footnote{\textcite{Lapuente2016Imprensa} aponta que a partir de 1930 a Escola do Annales faz críticas à concepção de que os jornais eram inadequados para o estudo do passado. A terceira geração dessa corrente historiográfica abraçou os novos aportes teóricos e abriu oportunidades para novas contribuições documentais, como os periódicos.}, durante o século XX. No Brasil até 1970, poucos trabalhos utilizavam desse material como fonte para discutir a história nacional. 

    As primeiras experiências do uso dos jornais, enquanto fonte primária, foram marcadas por desconfianças. Havia uma preocupação por parte dos estudiosos no tocante à qualidade do material oferecido por essas fontes. Não sabiam até que ponto esses periódicos sofriam intervenções e pronunciavam os interesses de instituições, grupos econômicos, financeiros e governamentais. 

    O século XX marcou um período de grande mudança para imprensa. O aprimoramento das técnicas permitirá a profissionalização dos jornais e a agilidade na impressão, bem como o barateamento dos mesmos.  Houve modificação na distribuição dos conteúdos, na estruturação das produções e surgiu a necessidade de mão de obra especializada como ``repórteres, desenhistas, fotógrafos, articulistas, redatores, críticos, revisores, além dos operários encarregados da impressão propriamente dita'' \cite[p.~138]{Luca2005Historia}. Em síntese, os aspectos físicos dos jornais foram modificando-se ao longo do tempo. A estruturação e divisão de conteúdo, as relações com a publicidade, mercado e público também são mutáveis, tendo em vista os objetivos a serem alcançados. Desta forma, vemos que as ``condições materiais e técnicas em si são dotadas de historicidade, mas que se engatam a contextos socioculturais específicos'' \cite[p.~138]{Luca2005Historia}.

    Com o avanço do tempo, verifica-se a importância dos jornais para o entendimento de novos objetos de estudos. Temos a apropriação de textos literários para a abordagem de assuntos sociais diversos, inclusive para veiculação de ideia de luta. Além de ``gênero, etnia, raça, identidade, modos de vida, experiência e prática políticas cotidianas, formas de lazer'' \cite[p.~119--120]{Luca2005Historia}, esse recurso servia como ``veículo privilegiado para divulgar seus manifestos'' \cite[p.~125]{Luca2005Historia}, fossem esses políticos ou sociais. Verificamos que é necessário associar cada material e produção ao seu contexto. Assim como é obrigatório problematizar aspectos como os recursos econômicos e tecnológicos disponíveis.  

    Considerando as possibilidades de análises históricas permitidas pela fonte jornalística, optamos pelo uso do periódico potiguar \textit{O Diário de Natal} para entender algumas informações que circularam sobre um evento que ocorreu no interior do Rio Grande do Norte, a implantação do Projeto Baixo Açu.  

    O Projeto Baixo-Açu foi uma política pública modernizadora\footnote{A ideia de política pública modernizadora dentro deste estudo, parte das discussões de \textcite{Andrade2007Caico}. Esta pensa as intervenções no espaço citadino de Caicó, cidade localizada no interior do Rio Grande do Norte, por meio da eletricidade, automóvel, imprensa. Em período correlato que Caicó buscava transforma-se, inspirado em grandes centros urbanos nacionais e internacionais, presenciava a permanecia de velhos costumes dado pelas secas e presença dos flagelados. Dessa forma, a modernização pensada para o Baixo-Açu é aquela que traduz o combate contra às secas no Nordeste por meio de intervenções espaciais realizadas pelos órgãos IOCS, IFOCS, DNOCS.  }, direcionada para a microrregião do Vale do Açu\footnote{A microrregião do Vale do Açu, fica localizada na mesorregião Oeste Potiguar e é composta por nove municípios: Assú, Alto dos Rodrigues, Itajá, Ipanguaçu, Jucurutu, Pendências, Porto do Mangue e São Rafael.}, interior do Estado do Rio Grande do Norte, que objetivava amenizar as consequências dos ciclos de estiagens do interior do estado. A ideia da construção da barragem Armando Ribeiro Gonçalves, remonta a primeira metade do século XX, quando em 1937, o atual DNOCS autorizou o início de estudos para a identificação de um espaço apropriado para tal obra. Em 1971, o vale do Açu foi o espaço indicado para a instalação da barragem. No entanto, o projeto Baixo-Açu saiu do papel somente em 13 de julho de 1975, com o Decreto de número 76.046, durante o governo do presidente Ernest Geisel. 

    Dividido em três etapas, essa ação implicou no barramento do leito do rio Piranhas-Açu\footnote{O rio Piranhas/Açu, nasce na Serra do Bongá, município de Bonito de Santa Fé, Estado da Paraíba, e desemboca no município de Macau, litoral do Rio Grande do Norte.  Seus principais afluentes são: rios Espinhara, Picuí e Seridó. }.A primeira fase do projeto, consistia na construção da barragem Engenheiro Armando Ribeiro Gonçalves, que atualmente é o maior reservatório artificial do estado. A segunda etapa caracterizaria a implementação de um programa de irrigação, e a terceira seria a instalação de um espaço voltado para atividades pesqueiras nas águas represadas. 

    A construção da barragem foi realizada entre os anos de 1979 e 1983, a sua inauguração contou com presença de consideráveis políticos do Estado e com o presidente da República da época, João Batista de Figueiredo\footnote{\textcite{Souza2011Teias} e \textcite{Pinheiro2018Vale} apontam o referido recorte de tempo, para a construção da barragem.}. Como notabilizava o projeto, algumas cidades foram atingidas e tiveram comunidades rurais inundadas, é o caso de Jucurutu, Açu e Ipanguaçu. A cidade de São Rafael, além de sofrer com a inundação dos espaços rurais, vivenciou a deslocamento do seu núcleo urbano. A execução desse projeto trouxe efeitos múltiplos aos indivíduos e espaços afetados, havendo assim a necessidade de compreender fatos específicos que tangem esse contexto que marcado por profundas discursões e incertezas.   

    Baseando-se na Análise do Discurso, o referido trabalho analisa publicações de cunho acadêmico e regional que investigam a temática Projeto Baixo Açu. O objetivo dessa pesquisa é pensar a circulação das notícias sobre essa obra e entender que tipo de discurso\footnote{Conceito discutido a partir da obra de \textcite{Foucault2002Ordem}. O texto retrata uma aula inaugural realizada no dia 2 de dezembro de 1970 no Collége da França. Fala sobre o discurso, como esse é construído, interpretado e repassado dentro de uma instituição e/ou meio social. O autor aponta que o discurso é um elemento marcado por controle, seleção, organização e distribuído por meio de procedimentos que procura conservar seus interesses e fugir dos perigos. Também lembra que esses procedimentos de interdição estão relacionados ao desejo e poder.} era veiculado pelo periódico \textit{O Diário de Natal}. Também analisaremos as ideias modernizadoras lançadas para o Vale do Açu, buscando entender que tipo de sertão\footnote{O conceito de Sertão será discutido a partir de reflexões e autores como \textcite{Amado1995Regiao}; \textcite{Moraes2003Sertao} e \textcite{Neves2003Sertao}, e versará sobre os aspectos simbólicos e ideológicos, desenvolvido ao longo do tempo, que pensa esse termo como categoria espacial, referenciando regiões e espaços marcados pela pobreza, seca e carentes de intervenções modernizadoras.} era apresentado por esse jornal.

    \section{Desenvolvimento}

    Desde do início do século XX, o Vale do Açu foi objeto de estudos\footnote{Em seus estudos, \textcite{Pinheiro2018Vale} aponta que em 1910 o IOCS publicou as investigações do geólogo e engenheiro Roderic Crandall sobre o Ceará, Rio Grande do Norte e Paraíba. Esse documento referência a bacia Piranhas-Açu e caracteriza um dos primeiros estudos sistemáticos, sob a responsabilidade do IOCS que falou sobre o Vale do Açu. O engenheiro Crandall apontou aquele espaço como um lugar potencialmente irrigável. Afim de entender cientificamente os sertões, a IOCS também contratou o engenheiro hidrólogo Geraldo A. Warring. Este percorreu o semiárido entre 1910 e 1912 e apontou duas áreas irrigáveis na bacia do rio Açu, a primeira ao longo do curso do próprio rio Açu e o outro no vale que segue aos longos dos rios Piranhas e do Peixe. } que investigavam as condições do solo semiárido brasileiro. Procuravam espaços que possibilitassem intervenções hídricas e que ajudasse no combate à seca, questão essa que ainda é considerado um dos maiores problemas nacionais, com destaque para a região Nordeste. Identificado a condição do solo favorável ao aproveitamento da agropecuária e reconhecido como um celeiro econômico no interior no estado\footnote{O reconhecimento do Vale do Açu, como um espaço propício para intervenções técnicas, foi inicialmente apontado pelos estudiosos da IOCS, posteriormente retomada na década de 1930 pelos técnicos da IFOCS e por fim, pelos profissionais do DNOCS.},  o Vale do Açu passou a ser alvo de políticas públicas ao longo do tempo. 

    Em 1910 a ``Inspetoria de Obras Contra as secas (IOCS)\footnote{Criada no contexto do governo de Nilo Peçanha (1909-1910), a IOCS foi transformada em 1919 em Inspetoria Federal de Obras Contra as Secas (IFOCS) e posteriormente no de 1945 em Departamento Nacional de Obras Contra as Secas (DNOCS).} publicou o primeiro relatório técnico que incluía discussões a respeito da Várzea do Açu. A ideia, entretanto, só viria a tomar corpo na década de 1940, com o projeto de construção da barragem de Oiticica'' \cite[p.~15]{Pinheiro2018Vale}. O projeto da barragem de Oiticica foi germinado no final da década de 1930. Os estudos retomados na região do Baixo Vale do Açu, objetivava definir o local para a construção de um grande reservatório que auxiliasse na disciplinaridade do rio Piranhas. Essa intervenção implicaria na garantia de água para todo o ano, combatendo assim o período de estiagem, a violência das enchentes durante o inverno e incitaria a produção da agricultura baseada na irrigação. Em 21 de outubro de 1954 foi publicado um Decreto Presidencial de nº 36. 370, o mesmo indicava a desapropriação de uma área de 143.063.500 m\textsuperscript{3} no município de Jucurutu, para a construção de uma barragem.  

    O projeto da barragem de Oiticica foi interrompido ``quer por falta de verbas, quer pela mudança de abordagem do DNOCS, ou até pela falta de empenho dos gestores públicos, e talvez por esses e mais outros conjugados, mas o interesse pela irrigação do Vale do Açu nunca foi de todo abandonado'' \cite[p.~130]{Pinheiro2018Vale}. Diante dos fatos, vimos outro surgir, o Projeto Baixo Açu e como já mencionado, este previa a construção da barragem Armando Ribeiro Gonsalves, próximo da cidade de Açu. Direcionado ao Vale do Açu, este projeto foi uma política pública que ficou marcado por um discurso de caráter modernizador. O mesmo objetivava amenizar as consequências dos ciclos de estiagens do interior do estado e a construção do lago artificial que aconteceu entre 1979 e 1983.  

    Retomando a fonte principal do nosso estudo, o jornal \textit{O Diário de Natal}, vemos que os discursos que rodeiam o Projeto Baixo Açu, tocam os interesses de instituições e carregam a carga simbólica construída para o sertão nordestino ao longo do tempo, no tocante ao combate do atraso da região. A partir deste periódico, analisaremos quais as principais notícias que circularam no Rio Grade do Norte, durante o período de 1975 até 1980.

    \textit{O Diário de Natal} é um jornal potiguar que foi fundado em 1939. Inicialmente intitulado de ``O Diário'', o mesmo teve como idealizadores Rivaldo Pinheiro, Waldemar Araújo, Aderbal França e Djalma Maranhão. A partir de 1945, o periódico passou a integrar o Diário Associados S/A\footnote{Diário Associados S/A é uma empresa de mídia da impressa fundada na década de 1920 no Brasil pelo jornalista Assis Chateaubriand. A trajetória desse grupo teve início quando Chateaubriand adquiriu o impresso O jornal. Este circulava no Rio de Janeiro.} e em 1947 teve o nome modificado para ``Diário de Natal''.  O jornal circulou por mais 70 anos e informava sobre fatos diversos. Por muito tempo, foi considerado uma grande referência editorial para diferentes áreas como administração pública, política, economia, artes. O Periódico foi extinto em 2012, em razão da nova realidade jornalística dada pela emergência da internet\footnote{Informação extraída do site: \url{http://ftp.editora.ufrn.br/handle/123456789/1456?subject_page=10}. Acesso em 29 out. 2020.}.

    Acerca do projeto Baixo Açu, o periódico trouxe notícias diversas. Essas tangiam todos os aspectos e falas dos envolvidos nesse evento que marcou a história da açudagem do Rio Grande do Norte. Intitulado ``DNOCS termina projeto para irrigar Piranhas''\footnote{Acervo Digital da Biblioteca Nacional [BNDigital] --- DNOCS termina projeto para irrigar piranhas. Diário de Natal, Natal. p. 5. 9 jan. 1975.} a matéria do ano 1975, expõe aos seus leitores o projeto em estudo. Constatamos a apresentação dos objetivos a serem alcançados. A notícia decorre sobre o tempo de estudo, as ações que serão realizadas pelo DNOCS, o tempo do processo e os benefícios oferecidos à população, como emprego. A empregabilidade anunciada paira pelo processo de execução do projeto e pelas atividades econômicas projetadas, como a agricultura irrigada e o setor da psicultura. 

    No mesmo ano, é anunciado a implantação de seis grandes projetos para o Estado que relacionava açudagem à agricultura irrigada. A edição 09870(1)\footnote{BNDigital --- DNOCS implantará seis grandes projetos no RN. Diário de Natal, Natal. p. 5. 19 jul. 1975.} afirmavam que todas as atividades pensadas para as cidades de Cruzeta, Pau dos ferros, São João do Sabugi, Apodi, São Rafael e Ceará-Mirim estariam concluídas no intervalo de quatro anos de 1975 até 1979. Focando no nosso assunto principal que é o Projeto Baixo Açu, afirmamos que a execução do mesmo extrapolou a data prevista, tendo em vista que a Nova São Rafael foi concluída no ano de 1983. Vale lembrar que a realocação desse espaço citadino estava envolvida no projeto principal. Os resultados destes, almejavam valores expressivos:

    \begin{quotation}
        [\dots]A barragem do Baixo-Açu terá 17 mil hectares irrigados, com um volume de 2,3 bilhões de metros cúbicos d'água --- duas vezes a capacidade do Itans [\dots] O projeto Cruzeta está em conclusão no Vale de Piranhas [\dots] O assentamento é de 24 famílias, devendo produzir 910 toneladas de banana, 156 de laranja, 62 de alho, 312 de tomate, 195 de cebola 105 de feijão, 281 de milho, 232 de oleaginosas, 20 de carne e 31 toneladas de leite. O projeto Ceará-Mirim [\dots] Serão irrigados 12 mil hectares para hortaliças, cereais e pastagens. O Projeto Itans-Sabugi está em fase final de implantação e será concluído em março de 1976. Vai ter uma área irrigada de 1.026 hectares, destinado a produzir 938 toneladas de uva, 805 de bananas, 138 de laranja, 355 de tomate, 1076 de cebola. O projeto Pau dos Ferros produzirá 222 toneladas de uva, 1860 de banana, 1040 de cebola, 260 de laranja, 1040 de tomate, 650 de cebola, 1768 de milho, num total de 8.938 toneladas de gêneros [\dots]\footnote{BNDigital --- DNOCS implantará seis grandes projetos no RN. Diário de Natal, Natal. p. 5. 19 jul. 1975.}
    \end{quotation}

    O Baixo Açu foi implantado durante um contexto político marcado por discurso desenvolvimentista, que apontava para eficiência técnica e administração moderna.  Como fruto desses pensamentos surgiram uma série de programas e planos para a sociedade brasileira, entre eles destacamos o II Plano Nacional de Desenvolvimento\footnote{Informação extraída de: \url{http://www.fgv.br/cpdoc/acervo/dicionarios/verbete-tematico/plano-nacional-de-desenvolvimento-pnd}. Acesso em 30 out. 2020.}. Implantado em 1975, este previa que até 1980, a sociedade brasileira alcançasse um patamar de industrialização e modernidade. Defendia que o desenvolvimento da sociedade, estaria atrelado à uma política de emprego, projetando o recebimento de salários, a elevação consumo e aumento da economia. A qualificação da mão de obra seria dada por meio de educação e treinamento profissional. Entre tantas ideias apresentadas e sensíveis ao nosso estudo, a integração do Brasil ao mercado mundial para a exportação de produtos manufaturados e primários, saltam os nossos olhos. Os produtos frutos dos sistemas de irrigação seria para o consumo nacional e internacional, escoado para fora do país, por meio desta relação econômica prevista. 

    A implantação desse projeto de açudagem no Rio Grande do Norte, re\-la\-ci\-o\-na-se com o discurso da problemática da seca no Nordeste. Tema sensível desse local, que possui a caatinga como vegetação predominante e está inserida no contexto do clima semiárido. Pensar o sertão dessa região, é permear numa categoria espacial, marcada por construções simbólicas e ideológicas. Durante o século XIX, ``sertão'' assumiu duas significações, ``um associado a ideia de semiárido; outra priorizando atividades econômicas e padrões de sociabilidade, articulado à pecuária'' \cite[p.~155--156]{Neves2003Sertao}. Ambos sentidos, traduzia uma ideia espacial e passou a referenciar espaços do interior, desérticos e pouco habitado.  

    Por caracterizar-se pela construção de um imaginário, o sertão reúne ``um conjunto de juízos e valores adaptáveis a diferentes discursos e a distintos projetos'' \cite[p.~3]{Moraes2003Sertao} intervencionistas, sobretudo aqueles que defendem a superação da ``condição sertaneja'' \footnote{Ideia exposta por \textcite{Moraes2003Sertao} em discursão do texto ``O sertão: um outro geográfico''.}. Desta forma, Morais afirma que, ``o sertão é qualificado para ser superado'' \cite[p.~4]{Moraes2003Sertao}. Diante do exposto, vemos que a Barragem Armando Ribeiro Gonçalves traduzia essa ideia de superação da seca do Nordeste. 

    O uso da imprensa escrita é fonte importante para o estudo da sociedade, como também para identificar os discursos que evocam o sertão. Em 1976, na edição 10148(1)\footnote{BNDigital --- Diretor do DNOCS vai dizer tudo que será feito no vale do Açu. Diário de Natal, Natal. p. 8. 24 set. 1976.} a implantação do setor de irrigação no Vale do Açu é novamente noticiada. A repetição da informação denota a complexidade do projeto que estava em andamento, mas também revela a necessidade de potencializar um espaço fértil no sertão, o Vale do Açu. Desse modo, o Projeto Baixo-Açu reflete a ideia de modernização que os técnicos tinham para o sertão nordestino, que era a açudagem junto ao setor de irrigação. 

    \begin{quotation}
        Paralelamente a construção da barragem, o DNOCS vai implantar um projeto de irrigação com a finalidade se quadruplicar a produção hort-fruti-granjeira do Vale do Açu --- um dos maiores vales secos mais produtivos do Nordeste. Ontem à noite em Açu, técnicos do DNOCS e da Secretaria de Trabalho explicaram os objetivos do projeto aos agricultores da região. O projeto denominado ``Baixo-Açu'' --- desapropriação de terras, construção de barragem e implantação de um programa de irrigação-está orçado em Cr\$ 720 milhões, segundo informações do ministro do Interior, Sr. Rangel Reis, quando esteve pela última vez no Estado. Será um dos maiores projetos do Governo Federal na região nordestina.\footnote{Ibidem.}
    \end{quotation}

    O sertão é um lugar assinalado por um conjunto de discursos. Esses elementos de anunciações, são marcados por controle, seleção, organização e é distribuído por meio de procedimentos que procuram conservar seus interesses e fugir dos perigos. Esses procedimentos de interdições, estão relacionados ao desejo e poder. Por vezes, esses mecanismos são perceptíveis na mídia, pois os jornais se adequam ao contexto histórico social no qual está inserido. Intitulada ``Ulisses critica projeto do Açu e pede mudanças''\footnote{BNDigital --- Ulisses critica projeto do Açu e pede mudanças. Diário de Natal, Natal. p. 13. 1 nov. 1977. }, em 1977 vemos um posicionamento crítico a esse projeto, essa notícia evidenciou no Diário de Natal pontos negativos que atingiria parte da população.  

    Ulisses Potiguar\footnote{Ulisses Bezerra Potiguar, nasceu em Parelhas (RN) e foi deputado federal entre 1975 até 1979. Filiou-se à Aliança Renovadora Nacional (Arena), após contexto da extinção dos partidos políticos através do Ato Institucional n° 2 e instauração do bipartidarismo. Informações extraídas do site: \url{http://www.fgv.br/cpdoc/acervo/dicionarios/verbete-biografico/ulisses-bezerra-potiguar}. Acesso em 2 nov. 2020.}, na época deputado do Rio Grande do Norte e membro da Arena\footnote{Partido intitulado de Aliança Renovadora Nacional (Arena). O mesmo era de caráter conservador e de sustentação ao regime militar brasileiro, que foi instalado no país em abril de 1964. Informações extraídas do site: \url{http://www.fgv.br/cpdoc/acervo/dicionarios/verbete-biografico/ulisses-bezerra-potiguar}. Acesso em 2 nov. 2020.}, proferiu um discurso na Câmara Federal, mostrando-se contrário ao modo como estava acontecendo a implantação do Projeto Baixo Açu.  Sua fala evidenciou problemas sociais graves que ocorreriam com o implante do Baixo Açu, caso não houvesse reformulações. Este dizia que parte do espaço afetado seria transformado em minifúndios, ``seguramente antieconômicos''. 

    Ulisses Potiguar defendia mudanças no projeto como ``o aproveitamento dos reservatórios já existentes na região: Mendubim, Piató, Ponta Grande, com 400 mil metros cúbicos, além do grande lençol freático, na extensão de 30 quilômetros em centenas de poços em funcionamento com 16 mil metros cúbicos horários''. O mesmo afirmou que essas revisões implicariam na redução de uma série de problemas como: destruição de fruteiras, carnaubeiras, erosão do solo, destruição de cidades e habitações, entre outros.  Vale pensar que o discurso do deputado em análise é significativo aos estudos que envolvem a temática do Projeto Baixo Açu, pois parte das problemáticas previstas por ele, são pontos de partida para entender reflexos negativos deste projeto. Vejamos: 

    \begin{quotation}
        Com esta reivindicação aceita, afirmam os colonos, evitaria um problema social mais extenso; destruição de cidades e habitações, construídas em padrões modernos; a destruição de milhares de fruteiras já produzindo e exportando somas vultuosas e a erradicação de milhões de carnaubeiras, produzindo anualmente cerca de um milhão e 200 mil quilos de cera de quatro tipos, exportando, criando divisas, dando emprego durante seis meses, acima do salário local a 60 por cento do vale. A medida evitaria também problemas de erosão; deslocamento sem destino de 30 mil cabeças de animais, bovinos, ovinos, caprinos, cavalares  e muares; desabrigo a 70m por cento da população, acima de 70 50 anos de idade e abaixo de 18, nos dois sexos; submersão  das jazidas de mármore e o prejuízo aos direitos humanos. Ulisses Potiguar concluiu seu pronunciamento na Câmara afirmando não possuir propósitos contestatórios: ``estou demostrando apenas minha viva repulsa a esses métodos e princípios, esperando que as reivindicações dos meus conterrâneos sejam atendidas pelas autoridades responsáveis desses Projeto tão danoso aos interesses as regiões.''\footnote{BNDigital --- Ulisses critica projeto do Açu e pede mudanças. Diário de Natal, Natal. p. 13. 1 nov. 1977.}
    \end{quotation}

    Os jornais possuem relação de poder. Eles estão imbricados com o público e por isso são formadores de opinião pública. Por esse motivo, nem sempre apresentam todas as informações que envolvem determinados fatos. Em abril do ano 1978\footnote{BNDigital --- Concorrência para barragem do Açu será esta semana. Diário de Natal, Natal. p. 4. 11 jul. 1978. } foi divulgado a concorrência para a construção da primeira etapa para execução do Projeto Baixo Açu. Segundo as notícias, várias firmas nacionais e internacionais estavam concorrendo à vaga. ``A firma vencedora da concorrência pública, arcará com financiamento total da obra, cuja barragem terá altura máxima de quarenta metros, inundando quarenta mil hectares de terra.''\footnote{Ibidem.}

    Os procedimentos que envolveram o concurso para a construção do reservatório foram suspensos por um ``mandado de segurança impetrado pelas construtoras Empresa Industrial Técnica S/A (EIT) e Queiroz Galvão, eliminadas do julgamento na segunda etapa.''\footnote{BNDigital --- Barragem do Açu: Conclusão da concorrência agora depende da justiça. Diário de Natal, Natal. p. 4. 19 jul. 1978.} Segundo o noticiário, essas teriam sido eliminadas ``na abertura dos envelopes contendo proposta técnicas e, insatisfeitas com o resultado, que beneficiou e a firma mineira Andrade Gutierez.''\footnote{Ibidem.} Esse evento implicou no retardamento do início da obra, pois:

    \begin{quotation}
        De acordo com as declarações de João Batista Marques de Souza, diretor geral adjunto do Departamento Nacional de Obras Contra as Secas (DNOCS), responsável pela obra, não existia ainda uma previsão para a conclusão da concorrência, vez que a comissão de licitação está aguardando a decisão judicial, já tendo enviado respostas à perguntas formuladas pelo juiz. As construtoras foram eliminadas na abertura dos envelopes contendo proposta técnicas e, insatisfeitas com o resultado, que beneficiou e firma mineira Andrade Gutierez, resolveram unir-se, impetrando o mandado de segurança. Para João Batista M Marques de Souza, realmente a firma mineira seria beneficiada com a eliminação antecipada de suas concorrentes, afirmando: ``pelo menos a idéia que se tem aqui é esta''. Enquanto o problema não se resolve, diminui cada vez mais a possibilidade de a obra ser construída dentro do prazo previsto.\footnote{BNDigital --- Concorrência para barragem do Açu será esta semana. Diário de Natal, Natal. p. 4. 11 jul. 1978. }
    \end{quotation}

    O ano de 1979 foi marcado por diversos debates que envolveu a construção do reservatório. Parte desses questionamentos e resistência, saíram de grupos defensores dos indivíduos diretamente afetados pelas realocações e indenizações, em virtude do alagamento da área habitada. A igreja católica do Rio Grande Norte foi um dos principais órgãos atuantes nessa luta, sobretudo nas causas que envolviam a cidade de São Rafael\footnote{A cidade de São Rafael foi totalmente realocada no ano 1983. A cidade foi totalmente construída e era citada no jornal como Nova São Rafael.}, espaço citadino totalmente realocado em virtude da construção da barragem.  

    Na edição 10649(1)\footnote{BNDigital --- DNOCS reafirma que barragem vem mesmo. Diário de Natal, Natal. p. 7. 13 jan. 1979.}, o chefe do 1º Distrito de Engenharia Rural do DNOCS, Carlos Queiroz afirmou que a construção do reservatório era irreversível e que ``com Mario Andreazza à frente do Ministério do Interior ninguém vai impedi-la''. Mas uma vez, constatamos a dominação sobre esse espaço, pois ``ultrapassar a condição sertaneja é a meta implícita dos discursos que buscam levantar e explicar a sua essência'' \cite[p.~4]{Moraes2003Sertao}.

    Nesse ano podemos ver o tamanho da força e interesse dos grupos que estavam à frente da implantação do Baixo Açu.  Em entrevista, Carlos Queiroz apontou que o governo federal estaria presente na obra, por meio de uma comissão executiva nomeada pelo Ministro do Interior, Mario Andreazza. Esse grupo seria formado pelo ``engenheiro Eldan Veloso (presidente) Carlos Queiroz, (substituto do presidente) e outro engenheiro do DNOCS, [que] montaria um escritório em Natal''\footnote{Ibidem.}, afim de facilitar o trabalho da comissão que seria fiscalizar e supervisionar a obra. Desse modo, o \textit{Diário de Natal} também noticiou a formação de uma comissão executiva para as obras da Barragem Armando Ribeiro Gonçalves e pontuou as novas estimativas de custos para a realização da mesma. Vejamos: 

    \begin{quotation}
        [\dots] uma comissão executiva nomeada pelo Ministro do Interior, Mario Andreazza,  em entrevista coletiva realizada na tarde de ontem, no 4º andar do Edifício Café Filho, nas Rocas, Carlos Queiroz acrescentou ainda que dentro de 30 dias deverá ser dado início à construção da Barragem Armando Ribeiro Gonsalves, a cargo da empresa ``Andrade Guiterez'', de Minas Gerais. O prazo de conclusão é de três anos e custo de estimativo fica de 1 bilhão e 400 milhões de cruzeiro   incluído o projeto de irrigação, colonização e indenização das terras, desapropriadas pelo DNOCS. 

        Uma comissão executiva, nomeada pelo Ministro do Interior, estará em Natal na próxima semana para supervisionar e fiscalizar a execução do projeto de irrigação no Rio Grande do Norte. A comissão terá todos os poderes sobre o andamento das obras da Barragem Armando Ribeiro Gonsalves e uma verba de 400 milhões de cruzeiro já está liberada para o início dos trabalhos[\dots]\footnote{BNDigital --- DNOCS reafirma que barragem vem mesmo. Diário de Natal, Natal. p. 7. 13 jan. 1979.}
    \end{quotation}

    A circulação das notícias sobre as etapas da execução do Projeto Baixo-Açu, era de suma importância para todos os envolvidos. Pois informava a população sobre os trâmites da obra e auxiliava na ratificação de ideias como a de superação dos problemas das secas por meio de intervenções técnicas. Desse modo, as notícias também serviram para evacuar discursos sobre a referida obra.  

    A construção do discurso manifesta a necessidade de mostrar uma verdade, e a propagação do mesmo é conduzido por forças, interesses e instituições. Assim, é necessário atentar para o espaço de atuação dos periódicos, considerando aspectos como contextos sociais e políticos que envolvem o Estado e o País.  Dessa forma, Foucault (2002) nos lembra que a política é um tema sensível, e o modo como o mesmo é abordado, pode gerar fortes impactos no âmbito social. Considerando a temática em estudo, vemos que  as notícias dadas pelo \textit{Diário de Natal} sobre o Projeto Baixo-Açu, causavam ebulição no meio social, sobretudo no  Vale do Açu, espaço diretamente afetado pelas intervenções.

    \section{Considerações finais}

    A mídia transforma fatos sociais em notícias. O profissional dessa área é responsável por organizar seu espaço de fala atendendo critérios como: contexto sociopolítico, postura política, ética profissional e o principal, seu público alvo, o leitor. Para todo modo, os jornais e os seus profissionais, exercem papeis fundamentais na sociedade, pois apresentam e discutem os mais variados temas que tocam o cotidiano das pessoas.  

    Os jornais impressos foi um dos veículos de comunicação mais utilizados pela sociedade, durante o período estudado. Através do \textit{Diário de Natal}, a população potiguar teve acesso as notícias diversas a respeito da execução do Projeto Baixo-Açu. Além dos interesses políticos, a população pôde manifestar as suas dificuldades e frustações no decorrer da execução da mesma.   

    Enquanto fonte histórica para a temática vimos que foi possível detectar os discursos que tocam o sertão, enquanto uma categoria simbólica, imaginária e geográfica, carregada de estereótipos. O sertão do Rio Grande do Norte, especificamente a região do Vale do Açu, era um espaço fértil dentro do espaço potiguar, que passou por estudos para ser potencializado. Projetaram assim, a superação da condição sertaneja de espaços circundantes ao rio Piranhas-Açu através da açudagem e perenização do curso da água do mesmo.  Assim, a condição discursiva desses enredos configurava-se por apresentar uma estrutura moderna no sertão potiguar.

    \nocite{Pesavento2007Cidades}
    \nocite{Silva2009Influencia}
    \nocite{Souza2010Escafandristas}
    \nocite{Weber2012Metodologia}

    \printbibliography[heading=subbibliography,notcategory=fullcited]

    \hfill Recebido em 15 mar. 2021.

    \hfill Aprovado em 15 abr. 2021.

    \label{chap:diarionatalend}

\end{refsection}
