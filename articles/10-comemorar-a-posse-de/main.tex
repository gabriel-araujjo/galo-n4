\begin{refsection}
    \renewcommand{\thefigure}{\arabic{figure}}

    \chapterTwoLines
    {Comemorar a posse de Thomaz de Araújo}
    {a construção de um lugar para o Seridó na memória histórica do Rio Grande do Norte}
    \label{chap:comemorar}
    
    \articleAuthor
    {Bruno Balbino Aires da Costa}
    {Doutor em História pelo Programa de Pós-graduação em História da Universidade Federal do Rio Grande do Sul (PPGH/UFRGS). Professor do Instituto de Educação, Ciência e Tecnologia do Rio Grande do Norte (IFRN), campus Canguaretama. ID Lattes: 6237.2531.8338.2621. ORCID: 0000-0003-3538-182X. E-mail: bruno.aires@ifrn.edu.br.}

    \begin{galoResumo}
        \marginpar{
            \begin{flushleft}
            \tiny \sffamily
            Como referenciar?\\\fullcite{SelfCosta2021}\mybibexclude{SelfCosta2021}, p. \pageref{chap:comemorar}--\pageref{chap:comemorarend}, \journalPubDate{}
            \end{flushleft}
        }
        O Instituto Histórico e Geográfico do Rio Grande do Norte foi criado no início do século XX, com o objetivo precípuo de construir um lugar para o estado na memória nacional. Uma das estratégias utilizadas pelos sócios da referida agremiação para a consecução desse projeto foi o de organizar e promover diversos atos comemorativos. Uma das datas celebradas foi o primeiro centenário da posse constitucional do capitão Thomaz de Araújo Pereira, no cargo de primeiro presidente da província. Nesse sentido, o presente artigo tem como objetivo analisar as condições de emergência desse ato comemorativo levado a cabo pela agremiação, evidenciando a topografia de interesses envolvidos nessa engenharia memorialística. Parte-se da hipótese de que os elementos políticos presentes no interior do instituto, e, também, fora dele, foram fundamentais para a conformação desse arranjo comemorativo.
    \end{galoResumo}
    
    \galoPalavrasChave{Instituto Histórico e Geográfico do Rio Grande do Norte. Memória. Comemoração.}
    
    \begin{otherlanguage}{english}
    
    \fakeChapterTwoLines
    {Celebrating the inauguration of Thomaz de Araújo}
    {the building of a place for Seridó in the historical memory of Rio Grande do Norte}

    \begin{galoResumo}[Abstract]
        The \textit{Instituto Histórico e Geográfico do Rio Grande do Norte} was created in the early twentieth century, with the primary objective of building a place for the state in national memory. Among the strategies used by that institute for the achieve their goal was to organize and promote various commemorative acts. One of the dates celebrated was the first centenary of the constitutional inauguration of Captain Thomaz de Araújo Pereira as the first president of the province. In that sense, this article aims to analyze the emergency conditions of that commemorative act carried out by the association, showing a topography of interests involved in that memorialistic engineering. We start with the hypothesis that the political elements present in the institute, and outside, were fundamental on shaping this commemorative arrangement.
    \end{galoResumo}
    
    \galoPalavrasChave[Keywords]{Instituto Histórico e Geográfico do Rio Grande do Norte. Memory. Commemoration}
    \end{otherlanguage}

    \section{Introdução}

    Thomaz de Araújo Pereira nasceu na região do \textit{Seridó}, mais precisamente, no atual município de Acari em 1765, e ali faleceu em 1847 \cite[p.~815]{Lyra1921Historia}. Descendia de um dos primeiros povoadores que --- vindos da Borborema, na Paraíba, no começo do século XVIII --- povoaram a região de Acari \footcite[p.~180]{DiscursoManoealDantas}. Seu avô, o português, Thomaz de Araújo Pereira, o primeiro dos três homônimos, fundou a fazenda de \textit{São Pedro} em Acari, formando uma numerosa descendência que, mais tarde, consolidou-se como uma elite agrária do sertão do Rio Grande do Norte \cite[p.~53]{Macedo2012Penultima}. Thomaz de Araújo Pereira, o neto, adveio dessa elite rural do sertão norte-rio-grandense, a qual, via de regra, compôs a própria aristocracia política da região \cite[p.~53]{Macedo2012Penultima}. Como corolário do status econômico e político da sua família, Thomaz de Araújo Pereira foi investido com uma patente militar das milícias, tornando-se, em 1799, tenente, e promovido, posteriormente, a capitão-mor da \textit{Primeira Companhia de Cavalaria de Ordenança da Vila do Príncipe}, hoje município de Caicó, em 1806 --- ``itinerário social comum à linhagem rica dos fazendeiros seridoenses.'' \cite[p.~54]{Macedo2012Penultima}.

    Em 3 de dezembro de 1821, Thomaz de Araújo Pereira foi eleito como um dos membros da primeira Junta governativa da província do Rio Grande do Norte \cite[p.~224]{Lyra1907Notas}. Com a organização do estado nacional, logo após a Independência do Brasil, Thomaz de Araújo Pereira foi nomeado presidente da província em 25 de novembro de 1823. Todavia, o político seridoense adiou o quanto pôde a cerimônia de sua posse, o qual se realizou apenas em 5 de maio de 1824.\footnote{``Dependeria de um bom inverno a posse do primeiro presidente da Província do Rio Grande do Norte. Para empreender a longa marcha --- em torno de 60 léguas --- a cavalo até a capital, o provecto fazendeiro da ribeira do Acauã condicionava a viagem a Natal ao volume abundante de capim para suas montarias.'' \cite[p.~51]{Macedo2012Penultima}.} Seu governo foi fugaz, durou apenas cinco meses. Tal efemeridade estava diretamente relacionada ao cenário político muito turbulento na província do Rio Grande do Norte desde a \textit{Revolução de 1817}. As forças políticas da província estavam frequentemente em rota de colisão devido às ferrenhas disputas pelo poder. Apesar do intento do presidente de província em promover a estabilidade da ordem pública, Thomaz de Araújo Pereira não conseguiu amenizar as desavenças entre os grupos políticos da província, levando as próprias tropas de linha ignorarem sua autoridade. Consoante Tavares de Lyra, três meses depois do início do seu governo, o batalhão de linha depunha na sua frente, o seu commandante, João Marques de Carvalho, nomeado a 19 de fevereiro de 1824: ``esse acto era o prenuncio de maiores e mais lamentaveis perturbações.'' \cite[p.~240]{Lyra1907Notas} Não conseguindo dominar a anarquia, Thomaz Pereira de Araújo demitia-se da presidência da província em 8 de setembro de 1824, retirando-se para Acari. \cite[p.~240]{Lyra1907Notas}

    A despeito da efemeridade do governo de Thomaz Pereira de Araújo e de sua pouca notabilidade política na presidência da província, o \textit{Instituto Histórico e Geográfico do Rio Grande do Norte} (IHGRN) decidiu comemorar o centenário da sua posse. Diante disso, uma questão imperiosamente se coloca: o que explica o interesse dos sócios do IHGRN em rememorar a posse de Thomaz Pereira de Araújo? O presente artigo tem como objetivo analisar as condições de emergência desse ato comemorativo levado a cabo pela agremiação, evidenciando a topografia de interesses envolvidos nessa engenharia memorialística. Parte-se da hipótese de que os elementos políticos presentes no interior do IHGRN, e, também, fora dele, foram fundamentais para a consecução desse arranjo comemorativo.
    
    \section{O significado político da comemoração}

    O IHGRN foi criado em 29 de março de 1902, com o objetivo precípuo de construir a memória histórico do Rio Grande do Norte.\footnote{Cf. \fullcite{Costa2017Casa}} Para isso, lançou mão de três estratégias: a escrita da História, a biografia e a comemoração. Para atender aos interesses específicos deste artigo, dedicarei, apenas, a essa última.   

    A comemoração estava na ordem do dia do IHGRN. Segundo o estatuto da instituição, especificamente, em seu artigo 59, capítulo 10, cabia aos sócios do sodalício ``solemnizar qualquer data historica'' \cite[p.~22]{EstatutosIHGRN1903}. Ao longo primeiros 25 anos de sua existência, o IHGRN organizou a comemoração dos centenários de nascimento de Duque de Caxias e de D. Pedro II, o 89º e 100º aniversários do fuzilamento de Frei Miguelinho e os centenários da \textit{Revolução Republicana}, da Independência nacional e a posse do presidente Thomaz de Araújo. De certa forma, cada ato comemorativo refletia um conjunto de interesses específicos que atendia as demandas políticas e sociais requeridas pelas instituições governamentais do Rio Grande do Norte e do Brasil \footnote{Cf. \fullcite{Costa2017Casa}.}. Com a comemoração da posse de Thomaz de Araújo não foi diferente.

    No dia 13 de abril de 1924, o presidente do IHGRN, Pedro Soares de Araújo, resolvera convocar uma sessão extraordinária para deliberar acerca da comemoração do 1º centenário da posse constitucional do capitão Thomaz de Araújo Pereira, no cargo de primeiro presidente da província do Rio Grande do Norte \cite[p.~264]{ActaSessao1925}. Para justificar a comemoração, Nestor Lima considerava que o Rio Grande do Norte, a exemplo de outros estados, deveria celebrar o início de sua existência constitucional como parte integrante da nação brasileira \cite[p.~265]{ActaSessao1925}. É preciso ressaltar que rememorava-se a data de criação da primeira Constituição do país, outorgada pelo imperador D. Pedro I, em 1824, apesar de tê-lo feito de modo autoritário, e que essa celebração fez parte da ``onda comemoracionista'' que invadiu o país, especialmente, na década de 20, com os eventos comemorativos dos centenários da Independência do Brasil em 1922, e o do natalício de D. Pedro II, em 1925. No início da década de 1920, a memória imperial já não representava mais uma ameaça ao regime republicano \cite{Rodrigues2013Releitura}. Nesse período, nenhum intelectual e/ou político cogitava a pertinência de uma restauração monárquica: ``No Brasil de 1922, o Império era uma nostalgia, jamais um projeto.'' \cite[p.~339]{Enders2014Vultos}. O fim do banimento da família imperial e a transladação dos restos mortais do imperador D. Pedro II e de Tereza Christina para o país, nos anos 20, significavam, ao mesmo tempo, que a memória monárquica não representava mais risco algum e o regime republicano poderia, de agora em diante, reintegrar o passado monárquico à memória nacional, ``fortalecendo, simbolicamente o próprio ideário republicano.'' \cite[p.~193]{Sandes2000Invencao}.

    Com efeito, a celebração da posse de Thomaz de Araújo Pereira tinha uma certa ligação com o comemoracionismo em torno da Constituição de 1824, mas não era apenas isso. Para os sócios do IHGRN, a questão dizia respeito a algo que ia além da semântica da representação política de 1824.  

    O interesse em comemorar a posse constitucional de Thomaz de Araújo possuía um significado simbólico importante para o IHGRN, uma vez que a celebração era uma forma de rememorar a origem do estado como uma unidade federativa independente, já que antes da emancipação do Brasil, o Rio Grande do Norte era uma capitania submetida a Paraíba, juridicamente, e a Pernambuco, economicamente e politicamente. Para todos os efeitos, comemorar o primeiro governo constitucional do Rio Grande do Norte e a posse do seu primeiro presidente significavam celebrar sua autonomia política. \cite[p.~176]{DiscursoManoealDantas}. Contudo, a comemoração possuía também um significado político.

    Na década de 20, políticos do \textit{Seridó}, especificamente, José Augusto Bezerra de Medeiros e Juvenal Lamartine\footnote{José Augusto Bezerra de Medeiros nasceu em 22 de setembro de 1884 no atual município de Caicó-RN. Em 1903, José Augusto Bezerra de Medeiros bacharelou-se pela \textit{Faculdade de Direito do Recife} (FDR), ocupando cargos públicos de Procurador da República, Fiscal de Governo Federal, diretor do Atheneu Norte-Rio-Grandense, Juiz de direito da comarca de Caicó, chefe de Política Interino e Secretário de estado no governo de Ferreira Chaves. Exerceu, ainda, mandatos na política estadual, na condição de deputado federal de 1913 a 1923, senador da república de 1928 a 1930 e na governadoria do Estado entre 1924 a 1927. Além de político, José Augusto era um intelectual. Escreveu vários livros, tomando como tema central o Seridó. O seu companheiro político, Juvenal Lamartine de Faria nasceu em 9 de agosto de 1874, no município de Serra Negro do Norte. Estudou no Atheneu-Norte-Rio-Grandense e graduou-se em direito pela FDR, em dezembro de 1897. Foi professor de geografia e vice-diretor do Atheneu Norte-Rio-Grandense em 1898. Exerceu vários cargos na magistratura pública e na política do estado. Foi juiz de direito, vice-governador do Rio Grande do Norte (1904--1906), deputado federal (1906), senador da república (1927) e governador do estado (1928--1930). Assim como José Augusto, Juvenal Lamartine escreveu vários livros sobre o Seridó, tornando-se, ao lado de Manoel Dantas, José Augusto, Oswaldo Lamartine, um dos grandes intelectuais que tomaram a referida região como objeto de estudo. \cite{MedeirosNeta2007Serido}.}, ascenderam ao governo do estado. Desde o início dos anos 10, os \textit{coronéis do Seridó} já despontavam como forças políticas em ascensão no Rio Grande do Norte, representando, inclusive, a principal contraposição à oligarquia Albuquerque Maranhão. \cite[p.~209]{Macedo2012Penultima}. Esta emergência dos \textit{coronéis do Seridó} no cenário político estadual foi possível graças ao enriquecimento das elites agrárias da região ligadas à produção algodoeira.   

    A Primeira Guerra Mundial possibilitou um reordenamento na economia do Rio Grande do Norte. Nas décadas de 10 e 20, o maior volume de riqueza do estado provinha do setor da cotonicultura, sendo a região do \textit{Seridó} a principal produtora do algodão do Rio Grande do Norte \cite[p.~50]{Spinelli1996Getulio}. O crescimento econômico da cotonicultura implicou diretamente no fortalecimento político dos coronéis da região, doravante, interessados em assumir a liderança do governo estadual.  

    Em 1913, o grupo dos Albuquerque Maranhão articulava-se para, mais uma vez, indicar um candidato que estivesse diretamente ligado aos interesses políticos da oligarquia. Todavia, as lideranças políticas seridoenses, reunidas em torno dos deputados José Augusto Bezerra de Medeiros e Juvenal Lamartine, opuseram a articulação orquestrada pela família Albuquerque Maranhão, não mais aceitando incondicionalmente a indicação proposta pelo último governador da oligarquia, Alberto Maranhão \cite[p.~208]{Macedo2012Penultima}. José Augusto e Juvenal Lamartine orquestraram um acordo dos \textit{coronéis do Seridó} em torno da candidatura de Joaquim Ferreira Chaves, contrariando a indicação dos Albuquerque Maranhão \cite[p.~208--209]{Macedo2012Penultima}. Apoiado pela elite econômica e política seridoense, Ferreira Chaves ganhou as eleições de 1913, administrando o estado entre 1914 a 1920, pondo fim a chefia dos Albuquerque Maranhão no governo estadual. Com a vitória de Chaves, o grupo político seridoense dava sinais claros de sua expressão no cenário político estadual.  

    Terminado o seu governo, Ferreira Chaves conseguiu emplacar a candidatura do seu sucessor, Antônio José de Mello e Souza, vitorioso no pleito de 1920. Todavia, a vitória não representou a consolidação do grupo de Ferreira Chaves no poder político do estado. Em 1923, o ex-governador não conseguiu dar continuidade as suas pretensões políticas no governo executivo. Nesse ano, as lideranças seridoenses articularam-se junto a Arthur Bernardes, o apoio à candidatura de José Augusto para o governo do Rio Grande do Norte, sepultando as pretensões de Ferreira Chaves de mais um mandato \cite[p.~20]{Spinelli1996Getulio}. Apoiado pelo presidente da República e pela coalisão de lideranças políticas do Seridó, José Augusto Bezerra de Medeiros fora eleito em 1923, inaugurando, ainda que por um tempo curto, a chefia seridoense no governo do Rio Grande do Norte.\footnote{José Augusto Bezerra de Medeiros governou o Rio Grande do Norte entre 1924 e 1927. Posteriormente, conseguiu eleger o seu sucessor político, Juvenal Lamartine que governou o estado entre 1928 e 1930, tendo sido deposto do poder devido à Revolução de 1930.}

    Assim como o grupo Albuquerque Maranhão, José Augusto Bezerra de Medeiros utilizou-se do passado como uma forma de legitimação política. A comemoração do centenário da posse de Thomaz de Araújo Pereira é um exemplo disso. Em outras palavras, a celebração da posse do primeiro presidente da província do Rio Grande do Norte representou um uso político do passado.

    \section{A comemoração da posse de Thomaz de Araújo Pereira e os usos políticos do passado}

    A comemoração do centenário teve como principal patrocinador o governo do estado do Rio Grande do Norte que, como em outras ocasiões, delegou ao IHGRN a tarefa de organizá-la.\footnote{Além do governo do estado, a Intendência do município de Natal também colaborou com o festejo. \cite[p.~172--173]{CentPosseTAPereira}.} É preciso acrescentar, ainda, que a celebração contou com as expensas da intendência municipal de Natal, a qual era governada pelo letrado e político seridoense, Manoel Dantas. Os poderes executivos do estado e da capital, dirigidos por seridoenses, estavam comprometidos em agenciar a celebração do centenário de posse do ancestral político do Seridó.  

    A solenidade contou com uma sessão magna, realizada pelo IHGRN, no salão nobre do Palácio do governo, e com a afixação de uma placa de bronze, contendo o nome da \textit{praça Thomaz de Araújo} e as datas de 1824 e 1924 \cite[p.~172--173]{CentPosseTAPereira}. As expensas da placa ficaram por conta do poder público estadual e a nomeação do antigo largo fronteiro ao quartel do exército para \textit{praça Thomaz de Araújo Pereira} ficou a cargo da intendência municipal da capital do estado \cite[p.~172--173]{CentPosseTAPereira}. A inauguração da placa reuniu autoridades políticas do Rio Grande do Norte e da capital, bem como representantes religiosos, militares e o povo, um ato de ``grande romaria cívica'', segundo os sócios do instituto \cite[p.~174]{CentPosseTAPereira}. Como de praxe, após o desencerramento da bandeira, um membro do IHGRN ficava responsável pelo pronunciamento do discurso em alusão à celebração. O vice-orador da agremiação, Nestor Lima, encarregou-se dessa empresa.  

    De antemão, Nestor Lima explicava aos seus ouvintes qual seria o enfoque do seu discurso: ``devo fazer aqui tão somente a justificação do motivo por que é este o local escolhido para guardar o nome e, mais tarde, o monumento do valoroso patriarca seridoense'' \cite[p.~194]{CentPosseTAPereira}. Segundo Nestor Lima, a praça havia sido palco de um levante, ocorrido em setembro de 1824, contra o presidente da província, Thomaz de Araújo Pereira. Não conseguindo dissuadir a tropa de linha, o político seridoense resolveu renunciar o poder em setembro daquele ano, voltando para a terra do seu berço, Acari \cite[p.~194]{CentPosseTAPereira}. É interessante notar que a praça é considerada como um marco não de luta, mas sim de renúncia. O que se destaca é a resignação de Thomaz de Araújo Pereira em defender o seu posto político. Para Nestor Lima, era dessa atitude do presidente de província que os norte-rio-grandenses deveriam rememorar: ``Foi na recordação desse gesto de desprendimento que a Intendencia de Natal, atendendo ao appello do Instituto Historico, deu o nome de «Thomaz de Araujo» á praça em que nos achamos'' \cite[p.~195]{CentPosseTAPereira}.

    Com efeito, o discurso de Nestor Lima tinha como escopo reabilitar a imagem de Thomaz de Araújo Pereira severamente criticada por Augusto Tavares de Lyra. Em seu artigo \textit{Algumas notas sobre a história política do Rio Grande do Norte}, publicado, em 1907, pela revista do IHGRN, Tavares de Lyra havia afirmado que Thomaz de Araújo Pereira não era o nome mais indicado para governar a província naquela ocasião, em grande medida, por causa da sua idade avançada, da sua cegueira e das ``ligações politicas que tinha, fazendo-o partidario intransigente, eram condições que contribuiam para não ser elle o preferido naquella quadra de paixões exaltadas, de odios e de desejos de desforras'' \cite[p.~240]{Lyra1907Notas}. Para Tavares de Lyra, a figura de Thomaz de Araújo Pereira era impotente para promover a estabilidade da ordem pública na província \cite[p.~240]{Lyra1907Notas}. Somado a isso, a força armada, as tropas de linha, sobrepunha-se à lei e a autoridade constituída. Nesse sentido, a ação de Thomaz de Araújo de Pereira foi lida por ele não como um ato de desprendimento, mas de anulação do seu próprio poder por uma força que era maior do que a autoridade nele investida. Para Tavares de Lyra, em vez de resignação, Thomaz de Araújo Pereira demitiu-se do cargo por querer fugir das responsabilidades ``que lhe adviriam de uma situação que se aggravava e que não podia remediar'' \cite[p.~240]{Lyra1907Notas}. A interpretação de Tavares de Lyra parece indicar que o primeiro presidente, além de inapto para o cargo, havia agido por um ato de covardia ou de medo. Esta leitura de Tavares de Lyra foi reforçada em seu livro \textit{História do Rio Grande do Norte}, publicado em 1921. Neste livro, especificamente, no capítulo 7, intitulado \textit{Acontecimentos que precederam e se seguiram á Independencia. Juntas Governativas. --- Confederação do Equador. --- Posse e governo do primeiro Presidente}, Augusto Tavares de Lyra reproduzia uma \textit{tradição oral} que supostamente afirmava que Thomaz de Araújo havia se ausentado de Natal dentro de um barril que fez transportar à cabeça de um escravo --- o qual conduziu-o até um lugar em que estaria a salvo e em condições de utilizar-se de um transporte em direção a Acari --- depois de sofrer algumas ameaças de índios de Extremoz ou de uma família chamada \textit{Matta-quiri}: ``essa tradição pode e deve ser verdadeira'' \cite[p.~533]{Lyra1921Historia}. A oralidade é convocada para provar o argumento de Tavares de Lyra de que Thomaz de Araújo receava o encontro com aqueles grupos na capital da província, antes mesmo de tomar a decisão de deixar a presidência. Mais uma vez, a narrativa de Tavares de Lyra parece sugerir que Thomaz de Araújo Pereira tinha uma tendência a capitulação em situações que lhe traziam alguma ameaça iminente. É possível que foi a partir dessa imagem de Thomaz de Araújo construída por Tavares de Lyra que Nestor Lima intentou desconstruir. Em contraposição ao possível pusilânime ou medroso, Nestor Lima conferiu ao primeiro presidente da província a imagem de abnegado. 

    Pela primeira vez, os sócios do IHGRN laureavam não a luta ou a vitória de um personagem norte-rio-grandense, mas a sua abnegação. Com o intento de tornar sagrada essa memória materializada na praça, Nestor Lima comparou-a ao gólgota, onde Jesus Cristo havia padecido, e a estátua de Tiradentes na cidade de Ouro Preto. Desde o início da República era comum construir um imaginário sagrado aos heróis republicanos. Não é por acaso que a imagem de Tiradentes esteve associada à de Cristo \cite{Carvalho1990Formacao}. O gesto de Nestor Lima é parecido com os republicanos dos primeiros anos do novo regime. A preocupação era semelhante: sacralizar a memória. Para Nestor Lima, o desprendimento de Thomaz de Araújo era equivalente ao ato do ``martírio cruento da cruz'' e a exposição da cabeça de Tiradentes em praça pública. A cruz de Cristo e a estátua de Tiradentes seriam a materialização sacra da memória dos dois mártires. Semelhantemente, a praça \textit{Thomaz de Araújo Pereira} rememoraria o ato sacrificial do ``brio do tradicional político sertanejo'' \cite[p.~533]{Lyra1921Historia}. O sacrifício do primeiro presidente de província do Rio Grande do Norte não consistia na sua morte em favor do povo norte-rio-grandense ou por um ideal, mas sim no custo do seu brio, de seu gesto de não resistir aos seus opositores. Nesse sentido, a inauguração da praça em decorrência da comemoração da posse do presidente de província era um ato de justiça para com a sua memória, um cumprimento de um dever dos rio-grandenses do norte do presente para com o seu passado \cite[p.~533]{Lyra1921Historia}. A comemoração estava associada ao dever de memória e ao ato de justiça do presente em relação ao passado. Dessa forma, o dever de memória é o dever de fazer justiça, pela lembrança, ao passado, isto é, a um outro que não a si \cite[p.~101]{Ricoeur2007Memoria}.

    Além do discurso de Nestor Lima, o IHGRN empreendeu mais uma alocução em homenagem ao centenário de posse de Thomaz de Araújo Pereira. Na sessão magna, o discurso ficou a cargo de Manoel Dantas. Além de ser o orador oficial da agremiação, Manoel Dantas possuía outras credenciais que o encaminhavam para a tarefa. O sócio do IHGRN era um letrado seridoense comprometido com a produção do saber sobre a região do \textit{Seridó} e do homem sertanejo.\footnote{Cf. \fullcite{Macedo2012Penultima}; e \fullcite{MedeirosNeta2007Serido}.} Thomaz de Araújo Pereira era uma personagem importante da memória seridoense, nesse aspecto, falar sobre ele era invocar um tipo representativo da própria memória histórica da região, uma vez que o avô do primeiro presidente da província tinha contribuído para a colonização e povoamento do Seridó. Além disso, é preciso citar que Manoel Dantas, José Augusto Bezerra de Medeiros e Juvenal Lamartine, enredados por lanços de parentescos, descendiam da família de Thomaz de Araújo Pereira \cite[p.~29]{MedeirosNeta2007Serido}. Assim, havia todo um interesse por parte desses três sócios do IHGRN de celebrar o centenário da posse do primeiro presidente da província. Comemorar a posse de Thomaz de Araújo era evidenciar o lugar do homem seridoense na construção da memória histórica do Rio Grande do Norte. É por esse motivo que Manoel Dantas, ao dirigir-se ao governador José Augusto Bezerra de Medeiros em seu discurso, fez questão de destacá-lo como ``descendente de Thomaz de Araujo, o primeiro filho da zona do Seridó que preside os destinos do estado no regime republicano.'' \cite[p.~177]{DiscursoManoealDantas}. Há uma clara associação entre o primeiro presidente da província e o primeiro governador seridoense a governar o estado. É aqui que percebemos nitidamente o uso político do passado. José Augusto Bezerra de Medeiros é colocado como um laço de continuidade entre o passado e o presente, evidenciando, a contribuição seridoense na própria história política do Rio Grande do Norte. Dessa forma, o poder político do presente era respaldado pela evidência histórica do passado, o qual apontava para a ancestralidade do governador seridoense. Em um momento de emergência dos políticos seridoenses no cenário político do estado, nada mais legitimador e simbólico do que mostrar a continuidade do passado no presente. A comemoração ganha uma significação importante, nesse processo de legitimação política. 

    Antes de tratar propriamente do objeto da celebração, Manoel Dantas esclarece aos convidados e aos seus consócios do IHGRN que o seu discurso não obedecia ao rigor histórico \cite[p.~177]{DiscursoManoealDantas}. O orador do Instituto estabelece, então, a diferença entre o discurso comemoracionista e o texto historiográfico. A distinção estabelecida por ele é simples: o trabalho historiográfico consiste no uso de documentos e do expediente da pesquisa para falar sobre um determinado acontecimento histórico. A comemoração trata do passado, mas, sem necessariamente, estar preso ao rigor do texto historiográfico. Manoel Dantas esclarece ao seu auditório: ``fica esta illustre assemblèa privada de ouvir e julgar um estudo rigorosamente histórico.'' \cite[p.~176]{DiscursoManoealDantas}. Manoel Dantas deixa claro que apesar da falta de pesquisa e documentos e do rigor do texto historiográfico, o seu texto comemoracionista não pretendia se enquadrar no domínio da fantasia \cite[p.~175]{DiscursoManoealDantas}. Em outros termos, Manoel Dantas evidenciava que seu discurso trataria do passado de maneira superficial, mas não ficcional. 

    Para Manoel Dantas, a comemoração do centenário da posse de Thomaz de Araújo considerava por um lado, a importância do fato em si e, por outro, a do indivíduo que o personificou \cite[p.~178]{DiscursoManoealDantas}. Isso significa dizer que a comemoração tratava da emergência do governo constitucional da antiga província e como esta foi possível a partir da personalidade do seu primeiro presidente. Consoante o orador, as divergências e o acirramento dos grupos políticos do Rio Grande do Norte, logo após a organização do estado nacional, levaram o governo imperial a nomear Thomaz de Araújo como o primeiro presidente da província.  Segundo Manoel Dantas, tal nomeação foi devida à personalidade e ao caráter de Thomaz de Araújo, demonstrados na eleição anterior para a junta governativa da província, em meados de 1823. Dessa maneira, a personalidade do político seridoense garantiu que o governo do Império pudesse conferir ao Rio Grande do Norte o seu primeiro presidente de província. É, nesses termos, que o fato em si e o indivíduo estavam diretamente entrelaçados. Contudo, Thomaz de Araújo Pereira não é somente o primeiro presidente da província do Rio Grande do Norte. Mais do que isso, ele é a evidência da determinação benéfica dos homens do \textit{Seridó} nos negócios públicos do Rio Grande do Norte \cite[p.~178]{DiscursoManoealDantas}. Aqui encontra-se o elemento central do discurso de Manoel Dantas, qual seja, construir o lugar para o Seridó na elaboração da memória histórica do estado. Além disso, Thomaz de Araújo Pereira representava a constituição \textit{sui generis} do povo seridoense.  

    Conforme Manoel Dantas, o núcleo de povoamento do \textit{Seridó} foi um dos últimos a ser formado no Rio Grande do Norte. Como já foi mencionado, o avô de Thomaz de Araújo Pereira teria sido um dos seus fundadores \cite[p.~178]{DiscursoManoealDantas}. Nesse processo de formação, os habitantes do Seridó teriam estabelecido um contato maior com a Paraíba e Pernambuco, o que redundou na adoção de hábitos mais pacíficos, desconhecendo as rivalidades da família que dariam origem ao \textit{cangaceirismo}, segundo Manoel Dantas. O contato com as capitanias vizinhas possibilitou uma certa cultura intelectual em relação às outras áreas do alto sertão. Manoel Dantas destaca a formação intelectual e liberal de alguns homens do sertão, especialmente, os padres, que se entrincheiraram nas revoluções \cite[p.~182]{DiscursoManoealDantas}. Conforme o orador oficial do Instituto, Thomaz de Araújo Pereira formara o seu caráter em contato com estes homens. Apesar de inculto, o primeiro presidente da província era um homem que tinha visão de instrução e de progresso: ``Tal era o homem, a quem o Governo do Imperio confiou a primeira presidencia do Rio Grande do Norte.'' \cite[p.~182]{DiscursoManoealDantas}.

    Depois de laurear a personalidade de Thomaz de Araújo Pereira, Manoel Dantas empenha-se em reabilitar a imagem do seu ancestral. A hesitação demonstrada por ele nas questões políticas relativas à província é explicada pela sua personalidade, isto é, pelo emprego do seu bom senso diante de um cenário político totalmente hostil, ao qual o Rio Grande do Norte se encontrava \cite[p.~184]{DiscursoManoealDantas}. O malogro da administração do primeiro presidente apontado por Augusto Tavares de Lyra é ligeiramente justificado por Manoel Dantas pela própria dificuldade inerente à instabilidade política da província e pelo \textit{modus operandi} com que geria o Rio Grande do Norte \cite[p.~185]{DiscursoManoealDantas}. Todavia, para o orador, o seu insucesso administrativo na presidência da província não apagara os atos sociais que realizara na zona do \textit{Seridó} \cite[p.~187]{DiscursoManoealDantas}. Ao contrário de Tavares de Lyra, Manoel Dantas concluía o seu discurso reforçando que Thomaz de Araújo, a despeito da falta de grandes feitos e importantes melhoramentos para a província, havia estabelecido um governo forte, másculo \cite[p.~193]{DiscursoManoealDantas}. Ora, a caracterização do governo de Thomas de Araújo Pereira como sendo uma expressão da sua virilidade ou masculinidade demonstra o diálogo de Manoel Dantas com a produção discursiva em torno da figura do nordestino, tipo regional esse que estava sendo gestado na década de 1920 \cite[p.~207--209]{AlbuquerqueJr2013Nordestino}. Assim como os discursos elaborados nesse enredo do nordestino, Manoel Dantas constrói uma imagem da personalidade de Thomaz de Araújo Pereira a partir dos elementos que identificariam esse tipo regional, apresentando-o como um homem imerso em uma sociabilidade tradicional e, acima de tudo, marcado pelos atributos masculinos. Thomaz de Araújo seria a encarnação da senilidade e da virilidade do homem público do Rio Grande do Norte, por essa razão que a leitura de sua postura pusilânime ou vacilante deveria ser desconstruída. Afinal de contas, o primeiro presidente da província representava o seridoense, expressão do tipo sertanejo.

    \section{Considerações finais}

    A comemoração do centenário da posse de Thomaz de Araújo demonstra o interesse dos seridoenses em construir um lugar para região na memória histórica do Rio Grande do Norte --- que estava sendo gestada no final do século XIX e início do século XIX. No limiar da República, letrados e políticos norte-rio-grandenses também se preocuparam em urdir narrativas que instituíssem um lugar para o Rio Grande do Norte na elaboração da memória nacional. O interesse por essa questão fez parte das estratégias políticas do grupo familiar que ascendeu ao governo do estado, no momento da Proclamação da República: os Albuquerque Maranhão --- liderados por Pedro Velho.\footnote{Cf. \fullcite{Bueno2002Visoes}; e \fullcite{Souza2008Republica}.} No final do século XIX e início do XX, a família Albuquerque Maranhão concebeu e mobilizou estratégias discursivas para a produção da identidade histórica, territorial e étnica do Rio Grande do Norte.\footnote{Para compreender as estratégias espaciais das elites norte-rio-grandenses do início do século XX, conferir: \fullcite[p.~13--36]{Peixoto2012Especialidades}.} Contudo, esse projeto identitário estava sendo disputado por três grupos familiares que exerciam uma espécie de domínio político em diferentes regiões do estado, a saber: Mossoró, Natal e o Seridó.\footnote{Ibidem p.~34.} No entanto, é o grupo político dos Albuquerque Maranhão que elabora, a partir da centralidade da cidade de Natal, a narrativa em torno do que seria a identidade histórica e espacial norte-rio-grandense, a despeito da existência de outras produções concorrentes, oriundas das classes políticas e intelectuais de Mossoró e do Seridó. Isso significa dizer que, assim como a memória nacional, a memória histórica potiguar estava em disputa.   

    A ala seridoense do Instituto, formada pelos sócios: José Augusto, Manoel Dantas, Juvenal Lamartine, encampou um projeto de construir um lugar para o Seridó na memória histórica norte-rio-grandense. Os discursos comemoracionistas mostram o movimento de desconstrução da própria historiografia produzida pelo IHGRN no início do século XX, mais especificamente, ao texto de Tavares de Lyra sobre Thomaz de Araújo Pereira. Nos anos de 1920, ele encontrava-se longe das atividades do IHGRN. Tavares de Lyra nunca se manifestou quanto à reabilitação da imagem de Thomaz de Araújo Pereira. Não houve qualquer debate em torno da figura do primeiro presidente de província. Os políticos seridoenses puderam, sem mais problemas, construir sua versão sobre um dos seus personagens históricos. A comemoração organizada pelo IHGRN foi uma ótima oportunidade para realizar tal empreendimento, afinal, a Instituição possuía um outro \textit{mecenas}, pela primeira vez, um governador oriundo do sertão norte-rio-grandense. Essa circunstância política era uma ocasião perfeita para se instituir uma outra narrativa para o passado do Rio Grande do Norte, dessa vez, destacando um lugar central para o Seridó na memória histórica do estado.

    \nocite{Spinelli2010Coroneis}

    \printbibliography[heading=subbibliography,notcategory=fullcited]

    \hfill Recebido em 31 mar. 2021.

    \hfill Aprovado em 16 abr. 2021.

    \label{chap:comemorarend}

\end{refsection}
