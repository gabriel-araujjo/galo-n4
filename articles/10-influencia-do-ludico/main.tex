\begin{refsection}
    \renewcommand{\thefigure}{\arabic{figure}}
    
    \chapterOneLine
    {A influência do lúdico no processo de aprendizagem na Educação Infantil}
    \label{chap:influencia-ludico}

    \articleAuthor
    {Conceição de Maria Gomes da Silva}
    {Graduada em Pedagogia pelo pela Universidade Vale do Acaraú --- UVA. Discente do Curso de Especialização em Educação Infantil do Instituto de Educação Superior Presidente Kennedy --- IFESP. Atualmente é professora no Centro de Educação Infantil (CMEI), Padre João Perestrello, Natal/RN. ID Lattes: 3471.0687.0512.7779. ORCID: 0000-0002-4457-8766. E-mail: conceicaomaria09@hotmail.com.}
    
    \articleAuthor
    {Nednaldo Dantas dos Santos}
    {Licenciado e bacharel em Ciências Biológicas; mestrado em Ciências Biológicas (UFRN), doutorado em Ciências da Saúde (UFRN). Pós-Doutor em Desenvolvimento de Produtos Nanotecnológicos. Professor formador Instituto de Educação Superior Presidente Kennedy (IFESP). Pós-doutorado no Programa de Pós-Graduação em Ciências da Saúde com o desenvolvimento de Produtos Nanotecnológicos na UFRN (2013). Atualmente é professor formador do Instituto de Educação Superior Presidente Kennedy (IFESP). ID Lattes: 3538.9403.5975.4089. ORCID: 0000-0003-2617-7261. E-mail: nednaldo@ifesp.edu.br.}
    
    \begin{galoResumo}
        \marginpar{
            \begin{flushleft}
            \tiny \sffamily
            Como referenciar?\\\fullcite{SelfSilvaAndSantos2021influência}\mybibexclude{SelfSilvaAndSantos2021influência}, p. \pageref{chap:influencia-ludico}--\pageref{chap:influencia-ludicoend}, \journalPubDate{}
            \end{flushleft}
        }
        Na Educação Infantil, a utilização de metodologias ativas em aula, possibilita compreender o desenvolvimento da criança pela forma e pela linguagem lúdica específicas da infância. É essencial, nesta fase, conhecer o significado de para interpretar o universo lúdico e reconhecer os elementos básicos da ludicidade, pelos quais a criança se comunica com o seu mundo pessoal e com o outro. O presente estudo teve o objetivo expor a influência de atividades lúdicas no desenvolvimento integral da criança que se encontra na Educação Infantil. A análise se deu com uma metodologia de abordagem quantitativo/qualitativa; com objetivos exploratórios e explicativos tendo como foco o procedimento documental. Os resultados apontaram que os diversos projetos pedagógicos desenvolvidos, no período de 2017 a 2019, possibilitaram uma interação e envolvimento, nas atividades propostas aos discentes matriculados, em turmas da Educação Infantil da Rede Municipal de Educação de Natal/RN. Os projetos pedagógicos, com abordagem ativa, demonstraram um potencial lúdico entre os discentes da unidade de ensino pesquisada. 
    \end{galoResumo}
    
    \galoPalavrasChave{Projetos. Crianças. Ensino.}
    
    \begin{otherlanguage}{english}

    \fakeChapterOneLine
    {The influence of play on the learning process in Early Childhood Education}

    \begin{galoResumo}[Abstract]
        In Early Childhood Education, the use of active methodologies in the classroom makes it possible to understand the child's development through the form and playful language specific to childhood. It is essential, at this stage, to know the meaning of in order to interpret the playful universe and recognize the basic elements of playfulness, through which the child communicates with their personal world and with others. The present study aimed to expose the influence of playful activities on the integral development of children who are in Kindergarten. The analysis took place with a quantitative/qualitative approach methodology; with exploratory and explanatory objectives focusing on the documental procedure. The results showed that the various pedagogical projects developed from 2017 to 2019 enabled interaction and involvement in the activities proposed to students enrolled in Early Childhood Education classes of the Municipal Education Network of Natal/RN. The pedagogical projects, with an active approach, demonstrated a playful potential among the students of the researched teaching unit. 
    \end{galoResumo}
    
    \galoPalavrasChave[Keywords]{Projects. Kids. Teaching.}
    \end{otherlanguage}


    \section{Introdução}

    A Educação Infantil é uma etapa essencial da Educação Básica, é a fase que antecede a entrada da criança no Ensino Fundamental. As crianças que frequentam essa etapa têm, em média, até 06 (seis) anos de idade. Ela é oferecida em creches na faixa etária até 3 (três) anos, sendo ofertada em pré-escolas para crianças de 4 a 6 anos de idade (BRASIL, 1996).  

    Essa etapa da educação básica é embasada na Lei de Diretrizes e Bases da Educação Nacional (LDB 9394/96) que apresenta um olhar mais específico para o atendimento às crianças menores de 06 (seis) anos. Compreende-se que a criança, enquanto ser atuante e capaz de construir seu próprio conhecimento, tem seu processo de desenvolvimento mais significativo a partir das vivências que lhe são propostas justamente nessa etapa do ensino. 

    As vivências na Educação Infantil permitem um maior acesso “ao campo de possibilidades para a imaginação, a criatividade, o desenvolvimento cognitivo e corporal, o reconhecimento da identidade do estudante e a interação social” \cite[p.~128]{CANDA2004Aprender}. De fato, esta etapa é um período em que as crianças são estimuladas, principalmente através de atividades lúdicas e jogos, a exercitar suas capacidades físicas, motoras, cognitivas e emocionais. 

    A compreensão sobre as vivências das crianças, permite uma ação do docente, mais direcionada a contribuir com os caminhos no processo de desenvolvimento destes, possibilitando a construção de novos saberes nesta etapa. Porém, para que isso ocorra, é essencial um novo olhar sobre a criança e suas especificidades, de forma a constituir uma prática docente mais verdadeira e significativa. 

    Nesse sentido, a importância das atividades lúdicas é destacada por \textcite{CARVALHO1992Brincadeira}, o qual afirma que: 

    \begin{quotation}
        [\dots] o ensino absorvido de maneira lúdica, passa a adquirir um aspecto significativo e afetivo no curso do desenvolvimento da inteligência da criança, já que ela se modifica de ato puramente transmissor a ato transformador em ludicidade, denotando-se, portanto, em jogo \cite[p.~28]{CARVALHO1992Brincadeira}. 
    \end{quotation}

    No entanto, nem sempre a educação infantil foi vista como uma etapa importante dentro da educação básica. Antes de estar garantida por lei, ela não tinha um caráter pedagógico. \textcite{BACELAR2009Ludicidade}, relembra o atendimento assistencialista do início das atividades da educação infantil no Brasil, e aponta que, à medida que esta foi se expandindo, foram surgindo perspectivas que visavam atender a outras necessidades que não fossem apenas assistenciais.  

    Em sua trajetória histórica, percebemos que sua evolução está diretamente ligada à história da sociedade, da família, dos conceitos sobre infância e criança, assim como também das políticas de assistência. Segundo \textcite{PASCHOALAndMACHADO2009história}: 

    \begin{quotation}
        As primeiras instituições na Europa e Estados Unidos tinham como objetivos cuidar e proteger as crianças enquanto as mães saiam para o trabalho. Desta maneira, a origem e expansão como instituição de cuidados à criança estão associadas à transformação da família, de extensa para nuclear \cite[p.~78]{PASCHOALAndMACHADO2009história}.
    \end{quotation}

    \textcite{DIDONET2001Creche}, da mesma forma, afirma que as creches, maternais e jardins de infância tiveram no seu início somente o objetivo assistencialista, cujo enfoque era a guarda, higiene, alimentação e cuidados físicos das crianças. 

    O avanço dos conceitos educativos, nessa fase, veio com a Constituição Federal (CF. 1988) e com a Lei de Diretrizes e Bases da Educação Nacional (LDB 9394/96), que regulamentou a Educação Infantil e a definiu como primeira etapa da Educação Básica, conforme já mencionado, com a finalidade de desenvolver integralmente a criança em seus aspectos físico, psicológico, intelectual e social, complementando o papel da família e da comunidade. 

    Neste contexto, o sistema municipal de educação de Natal/RN, passou por diversas modelos de ofertas de ensino infantil. Até meados da década de 1980, não havia estabelecimentos públicos exclusivamente para atendimento à educação infantil. Em um período em que o atendimento foi principalmente de caráter assistencialista, a prefeitura fez uma parceria com a Fundação Bernard Van Leer, da Holanda, que resultou no surgimento do Projeto Reis Magos. Em relação a isto, vale mencionar o que consta no Referencial Curricular Municipal de Educação Infantil de Natal \cite{SOUZAAndMORAES2008Referenciais}: 

    \begin{quotation}
        Em conformidade com a tendência assistencialista de atendimento as crianças oriundas de famílias de baixa renda, ainda predominantemente nesta época, foram firmadas, em 1986, um convênio com a Prefeitura da Cidade do Natal e a Fundação Bernard Van Leer, da Holanda, resultando na implantação do Projeto Reis Magos, que delineou alternativas para a implementação de uma prática pedagógica na Educação Infantil. Para apoiar tal projeto, foi criado o Centro Municipal de Educação Infantil Emília Ramos, a fim de operacionalizar as metas propostas com vistas ao conhecimento da realidade concreta da criança, partindo dos seus interesses e necessidades e da confiança na sua capacidade para a aprendizagem \cite[p.~14--15]{SOUZAAndMORAES2008Referenciais}.
    \end{quotation}

    Este projeto ofertou atendimento às crianças e às mães de baixa renda, que foram alfabetizadas e, com isso, puderam ser inseridas no mercado de trabalho. Algum tempo depois, outros Centros Municipais de Educação Infantil, foram inaugurados com uma proposta pedagógica voltada para o desenvolvimento da criança em seus aspectos sociais, motores, afetivos e cognitivos, em detrimento de um caráter apenas assistencial. 

    No ano de 2005, em conformidade com o Plano Nacional de Educação (Lei nº 10.172/01), o Conselho Municipal de Educação de Natal/RN, de acordo com o Plano Municipal de Educação (Lei nº 5.650/05), elaborou os Referenciais Curriculares para os sistemas públicos de ensino, os quais traziam uma organização curricular voltada para atividades pedagógicas que subsidiavam uma aprendizagem significativa às crianças. 

    Nas instituições de educação infantil, de fato, o trabalho deve ser realizado de forma a buscar a independência da criança, tornando-a mais segura e capaz de construir sua autonomia através de suas próprias decisões e iniciativas, de acordo com as habilidades cognitivas. As interações e atividades, que acontecem nesse contexto, são de extrema importância para aprendizado e desenvolvimentos de habilidades que permitam as crianças a lidarem com sentimentos, de frustrações e limites, e a definir suas preferências. A respeito disso, \textcite{ANTUNES2004Educação} afirma:  

    \begin{quotation}
        [\dots] brincar favorece a autoestima, a interação com seus pares e, sobretudo, a linguagem interrogativa, propiciando situações de aprendizagem que desafiam seus saberes estabelecidos e destes fazem elementos para novos esquemas de cognição \cite[p.~32]{ANTUNES2004Educação}. 
    \end{quotation}

    A criança produz, questiona, constrói seus próprios conceitos, junto com a mediação que o professor proporciona no contexto escolar:

    \begin{quotation}
        [\dots] o professor é mediador entre as crianças e os objetos de conhecimento, organizando e propiciando espaços e situações de aprendizagens que articulem os recursos e capacidades afetivas, emocionais, sociais e cognitivas de cada criança aos seus conhecimentos prévios e aos conteúdos referentes aos diferentes campos de conhecimento humano. Na instituição de educação infantil o professor constitui-se, portanto, no parceiro mais experiente, por excelência, cuja função é propiciar e garantir um ambiente rico, prazeroso, saudável e não discriminatório de experiências educativas e sociais variadas \cite[p.~30]{RCNEI1998}. 
    \end{quotation}

    Foi a partir da reflexão, acerca da história da Educação Infantil, que ocorreu o processo de construção de novos conceitos e concepções de escola, ensino, criança e infância, considerados como parâmetro atualmente. Retomando, o contexto histórico do século XVII, as crianças eram vistas como adultos em miniatura e tinham de viver como tal, participando de todas as atividades sociais. Antes disso, o conceito de criança esteve diretamente ligado a questões religiosas. Somente no século seguinte a sociedade passou a vê-las como seres pequenos, que estavam com seu pensamento em construção e que precisavam de uma mediação para a construção do seu saber: 

    \begin{quotation}
        Trata-se um sentimento inteiramente novo: os pais se interessavam pelos estudos dos seus filhos e os acompanhavam com solicitude habitual nos séculos XIX e XX, mas outrora desconhecida. [\dots] A família começou a se organizar em torno da criança e a lhe dar uma tal importância que a criança saiu de seu antigo anonimato, que se tornou impossível perdê-la ou substituí-la sem uma enorme dor, que ela não pôde mais ser reproduzida muitas vezes, e que se tornou necessário limitar seu número para melhor cuidar dela \cite[p.~12]{ARIÈS1981História}. 
    \end{quotation}

    A infância passou a ser vista como uma fase essencial do desenvolvimento humano, que tem as suas especificidades e necessidades próprias. Essa mudança no pensamento, resultou em uma mudança de paradigma, e a sociedade passou a perceber a infância como uma fase de aprendizagem sobre o mundo. “Somente em épocas comparativamente recentes veio a surgir um sentimento de que as crianças são especiais e diferentes e, portanto, dignas de ser estudadas por si só” enquanto conhecem o seu próprio espaço \cite[p.~10]{HEYWOOD2004Uma}. 

    \textcite{PEDROSA1996emergência}, em consonância com \textcite{VALSINER1988Ontogeny}, afirma que a criança, desde o seu nascimento, interage com um mundo de significados construídos historicamente, envolvendo-se em processos de significação de si, dos outros e dos acontecimentos de seu contexto cultural, construindo e reconstruindo ativamente significados. 

    Essas interações acontecerão através de atividades lúdicas, pois elas são essenciais ao universo infantil. Nesse contexto, destacam-se as brincadeiras e os jogos, ações que adquirem especificidades e conseguem promover grandes transformações de acordo com cada grupo. A linguagem do brincar é, no mundo infantil, uma linguagem universal, conforme demonstra \textcite{LUCKESI2004Estados}: 

    \begin{quotation}
        É pelo brincar que a criança aprende expressar ideias gestos emoções, a tomar decisões, a interagir e viver entre pares, a conhecer e integrar-se no seu ambiente próximo a elaborar imagens culturais e sociais de seu tempo e, em decorrência, desenvolver-se como ser humano dotado de competência simbólica \cite[p.~11]{LUCKESI2004Estados}. 
    \end{quotation}

    O lúdico permite internalizar, de forma prazerosa, divertida e natural, normas sociais e assumir comportamentos mais avançados de acordo com os contextos, aprofundando, assim, o conhecimento da criança sobre a vida social. Cabe aos educadores compreender essa relevância do brincar no processo de construção do conhecimento. 

    No entanto, conforme \textapud{PEREIRA2005Ludicidade}{SILVA2011Vivência},

    \begin{quotation}
        [\dots] as atividades lúdicas não se restringem ao jogo e à brincadeira, mas incluem atividades que possibilitam momentos de alegria, entrega e integração dos envolvidos [\dots] possibilita a quem as vivências, momentos de encontro consigo e com o outro, momentos de fantasia e de realidade, de ressignificação e percepção, momentos de autoconhecimento e conhecimento do outro, de cuidar de si e olhar para o outro, momentos de vida, de expressividade \apud[]{PEREIRA2005Ludicidade}[p.~20]{SILVA2011Vivência}.
    \end{quotation}

    Considerando todo o exposto, este trabalho teve como objetivo expor a influência de atividades lúdicas no desenvolvimento integral da criança que se encontra na Educação Infantil.  


    \section{Metodologia}

    Nesta seção apresentaremos os elementos estruturantes da metodologia da pesquisa: fundamentos teórico-metodológicos; o contexto e sujeitos da pesquisa; os procedimentos e percurso metodológicos e como foram tratamento os dados. 

    A metodologia é de abordagem quantitativo/qualitativa; de natureza aplicada; com objetivos exploratórios e explicativos e de procedimento documental, sob a perspectiva adotada por \textcite{LavilleAndDionne1999Construcao}.  

    A pesquisa foi realizada em fonte primária por meio de análises de projetos pedagógicos desenvolvidos no Centro Municipal de Educação Infantil (CMEI) Bom Samaritano, localizado no Bairro Quintas, na Zona Oeste do Município de Natal/RN, com 111 crianças matriculadas, divididas nos turnos matutino e vespertino. 

    Neste trabalho, estão sendo contemplados, os dados documentais, relacionados aos projetos pedagógicos desenvolvidos, entre os anos letivos de 2017, 2018 e 2019, com os estudantes matriculados em turmas de Educação Infantil.  

    Os instrumentos de coleta de dados foram: os dados primários, disponíveis nos projetos pedagógicos, e relatórios finais, dos projetos desenvolvidos no Centro Municipal de Educação Infantil (CMEI) Bom Samaritano.   


    \section{Resultados e discussões}

    Com base nos documentos disponibilizados, foi possível identificar que o CMEI Bom Samaritano, no período letivo entre fevereiro e maio de 2017, desenvolveu o projeto intitulado \textbf{“construindo a identidade e autonomia a partir das interações e vivências”}, que teve como público-alvo as crianças regularmente matriculadas no Nível III. Observou-se, através da descrição das atividades propostas para o projeto, o potencial lúdico, em vários momentos da rotina escolar, para o Centro de Educação Infantil. O projeto, tinha como objetivo, “trabalhar a identidade da criança, conhecendo sua história de vida, identificando dados pessoais e familiares, procurando estimular vivências nas quais as crianças desenvolver atitudes de respeito e valorização a si própria e as diferenças individuais dos que as cercam, observando também a importância e funcionalidade do CMEI, onde passam grande parte de seu tempo”.  

    Com base na análise do relatório final, do projeto \textbf{“construindo a identidade e autonomia a partir das interações e vivências”}, é possível sugerir que as ações desenvolvidas estimularam a promoção da identidade e da autonomia da criança, em todas as etapas das atividades desenvolvidas. De acordo com o RCNEI (2001), a identidade é um conceito do qual faz parte a ideia de distinção, de diferença entre as pessoas, a começar pelo nome, seguido de todas as características físicas, do modo de agir, de pensar e da história pessoal, ou seja, as atividades queriam mostrar que cada criança é um ser único, diferente do outro. As atividades desenvolvidas neste projeto, demonstraram um potencial lúdico significativo, pois promoviam, através da brincadeira, o reconhecimento do eu e a identificação das diferenças entre as crianças. 

    Referindo-se ao brincar, no Referencial Curricular Nacional para a Educação Infantil (1998), consta o seguinte: 

    \begin{quotation}
        Brincar é uma das atividades fundamentais para o desenvolvimento da identidade e da autonomia. O fato de a criança, desde muito cedo, poder se comunicar por meio de gestos, sons e mais tarde representar determinado papel na brincadeira faz com que ela desenvolva sua imaginação. Nas brincadeiras as crianças podem desenvolver algumas capacidades importantes, tais como a atenção, a imitação, a memória, a imaginação. Amadurecem também algumas capacidades de socialização, por meio da interação e da utilização e experimentação de regras e papéis sociais. \cite[p.~22]{RCNEI1998}.
    \end{quotation}

    As atividades que estimulam brincadeiras são fundamentais para o desenvolvimento integral do indivíduo. Para \textcite[p.~25]{LUCKESI2004Estados}, “no estado lúdico, o ser humano está inteiro, ou seja, está vivenciando uma experiência que integra sentimento, pensamento e ação, de forma plena. A vivência se dá nos níveis corporal, emocional, mental e social, de forma integral e integrada”. Essa experiência se processa interiormente e de forma peculiar em cada história pessoal, ou seja, cada criança reage à sua maneira nos momentos em que lida com a ludicidade.  

    As atividades do projeto \textbf{“construindo a identidade e autonomia a partir das interações e vivências”}, foram organizadas de forma a propiciar um ambiente rico em interações, que acolhiam as particularidades de cada indivíduo, promovendo desta forma o reconhecimento e o respeito das diversidades, ao mesmo tempo em que contribuiu para a construção da unidade coletiva, favorecendo a estruturação da identidade, bem como de uma autoimagem positiva de si mesmo. Foi identificado, o uso de cantigas de roda que permitiram trabalhar o nome da criança por meio da música e do crachá. Com base nos registros dos relatos das crianças, as atividades de roda permitiram, a expressão de suas preferências de brincadeiras, comidas, colegas, entre outras. O trabalho com cantigas de roda, além do potencial lúdico, permitiu, segundo dados dos relatórios, explorar movimentos e expressões do corpo em interação com o mundo, ou seja, potencializa a expressão corporal de cada indivíduo envolvido.  

    Foi utilizado nas atividades, deste projeto, a massa em atividade de modelagem, o que potencializou ações que abordaram a composição da família. Além dessa ferramenta, foi identificado nos registros, o uso de outros instrumentos, brinquedos, para representar a rotina familiar de cada criança que foi atendida com o projeto no CMEI. Acrescido a isto, o projeto, ainda possuía um circuito motor com diversas atividades de obstáculos, oportunizando testar, com segurança e sob o acompanhamento da professora, limites e habilidades motoras. O projeto teve, durante seu desenvolvimento, avaliações contínuas das atividades desenvolvidas, que envolveram a interação das crianças entre si e com os professores. 

    Foi observado que no segundo trimestre letivo de 2017, foram realizados 02 (dois) projetos. O primeiro foi o projeto intitulado \textbf{“homem do campo”}. Nele, é possível identificar pelos relatórios, que foram trabalhados assuntos pertinentes à temática campo, por meio de estratégias com potencial lúdico, com vistas a conhecer e valorizar o homem do campo. O projeto foi desenvolvido no período de maio e junho de 2017, com as crianças do Nível III. Existia no projeto a perspectiva que sensibilizar os estudantes o gosto pelas festas juninas, oferecendo-lhes oportunidade de descontração, socialização e ampliação de seu conhecimento através de atividades diversificadas, brincadeiras, pesquisa e apresentações características, desses festejos que fazem parte do folclore brasileiro, ressaltando seus aspectos popular, social e cultural. Os registros documentais do projeto, demonstram que as crianças puderam conhecer um pouco sobre a vida do homem do campo e, a partir desses conhecimentos, refletir sobre como os alimentos eram cultivados e como chegavam às suas casas. Conheceram também as festas juninas e os aspectos que permeiam essa tradição integrante da nossa cultura e que permite um momento rico de interação. É muito importante trabalhar a cultura desde a educação infantil. As Diretrizes Curriculares para Educação Infantil (2010), estabelecem como eixos norteadores do currículo, as interações e a brincadeira (idem, Art. 9º). No inciso XII, estabelecem que devem ser garantidas na Educação Infantil experiências que “propiciem a interação e o conhecimento pelas crianças das manifestações e tradições culturais brasileiras”. Sendo assim, o trabalho com as festas juninas, foi significativo por levar as crianças a refletirem sobre aspectos que fazem parte da nossa cultura.   

    Entre as propostas do projeto, intitulado \textbf{“homem do campo”}, foi observado as brincadeiras de faz de conta, representando o homem do campo e suas atividades rurais, por meio de argila e modelagem. Além disso, foram utilizados flocos de milho para preparar cuscuz, pelas crianças, posteriormente, degustado em sala. Estas vivências culinárias foram registradas através de desenhos pelos estudantes. Acrescido a isto, as crianças tiveram a oportunidade de dançar músicas juninas, em pares, sozinhas e em grandes e pequenos grupos. Estas atividades abordando a expressão corporal, possui um potencial lúdico, e com inúmeras potencialidades para o processo de aprendizagem integral. Os docentes que atuam na educação infantil, precisam reconhecer que, para as crianças, o movimento significa também comunicação. Sobre esse aspecto, \textcite[p.~1]{KUHNEAndSILVA2006corpos}, afirmam que “a corporeidade da criança se constitui a partir do ato de brincar sendo a linguagem primeira da qual ela lança mão para se relacionar com os outros, com os objetos e consigo mesma”. Dessa forma, a criança utiliza o movimento, enquanto linguagem, desde o ato de brincar e podem comunicar-se através do movimento.  

    O projeto realizado no segundo trimestre letivo de 2017, intitulado \textbf{“brinquedos e brincadeiras”}, aconteceu no período entre setembro e outubro de 2017, com as crianças do Nível III do turno matutino. O projeto partiu de algumas indagações, tais como: quais as brincadeiras do tempo da vovó? Quais os brinquedos que vocês conhecem? Quais os nomes das brincadeiras que vocês conhecem? Será que na época dos seus pais eles brincavam com os mesmos brinquedos que vocês brincam? Vocês conhecem algumas das brincadeiras que seus pais brincavam na infância? Será que as brincadeiras de hoje são iguais às de antigamente? O que é folclore? A partir dessas indagações, presentes no projeto pedagógico, foram criadas as estratégias utilizadas para o desenvolvimento das atividades. Os registros demonstram que a escola procurou enfatizar a importância do folclore para a cultura popular brasileira, resgatando antigos brinquedos e brincadeiras populares, bem como uma reflexão acerca das influências da tecnologia no processo de sensibilização sobre folclore na Educação Infantil.  

    Os brinquedos e brincadeiras configuram-se como um objeto de prazer e, simultaneamente, mesmo que de forma inconsciente, de aprendizado para a criança. A brincadeira tem um papel fundamental no desenvolvimento infantil, pois permite construir conceitos sobre a realidade. As Diretrizes Curriculares Nacionais para Educação Infantil \cite{Resolução5-2009}, no artigo IX, mostram que as práticas da educação infantil:

    \begin{quotation}
        [\dots] devem ter como eixos norteadores as interações e a brincadeira, garantindo experiências que [\dots] promovam o conhecimento de si e do mundo por meio da ampliação de experiências sensoriais, expressivas, corporais que possibilitem movimentação ampla, expressão da individualidade e respeito pelos ritmos e desejos da criança \cite[p.~25]{Resolução5-2009}. 
    \end{quotation}

    O trabalho pautado no resgate das brincadeiras populares auxilia na continuidade da cultura local e regional, contribuindo também, desse modo, para uma reflexão sobre a supervalorização que atualmente é dada ao brinquedo comercialmente produzido. O conhecer, construir e, sobretudo, o refletir acerca do brinquedo, da brincadeira, da tecnologia e dos seus respectivos papéis na cultura popular foi o eixo direcionador do projeto. 

    O projeto \textbf{“brinquedos e brincadeiras”}, trabalhou com um conjunto de objetivos atitudinais com relação ao brinquedo, tais como: respeitar e conscientizar a respeito das brincadeiras e brinquedos de antigamente; socializar seus conhecimentos prévios acerca dos tipos de brinquedos e brincadeiras, a partir das vivências dos seus pais, respeitando o espaço e o tempo do outro; conhecer e experimentar os diversos tipos de brinquedos e brincadeiras em várias gerações, enfocando o que a tecnologia trouxe de mudança nesse aspecto; perceber a importância da troca de experiências entre as pessoas (resgate do folclore como uma ação de repasse de saberes através dos tempos); vivenciar tradições da nossa cultura para, gradativamente, reconhecer e valorizar os brinquedos e brincadeiras utilizadas em outras gerações. Os outros objetivos, foram específicos dentro das áreas de conhecimento, conforme os referenciais curriculares. Dentro das atividades, com potencial lúdico, encontradas no projeto analisado, é possível citar os momentos de brincadeiras livres com manipulação de materiais, objetos e brinquedos diversos que contribuíram para o aperfeiçoamento das habilidades manuais das crianças. 

    No decorrer do projeto, segundo os relatórios, aconteceram momentos de contações de histórias de formas variadas, com manuseio de livros, fantoches, dedoches ou os próprios bonecos de personagens. As crianças puderam vivenciar a confecção de painéis e murais com imagens de diversos brinquedos e brincadeiras, imagens em que apareciam adultos e crianças brincando. O objetivo era estimular o relato sobre os momentos de brincadeiras em casa com a família. Foram solicitados diálogo no momento de roda e fotografias dos pais brincando quando crianças, a fim de criar um painel valorizando as brincadeiras que eram realizadas pelos pais em sua infância. As crianças também tiveram a oportunidade de registrar e expor, através da pintura em tela, suas brincadeiras favoritas.  

    No terceiro trimestre do ano de 2017, o primeiro projeto foi intitulado \textbf{“vem brincar que o circo já chegou”}. Ele teve como base o imaginário e as brincadeiras e buscou estimular a criatividade dos estudantes. As crianças participaram de jogos, brincadeiras e atividades para perceber as noções de tamanho, altura, comprimento, espessura, capacidade, massa, temperatura, espaço e velocidade. 

    Estavam previstos como objetivos do projeto \textbf{“vem brincar que o circo já chegou”}: explorar movimentos com o corpo, promover o desenvolvimento da motricidade ampla, equilíbrio e lateralidade; conhecer, nomear e identificar personagens e animais do circo, favorecendo o desenvolvimento da oralidade e fluência; proporcionar o contato com materiais e texturas variados e aprimorar a motricidade fina através de trabalhos manuais e manuseio de materiais. No decorrer do trabalho, as crianças também puderam conhecer um pouco sobre a história do circo, através de brincadeiras livres, do jogo, do faz de conta e nos momentos de pintura facial entre as crianças interpretando os palhaços. \textcite{MODESTOAndRUBIO2014importância} afirmam que: 

    \begin{quotation}
        É brincando que a criança constrói sua identidade, conquista sua autonomia, aprende a enfrentar medos e descobre suas limitações, expressa seus sentimentos e melhora seu convívio com os demais, aprende entender e agir no mundo em que vive com situações do brincar relacionadas ao seu cotidiano, compreende e aprende a respeitar regras, limites e os papéis de cada um na vida real; há a possibilidade de imaginar, criar, agir e interagir, auxiliando no entendimento da realidade \cite[p.~3]{MODESTOAndRUBIO2014importância}. 
    \end{quotation}

    Dessa maneira, o professor deve sempre promover atividades que levem a criança a construir seus conhecimentos e sua autonomia, dando destaque para a imaginação, de acordo com \cite{OLIVEIRA1998Vygotsky}: 

    \begin{quotation}
        A promoção de atividades que favoreçam o envolvimento da criança em brincadeiras, principalmente aquelas que promovem a criação de situações imaginárias, tem nítida função pedagógica. A escola e, particularmente a pré-escola poderiam se utilizar deliberadamente desse tipo de situação para atuar no processo de desenvolvimento das crianças \cite[p.~67]{OLIVEIRA1998Vygotsky}. 
    \end{quotation}

    No final do terceiro trimestre de 2017, o projeto intitulado \textbf{“por um natal de amor”}, ocorreu especificamente no período entre 01 de novembro de 2017 e 22 de dezembro de 2017. O tema do Natal, foi trabalhado a partir da importância das interações pessoais para o bom convívio em sociedade. O projeto teve como objetivo geral “despertar nos alunos o verdadeiro sentido do Natal através da participação de atividades alegres e espontâneas, enfatizando um ambiente festivo, perceptivo a solidariedade e amor ao próximo”. 

    Foram proporcionadas atividades que resgataram o real significado do Natal, bem como possibilitaram a construção de conhecimentos acerca das tradições da festa de forma participativa, descontraída, buscando integrar diversas áreas e permitir a livre criação, a interação e o diálogo, respeitando, porém, as diferenças individuais de cada criança. Isso permitiu o resgate da importância do perdão, do amor ao próximo, da fraternidade e da humildade, dentre outros quesitos que vêm sendo esquecidos na sociedade atual.  

    Comparado aos projetos anteriores desenvolvidos no CMEI, este projeto apresentou atividades com potencial lúdico em quantidade inferior. A proposta foi centrada no diálogo com as crianças, em atividades manuais e práticas sobre as questões já mencionadas. As crianças tiveram momentos de contação de história acompanhadas por uma Roda de conversa e, após esta, foram estimuladas a falar sobre a história, expressando seus sentimentos. Posteriormente, houve brincadeiras de faz de conta, representando um pouco da história contada utilizando materiais diversos organizados pela professora. Os momentos de brincadeira também foram organizados livremente, de forma que as crianças podiam brincar no pátio da escola com os brinquedos disponibilizados. Esses momentos são importantes, pois permitem que a própria criança organize a sua brincadeira, com autonomia e liberdade.  

    Em 2018, o primeiro projeto desenvolvido foi \textbf{“na escola, eu cuido de você, você cuida de mim”}, no período entre de fevereiro e maio. Este tema, esteve devidamente vinculado ao projeto anual do CMEI, que foi “Eu cuido de você, você cuida de mim e nós cuidamos do mundo”. O Projeto Pedagógico Anual permitiu que, no decorrer do processo, os docentes organizassem todas as atividades que foram desenvolvidas durante o ano letivo, de acordo com a faixa etária e a realidade das crianças. Nesta perspectiva, o projeto anual, além da turma do Nível III, envolveu todas as turmas da Instituição, através de subtemas trimestrais, contemplando os campos de experiências, através dos direitos de aprendizagem (conviver, brincar, explorar, participar, comunicar e conhecer-se), de maneira a trilhar um caminho para uma prática em que a criança fossem as protagonistas de sua aprendizagem. O objetivo do \textbf{“na escola, eu cuido de você, você cuida de mim”}, era “proporcionar momentos para que as crianças desenvolvessem uma imagem positiva de si mesma, ampliando sua autonomia, reconhecendo suas limitações e possibilidades, contribuindo assim para sua atuação no ambiente em que convive enquanto cidadão crítico e ativo”. Com base nos dados coletados, foi observado que a proposta foi desenvolvida a partir da coleta de dados prévios sobre o que as crianças já sabiam da temática. Foram trabalhadas livres, em que os estudantes puderam se expressar, representando momentos de sua rotina diária em casa e na escola. Nestes momentos, foi observada as atitudes de ajuda que as crianças ofereciam umas às outras, fruto dos diálogos que permearem todos os momentos de roda e discussão sobre a importância que cada pessoa possui.  

    O segundo projeto do ano de 2018, na turma do Nível III, foi intitulado \textbf{“Nós cuidamos do mundo”}, ocorrido no período entre maio e setembro. Para realização deste projeto, foi percorrido um caminho de conhecimento dos recursos naturais a fim de que todos atentassem para a importância deles e da preservação do meio ambiente para o bem comum. O objetivo era apresentar a natureza com suas belezas, curiosidades e fragilidades, mostrando os lados negativo e positivo da ação do homem e valorizando pensamentos e atitudes de preservação ambiental. As atividades desenvolvidas aconteceram embasadas nas propostas de vivências e experiências trazidas na BNCC \cite{BaNacCurEF2017}, que coloca a prática docente como ponto importante do processo de ensino e aprendizagem. 

    As experiências na educação infantil são fundamentais para que a criança tenha a autonomia de construir seu próprio saber, a partir de sua curiosidade em saber mais sobre o ambiente no qual está inserida, sendo importante que essas interações aconteçam através de momentos lúdicos e de brincadeiras. Conforme a BNCC \cite{BaNacCurEF2017}, 

    \begin{quotation}
        Na Educação Infantil, as aprendizagens essenciais compreendem tanto comportamentos, habilidades e conhecimentos quanto vivências que promovem aprendizagem e desenvolvimento nos diversos campos de experiências, sempre tomando as interações e a brincadeira como eixos estruturantes \cite[p.~42]{BaNacCurEF2017}.
    \end{quotation}

    Assim sendo, os momentos de brincadeiras possibilitam a oportunidade prazerosa de construção do próprio conceito sobre o mundo. Neste projeto, as crianças puderam discutir sobre os alimentos que são produzidos pela natureza e como eles chegam até as casas dos consumidores. A partir destas discussões, as professoras abordaram as comidas das festas juninas e as características do homem que vive no campo. Além disso, houve brincadeiras livres e orientadas, em pequenos grupos, em grandes grupos e individuais, com brinquedos variados, jogos, blocos lógicos e brinquedos construídos pelas próprias crianças, para que percebessem que o resíduo sólido, em muitos casos, pode ser reaproveitado para confeccionar algo tão importante para eles: um brinquedo.  

    A avaliação deste projeto, assim como de outros citados anteriormente, aconteceu de forma contínua, a partir das observações e registros do professor dos avanços apresentados pelas crianças. Tais registros são fundamentais, pois permitem ao professor acompanhar como está acontecendo e em que fase se encontra cada criança em seu processo de desenvolvimento, assim como planejar suas aulas para intervir de forma significativa:  

    \begin{quotation}
        Os registros, além de cumprirem um importante papel na formação dos professores, são parte do processo cuidadoso e contínuo de documentação da história dos processos de aprendizagem das crianças de modo a compartilhar a visão da criança como sujeito ativo e dar notícias às famílias sobre sua aprendizagem e desenvolvimento (OLIVEIRA, 2012, p.~382). 
    \end{quotation}


    A análise feita no projeto do terceiro trimestre do ano de 2018, que ocorreu no período entre  setembro a dezembro e foi intitulado \textbf{“nós cuidamos do mundo: por um natal de paz”}, permitiu identificar seu objetivo: despertar nas crianças o verdadeiro sentido do natal, refletindo sobre os sentimentos de respeito, amor, solidariedade, fortalecendo os vínculos afetivos e a autoestima de cada criança, através da participação em atividades alegres e espontâneas, enfatizando um ambiente saudável, bem cuidado e festivo, perceptivo à solidariedade e amor ao próximo. As atividades deste projeto apresentaram metodologia similar a desenvolvida nos projetos anteriores, envolvendo momentos de contação de história, brincadeiras livres e orientadas, com brinquedos e jogos, momentos de vídeo, de manuseio de materiais concretos como fantoches, dedoches e outros. Além disso, as crianças também tiveram momentos de modelagem com massa, pintura em diferentes objetos e registro livre dos momentos vivenciados na escola. 

    Analisando as atividades propostas, é possível observar que elas possibilitam uma aprendizagem prazerosa e significativa por meio do lúdico. A criança aprende descobrindo os significados do meio através de ações e interações com as outras crianças, explorando gestos e movimentos: 

    \begin{quotation}
        A instituição escolar precisa promover oportunidades ricas para que as crianças possam, sempre animadas pelo espírito lúdico e na interação com seus pares, explorar e vivenciar um amplo repertório de movimentos, gestos, olhares, sons e mímicas com o corpo, para descobrir variados modos de ocupação e uso do espaço com o corpo (tais como sentar com apoio, rastejar, engatinhar, escorregar, caminhar apoiando-se em berços, mesas e cordas, saltar, escalar, equilibrar-se, correr, dar cambalhotas, alongar-se etc.) \cite[p.~41]{BaNacCurEF2017}.  
    \end{quotation}

    Conforme a BNCC \cite{BaNacCurEF2017}, é através do lúdico que a criança explora o meio, interage com ele, o compreende e cria seu próprio saber sobre o espaço em que se encontra. 

    Com base no acervo documental analisado, é possível identificar que as propostas desenvolvidas pela unidade de ensino, assumem que atividades lúdicas são aquelas que abordam a prática de brincadeiras pelas crianças. É possível identificar a concepção que as crianças compreendem as brincadeiras através que trechos dos registros que os docentes realizaram das falas de alguns discentes, que terão os nomes verdadeiros substituídos por nomes fictícios. As brincadeiras que mais gostam de praticar: 

    \begin{quotation}
        \textit{Parque, parque de dinossauro, gosto de subir na escada e tomar banho no rio com a minha irmã. Gosto de brincar com carro MacQuem e com a pistinha de corrida. Tenho uma escolinha de brinquedo e também gosto dos dinossauros que andam sozinho} (Mário, 2019).  
    \end{quotation}

    Como é feito as brincadeiras no cotidiano das crianças fora da escola: \textit{“[\dots] quando está na caixa de areia pega o celular de brinquedo e o controle remoto. Também joga com bola de areia e constrói castelo de areia bem grande. Dentro tem o bonequinho que eu coloco [\dots]”}. Os relatos obtidos demonstram que as crianças possuem o conceito de lúdico desprendido dos brinquedos e não das brincadeiras. Essa constatação é importante, pois em alguns projetos que se apresentam com potencial lúdico não é identificado o uso de brinquedos nas atividades desenvolvidas pelos discentes. 


    \section{Considerações finais}

    Os projetos pedagógicos desenvolvidos no Centro Municipal de Educação Infantil (CMEI) Bom Samaritano, no período de 2017 a 2019, demonstraram o planejamento prévio como uma forte ferramenta de ação pedagógica para a compreensão sobre os aspectos que envolvem a ludicidade. Os documentos propostos com atividades com potencial lúdico analisados demonstram êxito em seus objetivos, uma vez que as crianças foram levadas a construir seu conhecimento de forma mais prazerosa e fazendo o que gostam, que é brincar. Ratifica-se o entendimento de que as crianças têm seu desenvolvimento mais significativo a partir das vivências propostas durante a educação infantil, na qual as atividades lúdicas têm papel importante. O uso da ludicidade, em atividades escolares, pode influenciar positivamente no processo de ensino e aprendizagem da criança da educação infantil.

    \nocite{LDB1996}
    \nocite{RIBEIRO2010afetividade}

    \printbibliography[heading=subbibliography,notcategory=fullcited]

    \label{chap:influencia-ludicoend}

\end{refsection}
