\begin{refsection}
    \renewcommand{\thefigure}{\arabic{figure}}
    
    \chapterOneLine
    {A concepção de pais e responsáveis por crianças de dois e três anos acerca do brincar na Educação Infantil}
    \label{chap:concepcao-pais-respo}

    \articleAuthor
    {Erineide Andrea Machado Dantas da Silva}
    {Especialista em Educação Infantil.  Instituto de Educação Superior Presidente Kennedy (IFESP). E-mail: erineideandrea@gmail.com.}
    
    \articleAuthor
    {Tereza Cristina Bernardo da Câmara}
    {Graduada em Educação Física. Mestra em Educação pela UFRN. Professora formadora do Instituto de Educação Superior Presidente Kennedy (IFESP). ID Lattes: 8591.8828.1363.7627. E-mail: terezacbcamara2016@gmail.com.}
    
    \begin{galoResumo}
        \marginpar{
            \begin{flushleft}
            \tiny \sffamily
            Como referenciar?\\\fullcite{SelfSilvaAndCâmara2021concepção}\mybibexclude{SelfSilvaAndCâmara2021concepção}, p. \pageref{chap:concepcao-pais-respo}--\pageref{chap:concepcao-pais-respoend}, \journalPubDate{}
            \end{flushleft}
        }
        Este artigo resulta de uma pesquisa de natureza qualitativa, exploratória, sobre a concepção dos pais e responsáveis por crianças de dois e três anos acerca do brincar no espaço escolar da Educação Infantil. Os dados foram analisados considerando as falas dos colaboradores no processo de desenvolvimento e aprendizagem das crianças, cujo objetivo era saber o que pensam os pais em relação à brincadeira na sala de aula. O referencial teórico, além dos documentos oficiais, é composto pelos teóricos Piaget, Vygotsky e \textcite{KISHIMOTO1999Jogo, KISHIMOTO2002Brincar}. O resultado da pesquisa surpreendeu a pesquisadora na medida em que os pais e responsáveis demonstraram reconhecer a importância da brincadeira no desenvolvimento de seus pequenos.
    \end{galoResumo}
    
    \galoPalavrasChave{Educação Infantil. Brincar. Desenvolvimento Infantil.}

    %% Missing abstract

    \section{Introdução}

    Ao longo dos anos, em especial a partir da homologação da Lei de Diretrizes e Bases da Educação Nacional (LDBEN), de nº 9394, no ano de 1996, a Educação Infantil, passa a ser reconhecida como primeira etapa da Educação Básica e vem sendo contemplada por uma política específica por parte do Estado, um olhar mais científico por parte dos educadores, além de uma maior atenção e envolvimento da sociedade civil. Como decorrência dessa nova realidade, tem havido o cuidado em melhorar as condições de trabalho e um maior investimento na formação acadêmica dos profissionais que atuam com as crianças pequenas, além de uma preocupação com os ambientes educativos e com os objetos de conhecimento desenvolvidos nessas instituições. 

    Além dos pontos destacados, é cada vez mais sentida a necessidade do envolvimento dos familiares responsáveis pelas crianças, para que se percebam como parceiros em um processo que tem como grande beneficiado alguém que é tão caro para ambas as partes, a criança, aqui entendida como sujeito de direitos e que, historicamente, anteriormente a promulgação da LDB 9394, vinha frequentando a escola de educação infantil, numa perspectiva assistencialista de atendimento. 

    Os Referenciais Curriculares Nacionais para a Educação Infantil (RCNEI) apontam a necessidade de se desenvolver na Educação Infantil, de maneira articulada, o tripé brincar, cuidar e educar, como fundamental para o desenvolvimento integral das crianças. Embora o documento reconheça a relevância de tal articulação, é inquietante perceber que muitos pais ainda consideram que tempo utilizado no espaço escolar para as brincadeiras, se constitui um mero passatempo.

    Por comumente escutar esse tipo de interpretação por parte de pais e responsáveis diante da vivência de jogos e brincadeiras, uma pesquisa de natureza qualitativa, do tipo exploratória, foi desenvolvida em uma instituição de Educação Infantil da rede privada na cidade de Natal/RN e o resultado da referida pesquisa resultou neste artigo.  

    A investigação teve como objetivo investigar qual a concepção dos pais e/ou responsáveis por crianças com idades entre dois e três anos acerca do brincar no processo de desenvolvimento e de aprendizagem na Educação Infantil. 


    \section{Brincadeira é coisa séria}

    Nesta seção será abordado o que afirmam alguns autores acerca da importância da brincadeira no processo de aprendizagem e desenvolvimento de crianças pequenas com idades entre dois e três anos. Participarão deste diálogo, além do que apontam os documentos oficiais, Piaget e Vygotsky, citados \textcite{KISHIMOTO1999Jogo, KISHIMOTO2002Brincar} e \textcite{MALUF2003Brincar}. 

    Segundo os Referenciais Curriculares Nacionais para a Educação Infantil (RCNEI),  

    \begin{quotation}
        A brincadeira favorece a autoestima das crianças auxiliando-as a superar progressivamente suas aquisições de forma criativa. Brincar contribui, assim, para interiorização de determinados modelos de adultos no âmbito de grupos sociais diversos, essas significações atribuídas ao brincar transformam-no em um espaço singular de constituição infantil \cite[p.~27]{RCN2001}. 
    \end{quotation}

    A organização e o uso dos brinquedos e das brincadeiras nas creches é de suma importância, dentre outros, para as interações entre as crianças. Necessário se faz planejar, selecionar e organizar os brinquedos e as brincadeiras, inclusive, se possível for, em conjunto com os pais, os responsáveis e a comunidade. É preciso considerar que todas as crianças são cidadãs com direito a uma educação de qualidade, e que devem ser educadas por meio de brincadeiras e a interação é o primeiro passo para adquirirem experiências ricas para explorar o mundo. 

    De acordo com as Diretrizes Curriculares Nacionais da Educação Infantil (DCNEI), em seu Artigo 9º, os eixos estruturantes das práticas pedagógicas dessa etapa da Educação Básica são as interações e a brincadeira, experiências nas quais as crianças podem construir e apropriar-se de conhecimentos por meio de suas ações e interações com seus pares e com os adultos, o que possibilita aprendizagens, desenvolvimento e socialização \cite{DCN2009}. 

    Portanto, não basta oportunizar a vivência de brincadeiras é preciso considerar a importância do ambiente educativo para as crianças pequenas, as condições físicas do local, a qualidade e a segurança para que toda a sua energia esteja voltada para o que ela vivência, de maneira plena. 

    Na Educação Infantil é possível observar a influência positiva das atividades potencialmente lúdicas\footnote{Segundo \textcite{HUIZINGA1990Homo}, dentre outras características o lúdico é liberdade, divertimento, fruição.} em um ambiente aconchegante, desafiador, rico em oportunidades e experiências para o crescimento sadio. Os primeiros anos de vida são decisivos na formação da criança, pois é o período no qual ela está construindo sua identidade e grande parte de sua estrutura física, sócio-afetiva e intelectual. 

    Acerca dessa formação, \textcite[p.~20--21]{MALUF2003Brincar}, reconhece que 

    \begin{quotation}
        [\dots] brincando-a (a criança) aprende novos conceitos, adquire novas informações e tem um crescimento saudável. Toda criança que brinca vive uma infância feliz, além de tornar-se um adulto muito mais equilibrado física e emocionalmente, conseguirá superar com mais facilidade, problemas que possam surgir no seu dia a dia. 
    \end{quotation}

    A dimensão social é algo inerente ao brincar e sendo a escola um espaço socializador em sua essência, não tem como não perceber o valor de tal prática no desenvolvimento dos que nela estão presentes. 

    A Base Nacional Curricular Comum (BNCC), considera que, 

    \begin{quotation}
        A interação durante o brincar caracteriza o cotidiano da infância, trazendo consigo muitas aprendizagens e potenciais para o desenvolvimento integral das crianças. Ao observar as interações e a brincadeira entre as crianças e delas com os adultos, é possível identificar, por exemplo, a expressão dos afetos, a mediação das frustrações, a resolução de conflitos e a regulação das emoções \cite[p.~35]{BaNacCurEF2017}. 
    \end{quotation}

    Além do exposto no documento, é importante notar que existem brincadeiras que estimulam cada fase do desenvolvimento infantil e é importante perceber quais são as brincadeiras indicadas para cada um desses períodos. O educador deve estar atento às necessidades da criança para que possa atendê-la de forma a contribuir com seu desenvolvimento. 

    Segundo Piaget, citado por \textcite{KISHIMOTO1999Jogo}, cada fase de desenvolvimento da criança se relaciona a um determinado tipo de jogo. Em sua classificação ele apresenta como primeira etapa, a sensório-motora, que contempla crianças de zero a dois anos. A essa etapa ele relaciona o Jogo de Exercício, que se caracteriza pela repetição do gesto, o prazer está em fazer o mesmo movimento repetidas vezes. Não fase seguinte, a pré-operatória, encontram-se crianças de dois a seis anos, que são as que este trabalho contemplou. Nesse período, o jogo característico é o Simbólico, ou faz-de-conta. Nele as crianças reproduzem ou modificam a realidade na qual estão inseridas. Por fim, surgem os Jogos com Regras, que estão presentes na fase operatória, que se estende dos seis aos doze anos.  

    No Jogo Simbólico, foco deste artigo, \textcite[p.~36]{KISHIMOTO1999Jogo} reconhece que “quando brinca, a criança assimila o mundo à sua maneira, sem compromisso com a realidade pois a sua interação com o objeto não depende da natureza do objeto, mas sua função a que a criança lhe atribui”. 

    Os pequenos vivenciam, de maneira intensa, o jogo de faz-de-conta e costumam brincar de professor ou professora, imitando os adultos; transformam casinhas em sorveteria; o lápis vira uma mamadeira, mas pode também se transformar em um foguete; a cadeira vira uma cama e a vassoura um cavalinho; os legos viram comida e o pente um microfone. 

    Representar, através da brincadeira, o que vivencia com os adultos em sua vida cotidiana é de extrema relevância no desenvolvimento infantil e sobre isso \textcite[p.~146]{KISHIMOTO2002Brincar}, reconhece que: 

    \begin{quotation}
        É importante a imitação de esquemas de adultos, mas não por intervenção direta. A influência indireta permite a observação, identificação, ação intencional da criança no sentido de repetir e de recriar, contribuindo para seu desenvolvimento. Oferecer oportunidades para visualizar diferentes formas de fazer estimula o surgimento de imitações e repetições de ações. Em situações de brincadeira a criança desenvolve a intencionalidade e a inteligência. 
    \end{quotation}

    Sendo assim, oportunizar o jogo simbólico, ou a brincadeira de faz-de-conta para crianças de dois a seis anos é dever de uma prática educativa comprometida com sua aprendizagem e desenvolvimento, pois considerando o que defende \textcite[p.~87]{KISHIMOTO2009BrincarÉDiferente}, 

    \begin{quotation}
        O brincar é importante para a criança expressar significações simbólicas. Pelo brincar a criança aprende a simbolizar. Ao assumir papéis, ao usar objetos com outras finalidades para expressar significações, a criança entra no processo simbólico. O brincar auxilia o desenvolvimento simbólico. Mas não se trata de entender o símbolo como exercício ou cópia de letras e números em práticas de uso do brinquedo no ensino formal. 
    \end{quotation}

    Outro aspecto que deve ser considerado diz respeito ao processo de socialização e o envolvimento da criança na comunicação consigo mesmo, com os outros e com o mundo, estabelecendo relações sociais e construindo conhecimentos. 

    O documento curricular de base da Educação Brasileira, a BNCC, apresenta seis direitos de aprendizagem e desenvolvimento para a criança da Educação Infantil, com vistas às competências gerais da Educação Básica, esses direitos asseguram 

    \begin{quotation}
        [\dots] as condições para que as crianças aprendam em situações nas quais possam desempenhar um papel ativo em ambientes que as convidem a vivenciar desafios e a sentirem-se provocadas a resolvê-los, nas quais possam construir significados sobre si, os outros e o mundo social e natural \cite[p.~35]{BaNacCurEF2017}.
    \end{quotation}

    Dentre esses seis direitos, encontra-se,  

    \begin{quotation}
        Brincar cotidianamente de diversas formas, em diferentes espaços e tempos, com diferentes parceiros (crianças e adultos), ampliando e diversificando seu acesso a produções culturais, seus conhecimentos, sua imaginação, sua criatividade, suas experiências emocionais, corporais, sensoriais, expressivas, cognitivas, sociais e relacionais \cite[p.~36]{BaNacCurEF2017}.
    \end{quotation}

    A brincadeira possibilita a integração das crianças com o meio, explorando o espaço, se apropriando dele. Para Vygotsky, citado por \textcite{KISHIMOTO1999Jogo}, o que define a brincadeira é a situação imaginária criada pela criança. Além disso, devemos levar em conta que o brincar preenche necessidades que mudam de acordo com a idade. “Um brinquedo que interessa a um bebê deixa de interessar a uma criança mais velha. Dessa forma, a maturação dessas necessidades é de suma importância para entendermos o brinquedo da criança como uma atividade singular” \cite[p.~60]{KISHIMOTO1999Jogo}.

    As Diretrizes Curriculares Nacionais da Educação Infantil (DCNEI) definem “a criança como um sujeito histórico e de direitos, que interage, brinca, imagina, fantasia, deseja, aprende, observa, experimenta, narra, questiona e constrói sentidos sobre a natureza e a sociedade, produzindo cultura” \cite[p.~94]{DCN2009}. Sobre a importância das brincadeiras, as DCNEI consideram que essas experiências promovem o envolvimento da criança com o ambiente e a conservação da natureza, ajudando a elaborar conhecimentos. 

    Desse modo, a diversidade de ações deve ser contemplada na Educação Infantil, favorecida por um espaço amplo, facilitador, que seja flexível e adequando, da melhor forma possível, à atividade proposta, respeitando as individualidades. 

    Ao brincar, segundo os RCNEI \cite[p.~27]{RCN2001}, 

    \begin{quotation}
        [\dots] as crianças transformam os conhecimentos que já possuíam anteriormente em conceitos gerais com os quais brinca. Por exemplo, para assumir um determinado papel numa brincadeira, a criança deve conhecer algumas de suas características. Seus conhecimentos provêm da imitação de alguém ou algo conhecido, de uma experiência vivida na família ou em outros ambientes, do relato de um colega ou de um adulto. 
    \end{quotation}

    É importante destacar que as crianças brincam de forma espontânea em qualquer lugar e com qualquer coisa, mas há uma diferença entre uma brincadeira livre e outra intencionalmente planejada com o objetivo de favorecer a autonomia e o aprendizado no processo educativo, contribuindo com o desenvolvimento infantil. 


    \section{O olhar dos pais e responsáveis acerca da brincadeira na educação infantil}

    Diante da inquietação acerca da maneira como os pais e os responsáveis pelas crianças se manifestam em relação às brincadeiras no espaço da Educação Infantil, convidamos seis deles para colaborar com a pesquisa que resultou neste artigo. Esse grupo representa 30\% (trinta por cento) do total de pais ou responsáveis das crianças de uma turma da escola que tem como docente a pesquisadora e autora deste trabalho. 

    A referida escola atende crianças de quatro meses a cinco, seis anos, do berçário ao nível V. Sua estrutura física é composta, no térreo, por uma recepção, um escritório, quatro salas de aula, uma sala de berçário com banheiro, dois banheiros infantis, um banheiro social, uma brinquedoteca, uma sala de vídeo, um refeitório, uma cozinha, um parque e uma piscina. No primeiro andar, tem espaço coberto, para eventos, que é onde são realizadas as atividades motoras, mais um escritório, dois banheiros sociais e uma cozinha.  

    Um diálogo foi estabelecido levando em conta as falas dos colaboradores, que são os pais e os responsáveis pelas crianças, os autores estudados e as autoras do artigo, na tentativa de entender a concepção dos pais ou responsáveis, acerca do brincar no processo de desenvolvimento e aprendizagem no espaço escolar da Educação Infantil. 

    Para tanto, foi adotada uma pesquisa de natureza qualitativa, na qual, segundo \textcite{GERHARDTAndSILVEIRA2009Métodos}, não se preocupa com representatividade numérica, mas sim, com o aprofundamento da compreensão de um grupo social de uma organização etc. A pesquisa qualitativa preocupa-se, portanto, com aspectos da realidade que não podem ser quantificados, centrando-se na compreensão e explicação da dinâmica das relações sociais. 

    A pesquisa qualitativa foi do tipo exploratória que visa a descoberta, o achado, a elucidação de fenômenos ou a explicação daqueles que não eram aceitos apesar de evidentes \cite{GONÇALVES2014Uma}. 

    Um questionário foi aplicado e, nele, os pais puderam expressar como percebem a brincadeira na escola e se consideram que elas são importantes no desenvolvimento e na aprendizagem de seus filhos. 

    Ao serem questionados se a escola de seus filhos favorece a vivência das brincadeiras, de modo sistemático, todos eles reconhecem que sim. É possível afirmar que isso ocorre porque a referida escola considera relevante e comunga com o que apresenta \textcite[p.~142]{TOURRETTE2009Introdução}, 

    \begin{quotation}
        Considera-se que, se não brinca, a criança está doente; portanto, a brincadeira parece ser indispensável para o seu desenvolvimento. É uma atividade geral que pode se desenrolar em qualquer lugar e tempo. Ela distingue-se da realidade no sentido de que as sequencias de comportamento não vão até seu termo (brinca-se em matar o outro, mas, na realidade, ele não morre), nem se orienta para o objetivo preciso (a criança brinca não para alcançar um resultado, mas para brincar). 
    \end{quotation}

    Por meio da brincadeira e seus simbolismos as crianças são capazes de aprender e reelaborar o universo ao seu redor. Também desenvolvem uma vida imaginária por meio de metáforas e de fantasias, brincando com temas próprios de sua realidade. Além disso, as brincadeiras funcionam como atividade prazerosa em si mesma e levam satisfação e realização pessoal para quem brinca, pois esta constitui uma das formas de relacionamento e recriação do mundo.  

    De acordo com \textcite[p.~21]{MALUF2003Brincar},  

    \begin{quotation}
        a criança é curiosa e imaginativa, está sempre experimentando o mundo e precisa explorar todas as suas possibilidades. Ela adquire experiência brincando. Participar de brincadeiras é uma excelente oportunidade para que a criança viva experiências que irão ajudá-la a amadurecer emocionalmente e aprender uma forma de convivência mais rica. 
    \end{quotation}

    As crianças têm uma enorme capacidade de exercitar a imaginação e a criatividade, e os brinquedos e as brincadeiras se configuram como favorecedores desse processo construtivo. Atuar sobre o mundo real, vivido, transformando-o ou preservando-o é algo que as brincadeiras possibilitam de maneira especial. Sobre essa condição do brincar, \textcite{KISHIMOTO1999Jogo} considera que,  

    \begin{quotation}
        o brinquedo coloca a criança na presença de reproduções: tudo o que existe no cotidiano, a natureza e as construções humanas. Pode-se dizer que um dos objetivos do brinquedo é dar à criança um substituto dos objetos reais, para que possa manipulá-lo \cite[p.~18]{KISHIMOTO1999Jogo}. 
    \end{quotation}

    Considerando o que dizem as duas autoras mencionadas e o que é observado no cotidiano da escola pelas autoras deste artigo, é possível reconhecer que as brincadeiras favorecem à criança a diferenciação do seu mundo interior, composto pelas fantasias, desejos e imaginação, do seu eu exterior, que é a realidade vivida em comunhão com seus familiares, colegas de escola e professores. 

    Ao expor tais reflexões e na caminhada em busca de alcançar o objetivo proposto pela pesquisa, ou seja, investigar a concepção dos pais e/ou responsáveis por crianças com idades entre dois e três anos acerca do brincar no processo de desenvolvimento e de aprendizagem na Educação Infantil. Estes foram questionados acerca de como percebem a brincadeira na escola.  

    As respostas obtidas foram as seguintes:

    \begin{enumerate}
        \item Na escola meu filho é bem interessado, gosta muito das brincadeiras e ele aprende muito. 
        \item Quando vejo minha filha cantando e brincando, imitando a professora em casa com as bonecas, quando ela traz atividades realizadas através de brincadeiras. 
        \item Eu percebo que cada dia mais ele conhece e vem com histórias e brincadeiras diferentes para nos mostrar com entusiasmo. 
        \item Quando ela está em casa ela brinca das mesmas coisas que brinca na escola. 
        \item Como uma ferramenta importante na ajuda do aprendizado. 
        \item Quando percebo minha neta querendo brincar das mesmas brincadeiras da escola, que através das brincadeiras educativas as crianças desenvolvem mais. 
    \end{enumerate}

    Diante das respostas fornecidas pelos pais e responsáveis, é possível analisar que elas se opõem ao que era esperado inicialmente, pois, a experiência da pesquisadora enquanto docente, em outras turmas trabalhadas, levava-a a considerar que os pais tinham uma visão de que brincar na Educação Infantil provocava dispersão e era perda de tempo. Nos depoimentos informais, orais, era comum ouvir que eles não entendiam o seu valor para o desenvolvimento integral de seus filhos, levando em conta os aspectos cognitivo, social, afetivo e psicomotor. 

    É importante observar o que afirmam os pais e responsáveis pelas crianças e estabelecer uma relação com o que apresenta a Base Nacional Comum Curricular (BNCC) em seus cinco campos de experiência, que “constituem um arranjo curricular que acolhe as situações e as experiências concretas da vida cotidiana das crianças e seus saberes, entrelaçando-os aos conhecimentos que fazem parte do patrimônio cultural.” \cite[p.~38]{BaNacCurEF2017}. 

    Os campos de que trata a Base Nacional Curricular Comum são: O eu, o outro e o nós; Corpo, gestos e movimentos; Traços, sons, cores e formas; Escuta, fala, pensamento e imaginação e; Espaços, tempos, quantidades, relações e transformações. Esses campos apresentam objetivos de aprendizagem e desenvolvimento específicos e, em vários deles, é possível encontrar a presença da brincadeira. 

    Aqui serão destacados alguns desses objetivos de aprendizagem de que trata a BNCC no que diz respeito às brincadeiras, focando apenas na faixa etária que corresponde ao das crianças alvo deste estudo.  

    No campo de experiência \textit{O eu, o outro e o nós}, é possível encontrar o seguinte objetivo de aprendizagem e desenvolvimento, “Respeitar regras básicas de convívio social nas interações e brincadeiras.” \cite[p.~44]{BaNacCurEF2017}. O campo de experiência \textit{Corpo, gestos e movimentos}, estabelece a necessidade de “Apropriar-se de gestos e movimentos de sua cultura no cuidado de si e nos jogos e brincadeiras” \cite[p.~45]{BaNacCurEF2017}.  

    No Campo Traços, sons, cores e formas, é possível encontrar, “Utilizar diferentes fontes sonoras disponíveis no ambiente em brincadeiras cantadas, canções, músicas e melodias” \cite[p.~46]{BaNacCurEF2017}.

    As respostas apresentadas pelos pais e responsáveis acerca da importância dos brinquedos e das brincadeiras, tais como: brincar é algo educativo; as crianças desenvolvem-se, aprendem mais; reproduzem o aprendizado em casa com a família; uma ferramenta importante ajudando no aprendizado; é interessante pois aprendem com gosto, prazer, espontaneidade contribuindo para o estimulo do desenvolvimento e a capacidade de aprendizagem das crianças, foi realmente surpreendente uma vez que, inicialmente, os discursos que utilizavam no cotidiano contrariavam o que foi apresentado por escrito nos questionários. 

    Acredita-se que as respostas dadas pelos pais às questões da entrevista, são decorrentes da preocupação da equipe gestora e pedagógica da escola, que ao identificar a necessidade de esclarecer acerca do valor dos brinquedos e das brincadeiras no processo de desenvolvimento e aprendizagem das crianças, promoveu reuniões pedagógicas para os devidos esclarecimentos, além do uso diário da agenda onde são apresentados os objetivos e procedimentos desenvolvidos em sala de aula. 

    Sendo assim, os pais tomam conhecimento dos projetos, das vivências realizadas e das atividades desenvolvidas que são construídas em conjunto pela equipe pedagógica e pelos docentes da escola, ao mesmo tempo em que vão percebendo os avanços na aprendizagem e no desenvolvimento de seus filhos. 

    Nesse sentido o RCNEI \citeyear{RCN2001} enfatiza o papel significativo das brincadeiras e orienta que o ato de educar significa propiciar situações de cuidados e brincadeiras organizadas em função das características infantis, de forma a favorecer o desenvolvimento e a aprendizagem. O documento reconhece que, 

    \begin{quotation}
        Por meio das brincadeiras, os professores podem observar e constituir um a visão dos processos de desenvolvimento das crianças em conjunto e de cada uma em particular, registrando suas capacidades de uso das linguagens, assim como de suas capacidades sociais e dos recursos afetivos e emocionais que dispõem \cite[p.~28]{RCN2001}.
    \end{quotation}

    Nos dias atuais, educadores se deparam com diversos problemas que interferem diretamente no processo de ensino-aprendizagem, algumas vezes em decorrência da ausência de limites das crianças ou mesmo pela falta de interesse dessas. As aulas tendem a se tornarem monótonas e sem motivação e desde bem cedo, muitas delas passam a frequentar a escola por obrigação, pelas necessidades que seus pais têm de deixá-las para poder trabalhar. 

    As brincadeiras, diante dessa realidade, podem trazer alegria e encantamento para o ambiente escolar o que poderia favorecer o gosto e o desejo de frequentar a escola para aprender e se desenvolver. 

    Na pesquisa realizada, os pais reconhecem, de maneira unânime, que as brincadeiras podem estimular o desenvolvimento e a aprendizagem das crianças. Esse resultado é atribuído ao fato de os pais participarem da vida escolar dos seus filhos, como por exemplo, os encontros pedagógicos e o diálogo com os professores através da agenda diária.  

    O Ministério da Educação elencou algumas recomendações para os pais e responsáveis por crianças em creches e pré-escolas, considerando que eles têm o direito de acompanhar a educação de seus filhos e que, para tanto, devem participar ativamente da vida escolar das crianças uma vez que isso interfere positivamente na qualidade do ensino. 

    Em um documento intitulado, \textit{O que verificar em relação à educação de sua criança se ela frequenta uma creche ou pré-escola}, é possível encontrar, no item 1, dentre outros, os seguintes pontos: a) A instituição tem proposta pedagógica em forma de documento?; b) Reuniões e entrevistas com familiares são realizadas em horários adequados à participação das famílias?; c) Há reuniões com familiares pelo menos três vezes por ano?; d) Os familiares recebem relatórios sobre as vivências, produções e aprendizagens pelo menos duas vezes ao ano?; e) A instituição permite a entrada dos familiares em qualquer horário? 

    No documento supracitado, no item 2, intitulado \textit{O que os familiares podem verificar com a criança sobre o atendimento na educação infantil}, é possível encontrar, dentre outros os seguintes pontos: a) Pergunte à criança o que ela mais gostou de fazer naquele dia; b) Incentive à criança a contar e a narrar situações vividas na instituição; c) quais brincadeiras aconteceram. 

    Diante do exposto, se confirma a necessidade fazer com que a relação de parceria entre os pais ou responsáveis pelas crianças e os que fazem a escola, seja dinâmica, viva e real, para que os pequeninos possam se desenvolver, aprender e exercer a sua cidadania.  


    \section{Considerações e recomendações}

    Concluir este artigo proporcionou o reconhecimento de que a pesquisa é uma fonte inesgotável de aprendizagem e a descoberta de que ela pode ser surpreendente. Ao iniciar o projeto de pesquisa, hipóteses eram levantadas e algumas ‘certezas’ adotadas e passíveis apenas de confirmação, mas eis que de maneira inesperada as respostas encontradas foram totalmente inusitadas, desconstruindo uma ideia que parecia certa. 

    Os pais e responsáveis demonstram, através do instrumento de coleta de dados, entender e valorizar a vivência das brincadeiras no processo de aprendizagem e desenvolvimento de seus filhos, o que remete essas pesquisadoras a novas inquietações, como por exemplo, por que quando esses mesmos pais tratam das questões pedagógicas, avaliam o brincar como mero Passa-Tempo? Por que parecem considerar que só houve ensino se o caderno ou o livro didático tiverem sido usados? Aqui é possível enxergar novas possibilidades de pesquisa, já que esse é um tema instigador e inquietante para as autoras. 

    Diante dessas questões fica a certeza de que adentrar no mundo da pesquisa estimula o educador a pesquisar cada vez mais, na tentativa de entender os porquês inquietantes da prática educativa e das relações que se estabelecem no espaço escolar. 

    A pesquisa desenvolvida foi de suma importância para o aprendizado destas autoras, que estão sempre buscando novos conhecimentos para melhorar suas práticas pedagógicas. Sendo assim, aqui é defendido que os/as profissionais da Educação Infantil tenham uma formação que oportunize a vivência de jogos, brincadeiras e outras atividades potencialmente lúdicas para que, diante do que sentem, possam desenvolver trabalhos dessa natureza em sala de aula. 

    Sendo um adulto brincante o educador infantil desenvolverá empatia com as crianças e segundo \textcite[p.~90]{ANNING2005brincar}, “O professor precisa estar intimamente envolvido com as crianças enquanto elas trabalham e brincam, ser capaz de ouvir em vez de falar para as crianças e de observar e analisar as evidências das aprendizagens”. 

    Também advogamos a necessidade de que educadores infantis se apropriem dos documentos oficiais específicos desse nível de ensino, dentre eles os Referenciais Curriculares Nacionais para a Educação Infantil (2001), as Diretrizes Curriculares Nacionais para a Educação Infantil (2009) e a Base Nacional Comum Curricular (2017) para que possam desenvolver o seu trabalho de maneira que as crianças alcancem o que se espera de cada uma delas em cada etapa de seu processo de aprendizagem e desenvolvimento. 

    Enfim, é possível perceber que os pais e os responsáveis pelas crianças de dois e três anos, alvos da pesquisa realizada, apresentam uma compreensão bastante satisfatória do que representa a brincadeira no processo de desenvolvimento e de aprendizagem de seus pequenos ao mesmo tempo em que fica evidenciada a importância do diálogo estabelecido entre a instituição educativa e as famílias, através das reuniões pedagógicas e da agenda de uso diário. 

    \nocite{LDB1998}
    \nocite{MEC_SN_OQueVerificar}

    \printbibliography[heading=subbibliography,notcategory=fullcited]

    \label{chap:concepcao-pais-respoend}

\end{refsection}
