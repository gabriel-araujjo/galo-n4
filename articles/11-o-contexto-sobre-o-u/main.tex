\begin{refsection}
    \renewcommand{\thefigure}{\arabic{figure}}
    
    \chapterOneLine{O contexto sobre o uso de substâncias lícitas e ilícitas em Caicó, Rio Grande do Norte}
    \label{chap:contextosubs}
    
    \articleAuthor
    {Allyson Iquesac Santos de Brito}
    {Graduado em História pela Universidade Federal do Rio Grande do Norte --- UFRN, campus CERES, em Caicó-RN. ID Lattes: 7011.1103.8491.3539. ORCID: 0000-0002-7249-632X. E-mail: allysonkesac@hotmail.com.}

    \articleAuthor
    {Helder Alexandre Medeiros de Macedo}
    {Historiador e professor de História. Atualmente é Professor do Departamento de História do CERES, da Universidade Federal do Rio Grande do Norte, lecionando no Campus de Caicó. Atua como Professor Permanente do Programa de Pós-Graduação em História do CERES-UFRN/Mestrado em História dos Sertões e como Colaborador do Programa de Pós-Graduação em História do CCHLA-UFRN/Mestrado em História e Espaços. ID Lattes: 8883.6377.0370.4518. ORCID: 0000-0002-5967-7636.}
    
    \begin{galoResumo}
        \marginpar{
            \begin{flushleft}
            \tiny \sffamily
            Como referenciar?\\\fullcite{SelfBritoAndMacedo2021}\mybibexclude{SelfBritoAndMacedo2021}, p. \pageref{chap:contextosubs}--\pageref{chap:contextosubsend}, \journalPubDate{}
            \end{flushleft}
        }
        O presente trabalho tem como norteamento analisar as medidas proibicionistas de substâncias e seus impactos na realidade de Caicó --- sertão norte-rio-grandense, no ano de 2017. Por entendermos que o sertão é um espaço que não está preso a um determinado tempo, o analisaremos através da contemporaneidade. Desse modo, contemplando o conceito de ``Sertão Contemporâneo''. Afim de construir um trabalho historiográfico sobre a contemporaneidade no Rio Grande do Norte, teremos como alguns objetivos, identificar quais elementos são desdobrados na sociedade caicoense em 2017, a partir do Proibicionismo, exercido pelo Estado brasileiro; problematizar a garantia de direitos sociais, de indivíduos, e civis perante a legislação e movimentos sociais. Como aporte metodológico, utilizaremos da Análise do Discurso para observar os desdobramentos encontrados nas respostas de um questionário disponibilizado online nas redes sociais do autor, nos apresentam. O questionário obteve sessenta (60) respostas referentes ao uso e consumo de substâncias consideradas atualmente como lícitas e ilícitas, como também, atenta para dados pessoais como orientação sexual, identidade de gênero, etnia, entre outros dados, se estes sofreram com algum tipo de repressão policial, de instituições privadas e/ou públicas e abuso de autoridade.
    \end{galoResumo}
    
    \galoPalavrasChave{Sertão norte-rio-grandense. Proibicionismo. Discurso. Contemporaneidade.}
    
    \begin{otherlanguage}{english}
    
    \fakeChapterOneLine
    {The context of legal and illegal substances usage in Caicó, Rio Grande do Norte}

    \begin{galoResumo}[Abstract]
        This work aims to analyze the prohibitionist measures of substances and their impacts on the reality of Caicó --- Rio Grande do Norte sertão, in 2017. As we understand that the sertão is a space that is not tied to a certain time, we will analyze it through contemporaneity. In this way, contemplating the concept of ``Contemporary Sertão''. In order to build a historiographic work on contemporary times in Rio Grande do Norte, we will have as some of the objectives, to identify which elements are unfolded in Caico's society in 2017, based on Prohibitionism the Brazilian State exercised; to problematize the guarantee of social, individual, and civil rights before legislation and social movements. As a methodological contribution, we will use Discourse Analysis to observe the developments found in the responses to a questionnaire available online on the author's social networks, they present us. The questionnaire obtained sixty (60) responses regarding the use and consumption of substances currently considered to be legal and illegal, as well as looking at personal data such as sexual orientation, gender identity, ethnicity, among other data, if they suffered from any type of police, private and/or public repression and abuse of authority.
    \end{galoResumo}
    
    \galoPalavrasChave[Keywords]{Sertão norte-rio-grandense.  Prohibitionism. Discourse. Contemporaneity.}
    \end{otherlanguage}

    \section{Introdução: a historicidade do proibicionismo a partir de 1960}

    Iniciamos realçando a importância da Constituição de 1988 que é estabelecida juntamente com a democracia e eleições diretas após a nação brasileira sofrer a cassação de direitos sociais, fundamentais e políticos com a Ditadura Militar por vinte e um (21) anos. Pretendemos evidenciar o proibicionismo, como um dos problemas do tempo presente que deve e merece atenção e discussões diante de seus contemporâneos, pois, não há como solucioná-los através de ideias e ideais de líderes do passado --- os quais demonstram com o tempo histórico, que suas medidas são paliativas e não pragmáticas e eficientes.

    Do olhar historiográfico acrescentamos a perspectiva do uso de substâncias, atualmente, consideradas ilícitas no Brasil. Os Estados Unidos também entram no debate deste trabalho por entendermos que as medidas governamentais deste país, sejam elas pautadas pela saúde, economia, pelos interesses políticos, entre outros, pelo racismo enraizado nos discursos de poder, são essenciais para percebermos como a conhecida ``guerra às drogas'' inicia-se e impulsiona, através de influência, países como o Brasil.

    A partir de 1960 encontramos nas resistências das juventudes norte-americanas, o fim pela guerra, principalmente, a do Vietnã. Assim como, o fortalecimento da contracultura organizada por movimentos sociais: das pessoas negras, LGBTQ+, das mulheres com o feminismo, e dos \textit{hippies} \cite{Grant2014Historia}.

    Chamamos a atenção para as medidas proibicionistas governamentais dos Estados Unidos por sua estreita relação ainda com o Brasil no período de 1970: Em plena ditadura militar brasileira enquanto na presidência norte-americana, estava Richard Nixon, declarador da guerra às drogas. Pautando a medida através do discurso de que as drogas deveriam ser consideradas inimigas da nação, Nixon clamou por mudanças. Ao considerá-las inimigas, novos alvos foram feitos como o combate ao tráfico e seus traficantes, o que desencadeou em processos judiciais e militares de enfrentamento ao uso de substâncias, de usuários, de suas autonomias como cidadãos e seus direitos fundamentais como humanos.  

    Por isso, analisaremos o conceito de discurso, trabalhando também sobre o interesse central de nossas pesquisas que são os sujeitos propagadores e receptores do ato de falar --- carregado de simbologia, constituídos ``por discursos historicamente produzidos e modificados; assim como o discurso, o sujeito está em constante produção, é marcado por movências.'' \cite[p.~16]{Fernandes2012Discurso}.

    Analisar o discurso nos propõe perceber essas mudanças constantes. Através do site oficial online da CNN, encontramos uma declaração de John Ehrlichman, ex-chefe de política interna de Nixon, ao jornal Dan Harun:

    \begin{quotation}
        Nós sabíamos que não poderíamos tornar ilegal ser contra a guerra (às drogas) ou negros, mas (poderíamos) fazer com que o público associasse os hippies à maconha e aos negros com heroína. E então criminalizando ambos fortemente, nós poderíamos perturbar essas comunidades.\footnote{Disponível em: \url{https://amp.cnn.com/cnn/2016/03/23/politics/john-ehrlichman-richard-nixon-drug-war-blacks-hippie/index.html?__twitter_impression=true}, acesso em 27 abr. 2019.}
    \end{quotation}

    Levando em consideração a declaração de John Ehrlichman, e a ação que o presidente estadunidense resolveu ter diante de um problema social complexo como o da proibição ou não de substâncias, ambas são analisadas pelo caráter político, racial, de classe e de inviabilização dos direitos humanos.

    Atentamos para o discurso feito por um presidente de uma nação poderosa militarmente e economicamente, para perceber o privilégio e autoridade que o mesmo exerce, pois, assim, pode-se realizar procedimentos externos de controle e delimitar o discurso. Neste caso, controlar e distinguir americanos de imigrantes, entre outras distinções, a de americanos brancos, de americanos negros.

    \begin{quotation}
        Por meio da interdição, são estabelecidos os direitos e as proibições em relação ao ato de falar e também ao que pode ser falado. O objeto do discurso define, assim, o lugar do dizer e o direito de falar privilegiado ou exclusivo de algum(ns) sujeito(s) em detrimento de outro(s). \cite[p.~48]{Fernandes2012Discurso}.
    \end{quotation}

    Tendo em vista os números relacionados ao encarceramento no período \cite[p.~182]{ColetivoDar2016Dichavando} e a fala do ex-chefe de política interna do governo supracitado, partimos da análise dos dados, para adentrarmos ao sujeito de poder que exerce sua autonomia perante sua vida e corpo --- Foucault denomina essa ação de biopoder \cite{Fernandes2012Discurso}, o qual se desdobra nos sujeitos em diversas formas encontradas, seja de um sujeito/governo sobre outro, seja as técnicas e cuidados de um sujeito sobre si.

    O discurso vem carregado de conceitos e palavras chaves que muitas vezes nos fazem nos perder em meio a sua complexidade. Assim, deixamos de nos aprofundar cientificamente, e tomamos tais discursos como verdadeiros, únicos e absolutos sobre determinado(s) assunto(s) e esquecemos que estes são criados por pessoas e desencadeados em outras.  

    Os impactos analisados por consequência do proibicionismo imposto, constituem interesses do âmbito religioso, judiciário, legislativo, executivo, farmacêutico, escolar, familiar, medicinal, penitenciário, social, entre outros, do próprio indivíduo com sua autonomia sob sua vida privada. 

    Por isso, acreditamos na multidisciplinaridade --- entendida como um campo que abarca diversas disciplinas como o Direito, a própria História, Ciências Sociais, Economia, Serviço Social, entre outras, áreas da saúde como Enfermagem e Medicina ---, pois observamos que

    \begin{quotation}
        o poder organiza-se em torno da vida; há, portanto, uma biopolítica investida de biopoderes [\dots] o poder implica também liberdade e possibilidade de resistência, cuja existência só é possível em sujeitos livres. \cite[p.~52]{Fernandes2012Discurso}.
    \end{quotation}

    \section{O proibicionismo em Caicó-RN}

    As interrogações iniciais que pautam o tema e a localidade escolhida, diz respeito às experiências pessoais do autor em 2014\footnote{No ano em questão, o autor estava no último ano do ensino médio e passou por uma das primeiras perdas familiares. O seu primo faleceu após ser covardemente assassinado por policiais.}, e assim, ao observar que existem áreas nas vidas dos cidadãos em que o Estado\footnote{Composto por burguesia, exército, legislação, administração e impostos. \cite{Koselleck2006Futuro}.} brasileiro, regido por uma Constituição Federal de 1988 --- pautando a igualdade entre os seus perante a Lei ---, pode causar mais problemas para sua sociedade do que propriamente resolvê-los. Mesmo sendo o grande controlador da vida privada de seus cidadãos, por assim vivermos em uma democracia, o Estado não pode ser um controlador totalitário dessas pessoas, e sim, parciais, mantendo o papel executivo, legislador e judiciário.

    Pretendemos identificar quais elementos são desdobrados na sociedade caicoense no ano de 2017, a partir da proibição, pelo Estado, de substâncias consideradas ilícitas. A percepção das medidas tomados pelo Estado brasileiro sobre a temática proposta será discutida através do conceito de proibicionismo --- entendido como um discurso ``narcofóbico'', assim proposto por Henrique \textcite{Carneiro2018Drogas}. Ou seja, um discurso antidrogas pautados por interesses de grupos sociais, sejam esses políticos, econômicos, e medicinais interligados à historicidade de acontecimentos de cunho nacional e internacional.

    Com a graduação iniciada em 2016 em licenciatura em História pela UFRN, campus CERES, na cidade de Caicó-RN, o autor esteve em contato com a história da cultura, da vida pública e privada, dos direitos fundamentais dos indivíduos, dos movimentos sociais como o da etnia negra, indígena, e de diversas outras do Ocidente ao Oriente. Com o olhar acadêmico direcionado ao proibicionismo para compreender a dualidade existente entre as substâncias, estas, divididas em lícitas e ilícitas, nos demonstrou o impacto causado pela nomenclatura e principalmente pela legislação brasileira, a qual vamos nos ater. 

    A partir dos fatos históricos apontados, nos debruçaremos em como a proibição de substâncias afetam a sociedade através do caráter social. Por meio dessa perspectiva, discutiremos os aspectos apontados e os desdobramentos das medidas proibicionistas a partir do tempo presente na cidade de Caicó, localizada no Rio Grande do Norte, Brasil.  

    Desse modo, evidenciamos um questionário disponibilizado online nas redes sociais do autor, por volta do ano de 2017, cujo os objetivos seriam o de identificar sujeitos anônimos usuários de substâncias lícitas e ilícitas na cidade supracitada. Assim como, colher dados e histórias pessoais referente ao uso dessas substâncias que agora servem como fontes para a construção deste artigo\footnote{Em 2017, o autor estava no seu segundo ano de graduação em História e percebeu que, pelo curso, conseguiria trabalhar com o tema da proibição intercalando com a realidade em que está vivendo na contemporaneidade da cidade em que mora (Caicó-RN). Sob orientação do professor Dr. Helder Alexandre de Medeiros Macedo, os primeiros passos foram dados para a construção final deste artigo, sendo a principal fonte utilizada, o questionário online.}.

    \section{Resultados}

    Para aproximar as informações obtidas de cunho internacional e nacional pelos estudos expostos acima, vamos intercalar a bibliografia estudada à história local e aos resultados de uma pesquisa realizada em Caicó, no Rio Grande do Norte no ano de 2017, por meio de um questionário online disponibilizado nas redes sociais dos autores que obteve sessenta (60) respostas. 

    A metodologia utilizada para compreender as respostas é a de Análise do Discurso (2005), o qual ``requer fazer aparecer os aspectos referentes à forma de existência social dos sujeitos tendo em vista os aspectos linguísticos, sociais e históricos que engendram sua constituição nas formações discursivas, na formação e transformação desses sujeitos e objetos que constituem''. \cite[p.~30]{Fernandes2012Discurso}.

    O questionário foi formulado com a finalidade de identificar e/ou mapear indivíduos que fizessem o uso de substâncias lícitas, ilícitas, se estes se consideram usuários (entendidos por fazerem o uso diário ou não as substâncias), suas relações com o ambiente jurídico, político e social no que diz respeito à repressão e opressão advinda de corporações policiais e instituições privadas, assim como, abuso de autoridade na cidade de Caicó-RN.   

    Outro importante objetivo que adquirimos através do questionário, foi o de também, aproximar a realidade social presente ao discurso acadêmico, e por isso, a Análise do Discurso se faz presente na metodologia a fim de discutir sobre a autonomia dos indivíduos, suas práticas, vivências, e o uso de substâncias. 

    Feita a discussão sobre a aproximação do discurso acadêmico ao popular, e das inovações culturais, políticas, entre tantas outras que carregam o sertão contemporâneo, apresentamos os dados sobre a pesquisa realizada em 2017, disponibilizada online, criada e executada pelo autor, para observar como a teoria se insere no ambiente da prática em moradores de cidades identificadas em que os mesmos (os usuários de substâncias ilícitas e lícitas), se encontram.  

    Portanto, devemos discutir as generalizações e não as receber como imutáveis, recaídas para o senso comum, mesmo que haja certos padrões que se repetem em relação a algum tema/problemática.  

    Mapeando a localidade dos entrevistados, 39 pessoas moram na cidade de Caicó, no Rio Grande do Norte. Ainda no interior do estado potiguar, 2 se encontram em Acari, 1 em Cruzeta, 2 em Currais Novos, 1 em Florânia, 1 em Jardim (sem identificar se seria Jardim do Seridó ou Jardim de Piranhas), 1 em Jardim do Seridó, 1 em São João do Seridó, 1 em Macaíba. Para a capital Natal tivemos 4 pessoas. Para territórios exteriores ao do Rio Grande do Norte, 1 no estado do Maranhão, 1 em Pedra Lavrada no estado da Paraíba, 1 em Piracicaba - São Paulo, 1 em São Paulo (sem identificar se seria o estado ou a capital), 1 em Salvador, 1 em Santa Luzia no estado da Paraíba, e por fim, 1 em Teófilo Antoni, Minas Gerais. 

    Partindo para a análise étnico-racial, 33 pessoas, que correspondem a 55\% do total da pesquisa, identificaram-se como brancas; 20 como pardas, enquanto 6 como negros, e 1 de etnia indígena. Como identidade de gênero tivemos 27 mulheres, 32 homens, 1 transgênero. Os entrevistados também foram questionados por sua orientação sexual a qual se resume a: 16 bissexuais, 13 homossexuais, 26 heterossexuais, 4 pansexuais, e 1 não definido.

    A repressão policial e de instituições também foram questionadas na pesquisa, com as seguintes situações: perguntamos aos entrevistados se sofreram abuso de autoridade vinda de um policial, e obtivemos que 34 pessoas responderam que não sofreram abuso. Em contrapartida, 26 pessoas afirmaram ter sofrido esse tipo de abuso vindo de membros das corporações policiais. Quando abrangemos a questão para a repressão vinda de um civil, autoridade ou instituição pública ou privada, 39 pessoas afirmam sentirem a repressão, enquanto 19 pessoas não chegaram a terem contato com esse autoritarismo e, 3 pessoas não conseguem identificar se passaram por esse caso. 

    Outro tipo de controle do discurso e exclusão, traz, não somente regras de entre os momentos em que alguém pode falar, mas sim, há rejeição, distinção e separação do que é um discurso considerado ``verdadeiro/normal'', de um discurso ``sem razão/anormal''. O exemplo dado é o da razão, tida como verdade e dita pelos ``normais'' em oposição à loucura, aos loucos. Logo, ``cabe à verdade, por exemplo, definir a loucura, identificar o louco, e, por conseguinte, justificar a interdição''. \cite[p.~48]{Fernandes2012Discurso}.

    Tendo o poder em todas as instâncias entre os sujeitos, o mesmo tanto implica e/ou requer a resistência. Por isso, acreditamos que as pessoas que utilizam de alguma substância ilícita em um contexto estatal proibicionista, como é o caso dos entrevistados, implica em ser um ato de resistência.

    \begin{quotation}
        ``Procurando focalizar o poder em micro instâncias, Foucault refere-se a formas de oposição ao poder, isto é, formas de resistência que constituem lutas antiautoritárias [\dots] Todas essas lutas contestam formas de poder e têm lugar no cotidiano dos indivíduos, pois são justamente o que os caracteriza em termos identitários e os tornam sujeitos. São também lutas contra a sujeição, contra formas de subjetivação e contra a submissão.'' \cite[p.~56--57]{Fernandes2012Discurso}. 
    \end{quotation}

    As pessoas entrevistadas no questionário online variam de faixa etária entre 16 e 29 anos. Entre elas, pelo menos duas começaram a utilizar das substâncias psicoativas a partir dos doze anos de idade enquanto treze pessoas iniciaram aos 14 anos, e onze pessoas iniciaram aos 15 anos. Nos referimos às últimas faixas etárias citadas com atenção para identificar que o proibicionismo, seus métodos de pertencimento no sistema de guerra às drogas, e as próprias substâncias chegam aos menores de idade com maior rapidez do que aos de maioridade penal, logo, demonstrando que tais medidas paliativas são ineficientes. 

    Quando questionados quais as primeiras substâncias psicoativas ilegais que utilizaram, obtemos a contagem de 49 pessoas para a maconha, 22 para o loló, 4 para cocaína, 7 para ecstasy, 1 para crack e 1 para uma outra substância que não estava entre as listadas acima, as quais estavam designadas para múltiplas escolhas.  

    Contudo, em busca de compreender se os entrevistados se consideram usuários diários ou não de produtos ilícitos, 50\% declararam que são usuários, enquanto 40\% afirmam que não são; 10\% ficaram em dúvida. Quando a pergunta é direcionada para os produtos lícitos como cigarro e álcool, o número de usuários diários em comparação ao ilícito aumenta 8\%, enquanto os não-usuários diminuem 9\%, e a dúvida sobre a questão permanece estável com 10\%.

    Logo, ``interessa ao analista do discurso refletir sobre como essas relações tão complexas integram os discursos, asseveram a constituição do sujeito discursivo e apontam para construções identitárias próprias aos sujeitos.'' \cite[p.~59]{Fernandes2012Discurso}.

    \section{Discussão: O sertão contemporâneo em destaque}

    As observações feitas sobre o espaço e tempo a serem trabalhados nos oferecem um olhar diferenciado tanto para a historiografia quanto para o conhecimento popular de que as complicações da legislação proibicionista não está intrinsicamente interligado somente aos grandes centros urbanos. 

    O estudo histórico aqui desenvolvido está ``muito mais ligado ao complexo de uma fabricação específica e coletiva do que estatuto de efeito de uma filosofia pessoal ou à ressurgência de uma `realidade' passada. É o produto de um lugar.'' \cite[p.~57]{Certeau1982Escrita}.

    As universidades devem ser defendidas como um lugar feito para discussões e debates sobre diferentes ideias baseadas pela expansão cientificista, longe da neutralidade e do silêncio. A dupla função da Universidade está interligada à discursos que são e que não são permitidos aos debates. Desse modo, ao excluí-los, fica representado ``o papel de uma censura com relação aos postulados presentes (sociais, econômicos, políticos) na análise.'' \cite[p.~63]{Certeau1982Escrita}.

    Não está mais para a História e para o ofício do historiador, ser construída/construir a narrativa apenas a partir do centro, de um modo totalizante, global. Assim, devemos observar outros elementos, fontes e fatos históricos sem cair nas generalizações, pois, tanto a História quanto o historiador mantêm sua essência: a de serem críticos \cite{Certeau1982Escrita}.

    Construiremos nossa narrativa histórica ao aproximá-la da realidade do discurso regionalista para começarmos o debate sobre a crise ``dos padrões tradicionais de sociabilidade que possibilitaram a emergência de um novo olhar em relação ao espaço, uma nova sensibilidade social em relação à Nação.'' \cite[p.~52]{AlbuquerqueJr2009Invencao}.

    Neste caso, os códigos de sociabilidades diante do século XIX e início do XX, em relação ao proibicionismo, devem ser questionadas na contemporaneidade a fim de percebermos se ainda são as melhores medidas diante das exigências sociais em detrimento da realidade, pois, buscamos nas partes, nas escalas territoriais-políticas-socais menores, ``a compreensão do todo, já que se vê a nação como um organismo composto por diversas partes, que deviam ser individualizadas e identificadas.'' \cite[p.~53]{AlbuquerqueJr2009Invencao}.

    Os usuários de substâncias ilegais estão sendo colocados às margens do centro irradiador de poder: sua autonomia e direitos estão sendo violados perante suas decisões sobre o seu corpo e vida, enquanto usuários de substâncias lícitas que muitas vezes também consomem substâncias ilícitas, mantêm suas autonomias intactas. Por isso, ``o discurso regionalista não pode ser reduzido a enunciação de sujeitos individuais, de sujeitos fundantes, mas sim a sujeitos institucionais.'' \cite[p.~61]{AlbuquerqueJr2009Invencao}. 

    Adentramos ao século XX, como o século definidor de um proibicionismo que atenta não somente para a proibição do contato de sujeitos com as drogas, mas também, que confronta de forma bélica, repressiva e autoritária tanto usuários quanto os comerciantes (taxados posteriormente de traficantes). Pouco menos de trinta anos após a abolição da escravidão no Brasil, em 1915, a maconha estava associada como uma vingança dos negros (africanos) sobre os brancos civilizados: Uma evidência do racismo estrutural em que o século XX estava inserido, e que ainda no século XXI, não está dissolvido. Em 1930, o terreno para a proibição da maconha tornava-se fértil ao passo que, em 1936 tivemos a criação da Comissão Nacional de Fiscalização de Entorpecentes (CNFE), subordinado ao Ministério das Relações Exteriores. \cite{Brandao2014Ciclos}.

    \begin{quotation}
        Destacamos que as necessidades econômicas levam inúmeros agricultores a se envolverem com o cultivo desta planta, notadamente em áreas marcadas pela baixa umidade e por poucas chuvas mal distribuídas ao longo do ano, ou seja, este cultivo representa uma alternativa real de manutenção financeira para quem vive no sertão nordestino. \cite[p.~5]{Brandao2014Ciclos}. 
    \end{quotation}

    Tendo sido feita a contextualização para chegarmos aos sertões, acrescentamos que segundo Janaína \textcite{Amado1995Regiao}, a categoria de ``sertão'' foi criada pelos portugueses que colonizaram o Novo Mundo, um destes mundos, em questão atual e específica, está o Brasil. Assim, o espaço em contraposição à Europa, foi denominado assim: ``espaços desconhecidos, inaccessíveis, isolados, perigosos, dominados pela natureza bruta, e habitados por bárbaros, hereges, infiéis, onde não haviam chegado as benesses da religião, da civilização e da cultura.'' (p. 149).

    Em contraposição a esta ideia eurocêntrica, chegamos ao conceito de sertões contemporâneos (ALBUQUERQUE JR, 2014), o qual trataremos de construir uma história visando a espacialidade sertaneja por meio da crise ``dos padrões tradicionais de sociabilidade que possibilitaram a emergência de um novo olhar em relação ao espaço, uma nova sensibilidade social em relação à Nação.'' \cite[p.~52]{AlbuquerqueJr2009Invencao}.

    Partindo dessa perspectiva do contemporâneo, do presente, reconhecemos o espaço e a temporalidade dos sertões nordestinos como plural, abarcador de multiplicidade de realidades, de diferenças e diversidades, e não, um espaço homogêneo, o qual não sofre mudanças sociais, políticas, econômicas, jurídicas, culturais e históricas através da coleta e análise das respostas do questionário.

    \begin{quotation}
        Atentar para o sertão como a definição feita acima, é, portanto, um gesto político da maior importância. É romper com as imagens e enunciados estereotipados, rotineiros, naturalizados, repetitivos, clichês sobre o sertão, a começar por enunciar a sua pluralidade interna. \cite[p.~43]{AlbuquerqueJr2016VedeSertao}.
    \end{quotation}

    Confrontar o olhar tradicional sobre o sertão, é o ato de observar criticamente o discurso proposto, a fim de discutir a quem este discurso dá poder e autonomia --- tais elementos se destacam em ``privilégios econômicos, políticos e sociais e repor dadas relações e hierarquias sociais, dentro e fora do espaço nomeado sertão.'' \cite[p.~43]{AlbuquerqueJr2016VedeSertao}. O proibicionismo encontrado nas fontes é a construção da base deste trabalho para observar os sertões como contemporâneos, assim como, explicitado acima.

    \section{Considerações finais}

    Vale ressaltar que os dados da pesquisa são referentes a sessenta pessoas, entre sua maioria, pessoas sertanejas que divididas por estados e municípios, cinquenta e quatro delas são do Rio Grande do Norte. Com este artigo acrescentamos algumas páginas a mais na História do Rio Grande do Norte e dos sertões do Brasil.

    \begin{quotation}
        ``Nesse sentido, o que está em questão não é o corpo, mas o sujeito de ação, produzido por uma exterioridade social, cultural e política. E isto se aplica a todo sujeito uma vez que a exterioridade atua sempre, por meio de discursos, na produção da subjetividade, e o sujeito é um efeito da subjetividade.'' \cite[p.~60]{Fernandes2012Discurso}.
    \end{quotation}

    Este questionário não tem a pretensão de ser um discurso generalizante diante da dimensão que é o território brasileiro e internacional, mas sim, estabelecer um diálogo de aproximação da Academia com as classes exteriores e interiores a ela. Os dados obtidos através do questionário, como os supracitados, nos revelam uma outra perspectiva de estudos direcionada aos sertões do Rio Grande do Norte, e de outras partes do Brasil. 

    Compreendemos os discursos são produzidos, administrados e interpretados, ou seja, não estão soltos sem finalidade e interesses de grupos sociais. 

    \begin{quotation}
        O discurso é assim palavra em movimento, prática de linguagem: com o estudo do discurso, observa-se o homem falando. [\dots] Essa mediação, que é o discurso, torna possível tanto a permanência e a continuidade quanto o deslocamento e a transformação do homem e da realidade em que ele vive. \cite[p.~15]{Orlandi1999Analise}.
    \end{quotation}

    Concluímos, agradecendo pela leitura, esperando que o estudo tenha gerado debate, afim de que, ocorram mais contribuições. Pretendemos continuar os estudos sobre a temática abordada, principalmente, por termos materiais que nos lançam para o poder do discurso, da contemporaneidade, do proibicionismo, e essencialmente, o da História.

    \nocite{AmadoAndFerreira2006Usos}
    \nocite{LoBianco2016}
    \nocite{Foucault2005Arqueologia}

    \printbibliography[heading=subbibliography,notcategory=fullcited]

    \hfill Recebido em 15 abr. 2021.

    \hfill Aprovado em 19 abr. 2021.

    \label{chap:contextosubsend}

\end{refsection}
