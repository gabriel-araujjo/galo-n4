\begin{refsection}
    \renewcommand{\thefigure}{\arabic{figure}}
    \renewcommand{\thetable}{\arabic{table}}
    
    \chapterOneLine
    {A concepção de pais e responsáveis por crianças de dois e três anos acerca do brincar na Educação Infantil}
    \label{chap:alfabetizacao-cartog}

    \articleAuthor
    {Fábio D'oliveira}
    {Prof. de Geografia da SEEC-RN, Especialista em Educação de Jovens e Adultos pelo Instituto Kennedy (IFESP). E-mail: fabioprofgeo@gmail.com.}
    
    \articleAuthor
    {Erineide da Costa e Silva}
    {Mestre em Desenvolvimento e meio ambiente. Professora no Instituto Federal do Rio Grande do Norte (IFRN). ID Lattes: 9354.8385.8752.3091. E-mail: erineide.silva@ifrn.edu.br.}
    
    \begin{galoResumo}
        \marginpar{
            \begin{flushleft}
            \tiny \sffamily
            Como referenciar?\\\fullcite{SelfDoliveiraAndSilva2021alfabetização}\mybibexclude{SelfDoliveiraAndSilva2021alfabetização}, p. \pageref{chap:alfabetizacao-cartog}--\pageref{chap:alfabetizacao-cartogend}, \journalPubDate{}
            \end{flushleft}
        }
        O artigo discute sobre a abordagem feita sobre a alfabetização cartográfica nos livros didáticos de Geografia utilizados na Educação de Jovens e Adultos, através da análise dos conteúdos de cartografia elencados na obra Maria Angélica Tozarini Teixeira “Caminhar e transformar: Geografia, Educação de Jovens e Adultos. Constata-se que os conteúdos propostos favorecem uma alfabetização cartográfica de forma parcialmente, enquanto outros não contemplam uma aprendizagem significativa, tendo em vista desconsiderar a realidade dos educandos, assim como o material imagético e contextualização são insuficientes. 
    \end{galoResumo}
    
    \galoPalavrasChave{Educação de Jovens e Adultos. Alfabetização cartográfica. Ensino de Geografia.}

    %% Missing abstract


    \section{Introdução}

    Esse artigo analisa, numa perspectiva exploratória/reflexiva, o livro didático utilizado nas turmas de Educação de Jovens e Adultos (EJA) das séries finais do Ensino Fundamental da Escola Estadual Djalma Aranha Marinho. Tal análise tem por objetivo investigar como a cartografia é abordada no livro didático “Caminhar e transformar: Geografia, anos finais do Ensino Fundamental” de autoria de Angélica Tozarini publicado pela editora FTD e assim verificar a eficácia de tal abordagem.  

    A necessidade da realização dessa investigação deu-se ao observar uma acentuada dificuldade dos alunos da EJA desse segmento nas escolas em operar com conceitos relacionados ou diretamente ligados a cartografia. 

    O livro didático apresenta a cartografia a partir dos seguintes conteúdos: tecnologia e espaço geográfico; variadas formas de orientação; pontos de orientação; rosa dos ventos e coordenadas geográficas que devem atingir o objetivo proposto pela autora que é formar um sujeito histórico capaz de aprender a ler mapas, perceber a relação entre texto contínuo e mapas, entender os conceitos de orientação e de coordenadas geográficas: latitude, longitude, paralelos, meridianos, e compreender a importância da tecnologia para a localização geográfica.  

    Partindo da proposta da autora, esse artigo tem por finalidade investigar se tal objetivo é alcançado diante da metodologia por ela apresentada e assim colaborar com os profissionais da Geografia que fazem uso desse livro como ferramenta mediadora em suas aulas. 


    \section{Metodologia}

    Trata-se de um estudo de caso, cuja análise abordagem sobre Cartografia no livro didático “Caminhar e Transformar: Geografia, anos finais do Ensino Fundamental” na Educação de Jovens e Adultos (EJA) de autoria de Angélica Tozarini publicado pela editora FTD.  

    Nesse sentido, a pesquisa sobre a abordagem da cartografia no livro didático envolve o ensino das técnicas cartográficas como um sistema de codificação e decodificação, que tem a intenção de oportunizar aos alunos uma precondição a sua leitura e escrita geográfica. Inicialmente a investigação foi feita acerca das atividades propostas no livro didático pela autora e em seguida averiguou-se relacionamento destas com o objetivo da autora e sua prática em sala de aula. 


    \section{Um breve histórico do ensino da geografia}

    Historicamente o ensino da Geografia escolar brasileira passou por mudanças em sua abordagem para refletir o momento da sociedade. O início do ensino e da produção geográfica no Brasil teve forte influência da escola francesa de Vidal de La Blache e portanto valorizava o papel do homem, mas numa proposta de análise da organização do espaço, território e lugar, a relação do homem com a natureza apenas como adaptações sem nenhuma intencionalidade ou ideologia, ou seja, era um estudo meramente descritivo das paisagens naturais e humanizadas de forma dissociada dos sentimentos dos homens pelo espaço \cite{ParâmetrosCurricularesGeografia2001}. 

    Nesse sentido os alunos eram estimulados a descrever e relacionar fatos naturais e sociais sem fazer uma análise crítica, mantendo-se neutros quanto às causas e consequências dos fatos por eles descritos. No período pós-guerra a realidade social tomou ares mais complexos e passaram a necessitar de estudos geográficos mais detalhados e análises mais profundas sobre as relações entre os seres humanos e a natureza \cite{ParâmetrosCurricularesGeografia2001}.  

    Fatos como o desenvolvimento do sistema capitalista, aprofundamento da urbanização, modificação dos espaços agrários com a intensa mecanização, avanços nas técnicas industriais e o aumento nas articulações entre os mais diversos locais do planeta corroboraram a necessidade de uma Geografia mais analítica que fosse um contraponto a Geografia tradicional de Vidal de La Blache, se tornara insuficiente para aprender a complexidade do espaço, pois a simples descrição já não apresentava os resultados esperados.  

    Nesse contexto era necessário realizar estudos voltados para a análise das ideologias políticas, econômicas e sociais. A partir desse descontentamento surge uma Geografia onde se buscava não somente explicar o mundo, mas transformá-lo. Dessa forma, procurou-se oportunizarão aluno a compreensão do processo de produção do espaço \cite{ParâmetrosCurricularesGeografia2001}. 

    Contudo, a “nova geografia” do pós-guerra com forte viés Marxista assim como a Geografia tradicional de La Blache negligenciaram a dimensão sensível da percepção do mundo e não se pode dissociar o imaginário nas análises do cotidiano tanto quanto das relações exteriores. Fica evidente a grande importância do uso dos mapas mentais na plena compreensão das relações homens-natureza \cite{ParâmetrosCurricularesGeografia2001}. 

    Na atual Geografia, buscam-se explicações mais plurais que recebam influência de outras ciências como a antropologia, a sociologia, a biologia, as ciências políticas, a filosofia entre tantas outras que buscam estudar as relações humanas. Com isso, pretende-se separar da ideia passada da geografia tradicional que meramente descrevia o espaço e da geografia marxista que se pautava na explicação exclusivamente política e econômica do mundo \cite{REGOAndCastrogiovanniAndKaercher2007Geografia}.  

    A Geografia moderna parte das explicações baseadas nas relações socioculturais da paisagem com os elementos físicos e biológicos para investigar as múltiplas interações entre eles estabelecidas na constituição dos lugares e territórios, com isso buscar explicar para compreender. A Geografia atual pauta-se nessa tendência conceitual para a sua abordagem no Ensino Fundamental. Portanto, ensinar Geografia no nível fundamental requer professores, alunos e materiais didáticos que estejam aptos a mudanças e novos estímulo \cite{TEIXEIRA2013Caminhar}. 


    \section{O que ensinar em geografia no Ensino Fundamental}

    Definir os conteúdos a serem trabalhados em sala de aula e como eles devem ser abordados é importante para o sucesso dos alunos na Geografia. Não se pode negligenciar temas tão importantes como as categorias de nação, território, lugar, paisagem e região. Não se deve entrar em modismos de temáticas atuais sem promover a compreensão dos múltiplos fatores que delas são causas ou decorrências, buscar a união da Geografia humana com a Geografia da natureza, e minimizar a “memorização” dos conteúdos buscando o pleno conhecimento do assunto abordado e suas relações \cite{ParâmetrosCurricularesGeografia2001}. 

    Os Parâmetros Curriculares Nacionais defendem a Geografia como uma área do conhecimento comprometida em tornar o mundo compreensível, explicável e passível de transformações, para tanto os temas que devem estar presentes nos livros didáticos devem também ser permeados da preocupação em buscar uma plena explicação dos acontecimentos e todos os seus desdobramentos no cotidiano dos alunos \cite{ParâmetrosCurricularesGeografia2001}. 

    Outra preocupação dos Parâmetros Curriculares Nacionais diz respeito aos conteúdos de Geografia serem ferramentas capazes de proporcionar o acesso à informação e a formação dos alunos para que possam compreender sua própria posição no conjunto de interações entre a sociedade e a natureza. Dessa forma, busca-se a democratização da escola, o bom convívio escolar e o estreitamento entre o que se é proposto pelos livros e a necessária abordagem plural que a Geografia moderna prega \cite{ParâmetrosCurricularesGeografia2001}. 

    Nesse sentido foram recuperados os conteúdos conceituais fundamentais, tornando-os bases para proposição de eixos-temáticos e valorizados os conteúdos procedimentais e atitudinais. 

    O objetivo principal da Geografia é estabelecer o processo histórico de relação das pessoas com a natureza a partir da leitura do lugar, do território e da paisagem. Nessa busca, a ciência geográfica utiliza como base para suas relações diferentes noções de espaço, tempo, fenômenos sociais, culturais e naturais característicos de cada paisagem, para então compreender o processo e a dinâmica das relações estabelecidas e a sua relação com as heranças no tempo \cite{ParâmetrosCurricularesGeografia2001}. 

    Para que uma boa análise possa ser feita é preciso que o estudo da paisagem foque nas dinâmicas de suas transformações e não simplesmente na descrição de um mundo aparentemente estático. Essa análise deve sempre perceber o espaço na Geografia como uma totalidade dinâmica onde ocorre a interação entre fatores naturais, sociais, econômicos e políticos \cite{ParâmetrosCurricularesGeografia2001}. 

    No contexto do ensino fundamental é importante estabelecer quais são as categorias mais adequadas para os alunos em relação a essa etapa da escolaridade e as capacidades que se espera que eles desenvolvam. Por esse motivo, os parâmetros curriculares nacionais estabeleceram o “espaço” como tema central de estudo e as categorias “território”, “região”, “paisagem” e “lugar” devam ser abordadas com o seu desdobramento \cite{ParâmetrosCurricularesGeografia2001}. 

    Ou seja, o aluno do Ensino Fundamental deve ser levado a fazer uma análise de um todo partindo de um ponto específico que é o espaço. Deve compreender que o espaço, a paisagem, o território e o lugar estão associados a força da imagem, que é tão explorada por veículos de comunicação e podem levar a formação mental de um modelo de mundo diferente da realidade \cite{TEIXEIRA2013Caminhar}.

    Por isso, a Geografia tem por objetivo intrínseco a formação de um ser pleno de cidadania, capaz de formar sua opinião crítica a partir da correta leitura da paisagem e do espaço. É no ensino fundamental que a Geografia vai auxiliar na formação de indivíduos conhecedores de seus direitos, deveres e participantes ativos na sociedade em que estão inseridos \cite{REGOAndCastrogiovanniAndKaercher2007Geografia}.  

    Observa-se que comumente os professores e professoras de Geografia tem usado para o ensino da ciência o livro didático e o seu próprio discurso está, geralmente, descontextualizado do lugar em que a escola está inserida. Aliada a essa prática elaboram exercícios de memorização que são avaliados de forma sistemática. 

    Na atual abordagem da Geografia a busca é pela mudança dessa prática pedagógica, onde os alunos tenham oportunidade de conhecer diferentes vivências com os lugares e dessa forma possam construir novas e mais complexas relações com os lugares. Com esse novo olhar espera-se que seja possível que os alunos desenvolvam a capacidade de identificar e refletir sobre diferentes aspectos da realidade, compreendendo a relação estabelecida entre a sociedade e a natureza.  

    Essa prática só será possível se forem garantidos os procedimentos de problematização, observação, registro, descrição, documentação, representação e pesquisa dos fenômenos sociais para que sejam formuladas hipóteses e explicações dessa realidade.  

    O convívio do(a) professor(a) com alunos em sala de aula é imprescindível e a vivência do aluno deve ser valorizada a fim de que ele possa perceber que a Geografia faz parte do seu cotidiano e que o professor é o elemento de ligação entre o conhecimento sistematizado e o conhecimento vivenciado. Por esse motivo é fundamental que o professor crie e planeje situações de aprendizagem em que os alunos possam conhecer os procedimentos de estudos geográficos.  

    A observação, a descrição, a analogia e a síntese são procedimentos importantíssimos no processo de aprendizagem da geografia. O espaço que o aluno vive deve sempre ser o ponto de partida dos estudos, porém não deve ser observada uma escala de importância entre o local, o regional, o nacional, o mundial e sim a relação entre os lugares. 

    Contudo, a complexidade das interações e das formas como ocorrem as transformações na sociedade impede análises mecânicas e lineares conduzindo os(as) professores(as) a uma abordagem mais ampla que privilegie o imaginário dos alunos que agora dispõe de grande quantidade de informações recebidas pelos meios de comunicação e internet de forma muito rápida.  

    Essa realidade também fornece subsídios aos professores para trabalhos com eixos temáticos que favoreçam a interdisciplinaridade nas análises de assuntos do cotidiano. O ensino de Geografia no nível fundamental deve intensificar a compreensão dos alunos sobre os processos de construção das paisagens, territórios e lugares não se restringindo somente à simples constatação e descrição, mas a busca por respostas sobre a realidade vivida a proposição de soluções para problemas encontrados e a manutenção de ações de sucesso \cite{ParâmetrosCurricularesGeografia2001}. 

    Esse tipo de postura vai favorecer ao aluno a compreensão de que ele mesmo é parte integrante e atuante dos processos de transformação das paisagens terrestres, entendendo paisagem como todo lugar que tenha contato. As noções de sociedade, cultura, natureza, trabalho, associadas a literatura tem fornecido base muito confiável para que a Geografia alcance um de seus objetivos que é obter informações para um pleno conhecimento de sua realidade em diferentes escalas temporais e dimensionais \cite{ParâmetrosCurricularesGeografia2001}.  

    Nesse contexto as artes visuais como o cinema, a fotografia e também a música se aliam à essa árdua tarefa de dar base a formação de um cidadão pleno de direitos e deveres que se preocupe em garantir a existência de uma sociedade justa e igualitária. 

    A Geografia ao trabalhar constantemente com imagens, utiliza-se de diferentes linguagens que possibilitam as melhores interpretações, proposições de hipóteses e formulação de conceitos. Por esse motivo, é de vital importância a implementação de uma cartografia conceitual que torne a localização e a espacialização uma referência na leitura das paisagens e seus movimentos. O estudo da linguagem gráfica contribui não somente para que os alunos possam ser capazes de usar a ferramenta básica da Geografia, os mapas, mas para que seja possível o desenvolvimento de capacidades relativas à representação do espaço e toda a sua dinâmica \cite{ALMEIDA2011Novos}. 

    O desenvolvimento da cartografia é muito antigo e desde a pré-história até os dias atuais continua em plena expansão e reconfiguração, essa dinâmica possibilita a sintetização das informações, a expressão do conhecimento e o estudo das situações sempre aliado a ideia de produção do espaço, sua organização e distribuição.  

    Infelizmente, a forma mais usual de se trabalhar a linguagem gráfica nas escolas é por meio da memorização feita a partir de atividades de colorização de mapas, cópias meramente mecânicas e preenchimento de nomes de rios e cidades sem a preocupação de contextualização das informações presentes no mapa. Não existe a preocupação por exemplo em informar/formar o aluno sobre as características básicas de em relação ao tipo de projeção e ao objetivo específico de cada mapa \cite{ParâmetrosCurricularesGeografia2001}.  

    A escola deve ser capaz de formar cidadãos capazes tanto de ler e interpretar as informações contidas num mapa, quanto se perceber como agentes codificadores e representantes do espaço. O “olhar” na Geografia tem grande importância na busca pela compreensão dos fatos do cotidiano e é nesse aspecto que as linguagens gráficas se apresentam como ótimas ferramentas de apoio ao trabalho dos professores. A observação permite aos alunos a compreensão de fatos sem a necessidade de grandes discursos e torna o aprendizado mais prazeroso \cite{ALMEIDA2011Novos}. 

    O ensino de Geografia no nível fundamental da educação deve ser capaz de fornecer conhecimentos relacionados ao mundo atual e suas diversidades nas construções das mais diferentes paisagens, identificação e avaliação das ações humanas em sociedade e suas consequências em diferentes espaços e tempos \cite{ParâmetrosCurricularesGeografia2001}. 

    Os conteúdos dos livros didáticos de Geografia devem estar abertos a um conjunto de eixos temáticos que se portem como norteadores do ensino/aprendizagem e esses serão importantes na busca por uma melhor interpretação das realidades, no conhecimento da relação espaço/tempo na construção da paisagem, na apropriação do aluno de conhecimentos que o tornem ator da sua própria vida \cite{ParâmetrosCurricularesGeografia2001}. 

    De posse desse conhecimento os alunos podem formular conceitos, estipular procedimentos e atitudes adequadas a cada situação. A escolha dos conteúdos deve ser relevante para a formação de um aluno que seja capaz de desempenhar as funções da cidadania conhecendo as características sociais, culturais e naturais do local onde vive, bem como as características de outros locais podendo assim formular comparações que expliquem, especializem e levem a compreensão das múltiplas relações das mais diferentes sociedades, nos mais variados recortes temporais \cite{ParâmetrosCurricularesGeografia2001}. 

    A aquisição desses conhecimentos permite a formação de uma consciência dos limites e responsabilidades da ação individual e coletiva com relação ao seu lugar e a contextos que vão da escala nacional até a escala mundial. A Geografia aliada as outras ciências permitem que o aluno desenvolva hábitos e valores para a vida em sociedade e também desenvolva o respeito a variedade de culturas, povos e etnias que compõem o estado brasileiro \cite{ParâmetrosCurricularesGeografia2001}. 

    Nesse aspecto os conteúdos devem auxiliar os alunos a uma compreensão mais complexa sobre o tempo e espaço e as relações com as transformações já ocorridas e as que ainda devem ocorrer.   

    O ensino e aprendizagem de Geografia na educação fundamental representa a base no processo contínuo de obtenção de conhecimentos que fortaleceram o aprendizado nas etapas seguintes da vida escolar dos indivíduos. Nessa etapa de ensino são recuperadas questões inerentes a relações dos indivíduos com a natureza, com os grupos sociais e com a construção do espaço.  

    A cartografia serve nesse sentido de apoio a compreensão mais ampla das relações estabelecidas entre a sociedade e a natureza, é possível fazer uma análise sobre as transformações que a natureza sofre devido as atividades econômicas, culturais, políticas, expressas de diferentes formas no próprio meio em que os alunos vivem \cite{ALMEIDA2011Novos}.  

    O estudo das informações inseridas em um mapa pode ajudar a elucidar ou entender e propor intervenções a questões, por exemplo, alimentares que se diferenciam entre o campo e a cidade, a divisão de atividades laborais entre zonas urbanas e zonas rurais e até mesmo entender a forma de lazer que cada sociedade possui. 

    O uso da cartografia escolar auxilia a compreensão por exemplo da estreita relação entre o desenvolvimento de certas atividades e determinados locais em detrimento a outros. Contudo, é necessário que o professor tenha a sensibilidade de conduzir seus alunos além daquilo que já conhecem e com isso possibilitar intervenções práticas ligadas a realidade estudada \cite{ALMEIDA2011Novos}. 

    Nessa etapa de ensino é necessário que o aluno aprofunde seus conhecimentos sobre os procedimentos da Geografia que são observar, descrever, representar cartograficamente por meio de imagens para então construir explicações.  

    Esse processo necessita do envolvimento do professor que se coloca como condutor de novos olhares dos seus alunos impedindo análises rasas sobre as questões levantadas, nesse sentido o professor expõe que a observação não é geograficamente apenas um olhar direto, mas sim um olhar intencional em busca de respostas nem sempre tão óbvias. 

    O aluno já possuidor do domínio da escrita e leitura deve ser estimulado a expor seus pensamentos e opiniões por escrito e em forma de imagens feitas por ele mesmo. Por esse motivo é importante que a imagem esteja presente como forma de representação da realidade.  

    Desenhar é uma forma peculiar a Geografia como um de seus procedimentos de aprendizagem que levam os alunos ao uso de noções de proporção, distância e direção fundamentais a compreensão da linguagem gráfica presente nas representações cartográficas.  

    O trabalho em sala de aula na construção da linguagem gráfica deve levar em consideração os referenciais que os alunos já possuem ao se localizar e orientar no espaço, como é o caso dos monumentos, formas de relevo, avenidas, ruas, praças e edifícios. Também é necessário o compartilhamento das informações com o intuito de uma nova esquematização que amplie as ideias de distância, direção e orientação.  

    Aliado a toda essa informação estão os mapas metais que serão base importante na construção do saber conceitualizado. Porém, é fundamental que o início da construção da linguagem gráfica por meio dos alunos seja feito de forma gradual, partindo da construção de mapas simples, onde os alunos possam em situações significativas propor soluções a questões pré-estabelecidas, dessa forma obtendo e interpretando informações.  

    Nessa fase de ensino o aluno deve ser estimulado a trabalhar com diferentes formas de representação cartográfica como diferentes tipos de mapas, atlas, globo terrestre, plantas cartográficas e até mesmo maquetes atualizadas para que ocorra uma interação entre o conteúdo sistematizado e a prática cada vez mais precisa e adequada de cada uma das representações estudadas \cite{ALMEIDA2007Cartografia}.  

    O estudo do meio com uso de imagens e representações de locais próximos e distantes do habitat dos alunos é um recurso didático muito interessante pois possibilita a construção e a reconstrução cada vez mais detalhada e estruturada da paisagem local e da sua relação com o global, fortalecendo vínculos de afetividade dos alunos com o lugar onde vivem.  

    Nesse sentido a Geografia pode e deve fazer um trabalho conjunto a história do lugar por meio de recortes temporais para auxiliar na interpretação da realidade atual como consequência das modificações ocorridas no tempo. 

    A geografia no ensino fundamental deve ser apresentada aos alunos como uma possibilidade de leitura e compreensão do mundo. Portanto, compreender o mundo na ótica da Geografia é condição de formação de indivíduos plenos de sua cidadania e passíveis de enfrentamento e resolução das questões do dia a dia.  


    \section{Andragogia: uma abordagem prática na EJA}

    Alinhado ao crescente número de alunos em turmas de educação de jovens e adultos, está também a crescente demanda por profissionais que estejam realmente habilitados a trabalhar de forma adequada com esse grupo de pessoas. Dessa forma fica evidente que os profissionais da educação de jovens e adultos devem procurar se cercar das mais diversas fontes de conhecimentos e de auxílio. 

    A partir dessa realidade um termo tem aparecido com maior frequência nos estudos relacionados ao processo educacional de adultos seja em instituições de ensino, como em ambientes laborais que praticam a formação continuada de seus trabalhadores. Esse termo é Andragogia, mas o que é a Andragogia? Sobre essa pergunta, \textcite{ROSSETTI2013Andragogia}, apresenta como resposta: ao usarmos a palavra Andrologia, estamos nos referindo a um conjunto de procedimentos de educação destinado ao ser humano amadurecido. Estamos falando de um caminho educacional que busca compreender o adulto aprendente, em sua natureza e capaz de tomar decisões como um ente psicológico, biológico e social seja homem ou mulher. 

    Esse novo caminho educacional direciona o profissional da educação de jovens e adultos a uma nova condição de facilitador de aprendizagens ante a sua condição anterior de educador tradicional. Dessa forma sua metodologia deverá ser alterada para gerar os resultados positivos esperados. Essa nova metodologia deve basear-se no entendimento que alfabetizar adultos difere bastante da alfabetização infantil pois os adultos têm motivações diferentes para aprender, buscam o sentido para a vida nas resoluções de questões inerentes a sua idade, cotidiano e tem a necessidade de se autodirigirem em busca do seu próprio conhecimento. 

    A abordagem prática dos conteúdos visa o desenvolvimento de atitudes e habilidades, já disponíveis nos adultos objetivando a melhora na experiência educacional em sala de aula. 

    Portanto, o conhecimento da Andragogia é importantíssimo na elaboração dos livros didáticos destinados a turmas de educação de jovens e adultos, evitando assim conteúdos insuficientes e/ou infantilizados que não irão favorecer uma formação adequada dos alunos. 


    \section{A importância do uso do livro didático como ferramenta educacional em turma de Educação de Jovens e Adultos}

    O cotidiano da sala de aula na Educação de Jovens e Adultos revela uma característica interessante quanto ao público que nela está presente, diferentemente de outros níveis de ensino as turmas de EJA geralmente são formadas por indivíduos que já carregam em si uma grande quantidade de conhecimentos obtidos pela sua própria vivência pessoal. É nesse cenário que o professor vai se tornar um estimulador ao pleno desenvolvimento do espírito crítico e da formação integral dos seus alunos. 

    Mas de que forma o professor poderá alcançar o seu objetivo que é formar cidadãos plenos dos seus direitos, de seus deveres e consequentemente sujeitos ativos na construção e reconstrução da realidade vivida por eles?  

    O livro didático ainda se mantém como um dos principais, senão o único, instrumento, em algumas realidades, para o professor alcançar tal objetivo. Diante disso, é fundamental que o professor conheça bem o livro didático que vai adotar para mediar suas aulas. Sobre essa questão, chamam atenção as autoras \textcite{SILVAAndSILVA2009abordagem}, ao afirmarem que o livro didático ainda se configurado como instrumento pedagógico muito presente, se não o único, na maioria das instituições de ensino. Logo, convém destacar a necessidade de o professor tornar-se cada vez mais atento para analisá-lo e conhecer as diferentes concepções paradigmáticas e metodológicas que embasam sua produção, conteúdos e conceitos trabalhados e as atividades propostas.  

    Nesse sentido, o conhecimento do professor quanto ao inteiro teor do livro didático é imprescindível para que sejam minimizadas eventuais lacunas em conteúdos que são por muitas vezes apresentados de forma bastante resumida sem ser dada a devida importância a temática. Cabe ao educador, principalmente em turmas de EJA, ter a capacidade em saber escolher o livro didático mais adequado a sua prática como também em saber usar da melhor forma esse instrumento de ensino. 


    \section{Análise dos conteúdos e atividades de cartografia no livro didático utilizado na Educação de Jovens e Adultos em turmas do Ensino Fundamental}

    O livro didático “caminhar e transformar” da autora Angélica Tozarini, no seu segundo capítulo intitulado de \textit{As novas tecnologias e as diversas leituras na Geografia} tem como objetivos, ensinar a leitura de mapas, estimular a percepção da relação entre textos contínuos e mapas, facilitar o entendimento dos conceitos de orientação e Coordenadas geográficas sobretudo os conceitos de latitude, longitude, paralelos e meridianos, além de fornecer aos educandos a compreensão da tecnologia nos meios de comunicação e na localização geográfica. 

    O capítulo inicia-se na página 25 com a apresentação de duas imagens de satélite: a primeira imagem refere-se a uma região da cidade do Rio de Janeiro e a segunda apresenta uma imagem do sul de Manhattan em Nova York nos Estados Unidos. Na mesma página a autora apresenta um pequeno texto lateral sobre a obtenção dessas imagens e sua qualidade, porém nenhum conteúdo conceitual sobre o uso da tecnologia na cartografia. Nelas há apenas um relato de como podem auxiliar na antecipação de fenômenos naturais. Na sequência das imagens, já na página 26, o livro apresenta a primeira atividade de leitura das imagens e possível relação com o texto lateral.  

    Contudo, não se pode esperar que os alunos já sejam plenamente capazes de fazer tais relações com tão pouco conteúdo conceitual, dessa forma as relações serão bastante rasas e devem dar a sensação de insuficiência e inutilidade aos alunos. Percebe-se que poderiam ser usadas imagens mais comuns aos alunos e que conceitos e explicações mais detalhados fossem mais bem explicitados no livro anteriormente a execução das atividades.  

    Na segunda atividade, propõe-se uma relação entre um texto sobre o corte ilegal de madeira na Amazônia e uma imagem criada na “mente” dos alunos, porém seria mais interessante que a autora tivesse inserido imagens de satélite das áreas degradadas e imagens de locais da própria Amazônia para que a relação fosse plenamente estabelecida. 

    Continuando na mesma página do livro didático, surge um texto isolado que parece querer remeter ao uso da tecnologia por parte das tribos indígenas dos estados de Rondônia e do Mato Grosso, nesse texto existe a afirmação que as tribos indígenas estão usando a tecnologia para elaboração de mapas. Mas como os alunos conseguiram compreender esse texto se eles não foram levados a produção de um mapa com uso de tecnologia? Não possuindo nenhuma atividade prática de construção de mapas, fica muito difícil perceber o uso da tecnologia e provavelmente o texto perderá seu objetivo.   

    Na página 27 do livro didático tem início os estudos sobre as coordenadas geográficas. Para tanto utiliza-se um texto que fala sobre o uso do GPS, a precisão de seus dados e questionando os alunos quanto ao uso de novas tecnologias para orientar as coordenadas geográficas.  

    A autora faz menção ao texto anterior sobre o uso da tecnologia na confecção de mapas pelas tribos indígenas e questiona sobre outras possíveis utilizações de recursos tecnológicos que os alunos possam saber. Ao término das apresentações conceituais, é proposta uma atividade com o uso do jogo da batalha naval que pode passar a ideia de infantilização da atividade tendo em vista que provavelmente a turma seja formada por sua grande maioria de alunos com uma idade mais avançada. Infelizmente grande parte dos alunos não consegue compreender o uso dessas atividades como relativas as suas idades.  

    Na sequência da atividade, o livro tem como proposta relacionar o jogo anteriormente realizado com os conceitos de latitude, longitude, paralelos e meridianos. Contudo falta a explicação de como foi feito o cálculo para o estabelecimento dos valores em graus das latitudes e longitudes. Informações como essa permitem que os alunos compreendam melhor a utilização prática dessas informações e que seja atingido um dos objetivos dos Parâmetros Curriculares Nacionais que versam sobre a facilitação do uso do conteúdo oficial na formação do saber cotidiano. 

    Com o estudo dos fusos horários, a autora parte para a explicação de como foi feita a padronização do horário no planeta. O texto do livro apresenta informações sobre como foram definidas a 24 faixas de 15 graus, representando uma hora cada e cita que essa divisão recebeu o nome de fuso horário. Não existe nenhuma atividade sobre fusos horários e com isso fica muito complicado que os alunos consigam levar para o seu dia a dia o entendimento dos fusos horários e não devem compreender também as adequações feitas em cada país. 

    Os conteúdos que se referem a orientação tanto por mapas, bússolas, Sol e Cruzeiro do Sul também estão dispostos de maneira muito superficial e não levam os alunos a elaboração de questionamentos quanto a sua funcionalidade prática em seu cotidiano. Fica evidente no livro didático que esse tema de enorme relevância para a Geografia e para o uso diário não foi devidamente abordado com a devida seriedade que lhe é inerente. 

    Conforme análise, apresenta-se na Tabela \ref{tabl:conteudo-livro-dida} resumo da análise nos conteúdos propostos.

    \begin{table}
        \centering
        \caption{Conteúdo do livro didático e objetivos propostos}
        \label{tabl:conteudo-livro-dida}

        \begin{tabular}[!hb]{p{5.5cm} P{1.7cm} P{1.7cm} P{1.7cm}}
            \toprule
            \multirow{2}{5.5cm}{Conteúdo do livro didático} & \multicolumn{3}{c}{Atingem os objetivos propostos} \\
            \cmidrule(lr){2-4}
            & Sim & Não & Em parte \\
            \midrule
            Leitura de imagens & & x & \\
            Coordenadas geográficas & & & x \\
            Fusos horários & & & x \\
            Orientação pelo sol  & & & x \\
            Orientação pelo cruzeiro do sul & & x & \\
            Orientação pelos mapas & & & x \\
            Pontos Cardeais & & x & \\
            \bottomrule
        \end{tabular}
        \caption*{Fonte: elaboração do autor}
    \end{table}

    Observa-se na Tabela \ref{tabl:conteudo-livro-dida} que a maneira como os conteúdos foram propostos, a maioria favorece atingir os objetivos apenas de forma parcial, enquanto ou outros não contemplam uma aprendizagem significativa, tendo em vista desconsiderar a realidade dos educandos, assim como o material imagético e contextualização são insuficientes. 


    \section{Considerações finais}

    A Educação de Jovens e Adultos tem se mostrado como uma das modalidades de ensino mais desafiadoras da educação. Ainda se percebe que existe um longo caminho a ser percorrido e que neste trajeto muitos desafios devem ser enfrentados. 

    A formação dos profissionais que nela estão é um dos grandes desafios a serem enfrentados e necessita de um olhar bastante criterioso para que seja evitada uma formação inadequada aos profissionais que desejem atuar em turmas de EJA. O profissional da educação hoje assume um papel de facilitador de aprendizagens e deixa de lado a função do professor tradicional 

    O livro didático destinado as turmas de educação de jovens e adultos, que em muitos casos é a única ferramenta educacional disponível para os educadores, também precisa ser melhor elaborado levando em consideração elementos que são bastante peculiares a essa modalidade tão especial de ensino. A elaboração dos livros didáticos deve ser pensada de acordo com o público que irá utilizá-lo evitando à repetição de erros comuns observados em obras utilizadas atualmente. 

    Os conteúdos trabalhados na EJA não devem ser instrumentalizados de maneira a serem confundidos com conteúdos ministrados em turmas de alfabetização infantil, uma vez que o compartilhamento desses conhecimentos deve estar adequado ao público jovem e em sua grande maioria adulto que frequenta as aulas e que necessitam de ações mais práticas e autodirigidas. 

    A obra analisada deixou algumas lacunas em relação aos conteúdos relacionados a cartografia e fica evidente que a ideia de Andragogia não foi levada em consideração na elaboração das atividades e na escolha dos textos sobre o assunto. Dessa forma, é urgente que os autores de livros didáticos destinados à EJA passem por uma requalificação em relação aos conceitos inerentes ao público para qual suas obras são destinadas. Esse novo olhar deverá ocasionar uma melhora significativa nos materiais ofertados as escolas e consequentemente uma formação mais adequada para todos os alunos da educação de jovens e adultos.  

    \printbibliography[heading=subbibliography,notcategory=fullcited]

    \label{chap:alfabetizacao-cartogend}

\end{refsection}
