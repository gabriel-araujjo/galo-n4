\begin{refsection}
    \renewcommand{\thefigure}{\arabic{figure}}

    \chapterTwoLines
    {O silencio dos caboclos}
    {notas sobre catimbozeiros perseguidos no Rio Grande do Norte}
    \label{chap:silencio}
    
    \articleAuthor
    {Rômulo Henrique P. Angélico}
    {Licenciado e bacharel em História pela UFRN, especialista em Ciências da Religião pela UERN, é servidor público na função de professor na SEEC-RN. ID Lattes: 4638.4633.1693.5766. ORCID: 0000-0002-0515-0222. E-mail: romulo\textunderscore{}livreiro@hotmail.com.}

    \begin{galoResumo}
        \marginpar{
            \begin{flushleft}
            \tiny \sffamily
            Como referenciar?\\\fullcite{SelfAngelico2021}\mybibexclude{SelfAngelico2021}, p. \pageref{chap:silencio}--\pageref{chap:silencioend}, \journalPubDate{}
            \end{flushleft}
        }
        O Catimbó-Jurema é uma tradição de matriz indígena oriunda do nordeste brasileiro. Ao longo dos séculos, índios juremeiros e seus descendentes caboclos foram vítimas de uma série de perseguições devido ao conjunto de crenças e práticas cerimoniais que compõem esse universo ser classificado, por autoridades católicas e laicas, feitiçaria. No território do Rio Grande do Norte a repressão aos cultos de matriz indígena parece ter sido considerável, entretanto, poucos relatos sobreviveram ao tempo. O presente artigo tem o objetivo de analisar casos que permaneceram na memória de veneráveis caboclos e em fontes específicas. O método utilizado foi o diálogo com juremeiros e a pesquisa bibliográfica. O artigo conclui que, embora a repressão ao Catimbó tenha se estendido do século XVI ao início XX, a tradição permanece viva em nosso Estado --- preservando elementos ancestrais autóctones.
    \end{galoResumo}
    
    \galoPalavrasChave{Jurema. Índios. Tradição. Ancestralidade.}
    
    \begin{otherlanguage}{english}
    
    \fakeChapterTwoLines
    {The silence of caboclos}
    {notes on persecuted catimbozeiros in Rio Grande do Norte}

    \begin{galoResumo}[Abstract]
        Catimbó-Jurema is a tradition of indigenous origin from the northeast of Brazil. Over the centuries, juremeiros indians and their caboclos descendants were victims of a series of persecutions due to the set of beliefs and ceremonial practices that make up this universe to be classified, by Catholic and secular authorities, witchcraft. In the territory of Rio Grande do Norte, the repression against indigenous cults seems to have been considerable, however, few reports have survived over time. This article aims to analyze cases that remained in the memory of venerable caboclos and in specific sources. The method used was dialogue with jurists and bibliographic research. The article concludes that, although the repression against Catimbó extended from the 16th to the beginning of the 20th century, the tradition remains alive in our state --- preserving indigenous ancestral elements.
    \end{galoResumo}
    
    \galoPalavrasChave[Keywords]{Jurema. Indians. Tradition. Ancestrality.}
    \end{otherlanguage}

    Se nos determinarmos a conversar, durante alguns dias, com um número considerável de pessoas, sobre as comumente chamadas ``religiões afro-brasi\-leiras'', perceberemos que boa parte dos indivíduos com os quais dialogamos não consegue diferenciar Candomblé, Umbanda e Catimbó. As pessoas geralmente consideram essas três religiões, surgidas em tempos e espaços distintos, como uma só e mesma coisa. 

    Sendo o Candomblé, em linhas gerais, uma tradição de matriz africana, chegada ao Brasil durante o período colonial, caracterizada pelo culto aos orixás (personificações das misteriosas forças da natureza e, ao mesmo tempo, ancestrais divinizados); e a Umbanda uma religião nascida no sudeste brasileiro no início do século XX (agregando em um único corpo ritualístico e cosmogônico o culto aos orixás e elementos provenientes de tradições indígenas, judaico-cristãs, kardecistas e orientais); é o Catimbó (também chamado Jurema Sagrada, Catimbó-Jurema, Jurema e Culto aos Senhores Mestres), por sua vez, legado indígena que sobrevive ao tempo e resiste às inovações e mudanças que o tempo traz. 

    O Catimbó, assim como o Candomblé, possui uma história marcada por perseguições e proibições ocorridas principalmente durante o período colonial --- época em que diversos pajés, índios, índias e caboclos, tiveram suas ocas queimadas, seus corpos mutilados e suas vidas condenadas por reverenciar, conforme seu universo cultural e conjunto de crenças, os espíritos que acreditavam proteger suas comunidades e por praticar um tipo característico de magia medicinal que envolvia, simultaneamente, a defumação com ervas, a ingestão de bebidas sacramentais e a evocação de forças da floresta. Essas práticas foram consideradas feitiçaria e não tardaram em ser condenadas pelo colonizador adventício. Porém, à margem da sociedade colonial, malgrado as proibições e perseguições sofridas, o Catimbó permaneceu vivo e, com o tempo, à medida que chegavam ao território brasileiro povos de origens diversas (cada grupo com seu respectivo conjunto de valores e crenças), os antigos cultos à Jurema assimilaram aspectos objetivos, subjetivos e simbólicos, de matrizes distintas --- tornando-se um culto, ou antes uma religião (uma vez que possui a capacidade de religar seus devotos às suas concepções de sagrado e ancestralidade) híbrida, cabocla, cujas raízes e principais práticas remontam aos povos autóctones do sertão nordestino, mas que ultrapassou os séculos, alcançando Império e República. 

    Por mais que as perseguições ocorridas, em sua maioria, tenham sido quase completamente esquecidas, deixaram fragmentos na memória de antigos mestres de Jurema (alguns dos quais conheci pessoalmente e enriqueceram, com suas memórias, este artigo). Ademais, excertos foram citados por autores que extraíram de poucas fontes escritas do passado determinadas notas relacionadas à tortura, prisão e mortes de índios e caboclos --- sendo objetivo deste artigo analisar alguns desses poucos casos, quase perdidos, especialmente os que ocorreram no território do Rio Grande do Norte. 

    Conforme o processo colonizador ganhava corpo, se expandia e fortalecia, os índios eram aldeados e submetidos à tutela estrangeira. Em paralelo à ``caboclização'' e integração de nativos de etnias distintas em comunidades sujeitas à autoridade da Igreja e da Coroa Portuguesa, elementos de origem africana e europeia chegavam em número crescente ao território brasileiro. Regiões como Pernambuco e Bahia, que possuíam portos apropriados à vinda de escravos africanos, receberam um grande número de negros que, com o tempo, tanto assimilaram elementos materiais e imateriais de origem indígena quanto transmitiram-lhes aspectos de sua própria cultura. 

    Na então Capitania do Rio Grande, o processo escravocrata se deu de modo singular: como aqui não havia porto apropriado à escravidão africana, os senhores de engenho e fazendeiros que pretendessem possuir ``negros da Guiné'' teriam que compra-los pelo dobro do preço a escravocratas de outras regiões. Por isso, senhores locais preferiam aprisionar e escravizar ``negros da terra'' (índios acusados de rebeldia ou de atentar contra o cristianismo e o Estado português) --- o que lhes saía menos custoso. Consequentemente, a quantidade de afrodescendentes no território do Rio Grande do Norte foi consideravelmente pequena, se comparada à presença negra em Pernambuco e Bahia e, por conseguinte, o culto à Jurema que se desenvolveu no estado em que vivemos foi, durante muito tempo, marcado principalmente por elementos indígenas, cristãos e judaicos. 

    Como a maioria dos pesquisadores que trataram do Catimbó-Jurema deu preferência ao estudo de suas manifestações conforme se formaram e ocorreram em Pernambuco, Paraíba e Ceará, o Catimbó norte-rio-grandense e seus processos de formação e desenvolvimento permanecem pouco conhecidos. Por outro lado, a presença de centros espíritas e de terreiros de Umbanda no estado do Rio Grande do Norte, a partir do final da primeira metade do século XX, com o subsequente surgimento e expansão das federações de Umbanda e Candomblé em nosso estado, exerceram consideráveis influências sobre os centros e casas que cultuavam a Jurema --- de modo que os rituais mais próximos das antigas pajelanças foram e continuam sendo, vagarosa mas progressivamente, substituídos por elementos de matriz africana e brasileira (compreendamos que o Candomblé é africano, a Umbanda é brasileira e o Catimbó é ameríndio --- sua presença é muito mais antiga que a formação das estruturas políticas e econômicas que definiram o Brasil). 

    Meus contatos iniciais com o Catimbó ocorreram entre os anos de 2005 e 2012 --- época em que tive a oportunidade de participar de inúmeras sessões de Catimbó-Jurema em terreiros de Canguaretama (município localizado no litoral sul do Rio Grande do Norte), visitar comunidades indígenas (o Amarelão, em João Câmara; o Katu dos Eleutérios, em Canguaretama, e a Aldeia Trabanda, em Baía Formosa, foram as aldeias visitadas na época), conversar com mestres, mestras e pajés (o sacerdote de Jurema, em ambiente não-indígena, é chamados ``mestre''; em aldeias, ``pajé''). Foi, por sinal, no final desse período que me tornei mestre e assumi a direção do Centro Cultural e Espiritualista Casa Sol Nascente do Rei Malunguinho (terreiro que se localizava no limite entre Parnamirim e Macaíba, cuja existência durou cerca de sete anos). 

    O início daqueles contatos ocorreu sem qualquer intenção de envolvimento afetivo: foram movidos por uma feira de ciências que tinha como objetivo apresentar as religiões existentes em Canguaretama (na época eu lecionava Cultura e Economia do Rio Grande do Norte, na Escola Estadual Juarez Rabelo). A Secretaria de Educação do município apresentava o tema (único a ser abordado por todas as escolas) e o assunto do ano letivo de 2005 foi ``religiões de Canguaretama''. Me surpreendi ao perceber que os grupos de todas as escolas decidiram abordar exclusivamente o cristianismo em suas vertentes católica e protestante. Então, em conversa com o diretor da citada escola, me prontifiquei a abordar o Candomblé (na época eu não conhecia as semelhanças e diferenças existentes entre o Culto aos Orixás, a Umbanda e o Catimbó) 

    Saímos, eu e um grupo de quatorze alunos e alunas, à procura de um Candomblé (que não encontramos). Porém, um professor nos levou a uma casa, localizada em área periférica, na qual, nos fundos, havia um centro em que ocorriam as ``mesas espíritas'' (``mesa'' é o nome que as sessões de Catimbó recebem em alguns centros). Chegamos no Centro Mestre Pena Branca e Estrela do Mar --- então dirigido por Maria Ivonete da Silva Santana, mais conhecida como Neta --- no momento em que ocorria uma sessão: a mestra estava por trás de uma mesa grande, coberta com toalha branca, sobre a qual havia livros, flores, terço, velas e um copo com água (chamado ``princesa'' nos terreiros). À sua esquerda havia outra mesa, com velas, rosas e estátuas representando os Mestres (as entidades espirituais que atuam nas sessões também são chamadas ``Mestres'', ``caboclos'' e ``encantados''). 

    Ao longo da sessão, os Mestres vinham e se manifestavam através da mestra Neta. Conversavam com as pessoas que se encontravam no local, receitavam plantas e tratamentos para doenças específicas, desmanchavam malefícios mágicos, transmitiam força aos doentes, bebiam cachaça e fumavam --- utilizando, inclusive, cachaça e fumaça de cachimbo em trabalhos de cura à guisa de antigos pajés (sempre que sentia ser necessário, um Mestre dava uma ``fumaçada'' em um doente objetivando a cura). Essas defumações mágico-medicinais, em Tupi Antigo (língua falada pelos nativos do litoral norte-rio-grandense à época inicial da colonização), são chamadas \textit{ka'átimbor}, (palavra que em português pode ser traduzida da seguinte forma: ka'á, ``mato'' e timbor, ``fumaça'', em outras palavras, ``fumaça de mato'', defumação). 

    Dentre as ervas utilizadas nas defumações, encontra-se, presente em todos os terreiros e em inúmeros trabalhos e cerimônias, desde épocas remotas, o Tabaco. Na Capitania do Rio Grande, foi de uso constante entre os povos do litoral e do sertão: vegetal reverenciado por seres humanos e entidades espirituais --- sendo ele próprio, nos mitos Kariri, chamado Badzé, reconhecido como ente espiritual descido dos céus em forma de planta, senhor da floresta e dos encantos. No litoral, os pajés Potiguara fumavam pra curar e para transmitir aos guerreiros o ``espírito da força''; os Tarairiú e Chumimy (Kariri que recusavam a conversão à fé cristã) evocavam espíritos, abençoavam casais e fertilizavam o solo com sua fumaça. 

    Mas dentre os vegetais sagrados presentes no Catimbó há um que está acima de todos: a Jurema Preta. Planta mágica por excelência, utilizada em diversos preparos medicinais do universo popular indígena-caboclo, cujo poder de cura e proteção espiritual parece não ter limites. Pinturas rupestres apontam à possibilidade de um culto à Jurema ancestral entre os índios do sertão; Olavo de Medeiros Filho e Luís da Câmara Cascudo disseram algo sobre uma bebida, produzida com o citado vegetal, capaz de levar os índios a visitar mundos espirituais e interagir com seus habitantes. Foi, provavelmente, essa característica, esse potencial da Jurema, que a tornou o ``vegetal mestre'' por excelência entre os povos indígenas e seus descendentes. Hoje, ``Jurema'', no universo mítico-cultural autóctone, além de vegetal sagrado é o nome de um Reino (um mundo espiritual) formado por reinos menores, cidades e aldeias, nos quais vivem as entidades espirituais que se manifestam nas sessões; é o nome de uma cabocla, ente espiritual e protetor; o nome de uma bebida sagrada, sacramental, comungada em determinadas sessões; e um dos nomes da tradição em si. 

    O encontro com Neta mudou minha vida. Graças a ele me encontrei. Dei início a uma pesquisa, bibliográfica e de campo, que não tem dia nem hora para acabar; e independente dos preconceitos que sofri, em diversos momentos e lugares, me assumi catimbozeiro. À proporção que interagia com casas, terreiros e comunidades indígenas e caboclas nas quais se cultuava a Jurema, conheci detalhes objetivos e subjetivos da tradição --- até que, após ter obtido sucesso em um trabalho de cura realizado com fumaça, um catimbó, fui reconhecido juremeiro por uma outra mestra: a senhora Zélia Maria, dirigente do Terreiro Tupinambá. 

    Resumindo o que até agora foi exposto, podemos dizer que o que atualmente chamamos Catimbó-Jurema, Jurema Sagrada, Culto aos Senhores Mestres ou Catimbó, é uma tradição de matriz indígena --- talvez uma das mais antigas do continente Americano --- que, independentemente dos influxos europeus e africanos sofridos ao longo do tempo, possui como principais características elementos de origem indígena: culto à Jurema Preta, considerada um vegetal sagrado em torno do qual gravitam outras plantas e seres espirituais; espíritos que, além de homens e mulheres, são pássaros, animais, seres marítimos e guardiões da floresta; defumações mágico-medicinais e evocatórias; comunhão de bebidas vegetais capazes de expandir a consciência, dentre outras. Particularmente, considero o Catimbó uma verdadeira religião. Utilizo o termo ``religião'' porque, como todas as outras, ele não deixa de ter seus meios de nos religar à Divindade (a Deus, à Natureza, às concepções caboclas de ``divino'' e ``sagrado''). Possui, além disso, uma série de ritos, inclusive breves liturgias, com base nos quais transcorrem as sessões (as chamadas ``mesas altas'', ``mesas baixas'', ``mesas rasteiras'' e ``giras de Jurema'', conforme o propósito do mestre, as necessidades dos devotos e os costumes do centro).  

    Como todas as religiões, os catimbozeiros cultuam entes espirituais. Esses seres são pássaros, cobras e outros animais sagrados, espíritos de plantas, almas de grandes pajés e índios guerreiros; além dos chamados mestres e mestras: inteligências de antigos curandeiros, raizeiros, parteiras, feiticeiros e bruxas, alguns dos quais oriundos de Portugal ou África, considerados seres humanos que, em algum momento de suas vidas, entraram em contato com a ``ciência do índio'' e passaram a trabalhar na Jurema. 

    Durante a colonização do território brasileiro, à margem dos diversos aldeamentos nos quais se aproximaram índios de etnias a princípio rivais; e devido à presença cada vez maior de judeus marranos, africanos de diversas nações, bruxas e feiticeiros degredados, sacerdotes católicos e missionários protestantes, os cultos à Jurema assumiram novas expressões conquanto assimilavam elementos adventícios. 

    Com o tempo, ao lado das pajelanças indígenas, o caldeamento colonial gerou as pajelanças caboclas --- cujas primeiras manifestações foram as chamadas ``santidades'': cerimônias realizadas por índios e portugueses nas quais pajés utilizavam trajes sacerdotais católicos e batizavam seguidores, bebiam, fumavam e entravam em êxtase. A Igreja Católica, por sua vez, perseguiu diversos ``descimentos'': evocações de entidades espirituais que ``desciam'' sobre os mestres e pajés mediante grandes defumações de tabaco e ingestão de bebida Jurema. Além desses ocorreram os ``adjuntos de Jurema'', realizados às escondidas, no meio das matas, por índios e caboclos; o ``ritual caboclo'' em que se bebia Jurema e cantava para santos católicos e seres encantados; o ``Ritual Tapuia'' no qual os caboclos ficavam nus, comiam carne crua com mel e corriam pelas matas de Canguaretama e Goianinha, durante o transe; e, finalmente, o Catimbó dos mestres juremeiros, em que antigos fundamentos de matriz indígena coexistem com práticas cabalísticas, bruxaria e feitiçaria ibérica, caracteres africanos e catolicismo popular. 

    O Catimbó-Jurema resistiu ao tempo. Algumas das inúmeras perseguições a que os membros de nossa tradição foram submetidos, desde o albor da colonização, ocorridas no território do Rio Grande do Norte, permaneceram nas memórias de alguns mestres e mestras --- alguns dos quais tive a honra de conhecer. Como quase nenhuma historiografia sobre esses acossamentos foram realizadas até então, esses fatos tendem a desaparecer para sempre, tanto da história quanto da memória popular. 

    Se a princípio os sacerdotes católicos designados à catequese não reprimiram com veemência os índios relutantes, com o tempo as perseguições se tornaram muito violentas --- uma vez que parte dos nativos resistia em abandonar costumes e práticas antigas e, por outro lado, indígenas convertidos tinham dificuldade em abrir mão de seus conjuntos originários de crenças. Os jesuítas, geralmente, não exerciam violência em sua repressão aos índios resistentes; já os padres barbadinhos, de origem italiana, catequistas dos sertões, reprimiram de modo muito violento os índios Kariri categorizados de feiticeiros. Nos Anais da Biblioteca Nacional há relatos de acontecimentos ocorridos em 1761, nas proximidades do Rio São Francisco, no território do Ceará, relacionados a índios mortos sem qualquer julgamento --- assassinados por missionários. Capuchinhos italianos, por sua vez, torturaram nativos batizados e aldeados, residentes nas missões, e queimaram seus corpos. Anos antes, em 1717, o índio João da Costa havia sido capado e açoitado, após o quê teve seu cadáver arrastado e queimado. As cinzas foram cobertas por terra. O mesmo destino tiveram as índias Theodora, Narciza, Francisca, Andreaza e Izabel \cite[p.~103--104]{Siqueira1978}. 

    Em precioso trabalho intitulado \textit{Religião como Tradução: missionários, Tupi e Tapuia no Brasil colonial}, Cristina \textcite[p.~379--406]{Pompa2003Religiao} apresenta, ao tratar da ação catequética nas aldeias do sertão nordestino, o modo de agir utilizado por padres de diversas ordens para reprimir indígenas que se esforçavam em preservar suas crenças e práticas mágico-medicinais e oraculares ancestrais --- assim como as festas em honra às antigas divindades: queimavam ocas sagradas, destruíam instrumentos e objetos ritualísticos, chicoteavam nativos e batiam-lhes com palmatória. Nesse contexto as fugas eram frequentes. Durante os séculos XVII e XVIII vários índios se embrenharam nas matas para, longe dos aldeamentos, tentar reconstruir locais de culto e sustentar valores e mitos originários. 

    Como sabemos, o território do Rio Grande do Norte, durante a colonização, foi habitado por três grandes grupos indígenas: no litoral viviam os Potiguara (partícipes do grande tronco linguístico-cultural Tupi); e nos sertões viviam as nações Tarairiú e Chumimy (ambas subdivididas em diversas comunidades que quase sempre recebiam os nomes de suas lideranças). Por mais que houvesse distinções e peculiaridades em seus universos cosmogônicos e espirituais, aqueles nativos possuíram crenças, ritos e práticas muito próximas cujos fragmentos permanecem vivos no Catimbó-Jurema dos dias atuais --- fato que observei na Jurema cultuada no litoral de nosso estado, principalmente nos municípios de Canguaretama e Goianinha, região sobre a qual me concentrei durante os sete primeiros anos de pesquisas. 

    Conforme sucedeu em outras regiões do nordeste brasileiro, as perseguições a índios e caboclos juremeiros ocorridas em território norte-rio-grandense ultrapassaram o período colonial. Se entre os séculos XVI e XVIII, 33 índios e mamelucos quedaram, de fato, presos pela Inquisição, apenas no oitocentos foram denunciados 273 índios e descendentes por diversas razões, principalmente por beber Jurema e ``descer demônios'' em meio a toques de maracás e cantigas nativas. Entre os denunciados se encontravam: a índia Antônia Guiragasu que ``tomava umas grandes fumaçadas de tabaco de cachimbo até ficar fora de si'' e ``invocava os demônios que lhe respondiam várias perguntas do outro mundo'' \apud[RESENDE, 2011][p.~55]{Angelico2020Espiritualidade}; e o Payaku Gaudêncio (tronco linguístico cultural Tarairiú), denunciado em 1756 por ``feitiçaria'' e por ter matado magicamente cerca de 50 pessoas --- com o auxílio de intérpretes (Gaudêncio não falava português) ele afirmou ser feiticeiro e disse que ``todas as vezes que bebia jurema ou angico lhe apareciam várias figuras horrendas'' \apud[CRUZ; SANTOS, 2010][p.~55]{Angelico2020Espiritualidade}.

    O Angico, assim como a Jurema, é uma das plantas sagradas mais citadas em cânticos ritualísticos e utilizadas, nas comunidades juremeiras, na manipulação de diversas medicinas tradicionais. Os povos autóctones preparavam, como visto, uma bebida enteógena com raízes da Jurema Preta; e com as sementes de Angico produziam um rapé (chamado Yopo) que, inalado, além de combater males relacionados a problemas respiratórios, também alterava o estado de consciência provocando determinadas visões.  

    A medicina aborígine, por estar relacionada à crença na atuação de espíritos (que a Inquisição categorizava de demônios), teve seu aspecto mágico-cerimonial combatido --- tendo sido as fórmulas tradicionais de confecção de diversos salvatérios, tão bem guardadas e escondidas pelos nativos que findaram perdidas. 

    Luís da Câmara Cascudo cita o caso do índio Antônio, falecido, confesso e sacramentado, em dois de julho de 1758, sepultado no adro da Igreja de Nossa Senhora da Apresentação. Antônio, que tinha aproximadamente 22 anos de idade, esteve preso na cidade do Natal em razão de sumário realizado contra os índios encontrados em ``adjunto de jurema, que se diz supersticioso'', na ``Aldea do Mepibú''. Sobre os adjuntos de Jurema, Cascudo escreveu: ``Uma festa secreta dessa indiada, no século XVIII, dizia-se `adjunto de jurema'. Adjunto é reunião, sessão, agrupamento. Faziam a bebida com a jurema e bebiam-na em meio de cerimônias que não deixaram rasto'' \citeyear[p.~27--28]{Cascudo1978Melagro}. 

    A memória popular afirma que, em algumas ocasiões, os juremeiros mortos eram desenterrados pela polícia e tinham seus corpos queimados. Se a causa para a queima era supersticiosa ou uma forma simbólica de repressão, não sabemos. No caso de Antônio, por mais que tenha sido enterrado, após confissão e sacramento, no adro de uma igreja, o triste fato de ter sido preso e morrer em consequência de um adjunto de jurema aponta para a existência de violentas ações repressivas aos juremeiros no Rio Grande do Norte. 

    Um dos casos mais interessantes de juremeiros perseguidos no Rio Grande do Norte, foi o ocorrido com Manoel Remígio (no bairro atualmente chamado Tirol), relatado no jornal A REPÚBLICA de 27 de outubro de 1900 --- citado por Sérgio Santiago em \textit{Ritual Umbandista} \citeyear[p.~15--16]{Santiago1973Ritual}, do qual transcrevo-o integralmente.

    \begin{quotation}
        Ontem por volta da meia noite a Polícia fez uma boa colheita. Foi o caso que o indivíduo de nome Manoel Remígio do Nascimento, antigo profissional de ``feitiçaria'', tinha convocado uma sessão para o esquisito local, próximo à lagoa, conhecida por a Lagoa de Manoel Felipe, cerca de meia légua distante desta cidade, o que efetivamente se realizou. 

        O velho pajé, vendo-se face a face com um agente policial corajoso e enérgico, assim desautorado e interrompido em meio da sessão magna, onde a alquimia de par com a encenação mágica tinha boquiaberta e presa toda a assistência, composta de onze pessoas, tentou nesse lance oferecer alguma resistência\dots 

        Preso o Remígio e mais os seus crentes foram conduzidos a esta Capital à presença do Dr. Francisco Carlos, que como salutar ensinamento ao Pajé natalense, apesar do seu misterioso saber, mandou-o repousar das fadigas, lá no palácio do Cabo André. 
    \end{quotation}

    Como pode ser percebido, durante cerca de quatrocentos anos, índios e caboclos não encontraram paz. Por mais que se esforçassem, não havia sossego à realização de ritos ancestres ou local no qual conseguissem realiza-los com o mínimo de segurança. A ridicularização e humilhação de Manoel Remígio e dos que assistiam sua pajelança é objetivamente exposta nas páginas do citado periódico.  

    Casos de perseguição policial como o acima citado parecem ter sido comuns no território norte-rio-grandense. O Catimbó entrava no rol das práticas de feitiçaria proibidas, formal e informalmente, pela justiça imbuída de valores judaico-cristãos coloniais que localizavam no campo do condenável e exterminável o conjunto de manifestações espirituais de matriz indígena e seus participantes. 

    Em Canguaretama tomei nota de caso análogo, ocorrido com dona Inácia Maria da Conceição, avó da mestra Neta (citada no início deste artigo). Segundo sua descendente, dona Inácia teria sido uma das primeiras pessoas do município a trabalhar nas ``mesas'' --- tendo começado a realizar suas práticas espirituais aos cinco anos de idade, sob luz de candeeiro (como Remígio, no início do século XX) Na época não existiam centros na cidade (as sessões ocorriam nas casas de pessoas doentes, nas próprias moradas dos juremeiros ou no meio das matas sob um pé de jurema preta), assim como não havia Umbanda ou Candomblé: a repressão policial coibia a realização de atividades do gênero. 

    Aos dez anos de idade, dona Inácia teria sido perseguida e presa pela polícia, porém, após adivinhar um problema familiar do delegado e ter deixado os policiais perplexos, foi posta em liberdade. O acontecimento fez com que Inácia jamais voltasse a ser procurada pela justiça. Segundo a mestra Neta, sua avó teria aprendido a ``Ciência da Jurema'' com uma cabocla (o que aponta para a existência de processos de transmissão oral muito antigos, sendo essa oralidade a principal ferramenta à preservação do que os caboclos chamam de ``fundamentos'') e desde os treze anos de idade realizava partos sem cirurgia, estancava hemorragias, fazia cair dente doente e puxava leite de peito através de rezas-fortes; adivinhava o sexo dos bebês, curava e benzia --- tendo trabalhado na Jurema até os 72 anos de idade.

    Nos dias de hoje, dona Neta não caminha mais entre nós. Para os juremeiros, ela se tornou uma Mestra da Jurema. Seu sobrinho, porém, é o atual dirigente do Centro Mestre Pena Branca e Estrela do Mar. Consegui acompanhar, durante alguns anos, o desenvolvimento do rapaz naquele centro: aos quatro anos de idade, tinha visões de pessoas falecidas e seres do Encanto, adivinhava a localização de indivíduos distantes e a existência de malefícios em pessoas participavam dos trabalhos dirigidos por sua avó; recebia, em alguns casos, Mestres de Jurema. Por volta dos sete anos, afirmava que seguiria o mesmo caminho da Mestra; e hoje é ele o responsável pelo Centro que, atualmente, possui um corpo de discípulos. 

    Conforme o exposto neste artigo, o Catimbó norte-rio-grandense, malgrado as severas perseguições e deturpações sofridas ao longo de mais de quatrocentos anos, resistiu aos desafios e dificuldades característicos da sociedade colonial. Os dados que possuímos, embora possam ser considerados poucos, apontam para atuações muito cruéis realizadas contra índios e caboclos catimbozeiros principalmente durante o século XVIII --- ainda que a oralidade e os parcos relatos remanescentes indiquem a continuidade repressora até o início do século passado. Hoje, século XXI, a polícia não mais persegue ou prende alguém por seu conjunto de crenças ou matiz espiritual. Porém, perseguições ideológicas permanecem vivas em nossa sociedade e os juremeiros, assim como umbandistas e candomblecistas, continuam sendo citados, em discursos radicais de proveniência cristã, como pagãos e adoradores do diabo.

    \nocite{Assuncao2006Reino}

    \printbibliography[heading=subbibliography,notcategory=fullcited]

    \hfill Recebido em 3 mai. 2021.

    \hfill Aprovado em 4 mai. 2021.

    \label{chap:silencioend}

\end{refsection}
