\begin{refsection}
    \renewcommand{\thefigure}{\arabic{figure}}

    \chapterTwoLines
    {Melhoramentos de São Paulo}
    {intervenções urbanas e as irmandades negras da capital}
    \label{chap:melhoramentossao}
    
    \articleAuthor {Alvaci Mendes da Luz}
    {Mestrando em História Social pela Pontifícia Universidade Católica de São
    Paulo, SP. Bolsista CAPES. Licenciado em filosofia pela Faculdade de
    Filosofia São Boaventura de Curitiba, PR e Bacharel em teologia pelo
    Instituto Teológico Franciscano de Petrópolis, RJ. Lattes ID:
    3967.2455.4903.4773. ORCID: 0000-0002-8929-1240. E-mail: alvaci@gmail.com}

    \begin{galoResumo}
        \marginpar{
            \begin{flushleft}
                \tiny \sffamily
                Como referenciar?\\\fullcite{SelfLuz2021}\mybibexclude{SelfLuz2021},
                p. \pageref{chap:melhoramentossao}--\pageref{chap:melhoramentossaoend},
                \journalPubDate{}
            \end{flushleft}
        } Partindo da análise do uso corrente do termo ``melhoramentos'' e da
        ``questão sanitária'' nas intervenções urbanas em São Paulo de meados
        do século XIX e início do XX, o presente artigo se propõe averiguar o
        quanto o urbanismo, que se configurava como saber, influenciou na
        redefinição do espaço urbano ao longo de quase um século. Desde o
        período colonial, as igrejas de irmandades católicas no centro da
        capital se destacaram como espaços de sociabilidade dos negros,
        marcando ao longo dos séculos os lugares de manutenção da religiosidade
        afro-brasileira, das heranças culturais advindas do período colonial e
        de certa autonomia administrativo-financeira da comunidade negra de São
        Paulo. No final do XIX e início do XX as igrejas pertencentes a estes
        grupos sociais, localizadas no triângulo central, bem como o seu
        entorno, sofrerão significativas intervenções da municipalidade.

    \end{galoResumo}
    
    \galoPalavrasChave{Confrarias católicas. Irmandades negras. Melhoramentos. Urbanismo. Higienismo.}
    
    \begin{otherlanguage}{english}
    
    \fakeChapterTwoLines
    {Improvements in São Paulo}
    {urban interventions and the black brotherhoods of the capital}

    \begin{galoResumo}[Abstract]
        Based on the analysis of the current use of the term ``improvements'' and the ``health issue'' in urban interventions in São Paulo in the mid-19th and early 20th centuries, this article aims to investigate how urbanism, which was characterized as a knowledge, influenced the configuration of urban space for almost a century. Since the colonial period, churches of Catholic brotherhoods in the capital downtown have stood out as spaces of sociability for the black people. What leads those spaces over the centuries to figure as places of preservation of Afro-Brazilian religiosity, the colonial period cultural heritage, and certain administrative autonomy by the black community. In the late 19th and early 20th centuries, the churches belonging to these social groups, located in the central triangle and surroundings, will suffer significant interventions by the municipality.
    \end{galoResumo}
    
    \galoPalavrasChave[Keywords]{Catholic brotherhoods. Black brotherhoods. Improvements. Urbanism. Hygienism.}
    \end{otherlanguage}

    \section{Introdução}

    \begin{quotation}
        \textit{O predomínio dos ``benedictos'' na que chamavam agora egreja de São Benedicto nenhuma vantagem lhe trouxe: andava esta suja e mal cuidada, por toda parte a desordem e o desleixo. No louvável intuito de obviar esses lamentáveis desconcertos, reuniram-se alguns lentes e antigos alumnos do ``Curso Jurídico'' e fundaram com a acquiescencia do Provincial, uma ``Irmandade de São Francisco'', que tinha por fim cuidar do culto e da conservação e asseio da igreja. Parece, porém, que este zelo dos ``doutores'' não foi de longa duração, continuando os ``benedictos'' a infelicitar o bello templo franciscano. Aos 5 de outubro de 1908, uma nova era abriu-se para a egreja do Convento de S. Francisco de S. Paulo. Em dependências da sacristia vieram morar os nossos confrades [\dots]. Fr. Basílio, que em seu tempo fez importantes melhoramentos na egreja, teve que sustentar tremenda luta com os menos disciplinados filhos de S. Benedicto.} \cite[p.~82]{Rower1922Provincia}.
    \end{quotation}

    Com estas palavras, o historiador e frade franciscano alemão, Basílio Röwer\footnote{Basílio Röwer chegou ao Brasil nas primeiras levas de frades alemães, logo após a Proclamação da República, vindos da Província de Santa Cruz da Saxônia. São conhecidos como ``restauradores'' pois seu objetivo foi o de revitalizar as Províncias Franciscanas em processo de decadência no Brasil desde o final do século XVIII. Ele é o responsável por recontar a história da antiga Província da Imaculada Conceição, bem como, foi o terceiro superior do Convento de São Francisco no período em que ele foi retomado das mãos da Irmandade de São Benedito.}, apresentou os argumentos para exaltar a ``nova era'' pela qual passava a igreja de São Francisco do centro da capital paulista. Ao historiar sobre os franciscanos, em seu livro elaborado para as comemorações do primeiro centenário da independência do Brasil em 1922, ele não poupou adjetivos para desqualificar a Irmandade de São Benedito\footnote{Irmandade de negros católicos instalada na capital paulista desde o século XVII no Convento franciscano do Largo São Francisco. Obteve aprovação oficial - civil e eclesiástica - em 22 de outubro de 1772. Com a saída dos franciscanos da capital em 1828, após a criação do ``Curso de Sciencias Jurídicas e Sociaes'' no Brasil (Direito), passam a administrar a Igreja de São Francisco. Esta administração durará até o retorno dos frades para a cidade de São Paulo. Um estudo mais amplo sobre esta irmandade está sendo desenvolvido por este pesquisador e tem como título: \textit{``Os pretos de São Benedito: a ascensão de uma irmandade negra na Imperial cidade de São Paulo'' (1854--1890).}} que por mais de oitenta anos (1828--1910) havia administrado aquela igreja localizada no triângulo central\footnote{Triângulo Central é a área urbana da capital paulista localizada entre os três principais conventos de religiosos católicos no período colonial, a saber: Mosteiro de São Bento (beneditinos), Convento do Carmo (carmelitas) e Convento de São Francisco (franciscanos). Há outras definições e delimitações sobre este espaço, convencionamos aqui usar este que se limita a área dos Conventos.}. Nas primeiras décadas do XX ela seria disputada pelos recém-chegados religiosos alemães e pelos irmãos leigos\footnote{Entende-se por ``leigo'' todo membro de determinada confraria que não faz parte do clero institucionalizado, ou seja, que não é sacerdote secular ou religioso do clero regular. Sobre os leigos nas confrarias católicas mineiras, assim se expressa Julita Scarano: ``Podemos dizer que nessas organizações é que se manifestava realmente o espírito religioso da população, que congregava os elementos das mais variadas categorias sociais. É interessante notar que tais elementos eram homens e mulheres que levavam vida comum, mas que patrocinavam o culto, construíam igrejas, paramentavam-nas, organizavam assim a vida católica local. Realmente, o leigo da irmandade mineira se considerava a própria igreja, julgando poder intervir em quase todas as questões eclesiásticas. Via no padre apenas aquele que tem capacidade de dizer missa e distribuir os sacramentos e somente nessas oportunidades se sobrepunha aos membros das irmandades. Estes sempre manifestaram atitude insubmissa em relação à autoridade eclesiástica, fato sentido mesmo pelos bispos''. \cite[p.~28]{Scarano1978Devocao}.} de São Benedito. Argumentos como a falta de asseio, o desleixo e a desordem, foram utilizados para depreciar a presença daquela associação naquele espaço físico, em contrapartida aos ``melhoramentos'' realizados no templo com a chegada dos frades europeus.  

    O final do século XIX e o início do XX foram decisivos para as confrarias\footnote{``As confrarias, divididas principalmente em irmandades e ordens terceiras, existiam em Portugal desde o século XIII pelo menos, dedicando-se a obras de caridade voltadas para seus próprios membros ou pessoas carentes não associadas. Tanto as irmandades quanto as ordens terceiras, embora recebessem religiosos, eram formadas sobretudo por leigos, mas as últimas se associavam a ordens religiosas conventuais (franciscana, dominicana, carmelitana), daí se originando seu maior prestígio. As irmandades comuns foram bem mais numerosas. Da metrópole se espraiou para o Império Ultramarino, o Brasil inclusive, o modelo básico dessas organizações''. \cite[p.~60]{Reis1991Morte}.} católicas em todo o território nacional. Dirigidas majoritariamente por leigos, as confrarias se instalaram no Brasil colonial desde o século XVII e ao longo do XVIII foram se organizando com estatutos próprios, critérios de adesão de cor, pureza de sangue, raça e status social.  Estavam alicerçadas nos ritos e costumes da tradição católica ibérica desde o século XIII no continente europeu. Este modelo barroco de catolicismo, afirma \textcite{Reis1991Morte}, adentrou o século XIX nas cidades brasileiras, mas será desde o início daquele período fortemente impactado pelos novos conceitos urbanistas, sanitaristas e higienistas\footnote{João José Reis faz uma importante análise deste fenômeno em seu livro \textit{A morte é uma festa}, tendo como base a revolta popular organizada pelas confrarias católicas de Salvador contra a inauguração do primeiro cemitério público da capital baiana em 1836. No capítulo 10 o autor pondera o quanto a medicina e suas teorias miasmáticas e microbianas influenciaram na decadência dos costumes das irmandades, entre eles, o enterro no interior das igrejas. Para mais informações sobre esse assunto ver \fullcite[p.~307--339]{Reis1991Morte}.}. 

    Em São Paulo verifica-se a presença de irmandades desde o século XVII, dentre elas, as compostas por negros --- escravos ou libertos --- mais conhecidas foram a Irmandade de Nossa Senhora do Rosário, a Irmandade de Santa Ifigênia e Santo Elesbão e a Irmandade de São Benedito. Ambas se organizavam em torno do santo de devoção, das festas anuais, das procissões, da compra de alforrias e aquisição de bens, e, particularmente na preocupação quanto ao local onde os irmãos seriam sepultados e as missas que seriam celebradas por sua alma\footnote{Sobre as confrarias de pretos e pardos ver: \fullcite{Quintao2019Contribuicoes}; \fullcite{Boschi1986Leigos}; \fullcite{Reis1991Morte}; \fullcite{Scarano1978Devocao}; \fullcite{Viana2007Idioma}}.

    A partir de meados do XIX, a área urbana de São Paulo que desde o período colonial se concentrava sobre o planalto de Piratininga, local onde se localizavam os espaços de sociabilidade dessas irmandades, começou a sofrer intervenções no seu traçado das ruas, avenidas, praças, bem como de suas edificações. Estas mudanças atingiram diretamente um dos lugares mais importantes dos negros na urbes: o território das igrejas das irmandades\footnote{Sobre os territórios de sociabilidade negra na capital paulista ver: \fullcite{Bertin2006MeiaCara}; \fullcite{Bertin2010Sociabilidade}; \fullcite{Comar2008Imagens}; \fullcite{Santos1998NemTudo}.}.

    \section{Melhoramentos de São Paulo: um debate historiográfico}

    Uma das palavras mais utilizadas para designar as intervenções urbanas em São Paulo foi a palavra ``melhoramentos'' e suas expressões correlatas como ``melhoramentos urbanos'', melhoramentos materiais, entre outros\footnote{Reforçam a afirmação do uso corrente do termo ``melhoramentos'' as pesquisas realizadas por \textcite{Bresciani2001Melhoramentos}, \textcite{Cerasoli2004Modernizacao} e \textcite{Borin2019Passeios}.}. A historiadora Maria Stella \textcite{Bresciani2001Melhoramentos} ao se referir a este termo analisa o quanto ele permaneceu em diferentes enunciados na cidade para designar, na grande maioria das vezes, os benefícios realizados em prol do ``progresso''. Ela parte da hipótese de que a palavra em questão, além de ser usada como lugar-comum, ou seja, relacionada com a ideia de um acréscimo positivo àquilo que se refere, também atua como metáfora e assim ``articula um sentido a uma representação, ou uma realização mental sob a forma de imagem'' \cite[p.~343--366]{Bresciani2001Melhoramentos} e pode ser ressignificada ao longo das décadas.

    Para pensar o trabalho das metáforas, \textcite{Bresciani2001Melhoramentos} recorre ao filósofo francês Paul Ricœur que entende ``o processo metafórico como cognição, imaginação e sentimento''. Por não haver uma teoria semântica da metáfora a ela se precisa acoplar uma teoria psicológica, da imaginação e do sentimento. Ricœur chamará isso de ``função pictórica do sentido metafórico'' e \textcite{Bresciani2001Melhoramentos} em sua hipótese sobre a palavra melhoramentos irá usá-la no sentido das semelhanças, das similaridades, como transferência de significados que a palavra pode trazer nesse sentido \cite[p.~350]{Bresciani2001Melhoramentos}. A adaptação do termo para além do lugar-comum, o manteve nos planos de reformas e nas pautas urbanas por tanto tempo. Assim, melhoramentos: 

    \begin{quotation}
        Refere-se sempre a objetos concretos, projeções de intervenções e/ou obras realizáveis que, pela dimensão imagética desenhada ou sugerida pela linguagem, são capazes de provocar em quem escuta, lê ou vê, o sentimento de serem partícipes, ou excluídos, de uma ação coletiva orientada no sentido de uma modelo ideal de cidade moderna, imagem essa que não se imobiliza numa dada representação, mas se desloca constantemente, acompanhando os sucessivos deslocamentos nas concepções de cidade ideal. \cite[p.~351]{Bresciani2001Melhoramentos}.
    \end{quotation}

    Em outras palavras, tendo como base a proposta de Ricœur, \textcite{Bresciani2001Melhoramentos} procura ``apreender a palavra melhoramentos, também, como metáfora aplicada a múltiplas situações portadoras de benefícios à cidade e a sua população''. Deste modo ``tomando melhoramentos como uma metáfora que põe ante nossos olhos uma cadeia ou sequência de semelhanças entre artefatos diferentes, pode-se entender a força explicativa (racional) e persuasiva (emocional) da palavra quando utilizada em um argumento''. \cite[p.~350--351]{Bresciani2001Melhoramentos}.

    Assim, melhoramentos pode ser usada pela municipalidade paulistana para descrever as intervenções urbanas implementadas desde 1850 até 1950 na cidade como lugar comum ou como metáfora. É neste período também que são criados alguns órgãos oficiais, que usam como base os conceitos do urbanismo que se configurava como saber. Dentre as atribuições destes órgãos estavam as de fiscalizar, punir e coordenar obras de pavimentação em ruas, calçadas e praças, bem como as de vistoriar lugares de moradia das classes trabalhadoras, como os cortiços\footnote{``O reconhecimento das más condições sanitárias de certas áreas da cidade e, em particular, das péssimas condições de asseio das moradias coletivas constitui presença constante nos relatórios de autoridades médicas desde pelo menos 1885, quando o médico da Câmara Dr. Eulálio da Costa Carvalho, dirige-se à Comissão de Justiça alertando-a da necessidade de normas que estipulassem critérios para a demolição dos cortiços `julgados inconvenientes ou prejudiciais à saúde dos seus habitantes' e, ao mesmo tempo, orientassem a manutenção da higiene dos existentes e dos que ainda fossem construídos.'' \cite[p.~19]{Bresciani2010Sanitarismo}.}.  

    Em um outro artigo, \textcite{Bresciani2010Sanitarismo} conduzirá sua reflexão para a compreensão da fundamental importância que as questões sanitárias adquiriram ao longo do século XIX e o quanto elas influenciaram na formação de um saber urbanístico. Entrecruzando-se com uma multiplicidade de conhecimentos, o urbanismo se constituirá como disciplina operativa e estará na base das intervenções do final do XIX e início do XX \cite[p.~15]{Bresciani2010Sanitarismo}. Afirma a autora:

    \begin{quotation}
        Um expressivo diálogo entre especialistas de diversas nacionalidades e formações --- médicos higienistas, engenheiros sanitaristas e legisladores --- dá lugar, no decorrer do século XIX, a um ``saber atuar'' sobre a materialidade dos núcleos urbanos e sobre o comportamento citadino, constituindo um campo de ação especializado. Não há para cada uma dessas especialidades um desenvolvimento interno próprio. A formação técnica dos especialistas constitui-se a partir de elementos que se cruzam com questões filantrópicas, religiosas e morais, tecendo um complexo campo de conceitos e de ``pré-conceitos''. \cite[p.~28]{Bresciani2010Sanitarismo}. 
    \end{quotation}

    Em uma rápida trajetória pelo urbanismo que se formava na Europa do dezenove, a autora nos traz elementos da multiplicidade constitutiva desse saber. Dentre eles, afirma que ``a coparticipação dos saberes do médico e do engenheiro nas primeiras intervenções nas cidades no século XIX encontra na conjunção industrialização e crescimento demográfico sua explicação mais evidente'' \cite[p.~28]{Bresciani2010Sanitarismo}. Densidade demográfica e industrialização, unidos a epidemias mortais naquele período, conforme análise do filósofo Fraçois Béguin, citado por \textcite{Bresciani2010Sanitarismo}, serão elementos fundamentais para a ``conscientização dos problemas sanitários e a formulação de uma prática intervencionista governamental nas cidades, prática apoiada nos saberes da medicina e da engenharia'' \cite[p.~28]{Bresciani2010Sanitarismo}.  

    Na Inglaterra, aponta ainda \textcite{Bresciani2010Sanitarismo}, uma série de medidas sanitárias preventivas adotadas pelo governo ao longo do dezenove, para melhorias de ambientes de água estagnada, lixo acumulado e esgotos em regiões das cidades habitados pela classe pobre trabalhadora, que aliavam a ``parceria duradoura entre o médico --- no cuidado dos corpos --- e o engenheiro --- nas ações de saneamento urbano'' \cite[p.~29]{Bresciani2010Sanitarismo} seriam elogiados pelo médico Jules Rochard, um dos primeiros a apontar o pioneirismo inglês na adoção dos princípios de higiene. 

    Em São Paulo, \textcite{Bresciani2010Sanitarismo} analisou o impacto que essas ideias causaram sobre os lugares de moradias populares, como os cortiços de Santa Ifigênia e o bairro de operários do Brás. Seu estudo aponta que tanto as classes mais populares quanto os letrados tinham conhecimento de todo processo urbanista que se configurava na Europa ao longo do dezenove. Ela afirma que ``descontados os aspectos estritamente técnicos do saber higienista [\dots], pode-se dizer que os preceitos da ``questão sanitária'' encontravam-se largamente difundidos entre a população.'' \cite[p.~19]{Bresciani2010Sanitarismo}. 

    Nos primeiros anos do regime republicano, ``as epidemias de febre amarela no interior paulista e na cidade de Santos põem em alerta as autoridades públicas que estabelecem programas de ``visitas domiciliares'' em áreas consideradas críticas'' \cite[p.~19]{Bresciani2010Sanitarismo}.  Os preceitos da questão sanitária aliados ao saber técnico justificarão as intervenções forçadas sobre os espaços de sociabilidade, moradias e trabalho, das classes menos favorecidas da população paulistana em uma cidade que crescia em rápida expansão populacional\footnote{``A partir da década de 1880, grandes levas de imigrantes impulsionaram o crescimento da cidade que, em 1886, passa a contar com 44.030 habitantes, concentrados em sua significativa maioria nos distritos centrais da Sé, Santa Efigênia e Consolação. Dobrara, portanto, o número de pessoas em comparação ao censo de 1872, que avaliara em 23.243 o número de habitantes na cidade. A explosão demográfica dar-se-ia nos anos subsequentes: 1890, com 64.934, e 1893, com 192.409 habitantes (MORSE, 1970, p. 238). Na virada do século XIX para o XX, a população da cidade seria estimada em mais de 200 mil habitantes.'' \cite[p.~20]{Bresciani2010Sanitarismo}.}. Assim:

    \begin{quotation}
        A ``higiene física'' conjugada a ``higiene social'' passava a exigir a aeração do tecido urbano muito denso, para isso contribuindo a presença de árvores e fontes e a implantação de equipamentos técnicos próprios a dar vazão aos mais variados fluxos --- água, esgoto, gás, veículos. Formava-se uma nova sensibilidade sensorial dos pontos de vista olfativo e visual que estabelecerá sólidos liames entre as intervenções nas cidades e a noção de embelezamento, a duradoura relação entre o belo, o estético e a limpeza. \apud[p.~66]{Bresciani2014Cidade}[p.~12]{Borin2019Passeios}.
    \end{quotation}

    Em outras cidades o processo era bem parecido. Sandra \textcite{Pesavento2001EraUmaVez} ao estudar os ``becos'' da capital gaúcha, localizados em sua maioria no centro e habitados por uma população pobre, negra e de pequenos comerciantes, verificará o quanto estes lugares passarão ao longo do dezenove por processos intervencionistas. Ao analisar os discursos da imprensa daquele período, a historiadora perceberá como os becos se ``tornarão lugares perigosos'', sendo usados como argumentos para intervenções sanitaristas municipais.  Afirma: 

    \begin{quotation}
        O beco passa a ser [em fins do século XIX] a designação que estigmatiza lugares malditos da urbe. O beco é sinistro, sujo, perigoso e feio. É o mau lugar, por onde circulam personagens perigosas praticantes de ações condenáveis. O beco é o reduto dos excluídos urbanos e corresponde, de forma exemplar, a uma bela demonstração do que poderíamos chamar a maneira conflitiva de construir o espaço urbano. \cite[p.~115]{Pesavento2001EraUmaVez}.
    \end{quotation}

    \textcite{Pesavento2001EraUmaVez} pontua que ``na voz dos jornais da época, os `becos' são sempre sórdidos, sujos, imundos. A designação alude à imagem da cidade que se quer destruir, é o opróbrio, velhice, feiura, crime e vício'' \cite[p.~19]{Pesavento1999Lugares}. Sendo assim, faz-se necessário intervenções sobre estes lugares pelo bem comum, pela boa higiene e para evitar doenças. Ela afirma que na virada do XIX para o XX o termo muda de significado e ``o sentido original, de natureza mais propriamente topográfico, de rua estreita, com ladeira [\dots] cede lugar a uma designação depreciativa que traduz uma avaliação ao mesmo tempo moral, estética e higiênica'' \cite[p.~115]{Pesavento2001EraUmaVez}.

    Esta mesma noção de corrupção dos lugares pobres é verificada no Rio de Janeiro por \textcite{Chalhoub1996Cidade}. De lá ele nos traz os indícios de o quanto as pressões exercidas sobre os moradores dos cortiços pelos atores deste processo de limpeza urbana afetaram os menos favorecidos da capital fluminense. Assinala:  

    \begin{quotation}
        Os corticeiros reclamavam que eram inexequíveis `as ordens continuadas' da Inspetoria de Higiene para fechamento de estalagens. Em primeiro lugar, porque não havia para onde remover os moradores, e não era correto sujeitar o `grande número de famílias ao vexame e às inconveniências de verem transferidos seus lares para a praça pública. \cite[p.~49]{Chalhoub1996Cidade}.
    \end{quotation}


    Enfim, no início do XX a ``maturidade do debate internacional centra as preocupações urbanísticas de modo mais sistemático nas questões relacionadas a moradia, ao trânsito, às áreas verdes e às grandes cidades'' \cite[p.~32]{Bresciani2010Sanitarismo} e evidenciam uma cidade doente, em que ``os lugares onde a patologia se manifesta são sempre as moradias operárias e dos pobres em geral'' \cite[p.~32]{Bresciani2010Sanitarismo}. Todo esse processo alicerçava os projetos urbanísticos daquele período e influenciava o caminho escolhido pelo Brasil nos seus planos de melhoramentos.

    \section{Igrejas de irmandades negras: intervenções, demolições e desapropriações}

    Alguns estudos recentes, bem como alguns já consagrados, procuraram dar visibilidade às confrarias negras na cidade de São Paulo, em contrapartida ao vasto material produzido sobre suas congêneres em cidades mineiras, baianas, fluminenses e pernambucanas\footnote{Alguns dos estudos mais recentes sobre as confrarias católicas em São Paulo, são: \fullcite{Comar2008Imagens}; \fullcite{Quintao2019Contribuicoes}; e \fullcite{Santos2020Igrejas}.}. É interessante notar a influência que essas associações, cada vez mais crescentes na colônia no século XVII e com maior relevância no século XVIII, tiveram sobre as populações negras nas cidades. Lucilene \textcite{Reginaldo2011Rosarios}, em seu trabalho sobre as irmandades e devoções africanas na Bahia setecentista, acentua que ``as associações leigas foram mais numerosas e influentes, do ponto de vista religioso e social, nos centros mais urbanizados'' \cite[p.~25]{Reginaldo2011Rosarios}.

    Em sua importante pesquisa, ao situar a relevância das confrarias nas cidades coloniais, tendo a Bahia como foco central, os estudos de \textcite{Reginaldo2011Rosarios} nos mostram o que ela chamou de ``espaços privilegiados de elaboração de uma nova religião no Atlântico: o catolicismo negro''. Citando Roger Bastide ela afirma que mesmo sendo uma imposição do regime escravista, este catolicismo impositivo acabou permitindo a criação de espaços de culto e reuniões mais ou menos autônomos, que aconteciam nas irmandades e confrarias negras.
    
    Também João José \textcite{Reis1996Identidade} afirma que ``entre as instituições em torno das quais os negros se agregaram de forma mais ou menos autônoma, destacam-se as confrarias religiosas, dedicadas a devoção dos santos católicos. Elas funcionavam como sociedades de ajuda mútua''. Assim ''a irmandade representava um espaço de relativa autonomia negra, na qual seus membros --- em torno das festas, assembleias, eleições, funerais, missas e assistência mútua --- construíam identidades sociais significativas.'' \cite[p.~44]{Reis1996Identidade}.
    
    No caso da capital paulista, Enidelce \textcite{Bertin2010Sociabilidade} que estudou lugares de sociabilidade negra na cidade, frisa que apesar de a população de cor estar sujeita a tentativas de controle ela ``encontrou meios de resistência na ocupação de alguns espaços como a região das igrejas de Nossa Senhora do Rosário, de Santa Ifigênia e de São Benedito, que eram sedes das irmandades''. \textcite{Bertin2010Sociabilidade} vai mesmo afirmar que ``é clássico na historiografia o uso das irmandades religiosas como lócus para observação da sociabilidade entre negros --- libertos e escravos'', por serem estes lugares, espaços de diferenciação de grupos étnicos africanos nas cidades \cite[p.~127]{Bertin2010Sociabilidade}.
    
    São estes locais de ligeira autonomia negra, de manifestação religiosa e até mesmo de certa emancipação financeira que serão alvo de intervenções, desapropriações e alijamento do centro urbano. Alegações como falta de asseio, barulho e perigo naqueles lugares serão comuns nos órgãos de imprensa e nas justificativas da municipalidade. \textcite{Bertin2010Sociabilidade} continua:  
    
    \begin{quotation}
        Os arredores daquelas igrejas, além da rua das Casinhas e da rua do Comércio eram, sem dúvida, os locais mais importantes da sociabilidade negra na cidade do oitocentos. A região atraia transeuntes, não apenas pelas irmandades, mas também por conta do comércio ambulante e informal. \cite[p.~129]{Bertin2010Sociabilidade}.
    \end{quotation}


    Umas das irmandades de negros mais estudadas em São Paulo, é aquela que se reunia na Igreja de Nossa Senhora do Rosário dos Pretos, localizada até o início do século XX no Largo do Rosário, hoje praça Antônio Prado. A professora \textcite{Quintao2002Irmandades} percorreu em sua pesquisa a história deste importante grupo social e trouxe elementos interessantes sobre sua relevância  na cidade, a adesão dos negros a esta confraria, os bens adquiridos ao longo de sua história, as festas e os enterros em seu cemitério particular \footnote{Para saber mais sobre a Irmandade do Rosário dos Pretos de São Paulo, ver: \fullcite{Quintao2002Irmandades}}.
    
    \textcite{Santos2020Igrejas}, outro estudioso sobre o Rosário dos Pretos, afirma que ao longo do XIX os administradores da irmandade foram adquirindo maior patrimônio imobiliário chegando a se consagrarem entre um dos vinte maiores detentores de bens na região urbana de São Paulo em 1809 \cite[p.~180]{Santos2020Igrejas}. Ao longo daquele século foram adquirindo imóveis e casebres nas proximidades da Igreja do Rosário o que chamou a atenção do poder legislativo paulista. Em 1858 a Câmara Municipal já cogitava a desapropriação dos terrenos e casas anexas a Igreja, o que foi efetivado com a lei de nº 670 de 1903 que tornou ``de utilidade pública, para o fim de serem desapropriados os terrenos e prédios necessários ao aumento do Largo do Rosário'' \cite[p.~183]{Arroyo1966Igrejas}, na gestão do prefeito Antônio Prado.  
    
    O relato de moradores do Largo do Rosário, elencados por \textcite{Arroyo1966Igrejas}, de cantorias noturnas de pretos e ritos fúnebres que seguiam até altas horas, podem ter sido usados como justificativas para algum tipo de sanção moral sobre eles \cite[p.~181--182]{Arroyo1966Igrejas}. Ao falar sobre a igreja do Rosário, \textcite{Santos2020Igrejas} afirma:

    \begin{quotation}
        Para a sociedade paulista, a expulsão das irmandades dos negros das regiões que se tornavam privilegiadas devido ao explosivo crescimento da capital, teria como um dos argumentos atender aos padrões de higienização propostos para São Paulo [\dots]. A memória dos negros seria apagada no ano de 1903, com a substituição da designação Largo do Rosário para Praça Antônio Prado e a construção da igreja na periferia da cidade. \cite[p.~181]{Santos2020Igrejas}.
    \end{quotation}

    Em 1886, o Código de Posturas Municipal\footnote{``O Código de Posturas Municipal era uma legislação bastante ampla, reunindo em um único documento diversas normativas relacionadas a ocupação, comportamento dos habitantes e manutenção da cidade. Ele trata tanto da ocupação física, determinando regras pera edificações e arruamentos, quanto das normas de convivências para realizações de festejos nas ruas, de funcionamento de estabelecimentos comerciais e de circulação para bondes e carroças, além de conter uma série de medidas dedicadas às questões sanitárias e higiênicas''. \cite[p.~7--8]{Borin2019Passeios}.} discorria em seus artigos sobre uma série de intervenções possíveis no espaço urbano, ``o título XVIII versa sobre `vagabundos, embusteiros, tiradores de esmolas e rifas'{}'', chegando a legislar até mesmo sobre as formas de se permanecer no espaço público. ``Outra regulação do uso da rua semelhante é feita no art. 257, que proíbe `os alaridos, vozerias e gritarias pelas ruas. O infrator incorrerá na multa de 5\$ ou 24 horas de prisão'{}''\footnote{\apud[São Paulo (Município). Código de Posturas do Município de São Paulo. Diário Oficial do Município, São Paulo, 6 out. 1886. Seção 1, p. 8503][p.~11]{Borin2019Passeios}}.  
    
    Importante pontuar que ao regular sobre as ruas, o Código de Posturas intervia sobre práticas comuns da população que desde o período colonial eram exercidas nas vias, praças e becos como estratégias de sobrevivência das populações negras e pobres. ``No ano anterior ao Código de Posturas, o jornal \textit{A Província de São Paulo} publicou uma reclamação sobre as quituteiras da Ladeira do Acu, posteriormente batizada Ladeira São João'', nele ``as queixas partiam do fato que essas mulheres ficavam sentadas no passeio com seus tabuleiros de fruta, e seguia reclamando da sujeira deixada pelos restos das mercadorias das quituteiras''. Como afirma \textcite{Borin2019Passeios}, ``as Posturas tencionavam velhos e novos embates na cidade, dialeticamente articulando repressão a antigos comportamentos e à projeção de um modelo de nova civilidade'' \cite[p.~12]{Borin2019Passeios}.
    
    Em uma outra igreja de irmandade negra na região central, a de Santa Ifigênia e Santo Elesbão, localizada próxima aos cortiços analisados por \textcite{Bresciani2010Sanitarismo} - que inclusive dá nome àquele bairro --- ``o fim do século XIX reservaria mais um fato adverso'', um ``decreto promulgaria o fim da irmandade e das disputas entre o vigário e os devotos daqueles santos negros em 1890''. A dissolução daquele grupo foi amplamente noticiada na imprensa da época e em 1912 a igreja passou ``então a se chamar Igreja Nossa Senhora da Conceição e Santa Ifigênia, dispondo a antiga padroeira num altar lateral'' \cite{Santos2020Igrejas}.

    Neste sentido, ao retirar os negros destes espaços se retirava deles os locais de manutenção da cultura e da identidade. \textcite{Bertin2010Sociabilidade}, vai dizer que ``a ocupação de alguns espaços da cidade pelos negros, pode ser relacionada com a resistência, na medida em que tais lugares tinham especial significação para as suas práticas culturais'', é por isso, por exemplo, que os entornos das três igrejas de negros citadas neste ensaio eram conhecidas como lugar de ``danças dos pretos'' \cite[p.~128]{Bertin2010Sociabilidade}. Assinala \cite{Santos1998NemTudo}:

    \begin{quotation}
        Unicamente as congadas, batuques, sambas, os moçambiques, ainda se realizavam pelas ruas, de ordinário no largo de S. Bento ou junto às igrejas de S. Benedito (que os documentos atestam pertencer a S. Francisco), e do Rosário, após o recolhimento das procissões: reprimidas por anacrônicas, foram substituídas pela dança dos caiapós, arremedo dos costumes daqueles silvícolas, sem valor étnico, organizado artificiosamente que era, de pretos crioulos da capital. \cite[p.~124]{Santos1998NemTudo} 
    \end{quotation}

    A irmandade de São Benedito do Largo São Francisco, das três citadas até aqui a menos estudada na historiografia paulistana\footnote{Fazemos esta afirmação baseados na escassa documentação bibliográfica encontrada sobre a dita Irmandade. As poucas pesquisas sobre o assunto podem estar relacionadas a dificuldade de acesso às fontes primárias, a imprecisa localização da Irmandade no espaço urbano e até mesmo às informações desencontradas dos historiadores franciscanos do início do século XX que a este grupo social dedicaram alguma pesquisa.}, passou por processo parecido de perda de patrimônio e apagamento de memórias. Teve sua aprovação oficial no final do século XVIII, mas foi ao longo do XIX que adquiriu certa autonomia e administrou uma igreja particular: a de São Francisco do centro da capital. A influência deste grupo no templo foi tamanha, que no início do século XX a igreja era conhecida como Igreja de São Benedito. 
    
    A saída dos religiosos franciscanos da cidade de São Paulo em dezembro de 1828 foi um marco decisivo para a história daquele grupo de negros. No espaço do Convento passou a funcionar a faculdade de Direito e naquele da igreja continuaram a se reunir os irmãos de São Benedito. Por décadas, fizeram do lugar seu espaço de culto, sociabilidade e autonomia. A aurora da República abriu lugar novamente ao retorno de religiosos europeus ao Brasil e os frades aqui chegados passaram a reivindicar conventos e igrejas que a eles pertenciam desde o período colonial\footnote{Para saber mais sobre os franciscanos no Brasil e a retomada de conventos no sul e sudeste do país, ver: \fullcite{Rower1947Ordem}; \fullcite{Willeke1974Missoes}. O professor Maurício de Aquino tem uma tese interessante sobre este momento histórico: \fullcite{Aquino2012Modernidades}. pp.~153--154.}.
    
    O jornal \textit{O commercio de São Paulo} no ano de 1909 assim noticiava:

    \begin{quotation}
        Os frades, segundo deprehende, mas do que nos foi informado, pretendem mudar o nome daquella egreja de São Benedicto para o de São Francisco, isso contra uma disposição episcopal contida no compromisso da irmandade, que, reconhece a egreja como sendo de São Benedicto.

        Achando que a irmandade está absolutamente no seu direito, de não consentir no esbulho que lhes querem fazer os frades, os quaes, segundo estamos também informados, nem sequer tem existência legal neste Estado, visto que são allemães, com residência em Santa Catharina, appellamos para s. exc. o sr. bispo afim de interceder a favor daquella irmandade e não permittir que a mesma seja perturbada no seu socego de mais de 40 annos, tanto mais se tratar de uma corporação composta em sua maioria de pretos indefesos já por falta de conhecimentos, já pela falta de recursos pecuniários, o que não acontece a tantas outras congêneres.\footnote{Acervo Digital da Biblioteca Nacional [BNDigital] --- O commercio de São Paulo, ano 1909, ed. 1176, fl 1.}
    \end{quotation}


    Não pretendemos aqui nos aprofundar nas discussões sobre os embates jurídicos entre irmandade e franciscanos, uma pesquisa simples em periódicos paulistas como o \textit{Correio Paulistano}\footnote{``Jornal lançado no dia 26 de junho de 1854 em São Paulo, tendo por fundador o proprietário da Tipografia Imparcial, Joaquim Roberto de Azevedo Marques. Foi seu primeiro redator Pedro Taques de Almeida Alvim. Nascido liberal, o jornal, segundo José Freitas Nobre, em pouco tempo tornou-se conservador: premido `por uma série de circunstâncias, especialmente as de caráter financeiro\dots teve que ceder à pressão política do Partido Conservador, a ele aderindo de maneira pública, perdendo um pouco do prestígio que conquistara na sua orientação independente'. Em fins da década de 1860, entretanto, rompida a conciliação entre liberais e conservadores, a linha editorial do jornal optou por aqueles. Fundado o Partido Republicano Paulista (PRP), o periódico tornou-se seu órgão de divulgação e em 1874 foi comprado por Leôncio de Carvalho, adotando uma linha reformista. Em 1882 assumiu a direção editorial Antônio Prado, que imprimiu ao jornal a orientação de defesa do abolicionismo, e posteriormente de defesa da ordem republicana. Nascido portanto como um órgão de imprensa liberal e independente, logo a seguir conservador e dependente do poder político oficial da província de São Paulo, novamente adepto da trilha liberal, abolicionista e republicana, o \textit{Correio Paulistano} tornou-se mais uma vez oligárquico e conservador depois do advento da República, atingindo neste período sua maioridade e prestígio juntamente com o PRP, então dirigido pelos oligarcas paulistas Manuel Ferraz de Campos Sales, Prudente de Morais, Antônio Prado e Francisco de Paula Rodrigues Alves, entre outros''. Disponível em: \url{http://www.fgv.br/cpdoc/acervo/dicionarios/verbete-tematico/correio-paulistano}. Acesso em: 6 fev. 2021.} fundado em meados do século XIX, já é o bastante para constatar a existência deste grupo social e sua relevância na cidade ao longo daquele período. Nos limitamos a lembrar que o pequeno espaço reservado a eles na historiografia oficial do início do XX foi dado por aquele grupo de frades europeus envolvidos nas disputas empreendidas\footnote{Ao citar a presença da Irmandade de São Benedito na Igreja de São Francisco, Frei Basílio Röwer irá reservar-lhes uma história recontada no espaço de poucas linhas, carregada por sua vez de adjetivos que desqualificam o tempo em que ela administrou a igreja. Para saber mais, ver: \fullcite{Rower1947Ordem}.}.

    Ao analisar fontes primárias sobre os irmãos de São Benedito na Igreja de São Francisco, percebe-se, por exemplo, o esforço dos irmãos negros para administrar, manter e reformar ao longo do XIX as dependências da igreja. Após um grande incêndio na Faculdade de Direito em 1880 que afetou boa parte do templo, coube a eles logo após o sinistro, as obras de reforma, realizadas a duras custas e a pedidos de esmolas. Nas fontes que chegaram até nós constata-se frequentemente obras de melhoria no templo. Como esta publicada no jornal \textit{Correio Paulistano}:

    \begin{quotation}
        O esplendor excepcional com que tem sido celebradas as cerimonias religiosas sobredictas teve por principal motivo a comemmoração da restauração do altar e capella-mor da egreja do convento de S. Francisco, destruídos por um incêndio, sabe-se, o restaurados, como também em tempo noticiamos, á custo de grandes sacrifícios da irmandade de S. Benedicto.\footnote{BNDigital --- Correio Paulistano, S. Paulo, 6 mai. 1883, ano 29, n. 08000, fl. 1.}
    \end{quotation}


    O que se sabe é que a irmandade foi extinta por decreto que o arcebispo de São Paulo, Dom Duarte de Leopoldo e Silva, lançou sobre ela em 24 de fevereiro de 1910. No dia 28 do mesmo mês ``nomeou uma comissão para arrecadar e administrar todos os bens da irmandade'' e mais, ``aos 28 de abril ordenou à mesma comissão retirasse as caixas de esmolas''. Decretava-se com aval episcopal a dissolução daquele grupo de negros e a consequente perda de todas as suas posses. Usando as palavras do franciscano Röwer (1957): ``ela própria cavou a sua sepultura'' \cite[p.~120]{Rower1957Paginas}.

    \section{Considerações finais}

    Ao que tudo indica, as ponderações de frei \textcite{Rower1922Provincia} ao relatar a passagem dos pretos de São Benedito pela igreja de São Francisco, estavam impregnadas das ideias urbanistas, higienistas e sanitárias do começo do século XX. Aliadas a diversos outros fatores, como apoio do clero e da municipalidade, elas podem ter influenciado nas decisões que foram tomadas a partir de então.  

    Afinal de contas a igreja ``andava suja e malcuidada'', reinava ``a desordem e o desleixo'', até se tentou diminuir ``estes lamentáveis desconcertos'' através dos ``lentes e alumnos do Curso Jurídico'' mas ``os benedictos continuaram a infelicitar o bello templo franciscano'', foi preciso ``uma nova era'' com a chegada dos franciscanos europeus para que ``importantes melhoramentos'' fossem realizados a seu tempo.

    \nocite{AydosAndKoehler2014Beco}
    \nocite{Bresciani1985Metropoles}
    \nocite{Bresciani2002Cidade}
    \nocite{Bresciani2009Cidades}
    \mybibexclude{Viana2007Idioma}
    \mybibexclude{Bresciani2014Cidade}

    \printbibliography[heading=subbibliography,notcategory=fullcited]

    \hfill Recebido em 27 jan. 2021.

    \hfill Aprovado em 16 abr. 2021.

    \label{chap:melhoramentossaoend}

\end{refsection}
