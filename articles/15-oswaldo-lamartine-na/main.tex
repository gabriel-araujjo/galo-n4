\begin{refsection}
    \renewcommand{\thefigure}{\arabic{figure}}
    \renewcommand{\thetable}{\arabic{table}}
    
    \chapterOneLine
    {Oswaldo Lamartine, um narrador da história natural seridoense}
    \label{chap:oswaldo-lamartine-na}

    \articleAuthor
    {Natália Raiane de Paiva Araújo}
    {Mestranda em História dos Sertões (UFRN/CERES --- Campus de Caicó). ID Lattes: 9012.3580.1940.6831, ORCID: 0000-0003-2904-0244. E-mail: natalia\textunderscore{}raianejprn@hotmail.com.}
    
    \articleAuthor
    {Evandro dos Santos}
    {Doutor em história pela UFRGS. Atualmente é professor adjunto de Teoria e História da Historiografia e do PPG em História dos Sertões na UFRN --- Caicó. ID Lattes: 7531.7665.8244.3713. ORCID: 0000-0003-2844-4810. E-mail: evansantos.hist@gmail.com.  }
    
    \begin{galoResumo}
        \marginpar{
            \begin{flushleft}
            \tiny \sffamily
            Como referenciar?\\\fullcite{SelfAraújoAndSantosAndrodos2021Oswaldo}\mybibexclude{SelfAraújoAndSantosAndrodos2021Oswaldo}, p. \pageref{chap:oswaldo-lamartine-na}--\pageref{chap:oswaldo-lamartine-naend}, \journalPubDate{}
            \end{flushleft}
        }
        Analisamos no presente artigo as relações naturais, sociais e culturais que envolvem o homem e a natureza, a partir dos ensaios de Oswaldo Lamartine de Faria, escritos sobre os sertão do Seridó, localizado no interior do Rio Grande do Norte. O sertanista em questão resgata em seus textos a história sertaneja por meio da paisagem natural do sertão, elencando discussões sobre ambientalismo, bioma, modernidade, entre outros elementos caracterizadores da natureza e da história dos sertão do Seridó.
    \end{galoResumo}
    
    \galoPalavrasChave{Sertão. Natureza. Modernidade. Seridó.}

    \begin{otherlanguage}{english}

        \fakeChapterOneLine{Oswaldo Lamartine, a Narrator of Seridó Natural History }

        \begin{galoResumo}[Abstract]
            In this article, we analyze the natural relationships, social and cultural aspects involving man and nature, from the essays by Oswaldo Lamartine de Faria, written about the sertão of Seridó, located in the interior of Rio Grande do Norte. The sertanista in question rescues in his texts the country´s history through the natural landscape of the sertão, listing discussions on environmentalism, biome, modernity, among other elements that characterize the nature and history of the Seridó hinterlands. 
        \end{galoResumo}
        
        \galoPalavrasChave[Keywords]{Sertão. Nature. Modernity. Seridó.}
    \end{otherlanguage}


    \section{Introdução}

    O sertão do Seridó estão localizados no interior do Rio Grande do Norte, um dos estados que compõem a região Nordeste do Brasil. A tessitura narrativa sobre esse espaço é acompanhada de dimensões simbólicas que o caracterizam, as quais se associam àquelas que constituem as representações nordestinas. A natureza está presente em todos esses escritos, principalmente quando se fala de secas, de vazio, de lugar inóspito e de natureza hostil. Sendo assim, apresentaremos uma perspectiva diferente, colocada por Oswaldo Lamartine de Faria em seus escritos sobre os sertão do Seridó. 

    As narrativas que constroem os sertão a partir de sua natureza tornam muitas vezes indissociáveis a figura do homem e a do meio no qual ele vive. Um ser natural, um bicho homem, mostrando que os discursos são feitos provenientes de interferências políticas, sociais, culturais e econômicas e apresentando, assim, a natureza sertaneja a partir das relações de poder entre indivíduo e meio ambiente. 

    Enquadramos este artigo na recente experiência de construção de uma nova área de estudos, desde a aprovação do Mestrado em História dos Sertões no âmbito da Universidade Federal do Rio Grande do Norte (UFRN). Nessa proposta de um mestrado acadêmico cuja área de concentração intitula-se História dos Sertões, a primeira turma --- à qual fazemos parte --- se deu em 2019, trabalhando com pesquisas que retratam os múltiplos sertões e as infinitas possibilidades de leituras e releituras deste espaço. 

    Dessa forma, abordamos parte dos escritos de Oswaldo Lamartine de Faria, publicados na obra \citetitle{FARIA1980Sertões} na construção, imagética ou não, deste espaço. Trata-se de uma escrita que enaltece o bioma local, ao invés de deturpá-lo, promovendo uma conscientização sobre o meio ambiente como caracterizador do espaço e tornando a região visível e dizível. Uma escrita de fora para dentro, pois o autor não morou efetivamente no sertão seridoense, apenas teve contato com eles ao longo da vida, como expõe Evandro dos Santos em seu artigo \citetitle{SANTOSAndrodos2018Estilo}. 

    Relacionando discussões que embasem o artigo, como as reflexões acerca da modernidade, da natureza, da historiografia, do sertão, e, em especial, da especificidade do Seridó, busca-se examinar certo entrelaçamento de estudos para obtenção de dados que mostrem que natureza e homem se diversificam ao longo do tempo. Suas relações não são imutáveis, portanto, havendo apenas um diferencial entre elas: a vida humana é finita, enquanto a natureza é infinita e pode ser transformada. 

    Nesse sentido, o historiador alemão Reinhart Koselleck, na obra \citetitle{KOSELLECK2014Estratos}, resgata o exame do processo histórico da inserção dos estudos sobre a natureza na construção da história, em um estudo que já se tornou um clássico. Além disso, incluímos no debate a discussão mais recente sobre os estudos do conceito de antropoceno --- desenvolvidos pelo historiador Rodrigo Turin ---, que tratam das interferências humanas no mundo, via capitalismo, o que tem imposto um repensar da permanência da espécie sobre a Terra.  

    O artigo, então, está subdividido em quatro partes, que apresentam a construção ao longo do tempo das relações entre homem e natureza e a presença dessas discussões no ambiente acadêmico, com a introdução dos estudos ambientais na construção histórica da humanidade. Ao longo do trabalho, as discussões sobre sertão do Seridó, espécie humana e natureza são exploradas, conjuntamente com uma apresentação sobre Oswaldo Lamartine. Por fim, relaciona-se os escritos do autor com estudos atuais sobre meio ambiente e o antropoceno. 


    \section{Um preâmbulo: a natureza como elemento das filosofias modernas da História Ambiental}

    As filosofias do século XVIII, tratadas por Reinhart Koselleck na obra \citetitle{KOSELLECK2014Estratos}, mencionavam as relações entre história e natureza, quando se falava, ainda que de forma muito geral, em uma ideia de disciplina histórica e de ciência histórica. Com as posteriores divisões entre as áreas do conhecimento de estudo e saberes, a natureza foi escanteada dos objetos de pesquisas históricas. Somente com o discurso ecológico, surgido em meados do século XX, o tema voltou a ser estudado mais diretamente na área de humanidades, sendo reinserido como um objeto para a história. 

    Segundo Koselleck, a configuração do conceito moderno de história acontece devido às experiências próprias da modernidade. O historiador alemão nos apresenta o percurso atribuído à palavra “história” desde o mundo antigo até sua ressignificação moderna a partir do último terço do século XVIII. Esse novo modo de vida torna a História um conceito mais abrangente e abstrato. Analisada como uma série de acontecimentos em alguns casos, a moralidade coletiva e a abstrata moderna também são inseridas nesse novo conceito, histórias coletivas e individuais passam a fazer parte desse processo e as análises de fatos meramente cronológicos passam a ser questionadas. 

    A natureza, então, passa a ter seu conceito historicizado, pois deixa de ser vista como algo permanente. Ou seja, a mudança faz parte da natureza, assim como da humanidade.  Francis Bacon e Voltaire vão defender a separação entre História e Natureza, mas filósofos como Kant vão historicizar o seu conceito, pois o mundo e o que há nele estariam sempre em modificação. No caso da natureza, esta passaria por modificações feitas pelo homem e pelo próprio ciclo natural da terra. Desse modo, \textcite[p.~6]{KANT2004Ideia} nos diz: 

    \begin{quotation}
        Para isso um homem precisa ter uma vida desmesuradamente longa a fim de aprender a fazer uso pleno de todas as suas disposições naturais; ou, se a natureza concedeu-lhe somente um curto tempo de vida (como efetivamente aconteceu), ela necessita de uma serie talvez indefinida de gerações que transmitam umas às outras as suas luzes para finalmente conduzir, em nossa espécie, o germe da natureza àquele grau de desenvolvimento que é completamente adequado ao seu propósito. E este momento precisa ser, ao menos na ideia dos homens, o objetivo de seus esforços, pois senão as disposições naturais em grande parte teria de ser vistas como inúteis e sem finalidade --- o que aboliria todos os princípios práticos, e com isso a natureza, cuja sabedoria no julgar precisa antes servir como princípio para todas as suas outras formações, tornar-se-ia suspeita, apenas nos homens, de ser um jogo infantil. 
    \end{quotation}

    A partir das colocações apresentadas por Kant, podemos perceber que o filósofo alemão do século XVIII coloca o homem como indivíduo manipulado pela natureza, ou seja, nesse cenário, o homem não ocuparia um \textit{status} superior, mas sim a natureza. Esta em suas formas sobreviveria sem o homem, entretanto este não vive sem aquela, o que nos leva a pensar que o homem, como agente biológico, tenta destruir a natureza ao longo do tempo, enquanto a natureza se reestabelece em alguns momentos. Já o homem, uma vez morto como espécie, não mais volta para habitar seu local de nascimento. 

    A História Natural faz uma análise historiográfica a partir dos marcos históricos, levando em conta as construções dos fenômenos naturais ao longo do tempo e as relações entre natureza e homem. Cada uma, história e natureza, desenvolve seus aspectos, e juntas formam novos moldes para historiografia. Uma escrita marcada por mudanças e que não sabemos como se findará se reportamos para o tempo presente. 

    Os fenômenos climáticos estão vinculados à História Natural e ao próprio processo de desenvolvimento das regiões no planeta, pois cada continente/país possui sua vegetação e clima diferenciados, mas o que ocorre é que, mesmo que cada lugar tenha suas particularidades geográficas e naturais, todo o planeta está em risco, devido à grande extração de recursos naturais e à poluição. Dessa forma, mesmo que a História Natural estude sobre a natureza, acaba por se ligar também aos estudos historiográficos da História humana, pois a partir do surgimento do homem na Terra e de suas adaptações ao clima e à vegetação das regiões, ele passou a criar sua cultura de desenvolvimento no planeta.  

    A História Ambiental também passou a fazer parte dos estudos historiográficos recentes, principalmente devido à interdisciplinaridade que ganhou espaço nas ciências nas últimas décadas. Os estudos das diferentes áreas podem ser equilibrados dentro da perspectiva de cada objeto de estudo. Desse modo, o ambiente natural é modificado pelo homem, e a forma natural de cada espaço está ligada ao ser humano em sua vida coletiva e individual, pois a sua intervenção neste lugar mostra as relações desenvolvidas entre o natural e o humano em determinada espacialidade. Como destacamos a seguir: 

    \begin{quotation}
        A história ambiental é, em resumo, parte de um esforço revisionista para tornar a disciplina da história muito mais inclusiva nas suas narrativas do que ela tem tradicionalmente sido. Acima de tudo, a história ambiental rejeita a premissa convencional de que a experiência humana se desenvolveu sem restrições naturais, de que os humanos são uma espécie distinta e “super-natural”, de que as consequências ecológicas dos seus feitos passados podem ser ignorados. A velha história não poderia negar que vivemos neste planeta há muito tempo, irias, pôr desconsiderar quase sempre esse fato, portou-se como se não tivéssemos sido e não fossemos realmente parte do planeta. Os historiadores ambientais, por outro lado, perceberam que não podemos mais nos dar ao luxo de sermos tão inocentes \cite[p.~2]{WORSTER1991Para}. 
    \end{quotation}

    Nós, humanos, não exercemos todos os domínios sobre a natureza, nem detemos todo o conhecimento a seu respeito, por isso que intervimos, ao passo que ela continua a nos surpreender. A partir do momento em que a usamos para nosso benefício, como através do desmatamento, para abrir estradas ou para criar cidades, é possível que estejamos afetando uma fauna e flora existentes naquele espaço, e não podemos saber com exatidão o que pode ocorrer de adverso com essa medida.  

    História e Natureza hoje se reúnem em estudos que partem de vieses diferentes, mas que se mesclam dependendo de sua intenção. Os estudos sobre o natural têm as suas disciplinas específicas, mas não deixam de contar sobre a humanidade e as relações de dominação do espaço, entre outras características. É nesse compilado que encontramos os estudos de Oswaldo Lamartine de Faria sobre o sertão do Seridó, local cuja identidade é marcada por sua natureza, como veremos a seguir. 


    \section{(Re)conhecer um apaixonado pelo sertão}

    Nesse processo de ascensão sobre os estudos ecológicos e históricos do planeta e suas subdivisões, nasce em Oswaldo Lamartine de Faria o desejo de escrever sobre a natureza da região do sertão do Seridó. Com o seu conhecimento acadêmico e cultural da região, passou a registrar e cartografar o espaço através de sua natureza, descrevendo os elementos consagrados dessa localidade. O autor narra através de suas obras os elementos que compõem a paisagem sertaneja, enaltecendo a natureza e o homem sertanejo, contando a história local e jogando luz sobre a região.  

    Oswaldo Lamartine nasceu na cidade do Natal, capital do estado do Rio Grande do Norte, no dia 15 de novembro de 1919, e faleceu em 2007 ao cometer suicídio. Lamartine foi filho caçula do casal Juvenal Lamartine de Faria e Silvina Bezerra de Araújo Galvão, ambos de linhagens renomadas no sertão do Seridó, com inclinações políticas --- seu pai, inclusive, foi governador do estado. Em Natal, onde passou grande parte da infância, já relatava seu contato com a natureza, nos quintais e mangueirais repletos de árvores frutíferas na companhia de amigos e colegas da rua Trairi.  

    Juvenal Lamartine de Faria (1874--1956), pai de Oswaldo Lamartine, tem linhagem na cidade de Serra Negra do Norte-RN, região do Seridó potiguar. O seu avô, Clementino Monteiro de Faria, foi chefe político da cidade, já desempenhando seu papel antecessor de coronel naquelas regiões, pois era dono de muitas terras, e o filho seguiu esse legado político e desempenhou as funções de deputado, senador e governador no estado do Rio Grande do Norte. Sua mãe, Silvina Bezerra de Araújo Galvão (1880--1961), também filha do coronel e chefe político na cidade de Acari-RN Silvino Bezerra de Araújo Galvão (1836--1922), não seguiu os passos familiares, mas esteve sempre ao lado do seu esposo. Nesse ínterim, podemos supor que o casamento também se deu por alianças políticas e econômicas, devido aos costumes da época. Desse modo, percebemos que Oswaldo Lamartine tinha uma família influente no estado do RN, sempre ao lado de pessoas ilustres e com boa educação.  

    Na infância foi alfabetizado, como muitos em sua época que pertenciam ao seu grupo social abastado, por professores particulares. Estudou no Colégio Dom Pedro II, fundado pelo professor Severino Bezerra (1888--1971) em Natal. Também estudou em internatos, no Ginásio de Recife do Padre Felix Barreto, e, posteriormente, no Instituto Lafayette, no Rio de Janeiro, entre os anos de 1929 e 1937. Por fim, conquistou formação superior na Escola Superior de Agronomia, em Lavras, Minas Gerais. A sua educação foi aferida aos filhos de pessoas influentes da época, visto que não se tinha educação para todos, só para aqueles com algum pecúlio. Já o gosto pelas coisas do sertão se relaciona muito com seu pai, que escreveu algumas obras também sobre este espaço: \citetitle{FARIA1965Patriarcas} e \citetitle{FARIA1965Velhos}. 

    A família Faria possuía várias fazendas no estado do Rio Grande do Norte, dentre as quais citamos duas localizadas na região Seridó: Fazenda Cacimbas (Serra Negra do Norte), onde Oswaldo Lamartine passou sua infância; e Fazenda Ingá (Acari). A Fazenda Lagoa Nova, em São Paulo do Potengi, na região agreste do estado, foi palco de experiências com alguns dos mestres de ofício que inspiraram seus escritos. Apesar de não ser sertão, foi nessa localidade que Oswaldo Lamartine viveu as experiências culturais rurais. Morou na Fazenda Acauã, município de Riachuelo, localizado na região oeste do estado, mas terminou seus dias em um apartamento na capital do RN, onde se suicidou. 

    Serviu ao exército brasileiro, foi pracinha de nº 1918, serviu a III Companhia de Metralhadoras do 16º Regimento Infantaria e depois veio a ser cabo e sargento. “Foram os mais saudáveis dias de farda com muita manobra, combate simulado, exercício de tiro e nenhum ferimento à bala. As baixas ao hospital corriam por conta da temida ração-de-guerra e das líricas venéreas, comuns aos exercícios de todas as terras” \cite[p.~46]{CAMPOS2001alpendres}. Tomou posse na Academia Norte-Rio-Grandense de Letras (ANRL) em 2001, na cadeira de número 12, a qual também foi ocupada por seu pai Juvenal Lamartine e tem como patrono Amaro Cavalcanti. Ademais, convém ressaltar que recebeu também o título de Doutor Honoris Causa da UFRN. Sobre o autor, Sanderson \textcite[p.~237]{NEGREIROSAnderson2001direção} nos diz: 

    \begin{quotation}
        Quem fala assim é Oswaldo Lamartine, técnico orientador de problemas de agricultura no Banco do Nordeste, conhecedor da “alma íntima” do grande sertão, pesquisador sério: da caça, da pesca, das abelhas, dos arreios, da conservação dos alimentos, do Sertão do Seridó. Do Seridó que começa com a subida da Serra do Doutor, logo depois de Santa Cruz de Inharé, de quem Oswaldo é o agrimensor do sonho e o pesquisador de tudo que forma corpo, vida e substancia das veredas do grande sertão. E os livros de Oswaldo e sua conversa e seu bate-papo são perorações antiverbalistas de sua lírica disponibilidade de viver --- Oswaldo tem jeito de um frade que procura um convento para morar e esse convento não existe --- é no melhor sentido uma busca do tempo perdido --- um Proust que escreve sobre sertão como poeta; e fala à maneira de um vaqueiro bem instruído. 
    \end{quotation}

    Oswaldo Lamartine de Faria foi, portanto, um técnico agrônomo e escritor norte-riograndense com atuação entre as décadas de 1940 e 2000. Apesar de dedicar grande afeto pelo sertão do Seridó e escrever sobre este, não teve fixação permanente neste espaço. Teve muitas moradas ao longo da vida, e seus escritos estão vinculados a esses percursos --- quiçá, não teria escrito tanto sobre este local se fosse um residente. Nas suas obras, há uma perspectiva de fora para dentro, um olhar de estradas percorridas e saudades do tempo vivenciado em sua infância, pois o fim de sua infância, a adolescência e a vida adulta foram vividos entre espaços distintos, sendo as visitas ao sertão reservadas a férias escolares e, posteriormente, de trabalho.  

    Nos 40 anos que morou no RJ, Oswaldo Lamartine conheceu os melhores sebos e livrarias, como também usou deste conhecimento para ligar-se ao sertão através de leituras sobre o tema e do desenvolvimento de suas pesquisas. Podemos citar Câmara Cascudo (1898--1986) e Rachel de Queiroz (1910--2003) como alguns dos autores com quem teve contato para empreender seus estudos. Citada por \textcite[p.~14]{CASTRO2015Areia}, Margarete Cardoso, dona de uma livraria frequentada por Oswaldo Lamartine, nos diz: 

    \begin{quotation}
        Ele foi cliente nosso durante muitos anos, de modo que nos tornamos amigos. Sempre deixava conosco, para venda, os livros que publicava. Era muito querido por todo mundo, justamente em função daquele jeito tranquilo, sertanejo, acima de tudo, sertanejo. Ele nos contava suas lembranças do sertão de antigamente e que hoje não era mais nada daquilo. Lamartine era um ótimo cliente, não digo isso do ponto de comprar coisas caríssimas, mas de sempre nos acompanhar e estar sempre presente. A preferência dele era, obviamente, o folclore nordestino, e tudo que se relacionasse com o Nordeste.  
    \end{quotation}

    Percebemos um amante do sertão em suas mais diversas metamorfoses, em especial emocionalmente, sempre estudando e produzindo intelectualmente para exercer com propriedade seu ofício, apesar de que a leitura e a escrita tenham sido desenvolvidas fora do seu ambiente específico de trabalho. Como profissional, trabalhou como administrador no Serviço de Colonização do Ministério da Agricultura, em Barra da Corda, no Maranhão. Durante muitos anos foi funcionário do Banco do Nordeste sediado no Rio de Janeiro. Dentro do processo de construção de sua vida profissional no RN, atuou na administração da Fazenda Lagoa Nova, em São Paulo do Potengi, e no Núcleo Colonial do Pium, como também foi professor da Escola Doméstica de Natal e da Escola Agrícola de Jundiaí.  

    Nesse sentido, o autor não deixou de ter contato com a natureza e com o homem em alguns de seus ofícios, e foi com essa bagagem que escreveu sobre o sertão do Seridó, em parte para manter sua memória viva, em outra para enfatizar as alterações que a modernidade e os meios econômicos e sociais causaram no ambiente sertanejo, sendo justamente essa a questão para o desenvolvimento desta pesquisa, apresentando o autor e as prerrogativas de seus textos. 


    \section{Entre o humano e o natural}

    Para se pensar sobre a relação entre natureza e a história humana, é preciso associar sociologia e epistemologia, como afirma \textcite{PÁDUA2010bases}, pois falar sobre natureza muitas vezes deixa os historiadores um pouco desconfortáveis para desenvolver suas pesquisas, principalmente quando se coloca a teoria ambiental e suas particularidades, como a separação entre ciência e política, uma relação que hoje é bastante evidente, já que o Estado financia as pesquisas científicas e promove ações ambientais.  

    A ecologia é uma das particularidades que encontramos quando estudamos sobre ambientalismo. Percebemos que a natureza não é só formada pelo que vemos, mas sim pelas partículas minúsculas imperceptíveis a olho nu, mostrando sua contribuição no ecossistema planetário e fazendo com que o mundo despertasse, a partir da difusão do assunto na mídia, para as ações que estão acontecendo em todo o planeta devido às intervenções humanas na natureza. Desse modo, 

    \begin{quotation}
        A ideia de “ecologia” rompeu os muros da academia para inspirar o estabelecimento de comportamentos sociais, ações coletivas e políticas públicas em diferentes níveis de articulação, do local ao global. Mais ainda, ela penetrou significativamente nas estruturas educacionais, nos meios de comunicação de massa, no imaginário coletivo e nos diversos aspectos da arte e da cultura. O avanço da chamada globalização, com o crescimento qualitativo e quantitativo da produção científico-tecnológico e da velocidade dos meios de comunicação, catalisou uma explosão de temas da vida e do ambiente na agenda política. A discussão ambiental se tornou ao mesmo tempo criadora e criatura do processo de globalização. A própria imagem da globalidade planetária, em grande parte, é uma construção simbólica desse campo cultural complexo \cite[p.~82]{PÁDUA2010bases}. 
    \end{quotation}

    Em termos globais, a partir dos estudos sobre ecologia, o mundo passa por um novo processo de reorganização, com os estudos agora divulgados na mídia e nas escolas e com os agentes políticos mundiais em ação sobre a retenção dos efeitos climáticos no planeta e a prevenção de um futuro desastre ambiental. Nesse sentido, a humanidade pode ser extinta, restando somente a própria natureza, com seus microrganismos que seriam capazes de se adaptarem aos novos moldes.  

    \textcite[p.~83]{PÁDUA2010bases}, destaca três pontos que devem ser analisados: a ação do homem na natureza; os marcos cronológicos do tempo; e a colocação da natureza como uma construção histórica, o que apresentamos e apresentaremos nesta pesquisa vinculado aos estudos de Oswaldo Lamartine. A habitação humana nos espaços construiu sua relação com o ambiente, sabendo que existe uma grande diversidade de ecossistemas e biodiversidades no planeta, então, nesse processo o homem passa a implantar suas práticas culturais e materiais e seu entendimento sobre esse processo resulta no desenvolvimento de sua própria existência, isto é, é a natureza que influencia na história humana.  

    Oswaldo Lamartine nos apresenta o sertão a partir das relações entre homem e natureza sob uma perspectiva diferenciada, expondo a caatinga como bioma nativo da região e mostrando que, apesar das prerrogativas, a vida no espaço é possível e muito aproveitada pelos sertanejos a partir dos recursos encontrados. A natureza é um elemento de apreciação e as interações entre homem e meio são satisfatórias, apresentando-se mais emblemáticas quando são inseridos os novos moldes de capitalização no sertão, que destroem os hábitos e costumes, juntamente com sua fauna e flora. 

    A descrição da paisagem do sertão é colocada pelo escritor em suas obras, evocando o sensível e construindo sua visualização. É a inserção do seu eu nos seus escritos que dá originalidade à referida obra, visto que o autor não escreve somente sobre o espaço sertanejo, mas se insere na escrita, sendo a sua obra marcada pelo saudosismo, pela paixão, pela subjetividade e por um narrador da história natural do sertão. O marco da sua escrita é a descrição etnográfica do espaço. A esse respeito, \textcite[p.~32]{MEDEIROS2019Estilo} afirma: 

    \begin{quotation}
        A metodologia utilizada por Oswaldo Lamartine em seus ensaios é o registro etnográfico, ou seja, a descrição, em campo, de artes de oficio, dos modos de fazer, ou da cultura material sertaneja que escolhe como objeto de pesquisa. Também realiza a pesquisa histórica, entenda-se, a consulta à documentação e bibliografia que lhe permitam obter informações sobre o objeto e/ou recorte espaço-temporal a que se dedica. Sua \textit{práxis} historiográfica é a identificação de uma origem universal para seu objeto de pesquisa, e uma origem ou primeira aparição e/ou uso deste objeto de que se tem indício no sertão seridoense. Assim, realiza uma descrição de “como é na contemporaneidade” a partir do registro etnográfico, criando assim a sensação de uma longa permanência do objeto no tempo, mas alertando sempre para sua iminente extinção --- recorrente justificativa dada como motivação de seu registro etnográfico. 
    \end{quotation}

    Eduardo \textcite{MEDEIROS2019Estilo}, descreve bem a forma como Oswaldo Lamartine elabora sua obra, tida como do gênero ensaístico devido ao seu formato, com trabalhos curtos e de fácil entendimento. A etnografia de seus estudos é facilmente percebida, pois toda a obra é bem detalhada. Podemos, também, enfatizar o fato de que sua produção exerce a função de denúncia, pois fala de um tempo que está se perdendo e de animais e plantas que estão ou estarão em extinção. Lamartine se caracteriza como autor sertanejo conceituado, um escritor do seu sertão que se torna nosso quando temos contato com sua obra, com uma escrita íntima e marcada pelo amor aos chãos sertanejos. Trata-se de um autor que vai além, quando escreve sobre o sertão do Seridó, um espaço longínquo, mas de grande estima para ele – é um ser no sertão. 

    Na apresentação de seu livro \citetitle{FARIA1980Sertões}, Francisco das Chagas Pereira vai definir que

    \begin{quotation}
        O sertão de Lamartine não existe como objeto exterior de pesquisa, distanciado do impessoal investigador. É espaço interior, vivenciado, incorporado ao mundo de valores, crenças e cuidados do escritor: até parece ter-se cristalizado no seu perfil aristocraticamente seco, tímido, quase ascético \cite[p.~14]{FARIA1980Sertões}.
    \end{quotation}

    Pereira ainda destaca que no sertão se vive de forma desigual, pois a natureza faz com que o homem viva de luta constante contra as secas que assolam o espaço e sua própria vegetação de pouca assistência ao homem. É importante ressaltar que Pereira usa de três elementos caracterizadores do espaço para descrever Oswaldo Lamartine, quando ele o define como seco, tímido e ascético, ou seja, Lamartine está ligado à natureza sobre a qual escreve. A imagem do sertão seridoense seria seca, pois está ligada às intempéries que acometem o lugar; tímida, porque nela não existe beleza exuberante, e ela não se modifica, continuando sempre no mesmo local, sem maiores perspectivas; e ascética, no sentido de que o desenvolvimento só se processa com uma ajuda religiosa, o que também constitui uma representação da região: a sua forte religiosidade para enfrentar o ambiente hostil. O estudioso também vai enfatizar que, devido às secas, a sobrevivência na região se torna difícil, sendo um dilema para aqueles que habitam e desejam sobreviver neste espaço.  

    Desse modo, isso nos leva a discussões mais ousadas em relação ao rompimento da tradição com a modernidade e o desenvolvimento das relações no sertão, sejam elas sociais, econômicas, culturais, entre outras, principalmente cientificamente, com aprimoramento de estudos sobre a natureza e a visão de uma perspectiva de proteção e preservação da natureza seridoense, o que abordaremos nas discussões de Oswaldo Lamartine e suas colocações sobre esta localidade. 


    \section{Uma historiografia do sertão}

    Oswaldo Lamartine, então, teve a visão do sertão do Seridó de dentro para fora e de fora para dentro, pois mesmo longe geograficamente, sempre esteve perto através das leituras, dos estudos e das ligações familiares e entre amigos. Um intelectual do sertão, deixando registrado este espaço de terras através dos bichos “animais” ou pé de bichos “plantas” e o bicho “homem”. Dentro destas relações, o autor se inclui como bicho-homem, mostrando que as ligações do homem com a terra são maiores do que podemos ver a olho nu. Assim Lamartine descreve sua jornada pelo sertão: 

    \begin{quotation}
        Vivi um bom pedaço no de vida no asfalto mas sempre me escapulindo para o sertão. Por mais impermeável que a gente seja sempre se lambuza. Mesmo assim sou, pra que negar, um bicho-do-mato. Daí ter ficado assim marginal que nem prostituta que deixou a zona --- nem a sociedade a recebeu e nem a zona a quis de volta \cite[LAMARTINE apud][p.~39]{CAMPOS2001alpendres}. 
    \end{quotation}

    Toda sua escrita está atrelada à sua experiência e ao seu conhecimento técnico acerca da natureza no sertão. Seu conhecimento está baseado em dados colhidos, leituras e prática, por isso, em suas obras podemos perceber um estudo antropológico, etnográfico, histórico e memorialístico, ou seja, uma interdisciplinaridade em seus escritos. \textcite{MEDEIROSNETA2007Sertão} pontua os escritos de Oswaldo Lamartine como formadores da paisagem espacial da região do Seridó, um discurso formador e difusor deste espaço para o mundo através das práticas culturais, usando da tradição regional para fortalecer a identidade cultural da região. Assim, 

    \begin{quotation}
        Oswaldo Lamartine coloca-se como o locutor do sertão de nunca mais, de práticas como a caça, a pesca e a conservação de alimentos. O homem é sempre um interventor junto à natureza e suas possibilidades. A natureza dos sertões do Seridó é a da paisagem da caatinga. Nas narrativas de seu sertão de nunca mais a caatinga é a paisagem composta como cenário. Nela as práticas e costumes como a caça, a pescaria e a criação de abelhas tomam corpo e são envolvidas pela tradição oral. A natureza, na obra de Oswaldo Lamartine apresenta-se como um cenário (d)escrito e cartografado em páginas sobre a fauna, a geografia e a topografia.
    \end{quotation}

    Desse modo, Oswaldo Lamartine levanta um estudo ambiental de viés historiográfico sobre a região do Seridó, em virtude de sua formação e de suas experiências culturais. Sua produção possui, então, diversos temas, estilos e tipos de narrativas, versões que nos levam a caminhos abertos para produzir releituras de suas obras e vieses diversos de conhecimentos. Como historiador, percebemos a busca pela origem do sertão seridoense e da sua cultura material. 

    Temos que levar em consideração que somos seres biológicos e geológicos atuando efetivamente sobre a natureza. Nesse sentido, a humanidade deixa de ser passiva para atuar como agente geológico no próprio ecossistema, pois devido à sua grandiosa intervenção na natureza, o ser humano passa a ser um agente dela, que por sinal é determinante para a existência de sua espécie no mundo, já que devido à sua intervenção o planeta vem caminhando para o fim, eliminando uma parcela da biodiversidade existente.  

    As transformações genéticas e culturais também fazem parte do processo de formação da humanidade e do planeta, e são agentes que devem ser levados em consideração, pois o inorgânico vai se adaptar às novas mudanças climáticas que venham a ocorrer, mas o ser humano pode ser extinto. Nesse sentido, a mudança sempre esteve e está presente nos processos biológicos e sociais, de modo que o advento da modernidade somente vai trazer mais elementos que contribuem para a exploração da natureza e dos modos de vida do homem no ambiente.  

    É nesse sistema que entra o poder que têm a política e o capitalismo industrial, que foram os meios que mais modificaram os espaços em busca de mercados e produção, sem levar em conta as consequências produzidas pelas suas exacerbações. São elementos que deterioraram o meio ambiente como um todo, trazendo drásticas consequências para os ecossistemas, e algo que vai além da poluição que enxergamos. Pensar nessas consequências no sertão seridoense seria um pouco difícil, visto que conscientização e caracterizações são elementos que muitas vezem eram apregoados por aqueles que tinham poder, como Estado e chefes de municípios.  

    Com seu conhecimento e uma visão de fora, Oswaldo Lamartine caracterizou os elementos formadores da paisagem sertaneja e pontuou aqueles que ao seu ver descaracterizariam o sertão. Sobre o modo como esse autor enxerga a modernidade presente em seu meio e as transformações que ela causa, Samara \textcite{SILVA2019Lugares} traz uma análise em sua dissertação sobre este contexto, pois elenca os espaços presentes na obra e vida de Oswaldo Lamartine. Dentre eles, encontramos a modernidade também como caminho de escrita para o autor, devido a questão sentimentais e ao mesmo tempo ecológicas, quando envolvem a destruição da natureza. Dessa forma, 

    \begin{quotation}
        O contexto de crise de 1930 foi um elemento catalizador para essa elite seridoense no momento em que parte desse grupo “exila-se” no Rio de Janeiro e passa a construir um discurso profundamente marcado pelo regionalismo e pelo ressentimento. Isso foi feito visando construir também uma diferença entre o tempo da tradição e o da modernidade, perversa e diluidora dos bons e sadios costumes. Essa “modernidade” pode ser intendida com uma analogia para todo o período posterior a “Revolução de 1930” com seus interventores “estrangeiros”, ou seja, temos também uma alteridade sendo criada entre os de nativos e os de fora. 
    
        Entretanto, a ideia do moderno como algo ruim, ou destruidor, se contrapõe ao discurso que essas mesmas elites exercitaram nas décadas de 10 e 20 que compreendia a modernidade como algo bom para o povo sertanejo, isso pode ser muito observado nos textos de Juvenal Lamartine (1916) e Eloy de Souza (1909) sobre o problema das secas do Nordeste. Nestes textos, o moderno é tomado do ponto de vista utilitário para resolver um problema de longa data \cite[p.~46--47]{SILVA2019Lugares}. 
    \end{quotation}

    Percebemos, nas análises de Silva, que a divisão entre tradição e modernidade tem seu início por volta da década de 30, período em que o país passa por diversas transformações, principalmente no que diz respeito às transições entre rural e urbano. A “elite” por ela colocada diz respeito aos grandes proprietários de terra do sertão, como também aos políticos e estudiosos que vão dar voz ao local, pois, dentro de um cenário de secas e fome, esses sujeitos apostaram na modernidade como uma locomotora de mudanças para o sertanejo. Mesmo que Oswaldo Lamartine venha desta folhagem e leitura, este prefere falar sobre o sertão dentro de outro viés, e tem na modernidade uma premissa de destruição deste espaço. Dessa forma, 

    \begin{quotation}
        Oswaldo Lamartine se coloca a favor dessa forma de pensar, mesmo que, conscientemente confessasse horror pela política. Ele toma 1930 como um marco da modernidade destruidora dos modos de vida do sertanejo, através da chegada dos automóveis, da eletricidade e do rádio. Nesse sentido, ele esquece que foi no governo de seu pai que esses elementos começaram a chegar nas zonas interioranas do Rio Grande do Norte, ou seja, ele novamente seleciona e oculta elementos para construir sua narrativa. O apego ao passado e as coisas do Sertão, é um posicionamento com o qual ele conviveu, ou seja, era um discurso disponível no seu contexto social e que devido aos fatos que se processaram na sua trajetória de vida, como as mudanças constantes, fez com que ele sentisse a necessidade de escrever e pesquisar sobre o assunto \cite[p.~47--48]{SILVA2019Lugares}. 
    \end{quotation}

    Como exposto, Oswaldo Lamartine seleciona fatos e oculta outros para compor suas narrativas. O amor ao sertão não era algo só seu, mas compartilhado por parentes, amigos e pelo contexto social da época. \textcite{SILVA2019Lugares} mostra que fatores familiares também o impulsionaram, seja para ocultar informações ou para expô-las. A modernidade interferiu diretamente nas suas lembranças e no modo como as práticas aconteciam, o que ocasionou a busca por uma forma de preservar essa memória sobre o sertão da sua infância, e foi na escrita que Oswaldo Lamartine assim o fez, eternizando o sertão natural e humano de seus dias de antanho.  

    Sendo assim, Lamartine teve todo um aparato e uma vivência sobre a chegada da modernidade e suas consequências, tanto para o sertanejo como para sua família. Não nos adentraremos nos anseios e faltas familiares em si, mas a tese de \textcite{CASTRO2015Areia} é um caminho para se entender melhor os eventos acontecidos em sua trajetória. Nas transições de sua vida, a década de trinta foi um divisor de águas, pois seu pai, então chefe do Estado do Rio Grande do Norte e contra o governo de Getúlio Vargas, foi perseguido, e é justamente nesse momento que o autor dá início a seu itinerário nos colégios internos, inicia sua trilha longe do sertão, e posteriormente começa a rabiscar seus escritos de saudades e curiosidades da região.  

    A modernidade na natureza vai representar a degradação que o homem ocasiona no ambiente, principalmente a partir das construções das cidades e da industrialização, mas levando em consideração que existem fatores macrohistóricos que vêm ocorrendo ao longo do tempo e sua grande questão seria a continuidade da vida humana no planeta. A representação do conceito de natureza para o homem ao longo dos anos afirma-se a partir da diferença entre sua conceituação e os outros elementos sociais, como a técnica e as relações sociais humanas. Nada seria possível sem os estudos científicos sobre o natural e suas particularidades, e hoje sabemos que a grande modificação que acontece se deu a partir das relações humanas com a natureza.  

    Podemos relacionar os estudos desenvolvidos por Oswaldo Lamartine às pesquisas sobre o Antropoceno, ou seja, o período do tempo que estamos vivenciando desde a Primeira Guerra Mundial, uma destruição do mundo por diversos vieses, entre eles o ambiental, seria o fim da humanidade e a reorganização do espaço natural sem humanos. Uma visão de perspectiva futura do planeta, no caso, do sertão do Seridó, que já vinha sofrendo com a degradação ambiental dentro de um pequeno espaço de tempo de transformações sociais e econômicas. Dessa forma,  

    \begin{quotation}
        Acreditamos não estar exagerando ao dizer que o Antropoceno, ao nos apresentar a perspectiva de um “fim de mundo” no sentido o mais empírico possível, o de uma mudança radical das condições materiais de existência da espécie, vem suscitando uma autentica angústia metafisica. Essa angústia, muitas vezes beirando o pânico, tem se exprimido em uma desconfiança perante todas as figuras do antropocentrismo, seja como ideologia prometeica do progresso da humanidade em direção a um Milênio sociotécnico, seja como pessimismo pós-modernista que celebra ironicamente o poder consistente do Sujeito ao denunciá-lo como inesgotável matriz de ilusões \cite[p.~45]{DANOWSKIAndVIVEIROSDECASTRO2014mundo}. 
    \end{quotation}

    A relação entre humanos e natureza nem sempre é harmoniosa, causando diversos danos ambientais. Por isso, os estudos do chamado antropoceno revelam indícios para o fim da humanidade, que pode ser causado por diversos fatores ambientais, inclusive por vírus dos quais não sabemos a existência e o efeito. A natureza é mais complexa do que podemos enxergar, e esses estudos estão muito à frente daquilo que não podemos conter/saber. 

    Rodrigo Turin estuda as relações entre etnografia e história entre os anos 1870 e 1900. Em relação ao uso do saber etnográfico para a reelaboração da história nacional brasileira, autores como José Veríssimo e Silvio Romero fizeram parte deste processo analisado por Turin, ou seja, autores que usaram suas obras com a conjunção dos saberes históricos e etnográficos. Os documentos etnográficos abordam uma ordem temporal mais ampla, abarcando processos migratórios e a caracterização os indivíduos física, linguística e culturalmente, ressaltando que a utilização desses tipos de documentos fazia sentido para determinadas pesquisas específicas, incluindo as mudanças sociais que fazem parte desses processos de estudos. Essa construção se dá por meio dos: 

    \begin{quotation}
        pares de oposições que marcaram a formação desses saberes, “oralidade” e “escrita”, “identidade” e “alteridade”, “espaço” e “tempo”, “consciência” e “inconsciência”, marcando a definição de seus objetos e de suas práticas, tornam-se, a partir de então, fluídos. A etnografia, ao tomar para si o objeto cujo domínio identificava o labor historiográfico (a formação nacional), mas aplicando os seus métodos, vai promover a construção de uma temporalidade própria, distinta daquele que vinha sendo trabalhada pela historiografia imperial. E é este deslocamento que irá produzir novos efeitos de conhecimento, enraizados, por sua vez, em novas representações políticas e em novas expectativas intelectuais \cite[p.~4]{TURIN2009história}. 
    \end{quotation}

    Essa interação entre o etnográfico e o histórico proporciona uma nova forma à organização política e à concepção de sociedade. Esses novos conceitos trouxeram uma nova escrita da história, e o tempo passa a ser ordenado modernamente, pois se analisa etnicamente as ligações entre os indivíduos, trazendo uma nova reestruturação nas concepções que antes foram escritas sobre “raça” ou “cor”, relacionando o físico, o intelectual, entre outros elementos que formam o ser. Tais abordagens relativas à histórias dos diferentes saberes e às disputas que se ocuparam de relacionar natureza e cultura são relevantes ao adequado exame dos escritos de Oswaldo Lamartine de Faria. 

    O Seridó, recorte privilegiado por Lamartine, é construído e evidenciado através da escrita, formando sua identidade local. Ele escreve sobre esse espaço enaltecendo o passado por ele vivenciado no sertão; aborda os âmbitos cultural e natural; redige com afetividade e por vezes utilizando de descrições imagéticas, carregadas do autor demarcando o espaço, criando um elo entre escritor e leitor a partir das memórias das paisagens por ele descritas em suas obras. 

    Lamartine atua como testemunho vivo das transições que aconteceram no sertão, e, por se colocar dentro de sua obra, obtém-se, então, a afirmação de que o autor faz uma autobiografia em suas narrativas. Acrescenta também em seus textos registros técnicos, estatísticos e literários, afirmando uma identidade a partir da geografia e da cultura, seu estilo de escrita. As obras desse autor são caracterizadas como ensaios, pois neste formato o autor pode ser livre para expor seu pensamento, mas levando em consideração que escreve apresentando a realidade do sertão, mesmo que tenha vivido longe do lugar e seus traços estejam entrelaçados à sua escrita. Escrevendo, assim, uma historiografia do sertão do Seridó. 

    Neste ínterim, entre escritos e vida de Oswaldo Lamartine, entrelaçam-se diversas vertentes de estudos historiográficos, sociológicos, naturais, modernos, entre outros, que revelam o sertão do Seridó pela sua marca geológica e social/cultural, relações que perpassam o meio ambiente e contam a trajetória do sertanista e de muitos sertanejos.  


    \section{Considerações finais}

    Esse entrelaçamento de escritos e das relações humanas com o meio constrói os espaços e os moldam. No caso do sertão, descrevem toda sua paisagem e tessitura, mostrando que se pode escrever histórias --- as mais diversas --- e as relações podem ser escritas e reescritas, em um processo de longas mudanças, e que do vazio tudo pode brotar. A natureza e suas infinitas formas sobrevivem e se adaptam ao longo dos tempos, sendo o homem quem a transforma, em alguns casos para sua sobrevivência no meio, daí a exploração e rarefação da fauna e flora. 

    Oswaldo Lamartine de Faria, no seu estudo mais específico do sertão do Seridó, molda o espaço pela sua escrita, a natureza como protagonista e o homem como agente biológico do próprio meio, mesmo que no decorrer do tempo o contato com a natureza tenha passado de sobrevivência para exploração. O sertanista faz uma descrição da paisagem sertaneja, através dos elementos naturais que a compõem e das práticas culturais que o homem desenvolveu com o meio. 

    Sendo assim, os estudos biológicos e ambientais sobre o sertão seridoense, conjuntamente com os estudos históricos que envolvem a humanidade e consequentemente o homem sertanejo, mostram que o sertão é uma região marcada por seu bioma natural brasileiro, a caatinga, e que as relações que o homem desenvolveu neste espaço não empobrecem, mesmo que o próprio bioma apresente as suas dificuldades, em relação principalmente às crises hídricas que acometem a região. Foi este o olhar de Oswaldo Lamartine sobre as terras sertanejas.  

    \printbibliography[heading=subbibliography,notcategory=fullcited]

    \label{chap:oswaldo-lamartine-naend}

\end{refsection}
