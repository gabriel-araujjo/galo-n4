\begin{refsection}
    \renewcommand{\thefigure}{\arabic{figure}}

    \chapter[Presídio de Açu: {\itshape{}entre territorialidades lusitanas e sertões indígenas}]{PRESÍDIO DE AÇU\\Entre territorialidades lusitanas e sertões indígenas\footnote{Transcrição modernizada empreendida pela Oficina Permanente de Paleografia, na qual adotou-se como sinais paleográficos o colchete para indicar uma palavra deduzida no documento, conquanto oculta por ações externas; e o sublinhado para desenvolver palavras abreviadas.}}%

    \label{chap:presidio-de-acu}
    
    \articleAuthor{Isabela Mendes Fechina}
    {Graduanda em História pelo Departamento de História da Universidade de Brasília (UnB), voluntária do projeto de extensão “Oficina Permanente de Paleografia”, da Universidade de Brasília. Lattes ID: 9502.5163.3482.6927. ORCID: 0000-0002-6443-2668. E-mail: isabelafechina@gmail.com.}

    \articleAuthor{Alexandre Bruno Barzani Santos}
    {Graduando em História pelo Departamento de História da Universidade de Brasília (UnB), bolsista do projeto de extensão “Oficina Permanente de Paleografia”, da Universidade de Brasília. Lattes ID: 4985.0840.4151.0551. ORCID: 0000-0001-8699-797X. E-mail: alexandrebarzan@gmail.com}

    \articleAuthor{André Cabral Honor}
    {Professor adjunto de História do Brasil Colonial da Universidade de Brasília, coordenador geral da Oficina Permanente de Paleografia, Lattes ID: 0590.8756.5936.9932, ORCID: 0000-0002-3665-129X. E-mail: cabral.historia@gmail.com.}

    \vspace{5mm}

    A massa \marginpar{
        \begin{flushleft}
            \tiny \sffamily
            Como referenciar?\\\fullcite{SelfFechinaEtAl2021Presídio}\mybibexclude{SelfFechinaEtAl2021Presídio},
            p. \pageref{chap:presidio-de-acu}--\pageref{chap:presidio-de-acuend},
            \journalPubDate{}
        \end{flushleft}
    } documental transcrita é composta por um processo promovido por uma carta escrita ao Rei de Portugal D. Pedro II pelo capitão-mor e governador Bernardo Vieira de Mello, político e militar brasileiro de carreira que participou do desfecho da Guerra dos Bárbaros e do desmantelamento do Quilombo dos Palmares. A fonte aborda a decisão dos moradores da Cidade de Natal da construção de um presídio no sertão do Açu, ainda povoado por resistentes grupos originários, responsáveis pela ausência do domínio lusitano em suas partes. 

    Datado de 1697, o documento está disposto no fundo do Arquivo Histórico Ultramarino (AHU), Administração Central (ACL), Conselho Ultramarino (CU), Seção 018 - Rio Grande do Norte, Caixa 1, documento 42. A transcrição desta carta e seus anexos viabiliza reflexões sobre as questões territoriais relacionadas aos sertões coloniais, bem como dos agentes históricos presentes nesses espaços, conceituados por Rafael Bluteau (1712 a 1721, p. 613) como “distantes do mar, apartados do controle e arbítrio portugueses”. 

    Anexados à solicitação para a construção do presídio estão acordos de paz com lideranças indígenas em razão dos seus recorrentes conflitos com os portugueses pela invasão e estabelecimento de sesmarias nas terras que originalmente ocupavam. Conquanto relacionados genericamente como tapuias na documentação, tratavam-se dos Ariús pequenos, Paiacus e Janduís, também conhecidos como Capela, todos pertencentes à mesma família Tarairiú \cite[p.~79]{ARAÚJO2007Muro}. Ao final da massa documental foram atestadas as certificações oficiais da Câmara, suas certidões de rendimentos e de seus antecessores.  

    Partindo-se do instrumental metodológico da paleografia e da bibliografia precedente, a fim de se empreender a análise e a problematização da fonte, buscou-se identificar o seu contexto de produção além dos usos dos conceitos-chaves nela presentes. A carta foi escrita em meio à Guerra do Açu, desdobramento de um maior conflito conhecido como Guerra dos Bárbaros5, que durou de 1650 a 1720, em várias regiões do Recôncavo Baiano e do semiárido das capitanias do Norte, após a expulsão da Companhia Holandesa das Índias Ocidentais (WIC) na Guerra Luso-Holandesa \cite[p.~64]{SILVA2015Ribeira}.

    Os grupos indígenas citados na correspondência possuíram ativa participação neste conflito, caracterizando-se por sua alta mobilidade nos interiores da Capitania do Rio Grande. É possível perceber suas atuações nas motivações diversas que permearam seus investimentos contra as forças portuguesas. A título de exemplo estão a reação ao sequestro de dois filhos do principal Janduí, fato que despertou animosidades entre os indígenas e os colonos \cite[p.~339]{MEDEIROS2008Povos}, como também os frequentes conflitos entre as reduções dos Paiacus com as tropas de paulistas \cite[p.~10]{MAIA2013Aldeias}.

    Dois conceitos são essenciais para a compreensão da carta: sertões e presídio. Tyego Silva apresenta-os como partes de um mesmo processo de territorialização colonial, sobretudo na Capitania do Rio Grande do Norte, cujo cerne está “na ideia de transformação de uma determinada espacialidade em um território com elementos que permitam o exercício do poder e a atribuição de valores” \cite[p.~19]{SILVA2015Ribeira}.

    O conceito de “sertões” na abordagem de Pedro Cardim se concebe como parte da dicotômica relação integrativa entre interior-litoral e barbárie-civi\-li\-da\-de. Dessa forma, pela óptica da Coroa e seus representantes, as áreas ainda não sujeitas a seu poderio eram compostas por "bárbaros", formando pontos instáveis a sua consolidação territorial \cite[p.~43]{DOMINGUES2019indígenas}. 

    Já “presídios”, conforme pontua \textcite[p.~28]{SILVA2015Ribeira}, significava “praça ou fortaleza com ‘gente de guarnição’; localidade protegida por soldados, com o objetivo de defendê-la de inimigos”. Destarte, a fonte transcrita também apresenta outro uso para a edificação, a de servir como assentamento básico para acolher os “novos moradores” que iriam estabelecer suas propriedades a esses sertões “desterrados”, que há muito estavam coabitados por identidades indígenas.

    \section{Transcrição: carta do Capitão-Mor, Bernardo Vieira de Mel\-lo, ao rei, D. Pedro II sobre a decisão de se construir um presídio no sertão do Açu}

    \noindent{}Pg. 1

    \vspace{1ex}

    \noindent{}[margem esquerda]

    \vspace{1ex}

    Haja visto o Procurador da Coroa Real de Lis\underline{bo}a de 697.

    Pelo q\underline{ue} respeita a fazenda, e q\underline{ue} me toca respondeu, parece-me justo o arbítrio do Cap\underline{it}ão-Mor no parágrafo 4 em sua relação, mandando-se continuar o pagam\underline{en}to do vigente desta Capitania na folha de Pernambuco, p\underline{ar}a q\underline{ue} com esta diminuição, possa haver nesta Capitania alguma sobra com q\underline{ue} se remedie a precisa necessidade q\underline{ue} se aponta, o que se deve mandar fazer por alguns anos.

    \vspace{1ex}

    \noindent{}[corpo principal] 

    \vspace{1ex}

    Senhor,  

    Em 29 de junho de 95 tomei posse desta Capitania do Rio Grande em que Vossa Maj\underline{estad}e por sua real grandeza foi servido p[mancha de tinta] por Capitão Maior dela, e achando em miserável estado aos moradores pela destruição, que em suas vidas, e fazendas lhes tinha feito o gentio bárbaro; desejando dar-lhes o castigo que os seus insultos mereciam me impediu achar uma carta de Vossa Majestade de 3 de dezembro de 94 na qual mandou Vossa Majestade seguisse em tudo a ordem do G\underline{overnad}or Geral deste estado Dom João de Lencastre, e esta fosse de que por todos meios procurasse reduzir o gentio a uma universal paz, e ter meu antecessor dado princípio a essa causa [em] algumas nações mais vizinhas a esta cidade, continuei nesta diligência com tão bom sucesso que tenho a todo o gentio desta Capitania reduzido a esta, sem até o presente haver a menor alteração; causando mais detrimento o conservá-los do que me pudera dar por guerra distingui-los, que por sua natureza são inconstantes, faltos de fé, [d]e nenhuma palavra nem agradecimento; mais lembrados do agravo que do benefício. Por cuja causa chamei a minha presença a Câ\underline{mar}a; e todos os moradores de mais suposição, p\underline{ar}a com seu parecer obrar o que visse ser mais conveniente para a segurança, e aumento das povoações, e todos votaram em que se fizesse no Sertão do Açu, que dista 40 léguas deste lugar, um presídio com gente que pudesse refrear qualquer impulso dos bárbaros. E como me achei sem efeitos da Real Fazenda de Vossa Majestade, fiz pelos mesmos um pedido de que resultou concorrerem todos com farinhas para sustento dos que nele assistissem por tempo de 6 meses enquanto se dava parte com o G\underline{overnad}or Geral, com os quais fiz assento cuja cópia como esta remeto a Vossa Maj\underline{esta}de e dando de tudo conta ao Gov\underline{ernad}or G\underline{er}al, que suponho a daria a Vossa Maj\underline{esta}de, o não fiz eu [n]o ano passado, por querer fazê-lo depois de ter subido naquele sertão, feito o presídio e reconhecido todo o desenho dos bárbaros, para com melhor fundamento dar de tudo conta a Vossa Maj\underline{esta}de: mas o inverno, foi tão calamitoso e crescerão tão copiosamente as inundações dos rios, que me puseram em sítio passante de 40 dias e me faltarão os mantimentos; por cuja causa experimentei grandes doenças e gastei mais tempo do que presumi, e se abreviou a partida das frotas; e como agora devo dar conta a Vossa Maj\underline{esta}de e sejam muitas as coisas de que devo fazer, o faço pelo papel que com esta remeto, para Vossa Maj\underline{esta}de ver os capítulos desse, e mandar o que for servido por sua Real grandeza. A Católica e Real Pessoa de Vossa Majestade guarde Deus, como este humilde e leal vassalo deseja. Rio Grande, 25 de abril de 1697. Bernardo Vieira de Mello. 

    %%%%%%%%%%%%%%%%%%%%%%%%%%%%%%%%%%%%%%%%%%%%%%%%%%%%%%%%%%%%%%%%

    \vspace{5mm}

    \noindent{}Pg. 2 

    \vspace{1ex}

    \noindent{}[margem esquerda]

    \vspace{1ex}

    nº 53 
    
    \vspace{1ex}

    \noindent{}[corpo principal] 

    \vspace{1ex}

    Rio Grande, 25 de abril de [1]697. Do Cap\underline{it}ão-Mor B\underline{e}r\underline{nard}o Vieira de Mello Em que dá conta de haver tomado posse daquela Capitania miserável, estado em que achou os moradores dela; e pedido q\underline{ue} se faz p\underline{ara} sustento do presídio do Açu, sobre o qual p\underline{ara} os custos mais remete os papéis inclusos. 

    %%%%%%%%%%%%%%%%%%%%%%%%%%%%%%%%%%%%%%%%%%%%%%%%%%%%%%%%%%%%%%%%

    \vspace{5mm}

    \noindent{}Pg. 3

    \vspace{1ex}

    \noindent{}[corpo principal] 

    \vspace{1ex}

    Cópia do Termo que se fez no adjunto da Câmara; e mais povo desta Capitania do Rio Grande, p\underline{ar}a se fazer o Presídio do Açu.  

    Aos dezesseis dias do mês de julho deste presente ano de mil e seiscentos e noventa e cinco, nesta cidade de Natal, Capitania do Rio Grande, nas casas de morada do Capitão Maior dela, Bernardo Vieira de Mello, donde por ele foram convocados os Oficiais da Câ\underline{ma}ra que de presente servem juízes, vereadores e procurador do Conselho, e todos os homens nobres que costumam servir na República também por ele chamados; a todos propôs que vindo a governar esta Capitania por Sua Majestade, que Deus guarde, e com desejos de se acertar no serviço do dito Senhor e bem deste, por se achar na mão do Capitão Maior a quem sucedeu uma carta do mesmo Senhor, em que dizia que no particular dos Tapuias levantados seguisse o que o Governador, e Capitão Geral do Estado Dom João de Lencastre lhe ordenasse. E logo vira duas cartas do mesmo Governador e Capitão Geral em que ordenava ao dito Capitão Maior, seu antecessor, que em toda a ocasião e por todos os meios tratasse de fazer paz com os ditos bárbaros, por ser esta a vontade de Sua Majestade; e o único meio para quietação e sossego destas Capitanias, as quais cartas assim de Sua Majestade, como do Governador Geral, fez presentes aos ditos congregados, dizendo-lhes que indubitavelmente havia de observar as ditas ordens, porque além de ser assim obrigado, tinha mostrado a experiência os impossíveis que havia para extinguir os bárbaros, e que só por meio da paz podia haver quietação, e que como achava esta principiada no modo possível, se achava também obrigado a observá-la com as condições nos capítulos dela contidos, que também lhes fez presente aos ditos congregados no Livro dos Registros, em que estavam lançados, e que julgava que agora o que convinha era conseguir o intento que tinha de povoar os sertões; especialmente o do Açu, metendo-se gados em todas as partes, porque desta sorte se aumentava logo em tudo esta Capitania, e subiriam de preço os dízimos reais, e que para este efeito estava resoluto a ir em pessoa ao Açu a dar forma às primeiras povoações, e fazer alguns quartéis fortificados, donde nestes princípios se recolhessem de noite os novos povoadores, até mostrar a experiência e amizade do gentio de que não há dúvida, vendo que nos alargamos a habitar entre eles desenganados de que os não tememos; e que este seu intento lhes comunicava aos ditos congregados, para ouvir os seus pareceres, e ajustarem o que melhor conviesse no particular das povoações; porque na matéria da paz não havia que resolver; por ser forçoso observá-la em razão das ordens q\underline{ue} haviam, e que para a tal observância encarregava a todos, e a cada um em particular tratassem os Tapuias novamente reduzidos com mostras de amizade pois sabiam de sua inconstância que pouco lhes bastava para desconfiarem, e q\underline{ue} para ajustá-lo ao modo das povoações pedia por serviço de El Rey aos Oficiais da Câ\underline{ma}ra e aos mais que presentes estavam, que o ajudassem com alguns mantimentos para socorrer a gente q\underline{ue} havia de levar consigo para o Açu, vista a impossibilidade da Fazenda Real em que de presente não havia um tostão para a despesa, e se estarem devendo mais de três mil cruzados caídos aos filhos das folhas do assentamento, e que concordada a viagem do Açu, pretendia que fosse logo a tempo de que os gados que vinham do Ceará Grande se pastassem para fazer q\underline{ue} deixassem logo naqueles campos algum gado dos que traziam; e que para isto lhe pertencia logo fazer aviso; porque sem perigo se podia atravessar aquele sertão, em razão da paz que

    %%%%%%%%%%%%%%%%%%%%%%%%%%%%%%%%%%%%%%%%%%%%%%%%%%%%%%%%%%%%%%%%

    \vspace{5mm}

    \noindent{}Pg. 4

    \vspace{1ex}

    \noindent{}[corpo principal] 

    \vspace{1ex}

    Também fez o Capitão Maior do Ceará com a nação dos Paiacus, e da destruição que estes fizeram aos Icós lá no interior do sertão onde habitam. Ouvido que foi pelos oficiais do Senado da Câmara, e os mais homens nobres desta Capitania que se acharão presentes ao proposto pelo dito Capitão Maior, todos de comum acordo disseram q\underline{ue} eram vassalos de Sua Majestade, e como tais não podiam faltar nunca em obedecer as suas reais ordens e aprovar as pazes que em razão delas se haviam feito com os Tapuias do sertão, sem embargo de conhecerem a sua inconstância, pouca palavra, e nenhuma razão em que vivem; prometendo observar, e em tudo guardar por sua parte o proposto e capitulado nas ditas pazes: e para que melhor conste da fidelidade com que todos desejam empregar-se no serviço do dito Senhor, se oferecem espontaneamente a assistirem o sustento necessário para os sujeitos que hão de assistir em defesa e segurança dos gados em que todos comumente concordaram se metessem nas ribeiras do Açu, a fim de que se povoasse, e as mais partes destes sertões que antigamente o foram; o qual sustento dariam por tempo de seis meses, enquanto o Governador Geral deste Estado manda socorrer com efeito aqueles lugares; para que se perpetuem neles as povoações, que naquela parte podem haver, as quais convém estejam com toda a segurança, pela pouca fé que costuma guardar o Tapuia a fim de que em nenhum tempo possam executar nos nossos descuidos e nas nossas poucas forças os seus danados intentos, mas antes tímidos delas observem as pazes celebradas. E que como todos os moradores desta Capitania vivem quase exaustos dos cabedais que possuíam, e por essa causa muito impossibilitados, se não atreviam a maior despesa, e se sujeitarão a essas fiados em que o Governador Geral do Estado mandará dentro dos sobreditos seis meses socorros convenientes a assegurar aquelas ribeiras, e todos estejam seguros nestas em que moram, e de outro modo não será possível viverem nelas largando suas casas e fazendas, para se recolherem às mais capitanias onde possam viver sem as soçobras q\underline{ue} lhes causam a inconstância dos Tapuias, o que não esperam do dito Governador Geral por conhecerem o grande zelo com que procura os aumentos de serviço de Sua Majestade, e o quanto é amantíssimo da observação dele, e por assim o esperarem da sua grandeza se ajustaram todos de comum acordo, e consentimento na forma referida; e para que a todo o tempo conste mandou o d\underline{it}o Capitão Maior Bernardo Vieira de Mello fazer este termo em que todos assinarão. Dia e era et supra. - Oficiais da Câm\underline{a}ra - João da Costa de Almeida Manoel Roiz Santiago - José Barbosa Leal. Domingos Gomes Salema Gonçalo Monteiro - Bernardo de Abreu e Lima - Francisco de Oliveira Banhos - O Padre Vigário Basilio de Abreu e Andrada - Estevão de Bezerril - Manoel Gomes Torres - Antônio Gomes Torres - Francisco Lopes - Manoel Vieira do Valle - Manoel de Abreu Friellas - Pedro da Costa Faleira - Antônio Gomes de Barros - Teodósio da Rocha - Manoel Roiz Coelho - José de Amorim - Pedro Martins Baião Filipe da Silva - Francisco de Ornelas - Sipriano Lopes Pimentel - Antônio Lopes Lisboa - Antônio Álvares Correa - Gonçalo da Costa Faleiro - Francisco Gomes - Manoel da Silva Vieira - Jacinto Veloso - Antônio Dias Pereira - Manoel Ribeiro de Carvalho - Manoel Fe\underline{rnande}s de Mello. Almox\underline{arifad}o da Fazenda Real. O provedor M\underline{ano}el Tavares Guer\underline{reir}o.  

    %%%%%%%%%%%%%%%%%%%%%%%%%%%%%%%%%%%%%%%%%%%%%%%%%%%%%%%%%%%%%%%%

    \vspace{5mm}

    \noindent{}Pg. 5

    \vspace{1ex}

    \noindent{}[corpo principal] 

    \vspace{1ex}

    O escrivão da d\underline{it}a Fazenda, Manoel Gonçalves Branco. O qual eu Manoel Eusébio da Costa trasladei bem, e fielmente do próprio que está lavrado no Livro Segundo, dos Registros da Secretaria deste Governo do Rio Grande, a folhas oitenta e sete que me reporto, e vai sem coisa que duvida faça o que sobredito escrevi. 

    %%%%%%%%%%%%%%%%%%%%%%%%%%%%%%%%%%%%%%%%%%%%%%%%%%%%%%%%%%%%%%%%

    \vspace{5mm}

    \noindent{}Pg. 6

    \vspace{1ex}

    \noindent{}[corpo principal] 

    \vspace{1ex}

    Cópia do Tratado da Paz feita com os Tapuias Ariús pequenos. 

    Aos vinte dias do mês de março deste presente ano nesta cidade do Natal, Capitania do Rio Grande, nas casas de morada do Capitão Maior dela Bernardo Vieira de Mello, e em sua presença se achou também o chamado Rei dos Tapuias Ariús pequenos, por nome Peca, que habitam nos confins desta Capitania, no mais íntimo destes sertões; o qual disse que vinha com sua própria pessoa ajustar a paz por estarem todas as nações mais vizinhas, e que residem no distrito desta Capitania, unidas na mesma paz e à nossa amizade; o qual disse que em nenhum tempo, por si nem por outrem dos seus haveria mais guerra com brancos, e se obrigava a fazê-la a todos aqueles que não quisessem admitir a nossa amizade; e prometia ser fiel vassalo do muito Invicto, e Poderoso Senhor Rei de Portugal nosso Senhor, a quem prometia servir e obedecer e aos seus Governadores e Capitães Maiores com pronta obediência como deve e é obrigado; e da sua parte pedia perdão da desobediência, e seus erros passados pelos quais prometia não só condescender a que se povoassem os sertões que a seu respeito se despovoaram; se não que com seus soldados ajudaria a fazer currais e casas para se meterem gados nas terras em que haviam, como o haviam feito os do Açu. E com isto o d\underline{it}o Capitão Maior lhe deu perdão dos seus erros passados e lhes segurou a paz que pediam tudo em nome do Governador e Capitão Geral deste estado Dom João de Lencastre, e conforme a sua ordem que para isto tinha: porém com as condições contidas nos capítulos seguintes: 1º Que descendo do sertão as nossas povoações não poderão trazer armas mais q\underline{ue} até os sítios que chamam do Taipu ou da Pirituba, ou do Iaçu, e vindo pela praia até a barra do Ceará-Mirim. 2º Que com os brancos que vão para o Sertão do Açu, ou para de onde eles habitam, a enviar seus gados terão toda a conformidade e os ajudarão para os benefícios dos mesmos gados, e condução deles pagando-lhes o seu trabalho. 3º Que se alguma outra nação se rebelar ou desobedecer irão com os brancos a fazer-lhes guerra até os reduzirem à nossa obediência. 4º Que não consistirão em sua companhia os escravos fugitivos dos moradores, antes os prenderão, e trarão abaixo, e se lhes pagará a sua diligência. 5º Que porquanto entre nós vive alguma gente da sua nação, machos e fêmeas, já domésticos, catequizados e batizados, que não pretenderão levá-los consigo para o sertão por não ser justo, que sendo batizados, e filhos da Igreja, tornem ao barbarismo de que saíram maiormente, porque estão todos voluntariamente contentes, e satisfeitos na companhia dos brancos. E porque na sua rudeza pode haver alguma incapacidade no aceitarem as di\underline{tas} condições lhe disse o d\underline{it}o Capitão Maior que nomeassem um branco, seu amigo, e confidente, para em seu nome aceitar as d\underline{ita}s condições, e prometerem a observância delas, o qual elegeu ao Capitão Antônio Ál\underline{vare}s Correa seu condutor, a quem buscaram por ser seu conhecido antigo, por ter terras aonde é sua habitação, e haver nelas tido gados que com o levante da guerra do dito gentio se destruíram; o qual vendo serem as condições todas racionais e toleráveis as aceitou e assinou este tratado em seu nome, em que também assinou com uma cruz o dito Rei Peca e um seu irmão por nome Capitão João Pinto Correa.  

    %%%%%%%%%%%%%%%%%%%%%%%%%%%%%%%%%%%%%%%%%%%%%%%%%%%%%%%%%%%%%%%%

    \vspace{5mm}

    \noindent{}Pg. 7

    \vspace{1ex}

    \noindent{}[corpo principal] 

    \vspace{1ex}

    E de tudo mandou o dito Capitão Maior fazer este assento, e que se registrasse de onde toca. Manoel Eusebio da Costa o fiz, ano de mil seiscentos e noventa e sete. Bernardo Vieira de Mello - Cruz do Peca - Cruz de João Pinto Correa - Antônio Álvares Correa. O qual eu Manoel Eusebio da Costa trasladei bem e fielmente do próprio que está lançado no Livro Segundo, dos registros da Secretaria deste Governo do Rio Grande, a folhas cento e quinze e verso, a que me reporto e vai sem coisa que dúvida faça o que sobredito escrevi. 

    %%%%%%%%%%%%%%%%%%%%%%%%%%%%%%%%%%%%%%%%%%%%%%%%%%%%%%%%%%%%%%%%

    \vspace{5mm}

    \noindent{}Pg. 8

    \vspace{1ex}

    \noindent{}[corpo principal] 

    \vspace{1ex}

    Cópia da Retificação de paz feita com os Tapuias Janduís da Ribeira do Açu  

    Aos vinte dias do mês de setembro deste presente ano nesta cidade do Natal, Capitania do Rio Grande, nas casas de morada do Capitão Maior Bernardo Vieira de Mello; e em sua presença se achou também o chamado Rei dos Tapuias Janduís por nome Taiá-Açu, o qual disse que vinha com sua própria pessoa a retificar a paz que pelos seus principais tinha mandado fazer, visto que de novo a havia o d\underline{it}o Capitão Maior mandado assegurar; enviando-lhe em sinal dela um seu bastão, e obrigado com isso vinha em pessoa, não só a retificar a mesma paz, se não a assegurar que em nenhum tempo por si nem por outrem dos seus haveria mais guerra com os brancos, e se obrigava a vi[r] em nossa companhia a fazê-la a todos aqueles q\underline{ue} não quisessem admitir a nossa amizade, e prometia ser fiel vassalo do muito Invicto e Poderoso Senhor Rei de Portugal; e Senhor nosso a quem prometia servir, e obedecer e aos seus Governadores, e Capitães Maiores com pronta obediência como deve, e é obrigado; e da sua parte pedia perdão da desobediência e seus erros passados, pelos quais prometia não só condescender a que se povoassem os sertões que a seu respeito se despovoam, se não que com seus soldados ajudaria a reedificar os currais e casas, como já dera princípio com os gados que agora haviam chegado do Ceará ao Açu, como dos mesmos homens que os haviam trazido constava; e que estava por todos os capítulos feitos na paz tratada com os seus enviados, que são os que abaixo se declaram - 1º Que descendo do sertão as nossas povoações não podem trazer armas mais que até os sítios que chamão dos Taipu, ou da Pirituba e vindo pela Praia até a barra do Ceará-Mirim - 2º Que com os brancos que vão para o Sertão de Açu a criar seus gados terão toda a união e conformidade; e os ajudarão para os benefícios dos mesmos gados, e condução deles, pagando-lhes o seu trabalho. - 3º Que se alguma outra nação se rebelar ou desobedecer irão com os brancos a fazer-lhes guerra, até os reduzirem à nossa obediência. - 4º Que não consentirá em sua companhia os escravos fugitivos dos moradores, antes os prenderão, e trarão abaixo e se lhes pagará a sua diligência - 5º Que porquanto entre nós vive alguma gente da sua nação, machos e fêmeas, já domésticos, catequizados e batizados, que não pretenderão levá-los consigo para o sertão, por não ser justo que sendo batizados e filhos da Igreja tornem ao barbarismo de q\underline{ue} saíram: Maiormente porque estão todos voluntariamente, contentes e satisfeitos na companhia dos brancos. E com isto o dito Capitão Maior lhe deu e segurou o d\underline{it}o perdão e paz que pediam tudo em nome do Governador, e Capitão Geral deste estado Dom João de Lencastre, e conforme a sua ordem que sobre este particular achou por carta de seu antecessor, o Capitão Maior Agostinho César de Andrade; e logo pelo dito Capitão Maior lhe foi admoestado o muito que lhe convinha, assim como se sujeitavam à obediência de vassalos de Sua Majestade, que Deus guarde, o abraçarem justamente a paz espiritual, querendo 

    %%%%%%%%%%%%%%%%%%%%%%%%%%%%%%%%%%%%%%%%%%%%%%%%%%%%%%%%%%%%%%%%

    \vspace{5mm}

    \noindent{}Pg. 9

    \vspace{1ex}

    \noindent{}[corpo principal] 

    \vspace{1ex}

    Aliar-se e aceitar sacerdote q\underline{ue} lhe administrasse os sacramentos, e ensinasse a doutrina cristã; ao que respondeu o chamado Rei que falaria com todos os mais para se aldearem, dando-se-lhe na Ribeira do Ceará-Mirim desta Capitania terras donde pudessem fazer suas plantas, por serem as do Açu muito secas para nelas se plantar roça; e o dito Capitão Maior lhe prometeu dar-lhes terras donde eles comodamente se pudessem aldear, e para maior capacitá-los lhes deu logo alguma ferramenta, mandando com eles pessoa que os fosse acomodar na parte mais conveniente: e para que bem constasse tudo o tratado acima mandou o d\underline{it}o Capitão Maior nomeasse homem branco mais seu confidente que por sua parte aceitassem as condições impostas e assinasse esse tratado como testemunha de tudo sobredito, que lhes foi lido e explicado pelo melhor modo q\underline{ue} possível foi para o poderem entender; para o que nomeou o dito chamado Rei ao Capitão Gaspar Freire de Carvalho, que com o d\underline{it}o Capitão Maior assignou perante muitas pessoas que presentes estavam, e do mesmo chamado Rei, e dos seus intérpretes que com ele se achavam, e mais Tapuias em sua companhia vieram; e de tudo mandou o dito Capitão Maior fazer este assento, e que se registrasse donde toca. Dia ut supra. João de Abreu Barreto o fez no ano de mil e seiscentos e noventa e cinco - Bernardo Vieira de Mello - Cruz de Taiá-Açu - Gaspar Freire de Carvalho. O qual eu Manoel Euzebio da Costa trasladei bem e fielmente do próprio que está lavrado no Livro Segundo dos Registros da Secretaria deste Governo do Rio Grande, a folhas cento e quinze a que me reporto, e vai sem coisa que dúvida faça o que sobredito escrevi. 

    %%%%%%%%%%%%%%%%%%%%%%%%%%%%%%%%%%%%%%%%%%%%%%%%%%%%%%%%%%%%%%%%

    \vspace{5mm}

    \noindent{}Pg. 10

    \vspace{1ex}

    \noindent{}[corpo principal] 

    \vspace{1ex}

    O Escrivão da Fazenda Real desta Capitania e Almoxarife dela passem ao pé desta certidão da importância do rendimento dos contratos desde o ano de seiscentos e noventa e cinco que tomei posse do governo desta Capitania; até o presente de seiscentos e noventa e sete, e juntamente quanto se paga aos filhos da folha desta Capitania em cada ano e a quantia que se deve a cada um deles dos anos passados que se lhe não pagou e porque causa não estão pagos, declarando-se que estes atrasados se lhe devem antes da minha vinda a esta Capitania, se depois que eu tomei posse dela tenho tudo com muita distinção e clareza para assim o remeter a Sua Maj\underline{esta}de, que Deus guarde. Cidade do Rio Grande, 6 de maio de 1697. 

    - Bernardo Vieira de Mello  

    Manoel Gonçalves Branco, escrivão da Fazenda Real, Alfândega e Almoxarifado nesta cidade de Natal, Capitania do Rio Grande, por Sua Majestade que Deus guarde V\!\underline{oss}a A\underline{lteza}. Certifico que em virtude da portaria acima, como Almoxarife da dita Real Fazenda, fizemos a conta de que está devendo este Almoxarifado aos filhos da folha dele, como também ao rendimento dos dízimos; e achamos constar do Livro das Arrematações haver-se arrematado os dois anos declarados na dita portaria a Manoel Rodrigues Areosa, ambos em preço e quantia de um conto trezentos e setenta mil réis. E vendo pelas folhas do apontamento a importância dos filhos da folha achamos pagar-se ao Capitão Maior duzentos mil réis; ao Provedor cinquenta; ao Almoxarife trinta; ao Reverendo Vigário duzentos e quarenta e quatro; ao Reverendo Coadjutor vinte e cinco; ao Tesoureiro da fábrica oito; ao Alferes da Fortaleza dos Santos Reis Magos noventa e seis mil réis. 

    %%%%%%%%%%%%%%%%%%%%%%%%%%%%%%%%%%%%%%%%%%%%%%%%%%%%%%%%%%%%%%%%

    \vspace{5mm}

    \noindent{}Pg. 11

    \vspace{1ex}

    \noindent{}[corpo principal] 

    \vspace{1ex}

    Que importa em cada um ano seis contos e cinquenta e três mil réis. E o que se deve até o presente é o seguinte: ao Capitão Maior que foi desta Capitania Pascoal Gonçalves de Carvalho se lhe ficou devendo de resto de seu ordenado do último ano de 688, noventa e quatro mil e quatrocentos; ao Reverendo Vigário que foi desta Capitania Paulo da Costa Barros se lhe ficou devendo de resto de seus ordenados dos anos de 688 e parte de 689, duzentos e setenta e seis mil réis; ao Reverendo Vigário que foi Eloi de Freitas de resto de seu ordenado trinta e sete mil réis. Ao dito Reverendo Coadjutor três anos e meio de 691 até parte de 695 oitenta e sete mil e quinhentos; e o resto de Vigário se lhe ficou devendo do ano de 690; ao Reverendo Coadjutor que foi no dito ano Manoel Dias Santiago vinte e cinco mil réis; ao Alferes da Fortaleza João Ferreira de Sousa se lhe deve cinco anos, de 692 até o presente que são quatrocentos e oitenta mil réis; ao Capitão Maior, atual Bernardo Vieira de Mello, se lhe deve de resto de seu ordenado cento e trinta mil réis; ao Provedor atual Manoel Fernandes Guerreiro se lhe deve de resto de seu ordenado trinta mil réis; e a mim Almoxarife atual, Manuel Fernandes de Mello, se me deve trinta mil réis; ao Reverendo Vigário atual Pedro da Rocha de Figueiredo um ano que há de fazer em julho próximo que se lhe deve duzentos e quarenta e quatro mil réis; ao seu Coadjutor Pedro Fernandes vinte e cinco; à fábrica oito; que tudo soma o que está devendo este Almoxarifado aos filhos da folha declarados atrás, e acima, um conto, quatrocentos e sessenta e seis mil, e novecentos réis. E não se pagou as pessoas nomeadas atrás nos anos em que serviram por nos ditos anos não haverem efeitos por se remeterem os dízimos em cada um deles que foram de seis para sete anos, a trezentos e dez mil réis, e só chegaram os dois últimos a estes trezentos e setenta e cinco; causa do gentio bárbaro pela destruição que fez no seu levante nos gados vacuns, e cavalares; e por nos ser mandado passar a presente a subscrevi eu, Manuel Gonçalves Branco, escrivão da Fazenda Real, e a fiz escrever. E assinei com o dito Almoxarife e vai na verdade. Cidade do Natal. 10 de maio de 1697.  

    \vspace{1em}

    \noindent{}- Manuel Gonçalves Branco - Manuel Fernandes de Mello
    
    \nocite{JORGEAndJorge2000500}

    \printbibliography[heading=subbibliography,notcategory=fullcited]

    \label{chap:presidio-de-acuend}

\end{refsection}
