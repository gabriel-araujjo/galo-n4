\begin{refsection}
    \renewcommand{\thefigure}{\arabic{figure}}

    \chapter[Os últimos desejos de um governador português na capitania do Rio Grande do Norte: {\itshape o testamento de Caetano da Silva Sanchez (1799)}]{OS ÚLTIMOS DESEJOS DE UM GOVERNADOR PORTUGUÊS NA CAPITANIA DO RIO GRANDE DO NORTE\\O testamento de Caetano da Silva Sanchez (1799)\footnote{A transcrição em tela é produto de um projeto de pós-doutorado cumprindo na Universidade Federal do Rio Grande do Norte com financiamento da CAPES.}}
    \label{chap:transcricao}
    
    \articleAuthor{Thiago do Nascimento Torres de Paula}
    {Pós-Doutor em Educação pela UFRN (2018), Doutor em História pela UFPR
    (2016), Analista de Ciência, Tecnologia e Inovação da FAPERN (Fundação de
    Apoio à Pesquisa do Estado do Rio Grande do Norte), Pesquisador do
    LEHS/UFRN (Laboratório de Experimentação em História Social da Universidade
    Federal do Rio Grande do Norte), titular da cadeira de nº 96 do IHGRN
    (Instituto Histórico e Geográfico do Rio Grande do Norte) e Professor
    Colaborador da Pós-Graduação Lato Sensu do IFRN (Instituto Federal de
    Educação, Ciência e Tecnologia do Rio Grande do Norte).
    Lattes ID: 1215.9127.7257.3170. ORCID: 0000-0002-4481-4327. E-mail:
    thiagotorres2003@yahoo.com.br.}

    \vspace{5mm}

    Os testamentos \marginpar{
        \begin{flushleft}
            \tiny \sffamily
            Como referenciar?\\\fullcite{SelfTorresDePaula2021}\mybibexclude{SelfTorresDePaula2021},
            p. \pageref{chap:transcricao}--\pageref{chap:transcricaoend},
            \journalPubDate{}
        \end{flushleft}
    } são documentos produtos da Idade Média. Assim, ao longo dos séculos XVI, XVII e XVIII a prática de elaborar testamentos se difundiu pela cristandade ocidental. Com isso, o objetivo deste trabalho, é disponibilizar a transcrição do testamento do governador portugueses Caetano da Silva Sanchez, elaborado e aberto no apaga das luzes do século XVIII, especificamente no ano de 1799 nas terras da freguesia de Nossa Senhora da Apresentação na capitania do Rio Grande do Norte. 

    O documento examinado e transcrito, atualmente encontra-se sub a guarda do arquivo do Instituto Histórico e Geográfico do Rio Grande do Norte, necessariamente integrando a seção de textos manuscritos. O testamento do ilustre Caetano da Silva Sanchez, trata-se de um texto assentado em um livro de notas pertencente a freguesia supramencionada, compondo uma coleção rara de 32 testamentos referentes ao litoral da capitania do Rio Grande do Norte. Documentos produzidos por homens e mulheres que viveram na Cidade do Natal, Vila Nova de Extremoz do Norte, povoação de São Gonçalo e Vila de São José do Rio Grande.  

    O material transcrito traz consigo marcas de sua trajetória, tais como: palavras apagadas, outras borradas, quando não sublinhadas por historiadores do passado. No entanto, o documento original encontra-se em excelente estado e conservação, constituindo-se em um manuscrito de três laudas, expondo manchas que sugere acidente com líquidos ao longo do tempo, porém, não há desgastes que comprometa a leitura. 

    O testamento do governador Caetano da Silva Sanchez, como tantos outros documentos testamentários do século XVIII, era um instrumento legal de função cartorial, porém, marcadamente redigido com um discurso religioso e por vezes com quebras de expectativas. Sendo assim, a primeira parte do texto é reservada pelo testador para um acerto de contas com o mundo celestial. Além de realizar a encomendação da alma e uma profissão de fé, relata aspectos sobre sua origem, casamento e filhos.  

    No entanto, o governador ao narrar as últimas vontades não descreveu como gostaria que fosse o seu funeral, nem os ritos do sepultamento, aliás, práticas recorrentes em outros testamentos época. Por outro lado, o testamento de Caetano da Silva Sanchez rompe com a expectativa do leitor, pois não há uma segunda parte em que o testador normalmente realiza a declaração dos bens moveis e imóveis3. A partir disto, o último governador setecentista da capitania do Rio Grande do Norte, expressou apenas duas vontades, a saber: que sua esposa fosse a testamenteira (procuradora pós-morte) e que fossem rezadas quatro capelas de missas, ou seja, 200 missas pelas almas dos seus pais.      

    Ao cabo, é consenso entre os historiadores que testamentos elaborados em outras temporalidades, apresentam-se como fontes importantes para compreensão das múltiplas dimensões da vida cotidiana. Os testamentos em suas estruturas são portadores de informações que podem ser examinadas por procedimentos qualitativos e quantitativos \cites{Marcilio1983Morte}{RodriguesAndDillmann2013Desejando}{Santos2013Historia}. Por fim, a transcrição exposta poderá servi como material para o processo de ensino e aprendizado de jovens pesquisadores, em cursos e seminários de Metodologia da Pesquisa Histórica, ou mesmo, como fonte para investigações no campo da História Sociocultural.

    \section{Transcrição}

    \subsection*{Testamento do capitão-mor governador Caetano da Silva Sanchez}

    Registro do testamento com que faleceu o capitão-mor governador que foi desta capitania Caetano da Silva Sanchez desta Freguesia de Nossa Senhora da Apresentação. 

    Em nome da santíssima trindade, padre, filho e espírito santo, três pessoas distintas e um só Deus verdadeiro. Saibam quantos este instrumento de testamento de cédula de ação, como em direito melhor nome haja de se chamar por sua validade virem que sendo no ano do Nascimento de Nosso Senhor Jesus Cristo de mil setecentos e noventa e nove, aos vinte três dias do mês de agosto do dito ano, nesta cidade do Natal, capitania do Rio Grande do Norte, na casas de minha residência eu, Caetano da Silva Sanchez, sargento-mor de infantaria paga e por agora governador da mesma capitania, estando em meu juízo perfeito e entendimento, com saúde que Deus é servido dar-me, e temendo a morte e por não saber o quando o mesmo senhor será servido levar-me para si, faço este testamento na forma seguinte. Primeiramente encomendo minha alma a santíssima trindade que a criou e a remiu, e a virgem Nossa Senhora, e ao anjo da minha guarda e ao santo do meu nome, aos de minha mais devoção e todos da corte do céu sejam meus intercessores, quando a minha alma deste mundo partir para que vá gozar da bem aventurança porque como fiel e verdadeiro cristão protesto viver e morrer na Santa Fé Católica, creio o que crê a santa Igreja romana mesma fé espero salvar a minha alma. Declaro que sou natura da Freguesia de Casais da Europa , filho legítimo do capitão Francisco da Silva Sanchez, de Maria Joaquina. Declaro que sou casado com Dona Maria Francisca do Rosário Lopes, filha legítima do sargento-mor Francisco Gonçalves, natural de Pernambuco, de cujo matrimônio tivemos dois filhos, um por nome Pedro que em poucos dias de nascido faleceu e outra por nome Dona Micaela Joaquina Sanchez, casada que foi com o capitão-mor Manuel Teixeira de Moura, que também já é falecida = Declaro que os bens que possuo no meu casal são uns poucos dez escravos os bens que deixaram por meu falecimento = Declaro que não tenho herdeiros forçados, ascendentes nem descendentes por terem meus pais e filhos falecido da vida presente, e a dita minha filha não deixar filho algum e não tenho esperança de ter mais filhos que sejam meus herdeiros descendentes e por esta razão deixo a minha dita mulher, Dona Maria Francisca do Rosário Lopes, por minha universal herdeira de todo [ilegível]onte de minha meação de minha fazenda que me ficar por meu falecimento, o qual dou todos os meus poderes que em direito posso par apor e dispor e vender e tomar posse de tudo que me tocar, como seu que fica sendo por meu falecimento e não disponham no meu enterramento por confiar nela o fará conforme as feitas que puder, e tudo quanto determinar o seu arbítrio me satisfaço e dou por bem determinado, o qual minha mulher dita também a nomeio por minha testamenteira, e lhe rogo queira assistir para da minha fazenda de meação mandar quatro capelas de missas, duas pela alma de meu pai, duas pela alma de minha mãe, que é o único legado que disponho, e lhe peço que faça não porque não confie nela que deixasse de fazer havendo lembrança, porém, porque o dirá meu pai fiquem sem esse sufrágio, pela obrigação que tenho e amo que conservo as suas almas e pela mesma confiança que faço da dita minha mulher de que há de cumprir este legado que só faço esta declaração para lembrança, não será ela obrigada por ela e por mais coisa alguma de dar contas com juízo desta minha última vontade e rogo as justiças de Sua Majestade Fidelíssima faça inteiramente em cumprir e guardar este testamento na forma que nele se contém e declaro tenho, digo, tanto secular como eclesiásticas. E deixo por revogado outro qualquer testamento ou codecilho que antes deste tenha feito, porquanto este é minha última vontade quero que seja o que valha para se lhe dar inteiro dito que por verdade poder o sargento-mor Antônio de Barros Passos o escrevesse o que lhe foi ditado por mim e lido o achei estar conforme a minha determinação e vontade [ilegível] assino nele com a minha firma de nome inteiro que costumo que também o que escreveu como testemunha, nesta dita cidade e no dito dia e mês e ano retro declarado = Caetano da Silva Sanchez = como testemunha que escrevi = Antônio de Barros Passos = Aprovação Saibam quantos este público instrumento de aprovação de testamento de derradeira e última vontade virem que no ano do nascimento de Nosso Senhor jesus Cristo de mil setecentos e noventa e nove anos aos vinte e quatro dias do mês de agosto do dito ano, nesta cidade do Natal, capitania do Rio Grande do Note, em casas de residência do governador desta cidade, par aonde eu tabelião adiante nomeado fui vindo e sendo aí apareceu o ilustríssimo senhor governador da dita cidade, Caetano da Silva Sanchez, de que sem moléstia alguma em seu perfeito juízo e entendimento que Nosso Senhor foi servido dar-lhe, pessoa que reconheço pela mesma de que se trata, de que dou fé, e por ele me foi dado este papel de sua mão a minha dizendo que era o seu solene testamento e que havia mandado escrever pelo sargento-mor Antônio de Barros Passos, dotando ele testador e que depois de escrito os mandara ler pelo achar conforme ele dito testador o havia ditado se assinara com o dito sargento-mor Antônio de Barros Passos este como testemunha que o serviu requerendo-me o aprovasse, porquanto ele testador o aprovava sendo outro qualquer testamento ou codecilho [ilegível] feito [ilegível] e rogava as justiças de Sua Majestade Fidelíssima assim o cumprisse e guardasse como nele se contém declarado [ilegível] inteiro vigor, cujo testamento tomando eu tabelião em minha, digo, em meu escritório verbum adverbum, achei limpo, sem vício algum nem borrão ou entrelinha que dúvida faça e estava assinado o dito testador com o dito sargento-mor Antônio de Barros Passos como testemunha que escreveu e estava escrito em duas laudas e meia de papel que acaba aonde eu tabelião principio esta aprovação, cujo testamento o aprovo e hei por aprovado tanto quanto em direito posso em razão do meu ofício sou obrigado, sendo em tudo presentes por testemunhas o reverendo padre coadjutor Francisco Oliveira, o capitão Antônio José de Souza Oliveira, o capitão Fidelis José da Rocha, o capitão Luís José Rodrigues Pinheiro, o tenente Antônio José de Vasconcelos = o alferes João Manuel Carvalho = o sargento-mor Antônio de Barros Passos que todos assinaram com o dito testador, pessoas todas de mim tabelião reconhecidas pelas mesmas de que se tratam de que dou fé eu, Patrício Antônio de Albuquerque, tabelião do público judicial e notas desta dita cidade do Natal, capitania do Rio Grande do Norte e seu termo pela Rainha Fidelíssima Nossa Senhora que  Deus Guarde, escrevi e assinei esta aprovação com o meu sinal público do que uso em dia e era retro no princípio desta declaração em fpe de verdade = Caetano da Silva Sanchez, Patrício Antônio de Albuquerque, Francisco Alves de Melo, Antônio José [ilegível], Fidelis José da Rocha = Luís José Rodrigues Pinheiro, Antônio José de [ilegível], Manuel de Carvalho = Antônio de [ilegível] quatorze de maro de mil e oitocentos abri este testamento pelo que era fechado e lacrado na forma do estilo, sem vício, cidade do Natal dia e ano como acima. Feliciano Dorneles. E não se continha mais em o dito testamento que eu bem e fielmente [ilegível] verbo adverbum [ilegível] coisa que dúvida faça [ilegível] em juízo [ilegível] o qual [ilegível] me [ilegível] que [ilegível] para [ilegível] 16 [ilegível] Manuel ´[ilegível] escrevi e assinei.

    \vspace{5mm}

    \hfill Manuel [ilegível]


    \printbibliography[heading=subbibliography,notcategory=fullcited]

    \hfill Recebido em 6 abr. 2021.

    \hfill Aprovado em 16 abr. 2021.

    \label{chap:transcricaoend}

\end{refsection}
