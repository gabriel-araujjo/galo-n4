\begin{refsection}
    \renewcommand{\thefigure}{\arabic{figure}}

    \chapter[Política(s) e modernização: {\itshape a implantação do programa “alimentos para a paz” e as frentes de trabalho no sertão do Seridó-RN (1968--1976)}]{POLÍTICA(S) E MODERNIZAÇÃO\\A implantação do programa ``alimentos para a paz'' e as frentes de trabalho no sertão do Seridó-RN (1968--1976)\footnote{O texto apresentado para a Revista Galo em formato de Projeto de Pesquisa foi aprovado para ser desenvolvido como pesquisa de mestrado no Programa de História dos Sertões no CERES-UFRN na cidade de Caicó-RN.}}
    \label{chap:politicamoder}
    
    \articleAuthor{João Paulo de Lima Silva}
    {Graduado em História (UFRN-CERES, Caicó), Especialista em História dos Sertões
    (UFRN-CERES, Caicó), mestrando no Programa de Pós-Graduação em História dos
    Sertões do (UFRN-CERES, Caicó). ID Lattes: 8111.2333.0951.3952.
    ORCID: 0000-0002-4254-8571. E-mail: joaopaulojp31@hotmail.com.
    Sob orientação da Prof.ª Drª. Jailma Maria de Lima.}

    \begin{galoResumo}
        \marginpar{
            \begin{flushleft}
                \tiny \sffamily
                Como referenciar?\\\fullcite{SelfSilva2021Politica}\mybibexclude{SelfSilva2021Politica},
                p. \pageref{chap:politicamoder}--\pageref{chap:politicamoderend},
                \journalPubDate{}
            \end{flushleft}
        } Este plano de trabalho traz como proposta de investigação o Programa Alimentos para a Paz, entre a segunda metade da década de 1960 e meados de 1970. O objetivo do trabalho é perceber como os programas propagandeados como sendo de ajuda humanitária realizados no Sertão nordestino se tornaram decisivos na sobrevivência dos moradores da região. As consequências da seca para a população eram extremas, uma das alternativas encontradas pelos governantes foi a criação das Frentes de Emergência. Uma das políticas públicas implementadas pelo Estado através da Aliança para o Progresso, um acordo entre Brasil e Estados Unidos com a intenção de minimizar os impactos sociais decorrentes dos grandes períodos de estiagem através da doação de excedentes americanos sob a garantia de pagamento em longo prazo. O governo do estado do Rio Grande do Norte objetivava com isso, conter parte da população flagelada nos seus lugares de origem, como também, minimizar a fome e o descontrole na economia local gerado por esse fenômeno climático. Esse plano de trabalho utiliza-se de documentos como relatórios, formulários, livros e jornais, fontes que estão disponibilizados no site da Hemeroteca Digital Brasileira. Publicações dos jornais, Diário de Natal, e O Poti, com data de circulação no período exposto, nos mostram que, mais do que as notícias de progresso, calamidade e assistencialismo, também houve uma constante insatisfação a partir de fatores que desestabilizaram cada vez mais a região, seriam esses, a fome, doenças, indústria da seca e, muitas vezes a morte. Ao longo do período estudado, foram postos em prática tanto o Programa Alimentos para a Paz, como as Frentes de Trabalho, ações voltadas para os sertanejos como atividades primordiais na sobrevivência da multidão, que tinha em troca da mão de obra na construção de obras emergenciais, um pagamento e pouca alimentação a serem repassados através dos convênios firmados entre os governos do Rio Grande do Norte e o governo dos Estados Unidos da América.
    \end{galoResumo}
    \mednobreak
    \galoPalavrasChave{Sertão nordestino. Flagelados. Frentes de trabalho.}

    \section{Introdução}

    \subsection{Delimitação do objeto}

    Este projeto tem como principal característica, a investigação sobre como ocorreu a implantação do ``Programa Alimentos para a Paz'', instituído no Brasil através da Aliança para o Progresso, um programa norte americano propagandeado como sendo de ajuda humanitária e as frentes de trabalho criadas na metade da década de 1960 e meados de 1970 no Sertão do Nordeste, como possível solução para o enfrentamento das intempéries climáticas.

    Entre esse período, o Brasil vivenciou uma conjuntura política instável que se refletiu em diversas disputas que resultaram no golpe civil-militar de 1964 e na decretação do Ato Institucional nº 5 de dezembro de 1968, que consequentemente resultou no fechamento do regime. Durante todo esse contexto, os sertões atravessaram momentos de grandes dificuldades e conflitos gerados pela seca. 

    A escolha do tema surgiu na pós-graduação em História dos Sertões, quando percebemos que a criação desses programas era algo muito recorrente para amenizar a fome causada como consequência do constante agravamento das secas e suas consequências nesse Sertão. 

    Ao apresentar no título as palavras política(s) e modernização, a pesquisa aborda a intervensão dos governos norte americano, brasileiro e do Rio Grande do Norte, que se apresentavam com a proposta de modernizar o Sertão, sendo que muitas vezes essa dita modernização teve caráter emergencial. Sertão esse que o autor Antonio Carlos Robert de Moraes nos apresenta como: 

    \begin{quotation}
        Um qualitativo de lugares, um termo da geografia colonial que reproduz o olhar apropriador dos impérios em expansão. Na, verdade, tratam-se de sertões, que qualificam caatingas, cerrados, florestas, campos. Um conceito nada ingênuo, veículo de difusão da modernidade no espaço. \cite[p.~5]{Moraes2003Sertao}. 
    \end{quotation}

    Desse modo, as falas e ações políticas surgem como fundamentais para percebermos a relação de poder existente em um cenário caótico e formulador de uma história regional muitas vezes estereotipada. E, algumas vezes nos remete ao ``Discurso oculto das Lideranças'', onde \cite[p.~72]{Neves2013Discurso} nos mostra intuitos políticos bem-sucedidos a partir do aproveitamento da ingenuidade do povo. 

    Nos relatórios elaborados pelos responsáveis pelo programa, os trabalhadores aparecem como personagens extremamente satisfeitos com o lugar e ações a que foram expostos, no entanto, nos jornais da época eram frequentes os discursos sobre as péssimas condições de moradia, epidemias e riscos aos quais todos esses flagelados foram submetidos. 

    Portanto, a nossa problemática permeia nesses aspectos, investigaremos qual o verdadeiro perfil desses trabalhadores e quais as reais causas levaram ao encerramento das atividades nessas concentrações intituladas Frentes de Trabalho.

    \subsection{Discussão bibliográfica}

    A discussão bibliográfica foi feita considerando produções científicas que atenderam os critérios de elaboração dessa pesquisa. Debatemos através da obra \textit{``O governo do Monsenhor Walfredo Gurgel''} de José Daniel \textcite{Diniz2016Governo}, \textit{``O Nordeste e a Historiografia Brasileira''} \citeyear{Neves2012Nordeste} de Frederico Castro Neves, além de publicações do Jornal Diário de Natal do ano de 1966, qual foi participação de importantes políticos do período em questão como Aluízio Alves, perante a implantação do Programa Alimentos para a Paz no Nordeste.

    Outro fato importante percebido, é como ocorreu a transição de poder através da acirrada disputa política entre Monsenhor Walfredo Gurgel e Dinarte Mariz, uma vez que tal fato representou significativo episódio da história política de um Estado que foi posto em situação de calamidade não só pelos efeitos causados pelas condições climáticas incertas, mas também pelo enorme número de desempregados que se sucederam diante disso. 

    Explanaremos sobre as tentativas de solucionar tais problemas, como, por exemplo, o Encontro de Prefeitos do Seridó ocorrido em 1966, ocorrido nas cidades de Caicó e Currais Novos.  

    Discutiremos sobre a seca de 1969 ao abordar que é algo que exige, mas que ultrapassa as fronteiras naturais e nos desloca a outro cenário, muitas vezes elaborado a partir das práticas políticas e climáticas que constantemente se inserem no Sertão. As causas dessas condições reforçam a ideia de progresso, como nos mostra \textit{``Palavras que calcinam, palavras que dominam: a invenção da seca do Nordeste''}, \citeyear{AlbuquerqueJr1995Palavras} de Albuquerque Júnior, que nos levou a perceber as mudanças ocorridas mesmo que emergenciais ou de caráter paliativo, evidencia-se como se deu o processo de modificação desses espaços, muitas vezes por meio de obras governamentais. 

    Trabalhos como \textit{``A Nova Relação do Sertanejo com a Face Visível da Seca''} de José Messias Rangel e Fábio Freitas Schilling Marquesan \citeyear{RangelAndMarquesan2014Nova}, e \textit{``Populismo e Modernização no Rio Grande do Norte''} de Sérgio Luiz Bezerra \textcite{Trindade2004Aluizio}, que nos indicam que nesse período ocorreu um grande número de construção de barragens, açudes, estradas e rodovias,  nos quais nos permite abordar quem eram os trabalhadores que atuaram e se tornaram parte definitiva de paisagens locais em fase de crescimento, além da articulação governamental junto às alianças firmadas em uma tentativa de remediar os constantes transtornos causados.

    Esse período também ficou marcado por atitudes ilegais onde muitos lucravam desonestamente a partir dos recursos que deveriam ser atribuídos à assistência populacional. A autora Carla Monteiro Sales, com sua obra \textit{``Sertão Encantado: representações da paisagem nordestina no cinema da retomada''} \citeyear{Sales2014Sertao} nos levou a perceber que, esse cenário em determinado momento ficou conhecido através da mídia da época como se fosse uma característica da região, uma vez que os jornais traziam uma abordagem de que o Sertão era um local extremamente seco, ainda que isso não fosse uma realidade constante. E a autora justifica: 

    \begin{quotation}
        O Sertão nordestino é uma região com forte apelo visual, sua enunciação raramente é desassociada de um conjunto de imagens mentais que nos remete as suas principais características e compõem certa significação sobre essa porção espacial. Trata-se de uma concepção que aparece, muitas vezes, de forma enraizada, ratificando um imaginário socialmente compartilhado tão coerentemente repetido, que adquire nexos de verdade. \cite[p.~115]{Sales2014Sertao}.
    \end{quotation}

    E com isso, a ideia do Sertão geograficamente castigado e historicamente pobre deu à indústria da seca uma espécie de suporte, onde na maioria das vezes, as injeções de recursos realizadas para suprir as dificuldades, foram tão vistas como necessárias ao serem noticiadas, que se deixou de mostrar que, por trás de tudo aquilo havia personagens agindo em benefício próprio. 

    O Sertão nordestino surge como um ambiente conflituoso, os flagelados se mostravam insatisfeitos com as duras horas de trabalho, os alimentos não eram mais suficientes para saciar a fome dos trabalhadores, onde o desemprego assumiu grandes proporções, garantindo a ocupação e os meios de subsistência da população \cite[p.~33]{Duarte2002Seca}. E mesmo diante desse cenário, em \textit{``Seca e poder: entrevista com Celso Furtado''}, \citeyear{Furtado1998Seca}, Celso Furtado nos revela que essa população fragilizada via, mesmo diante da situação de escassez, uma forma de vencer a pobreza já existente e que de certo modo se alastrou nessas décadas, o que nos remete ao romantismo dito sobre os Sertões por Euclides da Cunha. 

    A partir disso uma multidão insatisfeita e temerosa pela falta de trabalho e atenção por parte das autoridades caminhava para um cenário de revoltas, através de invasões que, constantemente ocasionaram saques, como bem vimos no trabalho de Diêgo Nascimento de Souza, \textit{``Entre saques e multidões: efeitos da seca de 1953 no cenário urbano de Currais Novos''}, \citeyear{Souza2012Entre}. Pesquisa esta que, mesmo não condizendo com a temporalidade a ser abordada, nos dá uma estimativa de como a seca e suas consequências já eram consideradas um fenômeno social complexo bem antes desta discussão. 

    Relatamos também a realidade temporal de uma população doente, pessoas de todas as idades envolvidas nas frentes de trabalho, forçadas a conviver em um ambiente sem qualquer condição de higiene que fosse adequada para se viver. As más instalações de moradia e trabalho apresentaram a essas pessoas uma fome mais generalizada, sérias doenças e, em muitos casos uma morte prematura e desassistida pelas políticas públicas que, na grande maioria das vezes não conseguia conter tais problemas, fatos frequentes e constatados nos jornais da época. 

    \subsection{Justificativa}

    Nossa pesquisa justifica inicialmente, a contribuição que esperamos dar à historiografia que trata da realidade do Sertão do seridó norte-rio-grandense, apre\-sen\-tando-o em sua pluralidade, uma vez que, para muitos a palavra sertão remete a uma realidade embasada por aspectos físicos como clima e economia, ou nos aspectos simbólicos como do tipo, narrativas e identidades. Existem recortes espaciais com realidades e diferenças a serem questionadas \cite[p.~41--42]{AlbuquerqueJr2014Distante}.

    Pensar esse Sertão é, antes de tudo, compreender que tudo se trata de um espaço provido de experiências, conflitos, e disputas que, política ou popularmente arquitetam um novo rumo às mudanças que esse cenário apresenta. Boa parte da historiografia produzida academicamente ainda não se dedicou pontualmente ao que propomos, pois carece de trabalhos que se dediquem não somente ao Sertão, mas, estudos que apresentem os espaços, as ações existentes nesses lugares e, por fim, os protagonistas dessas histórias. Trabalhos sobre essa temática, podemos apontar, \textit{Criar ilhas de Sanidade: Os Estados Unidos e a Aliança Para o Progresso no Brasil (1961--1966)}, de Henrique Alonso de A. R. \textcite{Pereira2005Criar}; \textit{Caicó: uma cidade entre a recusa e a sedução}, de Juciene Andrade Felix \textcite{Andrade2007Caico}, possuindo recortes temporais anteriores ao que tratamos.

    Percebemos que, trabalhos que envolvam Sertão, convênios, frentes de trabalho e flagelados são em número muito reduzido. Nosso trabalho contribui para a sociedade de forma a expor que, o Sertão, por mais que não qualificado no ponto de vista clássico da geografia, visto apenas como um horizonte modernizável, constitui-se por suas atividades, grupos sociais e relações que o qualificam e destroem estereótipos. 

    Ao esmiuçar os conteúdos historiográficos da pesquisa nos deparamos constantemente com as ações das elites políticas tradicionais, essas, representadas através do surgimento das políticas públicas que, muitas vezes, geravam conflitos e mudanças no espaço povoado por uma classe fragilizada e dependente dessas ações, fossem elas efetuadas por meio do governo ou da igreja católica. Fato esse que faz com que esse trabalho se adeque perfeitamente aos itens que acolhem a proposta da Linha de Pesquisa I, intitulada Cultura Material, Sociedade e Poder nos Sertões. 

    É cada vez mais notável o distanciamento da palavra com o mundo e, no jornalismo regional isso tem a influência do provincianismo típico de cidades dominadas por organizações e famílias ``tradicionais''. Dessa forma, a partir da soma dos aspectos históricos, sociais e políticos, será possível retratar o panorama entre 1968 e 1976, período que apresenta falta de referência e contexto da historiografia local.

    \section{Objetivos}

    Gerais

    \begin{itemize}
        \item Analisar as políticas assistências implantadas nos sertões do Seridó-RN entre os anos de 1968--1976, destacando as ações dos Estados Unidos da América na contribuição com a distribuição de alimentos.
    \end{itemize}

    \noindent{}Específicos

    \begin{itemize}
        \item Analisar como o Programa ``Frentes de Trabalho com Alimentos para a Paz'' se estabeleceu no Sertão do Seridó-RN; 

        \item Demonstrar quais foram as reais consequências proporcionadas pelo programa para o povo mais carente da região; 
    
        \item Compreender os motivos que fizeram com que esse e outros programas fossem extintos. 
    \end{itemize}

    \section{Diálogos teóricos}

    Procurou-se utilizar um enfoque político-social, historiográfico para verificar prováveis mudanças no espaço sertanejo exercidos sobre os grupos sociais que ali viveram e quais as consequências decorrentes disto, uma vez que percebemos os espaços como produtos a partir das transformações humanas causadas por seus conflitos, dominações, resistências e negociações. 

    Dentro desse contexto, tratar da fase em que o Nordeste brasileiro e mais especificamente o Sertão sofria com as amarguras da seca e não abordar os programas que contribuíram para o que foi intitulado como ``a invenção da seca'', seria, sem dúvida, uma grande lacuna. Ainda mais quando esse período identifica muito bem a continuação do momento onde a política social tentava se fortalecer através de uma dita ``ação positiva para ajudar a América Latina'', visto também como o momento de se dar um novo esplendor à política de boa vizinhança. 

    Tal momento não se faz marcante apenas na história do Sertão, mas também na história política\footnote{Remónd conceitua a História Política que guarda ressonâncias com a que se produz nas pesquisas atuais, onde afirma o autor: ``É a história do Estado, do poder e das disputas por sua conquista e conservação, das instituições em que ele se concentrava das revoluções que o transformavam''. \cite[p.~15]{Remond2003Historia}.} do Brasil que oscilava entre um momento de extremas transições políticas e sociais. 

    Sérgio Trindade afirma que o receio da disseminação de ideias subversivas em sua área de influência geopolítica fez os Estados Unidos criarem mecanismos de auxílio às áreas subdesenvolvidas que pudessem ser alvos da presença comunista. Os fantasmas de Fidel Castro e de Ernesto Che Guevara assombravam os americanos. A Revolução Cubana criou certa desestabilização na América Latina. \cite[p.~198--199]{Trindade2004Aluizio}. 

    De acordo com Reichel, ``A criação da Aliança para o Progresso fazia parte dessa perspectiva norte-americana de frear o \textit{perigo vermelho}.'' \cite[p.~189--208]{Reichel2004Perigo}. Exposto a isso tudo, estava o Sertão, já comentado anteriormente como aquele possuidor de realidades e diferenças a serem questionadas. O Sertão não mais pode ser entendido como uma unidade homogênea, um recorte espacial presidido pela semelhança e pela identidade. \cite[p.~41--42]{AlbuquerqueJr2014Distante}. 

    O historiador tem agora a função de produzir sentidos, assim sendo, esta pesquisa tem a preocupação de situar os vestígios políticos e modernizadores que foram responsáveis pelas muitas ações e movimentos a que foram expostos os indivíduos que compunham esse espaço no recorte temporal indicado, fossem eles a elite ou as aglomerações flageladas.

    A SUDENE e USAID como instituições que atuaram sob o caráter de cooperação para o desenvolvimento, foram decisivas durante o processo de administração dos programas implementados no período em questão, não se pode descartar o fato de que nem sempre, as ações propostas funcionaram como esperado. O que acabou por causar divergências e conflitos, tanto administrativas, como populares. 

    Do ponto de vista da expansão territorial, a modernização tem dois sentidos principais: um que envolve a infraestrutura econômica, a base técnica e os meios de produção e outro que envolve os aspectos políticos e ideológicos. De acordo com \textcite{Hobsbawm1996Revolucao}, se a Revolução Industrial britânica forneceu o modelo para as fábricas, rodovias, cidades, infraestrutura, emprego das técnicas etc., a Revolução Francesa forneceu o modelo político e ideológico do processo de modernização. Para \textcite[p.~111]{Giddens1984Teoria}, ``a teoria da modernização está associada diretamente à teoria da sociedade industrial''.  

    O conceito de modernização, nesse sentido, é abrangente, já que está relacionado a um conjunto de transformações que se processam nos meios de produção, mas também na estrutura econômica, política e cultural de um território. Para se expandir espacialmente, a modernização entra no jogo dos debates teóricos e geralmente é justificada ideologicamente nas instituições acadêmicas, no universo político e nos meios de informação. Assim, modernização não se refere, única e exclusivamente, às transformações que se processam nos meios de produção e nas bases técnicas, pois envolve um conjunto de valores que, advindos de uma determinada classe social, se apresenta com forte caráter ideológico. 

    Trata-se da expansão da própria modernidade do ponto de vista territorial. No Sertão, sua expressão podia começar a ser observada na abertura de estradas, na construção dos reservatórios de água, nos sistemas de transporte, nos contrastes das cidades, etc. Nesse período jornais faziam debates em torno do que seria modernização, grande parte era concebida pelas elites como ``agentes civilizatórios'', isso tudo com a pretensão de mudar a realidade da sociedade, dando assim, ênfase às ações governamentais. 

    \section{Fontes e metodologia}

    Ao analisarmos o material memorialístico publicado sobre o período governamental da época, por exemplo, a obra de Jose Daniel Diniz \textit{``O governo do Monsenhor Walfredo Gurgel''} \citeyear{Diniz2016Governo}, nos deparamos com relatos de experiências vivenciadas durante seu governo, que ocorreu entre 1966 a 1971. Alguns desses estudos foram publicados, sobretudo por ex-auxiliares do governador, o que evidencia a necessidade de trabalhos que tragam em seu corpo, análises mais aprofundadas sobre o devido período. 

    De acordo com Paulo Antônio Rezende, a fonte jornalística permite ao historiador, além dos discursos informativos, trabalhar com anúncios que buscam seduzir e encantar os leitores \cite[p.~62]{Rezende1997Desencantos}. Assim, utilizamos para uma melhor percepção do nosso trabalho, a imprensa através de matérias do \textit{Diário de Natal}, \textit{O Poti}, \textit{RN Econômico} e \textit{Memorial da Democracia}, que nos auxiliaram a refletir sobre o período da década de 1960 e 1970 nos sertões norte-rio-grandenses, sobretudo no que se refere às ações norte-americanas voltadas para o combate à fome nos sertões. O Poti e o Diário de Natal presentes no acervo da Hemeroteca Nacional Brasileira. 

    A partir, ainda do diálogo de \textcite{Neves2012Nordeste}, com matérias dos jornais O Poti e Diário de Natal, presentes no acervo da Hemeroteca Nacional Brasileira, foram esboçados, o início do programa Alimentos Para a Paz em terras sertanejas, bem como a distribuição de alimentos vindos deste, e seu repasse para o povo por meio de órgãos ligados à Igreja Católica.  

    Para melhor discutir como ocorreu essa participação da igreja junto aos programas, utilizamos como fonte um formulário elaborado pela Cáritas Brasileira que tinha o objetivo de inscrever programas de alimentação infantil e atividades de auto ajuda ao Programa Alimentos para a Paz\footnote{O formulário datilografado encontra-se no Acervo de documentos da Paróquia da Diocese de Caicó. No primeiro andar do Centro Pastoral Dom Wagner, depositados em pastas plásticas e armários de ferro. Os mesmos não se encontram enumerados por se encontrarem em processo de catalogação. Acesso em 4 mai. 2018.}. Por fim, discutimos quais as causas que levariam ao enfraquecimento do programa e consequentemente seu fim no ano de 1974.

    Utilizamos exemplares do Diário de Natal disponíveis na Hemeroteca Digital Brasileira, além de relatórios manuscritos, pertencentes ao arquivo da Paróquia de Santana e elaborados a partir de dados funcionais dos que atuaram em algumas frentes de trabalho ou emergência, como assim está descrito nos documentos\footnote{Os documentos manuscritos (relatórios) encontram-se no Acervo de documentos da Paróquia da Diocese de Caicó. No primeiro andar do Centro Pastoral Dom Wagner, depositados em pastas plásticas e armários de ferro. Os mesmos não se encontram enumerados por se encontrarem em processo de catalogação. Acesso em 2 mai. 2018. }. Após uma breve coleta e futura seleção de dados expostos nesses itens citados, teremos uma clara demonstração de como funcionava a distribuição dos trabalhadores, quais suas funções, quanto receberiam por seu trabalho e até mesmo o perfil social destes homens. 

    As mudanças das cidades ocorreram, tanto no espaço físico como nas sensibilidades. Porém a situação de quem estava por trás desse progresso foi bem diferente daquela que estampou as páginas dos jornais, sempre reverenciando um espaço de total controle e ordem. Essas construções foram o paliativo para o sustento dos flagelados que ocupavam várias localidades do Sertão. 

    O documento expôs no topo da página o título ``Natureza do Trabalho'' e como subtítulo, ``Filosofia do Trabalho: Ser útil à comunidade''. Nesse ponto, percebe-se a introdução dos trabalhadores como agentes responsáveis por desenvolver um trabalho de participação junto ao Estado. Isso talvez como uma forma de fazer com que os mesmos desenvolvessem seu trabalho de forma mais devotada, uma vez que se sentiriam responsáveis pela melhoria das condições locais. 

    O relatório apresenta as atividades a serem gerenciadas pelos órgãos responsáveis pelas frentes de trabalho. Assim, o DER ficava responsável por desmatamento das faixas laterais das estradas; decomposição dos aterros; fazer brita para construção de pontes; elevação de aterro barragem; refazer aterros barragem arrancados e fazer paralelepípedos em tempo oportuno. O DNOCS tinha como suas funções; a limpeza e acostamento das estradas; serviço de canais; a fabricação de tijolos; calçamento das casas dos colonos, além da arborização de colônia. 

    A estrutura organizacional, desenvolvida pela classe de homens pobres, tor\-na-se vulnerável não apenas pelas condições climáticas, como também por um conjunto de fatores administrativos, que pouco a pouco demonstravam a fragilidade do programa, destacando-se a ausência de recursos suficientes e a falta de créditos assistenciais.

    \section{Cronograma}

    \begin{center}
        \begin{tabular}{ l c c c c }
            \toprule
            Atividades                 & 2020.1 & 2020.2 & 2021.1 & 2021.2 \\
            \midrule
            Cumprimento de disciplinas & x      & x      &        &        \\
            Pesquisas bibliográficas   & x      & x      &        &        \\
            Revisão bibliográfica      &        &        & x      & x      \\
            Escrita da dissertação     &        &        & x      & x      \\
            Exame de qualificação      &        &        & x      &        \\
            Reuniões de planejamento   & x      & x      & x      &        \\
            Mapeamento de fontes       &        & x      &        &        \\
            Apresentação em eventos    &        & x      & x      & x      \\
            Revisão                    &        &        &        & x      \\
            Redação final              &        &        &        & x      \\
            Defesa da dissertação      &        &        &        & x      \\
            \bottomrule
         \end{tabular}
    \end{center}

    \printbibliography[heading=subbibliography,notcategory=fullcited]

    \hfill Recebido em 30 abr. 2021.

    \hfill Aprovado em 14 mai. 2021.

    \label{chap:politicamoderend}

\end{refsection}
