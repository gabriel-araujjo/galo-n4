\chapter{Um balanço (editorial) da curta vida de Galo}

O ano de 2021 trouxe o primeiro ano de vida acadêmica da Revista Galo. Um ano de muito trabalho e conquistas significativas, nascemos em tempos difíceis, onde a maior crise sanitária do mundo acontece desde de março de 2020, nascemos no olho do furacão, em meio a maior pandemia dos último 100 anos, porém, colocamos na praça um produto científico de qualidade. 

Prezamos pela honestidade acadêmica, valorizamos os autores e prezamos pela excelência na comunicação com todas as pessoas que nos procuram. Ficamos felizes em olhar para trás e ver que construímos uma boa relação com o nosso corpo científico na figura dos editores e pareceristas \textit{ad hoc}, sem eles a tão almejada qualidade --- científica --- não seria possível.  

Quero agradecer o trabalho voluntário de Leonardo Claudiano e Gabriel Araújo, eles fazem a mágica acontecer nos bastidores, ambos são incansáveis na comunicação visual da revista, como também, no trabalho cansativo e de excelência na edição e editoração da Galo. 

Dando continuidade nesse balanço anual, quero celebrar a existência do primeiro ano festejando as nossas modestas conquistas, a exemplo: o número de ISSN, nosso portal na internet, os dois primeiros números publicados --- e citados em salas de aulas pelo Brasil a fora --- os dois anais publicados e, os dois primeiros \textit{e-books} do Selo Biblioteca Ocidente. E agora temos o prazer de lançar o dossiê organizado pelo pesquisador Thiago Torres sobre a história do Rio Grande do Norte, peço licença para anunciar os artigos e os autores das outras seções da revista, que são: \textit{Dispositivo moda: a roupa em processos artísticos contemporâneos} de Violeta Adelita Ribeiro Sutili; \textit{Melhoramentos de São Paulo: intervenções urbanas e as irmandades negras da capital} de Alvaci Mendes da Luz; \textit{A cidade em mapas, textos, imagens e imaginários} de Leonardo da Silva Claudiano; \textit{Sair da pirâmide e conhecer Além-Nilo: ensino de história do Egito Antigo na Educação Básica} de Bruno Miranda Braga. 

Na Seção “Transcrição de Documentos/Análise Iconográfica” temos uma transcrição de documento do próprio organizador deste número: \textit{Os últimos desejos de um governador português na capitania do Rio Grande do Norte: o testamento de Caetano da Silva Sanchez (1799)} de Thiago do Nascimento Torres de Paula. E por fim, temos a seção “Projeto de Pesquisa/Relatos de Pesquisa” com o texto \textit{Política(s) e Modernização: a implantação do programa “Alimentos para a paz” e as frentes de trabalho no Sertão do Seridó-RN (1968--1976)} do mestrando João Paulo de Lima Silva. 

Como disse o organizador deste número anteriormente, mesmo em tempos mais difíceis como os que estamos vivendo, eu, como editor, e em nome de toda a equipe editorial da Revista Galo, desejo que todos e todas tirem um tempo e possam se debruçar nas páginas da Galo.

\medskip

Uma ótima leitura!

\bigskip

Editor \textsc{Franciscv} Isaac D. de Oliveira

\bigskip

\hfill 04/05/2021.