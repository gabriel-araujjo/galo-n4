\chapter{Editorial}

O Dossiê destina-se à divulgação e disseminação de estudos oriundos de
pesquisas realizadas no programa de pós-graduação \textit{lato sensu} em nível
de especialização do Instituto de Educação Superior Presidente Kennedy ---
IFESP, com foco nas discussões, estudos e pesquisas realizadas em torno das
práticas e ensino na Educação Básica.

Assim, \textit{Formação docente: aproximações entre a pesquisa e o ensino na
educação básica} é uma iniciativa do Núcleo de Estudos, Pesquisa e Extensão ---
NEPE, em reunir estudos concluídos nos três últimos anos com temáticas
desenvolvidas nos seis cursos de especialização: Gestão de Processos
Educacionais; Educação Infantil; Ensino de Língua Portuguesa; Educação
Ambiental; Educação de Jovens e Adultos: saberes e práticas na formação
docente; Educação Matemática: teoria e prática no ensino fundamental.

Nossos cursos atendem à uma demanda da formação continuada de professores e
gestores da rede pública de ensino da Educação Básica do estado do Rio Grande
do Norte. E nesse sentido, as pesquisas desenvolvidas têm como foco discutir
problemas relacionados à gestão escolar, à formação de professores, ao ensino,
à aprendizagem e às políticas públicas de promoção à educação.

O dossiê \textit{Formação docente: aproximações entre a pesquisa e o ensino na
educação básica} está composto por 14 capítulos. O primeiro capítulo,
\textit{“Leitura literária e o leitor na era digital”} de autoria de
Delcimar Francisco de Medeiros e Arandi Róbson Martins
Câmara, traz reflexões sobre a leitura literária nos ambientes tecnológicos da
informação e as práticas de leitura na formação do leitor na perspectiva do
letramento digital e literário.

No segundo capítulo, \textit{“O olhar da gestão frente à importância da leitura
compreensiva nos anos finais do ensino fundamental”} de autoria de José
de Arimatéia da Paz Albuquerque e Rozicleide Bezerra de Carvalho, o
objetivo foi apresentar de que forma a gestão de uma escola pública compreende
a importância da leitura compreensiva como responsabilidade das áreas de
conhecimento dos anos finais do Ensino Fundamental.

O terceiro capítulo, \textit{“Contribuições do software GeoGebra para
o ensino aprendizagem de função polinomial do 2º grau”} de Sarah Mara
Silva Leôncio e Maria José Lima dos Santos, discute as
potencialidades do software GeoGebra como recurso de ensino e
aprendizagem na construção de gráficos da função polinomial do 2º grau e a
compreensão dos conceitos matemáticos de forma motivadora, criativa e dinâmica.

O quarto capítulo, \textit{“Matemática na arte: uma proposta interdisciplinar
para o ensino de geometria”} dos autores Charles Ricardo Lemos de Melo
e Wguineuma Pereira Avelino Cardoso, apresenta uma reflexão de como a
Arte pode contribuir no ensino e aprendizagem da Geometria numa proposta
interdisciplinar de ensino que utiliza imagens de telas de pintura a óleo como
veículo de investigação dos elementos geométricos presentes nestas.

O quinto capítulo, \textit{“O material dourado como recurso didático no ensino
do algoritmo da subtração: uma experiência em sala de aula”} de autoria de
Verônica Umbelino Souza de Carvalho e Anilda Pereira da Silva
Guimarães, contribui para os debates sobre a importância dos materiais
manipuláveis, em particular o Material Dourado, para uma melhor compreensão nos
aspectos ensino/aprendizagem da Matemática. De modo a facilitar a compreensão
do desagrupamento na subtração e relações numéricas abstratas.

No sexto capítulo, \textit{“A utilização de modelos e analogias como
estratégias de ensino e aprendizagem nas aulas de química”} as autoras
Keila Barbosa da Fonseca e Lorena Gadelha de Freitas Brito
refletem sobre uma proposta didática com uso de modelos e analogias como
estratégias de ensino podem contribuir com a aprendizagem em diversos aspectos,
proporcionando em sala de aula momentos de reflexão, discussão e participação
sobre os fenômenos que compõem a natureza.

O sétimo capítulo, \textit{“Uma reflexão diante do pedagogo empresarial como
gestor da ouvidoria”} de autoria de Tâmara Maria Soares de Medeiros de
Cavalcanti e Ilsa Fernandes de Queiróz, discute sobre a atuação do Pedagogo em
um ambiente empresarial na gestão da Ouvidoria de uma Instituição Financeira
Pública. Apresenta os resultados de uma pesquisa de campo por meio de
observação nos canais de comunicação da ouvidoria, aponta a importância da
presença do pedagogo contribuindo para um trabalho mais humanizado, com
possibilidades de análises mais seguras, buscando colocar à disposição do
cidadão um serviço de qualidade.

No oitavo capítulo, \textit{“Reflexões sobre o fazer pedagógico dos professores
do Centro Estadual de Educação Profissional Dr. Ruy Pereira dos Santos”} das
autoras Roseane Idalino da Silva e Mariza Silva de Araújo, aponta para um fazer
pedagógico que busca atender a uma dimensão do trabalho, da ciência, da cultura
e da tecnologia, visando desenvolver um trabalho que pense na perspectiva da
formação humana integral, contribuindo significativamente nesse processo com
ações de reflexão e vivências diferenciadas no espaço escolar.

No nono capítulo, \textit{“Currículo na Educação Infantil: desafios encontrados
pelos professores”} as autoras Josefa Maria de Medeiros e Antônia Zélia de
Assis Dantas, discutem referenciais bibliográficos e apresentam resultados de
uma pesquisa de Campo, que procuram conhecer os principais desafios encontrados
quanto à elaboração de currículos no seu nível de concretização da sala de
aula. Os estudos realizados apontam a necessidade de dar voz às crianças nessa
construção e a necessidade de repensar as posturas dos professores.

O décimo primeiro capítulo \textit{“A influência do lúdico no processo de
aprendizagem na educação infantil”} de Conceição de Maria Gomes da Silva e
Nednaldo Dantas dos Santos, objetiva expor um estudo sobre a influência de
atividades lúdicas no desenvolvimento integral da criança que se encontra na
Educação Infantil. Conforme o discurso das crianças envolvidas no estudo, as
atividades proporcionam satisfação e alegria no processo de aprendizagem.

O décimo segundo capítulo \textit{“A concepção de pais e responsáveis por
crianças de dois e três anos acerca do brincar na educação infantil”} das
autoras Erineide Andrea Machado Dantas da Silva e Tereza Cristina Bernardo da
Câmara, revela os resultados de uma pesquisa em que os pais e responsáveis por
crianças demonstram entender e valorizar a vivência das brincadeiras no
processo de aprendizagem e desenvolvimento de seus filhos, o que surpreende
essas pesquisadoras, visto que na prática cotidiana da escola esses mesmos pais
muitas vezes tratam das questões pedagógicas e avaliam o brincar como mero
passa-tempo.

O décimo terceiro capítulo \textit{“Educação Ambiental nos livros de geografia:
uma análise temática”} de Josemar Souza de Lima e Dayanne Chianca de Moura traz
um levantamento envolvendo uma amostragem de nove volumes de livros didáticos
de geografia do PNLD, triênio 2018--2020 cujo objetivo foi analisar quanto e
como é abordada a temática ambiental. O estudo pôde apresentar um panorama das
questões ambientais em uma disciplina que a priori está diretamente interligada
com o meio ambiente em suas várias vertentes de estudo e discussões.

O décimo quarto capítulo \textit{“A importância do espaço escolar para o
ensino-aprendizagem a luz de um estudo de caso”} de autoria de Ângela Maria
Paulo e José Avelino da Hora Neto, apresenta um estudo de caso em que analisa
as noções sobre o conceito de espaço físico escolar e como se percebe esse
espaço e a influência exercida por ele na cultura organizacional, na gestão
escolar e no ambiente escolar.

As organizadoras e autores(as) reconhecem a importância da multiplicidade de
olhares em torno das pesquisas realizadas favorecendo a ampliação das
discussões e reflexões em torno das temáticas apresentadas nos estudos. Boa
leitura!

\bigskip

\hfill Natal, dezembro de 2021.
