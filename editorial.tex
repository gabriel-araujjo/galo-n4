\chapter{Editorial}

O Rio Grande encontra-se elencado entre as regiões mais antigas de colonização europeia das Américas. Ainda no século XVI recebeu em seu litoral marco de posse da Coroa portuguesa, já no século XVII foi território da América holandesa, chegando ao século XVIII como teatro do extermínio indígena durante a Guerra dos Bárbaros. Em meados dos anos setecentistas a região foi nomeada \textit{Rio Grande do Norte}, consequência dos acordos diplomáticos entre as Coroas ibéricas que disputavam territórios no sul do continente. 

Apesar de não ter sido um produtor de açúcar em larga escala e não possuir um porto atlântico, o Rio Grande do Norte conheceu a pratica da escravidão africana no litoral e nos sertões, sob o domínio administrativo e judiciário de outras capitanias a região supracitada esteve representada em sedições liberais como, a Revolução Pernambucana de 1817 e a Confederação do Equador em 1824. Revoltas de cunho republicano que se consubstanciaram diferentemente da afamada Inconfidência Mineira de 1789.  

Durante o agitado e conflituoso século XIX, a terra onde nasceu Dom Antônio Felipe Camarão, vivenciou epidemias de cólera que assolavam todo o território imperial do Brasil, além do surgimento dos cemitérios, a construção da torre da igreja matriz da cidade do Natal, o desenvolvimento da cotonicultura e da malha ferroviária. Para mais, o fluxo de miseráveis que fugiam das secas nos sertões em direção ao litoral, seguido dos motins da fome na povoação de Mossoró. Neste último caso, era o povo depauperado do Rio Grande do Norte demonstrando sua capacidade de organização, pressão e negociação com os representantes do Imperador Dom Pedro II. 

Os anos oitocentistas apagariam suas luzes com a proclamação da República (1889), que traria o sol da modernidade para província do Rio Grande do Norte, especificamente para capital: o teatro, o bonde elétrico, a ponte metálica e expansão planejada da Cidade do Natal, mas sobretudo a multiplicação dos pobres, dos miseráveis e excluídos descendentes escravos e de indígenas secularmente explorados, que se espalhariam pelos morros, arrabaldes e periferias por todo século XX. 

A partir desta longa história marcada por ocupações, conflitos, exploração e resistência, somada as múltiplas possibilidades de interpretações do passado humano, vinculadas a novas metodologias e propostas teóricas, é que a Revista Galo brinda o público leitor com o Dossiê: História do Rio Grande do Norte: temporalidades e interpretações. Dentre as publicações que integram a coletânea em tela, destaca-se os seguintes artigos: \textit{O patrimônio da Companhia de Jesus na Capitania do Rio Grande do Norte: bens como sustento da fé (1600--1759)} de Ana Lunara da Silva Morais; \textit{Farinha e Carne no Sertão. Fome e carestia no litoral: aspectos do mercado interno no Rio Grande do Norte (séc. XVIII a XIX)} do professor Thiago Alves Dias; \textit{Não ao peso, não ao recrutamento: os Quebra-quilos e as autoridades públicas no Rio Grande do Norte (1874--1875)} de João Fernando Barreto de Brito; \textit{De como as letras formam um cidadão: os ritos e símbolos da Primeira República na cidade de Parelhas-RN (1928--1930)} de Laísa Fernanda Santos de Farias e Sebastião Genicarlos.

Quanto ao Rio Grande do Norte contemporâneo, realça-se: \textit{A institucionalização da matriz de Santa Luzia na cidade de Mossoró-RN} de Arthur Ebert Dantas dos Santos, Jackson Luiz Fernandes Adelino, Lara Raquel de Souza e Maia e Valdeci dos Santos Júnior; \textit{“Vencido o \emph{new look}”: resistências femininas a Christian Dior e as suas modas (Natal/RN, 1948--1953)} de João Vieira Neto e Joel Carlos de Souza Andrade; \textit{Os caminhos e os desdobramentos da vida: trajetória política e dos discursos e pronunciamentos de Dinarte Mariz} de Larisse Santos Bernardo e Jailma Maria de Lima; \textit{Frentes de trabalho e Ligas camponesas: movimentos populares, conflitos e sobrevivência (1960--1976)} de João Paulo de Lima Silva;  

Por fim, \textit{O Diário de Natal: o papel da imprensa potiguar na circulação das notícias do Projeto Baixo-Açu (1975--1979)} de Maiara Brenda Rodrigues de Brito; \textit{Comemorar a posse de Thomaz de Araújo: a construção de um lugar para o \emph{Seridó} na memória histórica do Rio Grande do Norte} de Bruno Balbino Aires da Costa e \textit{O contexto sobre o uso de substâncias lícitas e ilícitas em Caicó, Rio Grande do Norte} de Allyson Iquesac Santos de Brito e Helder Alexandre Medeiros de Macedo; e o último texto dentro do dossiê é \textit{O silêncio dos caboclos: Notas sobre catimbozeiros perseguidos no Rio Grande do Norte} de Rômulo Henrique P. Angélico. 

Os textos do atual conjunto, falam um pouco da produção historiográfica e das múltiplas visões dos historiadores e historiadoras sobre a nossa história. A cabo de tudo, apesar das contingencias impostas ao povo brasileiro, deseja-se uma excelente leitura a todos. 

\bigskip

Thiago do Nascimento Torres de Paula

\bigskip

\hfill Natal, maio de 2021.
